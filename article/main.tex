\documentclass[a4paper,12pt]{article}
\usepackage[brazil, english]{babel}
\usepackage[utf8]{inputenc}
\usepackage[T1]{fontenc}
\usepackage{geometry}
\usepackage{setspace}
\usepackage{titlesec}
\usepackage{hyperref}
\usepackage{graphicx}
\usepackage{caption}
\usepackage{subcaption}
\usepackage{fancyhdr}

\hypersetup{
    colorlinks=true,% make the links colored
    linkcolor=blue
}

\usepackage{setspace}
\setstretch{1.2} % Altere o valor 1.2 para o valor desejado

\newcommand{\printingbibliography}{%

    \pagestyle{myheadings}
    \markright{}
    \sloppy
    \printbibliography[heading=bibintoc, % add to table of contents
                   title=Bibliography % Chapter name
                  ]
    \fussy%
}
\PassOptionsToPackage{table}{xcolor}

\pagestyle{fancy}
\fancyhf{}
\renewcommand{\headrulewidth}{0pt}
\fancyhead[R]{\thepage}

\geometry{a4paper,top=30mm,bottom=20mm,left=30mm,right=20mm}

\titleformat*{\section}{\bfseries\large}
\titleformat*{\subsection}{\bfseries\normalsize}

\title{\large \textbf{A Influência da Escola Austríaca no Mercado Financeiro: 
Um Estudo sobre o Comportamento na Decisão de Investimentos}}
\author{ANDRÉ VIEIRA DA SILVA}
\date{\today}

\begin{document}

\maketitle

\selectlanguage{brazil}

\begin{abstract}
Este estudo examina o impacto da Escola Austríaca de Economia no mercado financeiro,
com foco na compreensão do comportamento dos investidores. A Escola Austríaca, 
conhecida por seus princípios de individualismo e livre mercado, desempenhou um papel 
significativo na teoria econômica. Este estudo analisa como os conceitos e ideias da 
Escola Austríaca influenciam as decisões de investimento, levando em consideração 
fatores como preferências de curto prazo, incerteza e racionalidade limitada. 
Além disso, esta pesquisa investiga como as teorias austríacas se relacionam com o 
ambiente atual do mercado financeiro e como os investidores incorporam ou modificam 
essas teorias em suas estratégias. A análise é conduzida por meio de revisão bibliográfica, 
estudos de caso e análise de dados históricos. Os resultados indicam que a Escola Austríaca 
exerce influência notável no mercado financeiro, moldando a visão dos investidores sobre a 
economia e as políticas governamentais. Entender essa influência é fundamental para 
compreender o comportamento dos investidores e suas decisões no mercado financeiro. Esta pesquisa 
contribui para a compreensão mais profunda das dinâmicas entre a teoria econômica e 
a prática dos investimentos.
\textbf{PALAVRAS CHAVE: ...}.
\end{abstract}

\selectlanguage{english}
\begin{abstract}
This research sheds light on the profound impact of the Austrian School of Economics 
on the financial landscape, particularly by elucidating investor behavior. The Austrian School, 
renowned for its principles of individualism and free-market ideology, has played a pivotal 
role in shaping economic theory. This study meticulously scrutinizes how the tenets and ideologies 
espoused by the Austrian School reverberate within the realm of investment decisions.
The examination delves into various facets of this influence, considering factors such as 
investors' short-term preferences, the pervasive uncertainty that characterizes financial 
markets, and the acknowledgment of human limitations in rational decision-making processes. 
Furthermore, this research probes into the dynamic interplay between Austrian economic theories 
and the contemporary financial market milieu. Intriguingly, the study investigates how investors 
assimilate, adapt, or even amend Austrian economic theories to craft their investment strategies. 
This intricate relationship between theory and practice is unveiled through a multifaceted 
methodology that encompasses exhaustive literature reviews, comprehensive case studies, 
and meticulous analysis of historical data. The culmination of this extensive inquiry reveals 
a conspicuous and lasting impact of the Austrian School on the financial landscape. Austrian economic 
theories continue to mold investors' perspectives on economic trends, government policies, and their 
consequent investment choices. Recognizing and comprehending this enduring influence becomes paramount 
for any comprehensive understanding of investor behavior and decision-making in the complex 
and ever-evolving financial markets. 
\textbf{KEYWORDS: Austrian School of Economics}.
\end{abstract}

\selectlanguage{brazil}

%\hypersetup{linkcolor=blue}
\tableofcontents

\section{Introdução}
\hspace{0.5cm}Os investidores tomam decisões complicadas no mercado financeiro, que afeta an economia 
em geral e seus próprios interesses. Em tal situação, as perspectivas e estratégias dos 
investidores são fortemente influenciadas pelas teorias econômicas. A Escola Austríaca de 
Economia é uma das teorias mais conhecidas por seu foco no individualismo, no livre mercado 
e em uma compreensão única dos princípios econômicos.

Este estudo examina como a Escola Austríaca de Economia afeta o comportamento dos 
investidores e aumenta nossa compreensão do mercado financeiro. Desde seus fundadores, 
Ludwig von Mises e Friedrich Hayek, an Escola Austríaca tem influenciado a teoria econômica 
e a política econômica. Os princípios que sustentam têm impactado o mercado financeiro, 
afetando as escolhas de investimento, as perspectivas econômicas e as políticas governamentais.

No decorrer deste estudo, exploraremos como os conceitos e ideias da Escola Austríaca 
permeiam as decisões de investimento, considerando fatores como as preferências de curto prazo, 
a incerteza inerente ao mercado e a racionalidade limitada dos investidores. Além disso, 
investigaremos como as teorias austríacas se relacionam com o ambiente contemporâneo do 
mercado financeiro e como os investidores incorporam ou adaptam essas teorias em suas estratégias.

Usaremos uma abordagem multidisciplinar para atingir esses objetivos. Isso inclui revisões 
bibliográficas aprofundadas, análises de estudos de caso representativos e análises detalhadas 
de dados históricos. Esperamos que este estudo possa auxiliar na nossa compreensão da complexa relação 
entre a prática dos investimentos e a teoria econômica. Ao mesmo tempo, também queremos enfatizar 
o papel constante e significativo da Escola Austríaca de Economia no mundo financeiro atual.

\subsection{Escola Austríaca}
\end{document}
