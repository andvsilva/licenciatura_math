\documentclass[a4paper,12pt]{article}
\usepackage[brazil, english]{babel}
\usepackage[utf8]{inputenc}
\usepackage[T1]{fontenc}
\usepackage{geometry}
\usepackage{setspace}
\usepackage{titlesec}
\usepackage{hyperref}
\usepackage{graphicx}
\usepackage{caption}
\usepackage{subcaption}
\usepackage{fancyhdr}

\hypersetup{
    colorlinks=true,% make the links colored
    linkcolor=blue
}

\usepackage{setspace}

\newcommand{\printingbibliography}{%

    \pagestyle{myheadings}
    \markright{}
    \sloppy
    \printbibliography[heading=bibintoc, % add to table of contents
                   title=Bibliography % Chapter name
                  ]
    \fussy%
}
\PassOptionsToPackage{table}{xcolor}

\pagestyle{fancy}
\fancyhf{}
\renewcommand{\headrulewidth}{0pt}
\fancyhead[R]{\thepage}

\geometry{a4paper,top=30mm,bottom=20mm,left=30mm,right=20mm}

\titleformat*{\section}{\bfseries\large}
\titleformat*{\subsection}{\bfseries\normalsize}

\title{ \textbf{\large Monitoramento e Análise das Métricas da Blockchain do BTC}}
\author{ANDRÉ VIEIRA DA SILVA}
\date{\today}

\begin{document}

\maketitle

\selectlanguage{brazil}

\begin{abstract}
\textbf{PALAVRAS CHAVE: }.
\end{abstract}

\selectlanguage{english}
\begin{abstract}

\textbf{KEYWORDS: A}.
\end{abstract}

\selectlanguage{brazil}

\newpage
%\hypersetup{linkcolor=blue}
\tableofcontents

\setstretch{1.3} % Altere o valor 1.2 para o valor desejado

\section{Introdução}
\hspace{0.5cm}A blockchain do Bitcoin(BTC) é uma rede descentralizada que permite 
a realização de transações financeiras sem a necessidade de intermediários. 
Para monitorar e analisar as métricas da blockchain do BTC, é possível utilizar 
as métricas de blockchain, que são dados brutos on-chain extraídos de APIs que 
mostram as atividades em uma blockchain.

As quatro métricas cruciais de blockchain na avaliação de criptomoedas e 
blockchain são: taxa de hash, endereços ativos, valores de transação e taxas.
A taxa de hash, ou hash rate, de uma blockchain é o poder computacional 
combinado que os mineradores de criptomoedas usaram para realizar cálculos 
em uma blockchain de Prova de Trabalho (PoW) a fim de produzir novos blocos, 
ou seja, minerar novos tokens.

Além disso, a análise on-chain consiste em observar a Blockchain e os 
movimentos dos usuários para obter informações, insights e indicadores 
sobre o preço de um criptoativo. 
Por exemplo, a Glassnode fornece o Bitcoin: Balanced Price, que 
"representa a diferença entre o preço realizado e o preço transferido" do BTC. 

O indicador fornece um valor nominal para o Bitcoin, e serve como um norte para 
saber se o ativo está próximo ou não do seu "valor justo".
Outro indicador útil é a Dominância do Bitcoin, que mede a participação total 
do BTC em relação a todo o mercado de criptoativos.

A métrica é calculada dividindo o valor da capitalização do Bitcoin pela 
capitalização total das maiores criptomoedas, multiplicado por 100.
Um estudo recente da MIT Sloan mostrou que a realidade da rede Bitcoin 
é diferente do modelo idealizado e descentralizado que é frequentemente 
apresentado pelos entusiastas de criptomoedas. O estudo revelou que a 
rede ainda é dominada por grandes jogadores concentrados, e que a estrutura 
dos principais participantes da rede é diferente do que se imagina.

Em resumo, a análise das métricas da blockchain do BTC é uma ferramenta 
importante para entender o comportamento da rede e dos usuários, bem como 
para avaliar o valor do Bitcoin e outras criptomoedas. As métricas de blockchain 
e a análise \textit{on-chain} podem fornecer informações valiosas para investidores e 
entusiastas de criptomoedas.

\end{document}
