\documentclass[a4paper,12pt]{article}
\usepackage[brazil, english]{babel}
\usepackage[utf8]{inputenc}
\usepackage[T1]{fontenc}
\usepackage{geometry}
\usepackage{setspace}
\usepackage{titlesec}
\usepackage{hyperref}
\usepackage{graphicx}
\usepackage{caption}
\usepackage{subcaption}
\usepackage{fancyhdr}

\hypersetup{
    colorlinks=true,% make the links colored
    linkcolor=blue
}

\usepackage{setspace}

\newcommand{\printingbibliography}{%

    \pagestyle{myheadings}
    \markright{}
    \sloppy
    \printbibliography[heading=bibintoc, % add to table of contents
                   title=Bibliography % Chapter name
                  ]
    \fussy%
}
\PassOptionsToPackage{table}{xcolor}

\pagestyle{fancy}
\fancyhf{}
\renewcommand{\headrulewidth}{0pt}
\fancyhead[R]{\thepage}

\geometry{a4paper,top=30mm,bottom=20mm,left=30mm,right=20mm}

\titleformat*{\section}{\bfseries\large}
\titleformat*{\subsection}{\bfseries\normalsize}

\title{ \textbf{\large A Influência da Escola Austríaca no Mercado Financeiro} \\
\large Um Estudo sobre o Comportamento na Decisão de Investimentos}
\author{ANDRÉ VIEIRA DA SILVA}
\date{\today}

\begin{document}

\maketitle

\selectlanguage{brazil}

\begin{abstract}
Este estudo examina o impacto da Escola Austríaca de Economia no mercado financeiro,
com foco na compreensão do comportamento dos investidores. A Escola Austríaca, 
conhecida por seus princípios de individualismo e livre mercado, desempenhou um papel 
significativo na teoria econômica. Este estudo analisa como os conceitos e ideias da 
Escola Austríaca influenciam as decisões de investimento, levando em consideração 
fatores como preferências de curto prazo, incerteza e racionalidade limitada. 
Além disso, esta pesquisa investiga como as teorias austríacas se relacionam com o 
ambiente atual do mercado financeiro e como os investidores incorporam ou modificam 
essas teorias em suas estratégias. A análise é conduzida por meio de revisão bibliográfica, 
estudos de caso e análise de dados históricos. \newline
\textbf{PALAVRAS CHAVE: Escola Austr\'iaca de economia, mercado financeiro, liberdade}.
\end{abstract}

\selectlanguage{english}
\begin{abstract}
This research sheds light on the profound impact of the Austrian School of Economics 
on the financial landscape, particularly by elucidating investor behavior. The Austrian School, 
renowned for its principles of individualism and free-market ideology, has played a pivotal 
role in shaping economic theory. This study meticulously scrutinizes how the tenets and ideologies 
espoused by the Austrian School reverberate within the realm of investment decisions.
The examination delves into various facets of this influence, considering factors such as 
investors' short-term preferences, the pervasive uncertainty that characterizes financial 
markets, and the acknowledgment of human limitations in rational decision-making processes. 
Furthermore, this research probes into the dynamic interplay between Austrian economic theories 
and the contemporary financial market milieu. Intriguingly, the study investigates how investors 
assimilate, adapt, or even amend Austrian economic theories to craft their investment strategies. 
This intricate relationship between theory and practice is unveiled through a multifaceted 
methodology that encompasses exhaustive literature reviews, comprehensive case studies, 
and meticulous analysis of historical data.\newline
\textbf{KEYWORDS: Austrian School of Economics, Financial Market, Liberty}.
\end{abstract}

\selectlanguage{brazil}

\newpage
%\hypersetup{linkcolor=blue}
\tableofcontents

\setstretch{1.3} % Altere o valor 1.2 para o valor desejado

\section{Introdução}
\hspace{0.5cm}Os investidores tomam decisões complicadas no mercado financeiro, que afeta a economia 
em geral e seus próprios interesses. Em tal situação, as perspectivas e estratégias dos 
investidores são fortemente influenciadas pelas teorias econômicas. A Escola Austríaca de 
Economia é uma das teorias mais conhecidas por seu foco no individualismo, no livre mercado 
e em uma compreensão única dos princípios econômicos.

Este estudo examina como a Escola Austríaca de Economia afeta o comportamento dos 
investidores e aumenta nossa compreensão do mercado financeiro. Desde seus fundadores, Carl Menger, 
Murray Rothbard, Ludwig von Mises e Friedrich Hayek, a Escola Austríaca tem influenciado a teoria econômica 
e a política econômica. Os princípios que sustentam têm impactado o mercado financeiro, 
afetando as escolhas de investimento, as perspectivas econômicas e as políticas governamentais.

No decorrer deste estudo, exploraremos como os conceitos e ideias da Escola Austríaca 
permeiam as decisões de investimento, considerando fatores como as preferências de curto prazo, 
a incerteza inerente ao mercado e a racionalidade limitada dos investidores. Além disso, 
investigaremos como as teorias austríacas se relacionam com o ambiente contemporâneo do 
mercado financeiro e como os investidores incorporam ou adaptam essas teorias em suas estratégias.

Usaremos uma abordagem multidisciplinar para atingir esses objetivos. Isso incluir\'a revisões 
bibliográficas aprofundadas, análises de estudos de caso representativos e análises detalhadas 
de dados históricos. Esperamos que este estudo possa auxiliar na nossa compreensão da complexa relação 
entre a prática dos investimentos e a teoria econômica. Ao mesmo tempo, também queremos enfatizar 
o papel constante e significativo da Escola Austríaca de Economia no mundo financeiro atual.

\subsection{Escola Austríaca}
\hspace{0.5cm}A Escola Austríaca de Economia está entre as escolas de pensamento econômico mais notáveis e
distintas da história. Ela surgiu na Áustria no final dos anos 1800 e início dos 1900s e é 
conhecida por seus métodos oposicionistas e sua ênfase na liberdade individual, mercado livre 
e ações humanas.

A Escola Austríaca enfatiza o indivíduo como uma unidade básica de análise econômica, ao contrário 
das abordagens tradicionais. Eles reconhecem que as escolhas e as percepções subjetivas de cada indivíduo 
desempenham um papel importante na formação dos fenômenos econômicos. O foco desta escola de pensamento 
é na liberdade individual, no mercado livre e na crítica à intervenção do governo na economia.

Este artigo examina a Escola Austríaca de Economia em detalhes. Examinaremos sua abordagem à teoria do valor,
sua percepção dos ciclos econômicos, seu compromisso com o liberalismo clássico, sua crítica ao socialismo e
a limitação do papel do Estado na economia. Além disso, investigaremos os meios pelos quais as ideias da Escola 
Austríaca continuam a influenciar o pensamento econômico moderno e o impacto das políticas públicas em todo o 
mundo.

A visão da Escola Austríaca sobre a economia é única e muitas vezes provocadora, compreender suas teorias e 
contribuições é essencial para uma apreciação mais profunda da variedade de pensamento econômico que enriquece 
o cenário acadêmico e político global. Solicitamos que você se junte a nós nesta viagem intelectual para descobrir 
os fundamentos e os ensinamentos eternos dessa escola distinta de pensamento econômico.

\subsection{Princ\'ipios Econ\^omicos}

\hspace{0.5cm}A abordagem única e impactante da Escola Austríaca de Economia é moldada por uma coleção de 
princípios econômicos fundamentais. A compreensão dos mercados, o comportamento humano e o papel do Estado 
na economia estão intimamente ligados an esses princípios.

\subsubsection{Valor Subjetivo}
\subsubsection{Ação Humana}
\subsubsection{Mercados Livres }
\subsubsection{Ciclos Econômicos } 
\subsubsection{Limitação do Papel do Estado}
\subsubsection{Propriedade Privada}
\subsubsection{Crítica ao Socialismo }
\subsubsection{Ordem Espontânea}

\hspace{0.5cm}A Escola Austríaca de Economia foi construída sobre esses fundamentos econômicos, que ainda têm um impacto 
na filosofia política e econômica em todo o mundo. Eles enfatizam que, para uma análise econômica bem-sucedida,
a compreensão das complexidades da ação humana, a liberdade individual e o mercado livre são essenciais.

\section{Escola Austr\'iaca aplicada ao Mercado Financeiro}

\section{Metodologia de An\'alise dos Dados Econ\^omicos}

\section{An\'alise de Sentimento dos Dados das Atividades Econ\^omicas}

\section{Resultados}

\section{Conclus\~oes}

\end{document}
