%%%%%%%%%%%%%%%%%%%%%%%%%%%%%%%%%%%%%%%%%%%%%%%%%%
% These are some new commands that may be useful 
% for paper writing in general. If other new commands
% are needed for your specific paper, please feel 
% free to add here. 
%
% The currently available commands are organized in: 
% 1) Systems
% 2) Quantities
% 3) Energies and units
% 4) particle species
% 5) Colors package
% 6) hyperlink
%%%%%%%%%%%%%%%%%%%%%%%%%%%%%%%%%%%%%%%%%%%%%%%%%%

\usepackage{amsmath}
\usepackage{amssymb}
\usepackage{upgreek}
\usepackage{multirow}
\usepackage{setspace}% http://ctan.org/pkg/setspace
\usepackage{fancyhdr}
\usepackage{datetime}

% 1) SYSTEMS
\newcommand{\btc}               {\textbf{BTC}}
\newcommand{\btcspace}          {\textbf{BTC} }
\newcommand{\pow}               {\textbf{PoW}}

% 4) definition to references, biblatex and hyperlink
\usepackage[backend=bibtex, 
style=nature,  %style reference.
sorting=none,
firstinits=true %first name abbreviate
]{biblatex}

\usepackage{hyperref}
\hypersetup{
    colorlinks=true, %set "true" if you want colored links
    linktoc=all,     %set to "all" if you want both sections and subsections linked
    linkcolor=blue,  %choose some color if you want links to stand out
    citecolor= blue, % color of \cite{} in the text.
    urlcolor  = blue, % color of the link for the paper in references.
}

% 5) Tikz and figures
\usepackage{epsfig}
\usepackage{lmodern}
\usepackage{mathtools}
\usepackage[utf8]{luainputenc}
\usepackage{xspace}
\usepackage{tikz}
\usepackage{pgfplots}
\pgfplotsset{compat=newest}

\usetikzlibrary{positioning}
\usepackage{subcaption}

% 6) colors:
\usepackage{xcolor}
\definecolor{ao(english)}{rgb}{0.0, 0.5, 0.0} % dark green

% 7) Add lines numbers
%\usepackage{lineno}

% add pdf file to thesis:
\usepackage{pdfpages}