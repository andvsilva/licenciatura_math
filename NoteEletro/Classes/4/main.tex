\documentclass[a4paper,12pt]{article}
\usepackage[brazil, english]{babel}
\usepackage[utf8]{inputenc}
\usepackage[T1]{fontenc}
\usepackage{geometry}
\usepackage{setspace}
\usepackage{titlesec}
\usepackage{hyperref}
\usepackage{graphicx}
\usepackage{caption}
\usepackage{subcaption}
\usepackage{fancyhdr}
\setlength{\headheight}{15pt}
\addtolength{\topmargin}{-2.5pt}
\usepackage{xcolor}
\usepackage{amsmath, amssymb, bm}
\usepackage{mathtools}
\usepackage{cancel}
\usepackage{tikz}
\usepackage{newunicodechar}
\usepackage{ragged2e}
\usepackage{setspace}
\usepackage{tikz-3dplot} % Necessário para coordenadas 3D
\usetikzlibrary{intersections}
\usepackage{siunitx}
\usetikzlibrary{3d, arrows.meta}

\usepackage{color}
\definecolor{myblue}{rgb}{.8, .8, 1}

\usepackage{amsmath}
\usepackage{empheq}

\newlength\mytemplen
\newsavebox\mytempbox

\makeatletter
\newcommand\mybluebox{%
    \@ifnextchar[%]
       {\@mybluebox}%
       {\@mybluebox[0pt]}}

\def\@mybluebox[#1]{%
    \@ifnextchar[%]
       {\@@mybluebox[#1]}%
       {\@@mybluebox[#1][0pt]}}

\def\@@mybluebox[#1][#2]#3{
    \sbox\mytempbox{#3}%
    \mytemplen\ht\mytempbox
    \advance\mytemplen #1\relax
    \ht\mytempbox\mytemplen
    \mytemplen\dp\mytempbox
    \advance\mytemplen #2\relax
    \dp\mytempbox\mytemplen
    \colorbox{myblue}{\hspace{1em}\usebox{\mytempbox}\hspace{1em}}}
\makeatother

\usepackage[most]{tcolorbox}

\newtcbox{\mymath}[1][]{%
    nobeforeafter, math upper, tcbox raise base,
    enhanced, colframe=blue!30!black,
    colback=blue!30, boxrule=1pt,
    #1}

\tcbset{
    highlight math style={
        enhanced,
        colframe=red!60!black,
        colback=yellow!50,
        arc=4pt,
        boxrule=1pt,
        drop fuzzy shadow
    }
    }

\usepackage{physics}
\usepackage{pgfplots}
\pgfplotsset{compat=1.17}

\linespread{1.5}

\definecolor{ao(english)}{rgb}{0.0, 0.5, 0.0}
\definecolor{byzantium}{rgb}{0.44, 0.16, 0.39}
\newunicodechar{∘}{\circ}

%%%%%%%%%%%%%%%%%%%%%%%%%%%%%%%%%%%%%%%%%%%%%%%%%%
% These are some new commands that may be useful 
% for paper writing in general. If other new commands
% are needed for your specific paper, please feel 
% free to add here. 
%
% The currently available commands are organized in: 
% 1) Systems
% 2) Quantities
% 3) Energies and units
% 4) particle species
% 5) Colors package
% 6) hyperlink
%%%%%%%%%%%%%%%%%%%%%%%%%%%%%%%%%%%%%%%%%%%%%%%%%%

\usepackage{amsmath}
\usepackage{amssymb}
\usepackage{upgreek}
\usepackage{multirow}
\usepackage{setspace}% http://ctan.org/pkg/setspace
\usepackage{fancyhdr}
\usepackage{datetime}

% 1) SYSTEMS
\newcommand{\btc}               {\textbf{BTC}}
\newcommand{\btcspace}          {\textbf{BTC} }
\newcommand{\pow}               {\textbf{PoW}}

% 4) definition to references, biblatex and hyperlink
\usepackage[backend=bibtex, 
style=nature,  %style reference.
sorting=none,
firstinits=true %first name abbreviate
]{biblatex}

\usepackage{hyperref}
\hypersetup{
    colorlinks=true, %set "true" if you want colored links
    linktoc=all,     %set to "all" if you want both sections and subsections linked
    linkcolor=blue,  %choose some color if you want links to stand out
    citecolor= blue, % color of \cite{} in the text.
    urlcolor  = blue, % color of the link for the paper in references.
}

% 5) Tikz and figures
\usepackage{epsfig}
\usepackage{lmodern}
\usepackage{mathtools}
\usepackage[utf8]{luainputenc}
\usepackage{xspace}
\usepackage{tikz}
\usepackage{pgfplots}
\pgfplotsset{compat=newest}

\usetikzlibrary{positioning}
\usepackage{subcaption}

% 6) colors:
\usepackage{xcolor}
\definecolor{ao(english)}{rgb}{0.0, 0.5, 0.0} % dark green

% 7) Add lines numbers
%\usepackage{lineno}

% add pdf file to thesis:
\usepackage{pdfpages}

\hypersetup{
    colorlinks=true,% make the links colored
    linkcolor=blue
}

\usepackage{setspace}
\addbibresource{bibliography.bib}

\newcommand{\printingbibliography}{%

    \pagestyle{myheadings}
    \markright{}
    \sloppy
    \printbibliography[heading=bibintoc, % add to table of contents
                   title=Refer\^encias % Chapter name
                  ]
    \fussy%
}
\PassOptionsToPackage{table}{xcolor}

\pagestyle{fancy}
\fancyhf{}
\renewcommand{\headrulewidth}{0pt}
\fancyhead[R]{\thepage}

\geometry{a4paper,top=30mm,bottom=20mm,left=30mm,right=20mm}

\titleformat*{\section}{\bfseries\large}
\titleformat*{\subsection}{\bfseries\normalsize}

\title{ \textbf{\large Eletromagnetismo I }}
\author{Andr\'e V. Silva}
\date{\today}

\begin{document}

\noindent\rule{\linewidth}{0.8pt}\\
\begin{center}
    \textbf{UNIVERSIDADE FEDERAL DO RIO DE JANEIRO}\\
    \textbf{INSTITUTO DE FÍSICA}\\
    \textbf{\textcolor{blue}{Segunda lista complementar de Eletromagnetismo 1}}\\
    \textbf{Maio de 2025}\\
    \vspace{0.5cm}
    Prof. João Torres de Mello Neto\\
    Monitor: Pedro Khan
\end{center}
\noindent\rule{\linewidth}{0.8pt}\\
\maketitle


\noindent\rule{\linewidth}{0.4pt}\\

\justifying

\begin{flushleft}
\textbf{\textcolor{blue}{\Large Problema 1}}\\

Dois planos condutores aterrados ao longo dos eixos \(x\) e \(y\) se 
interceptam na origem, conforme mostrado na figura. Uma carga \(q\) 
é colocada a uma distância \(b\) acima do eixo \(x\) e a uma distância 
\(a\) à direita do eixo \(y\). Determine a força sobre a carga.\\

\noindent
\textbf{Sugestão:} as cargas imagens devem fazer com que as condições de 
contorno sejam mantidas nos dois planos simultaneamente.

\vspace{0.5cm}

\begin{center}
\begin{tikzpicture}[scale=1.2]
  % Eixos
  \draw[->, thick] (0,0) -- (4,0) node[below right] {\(x\)};
  \draw[->, thick] (0,0) -- (0,4) node[above left] {\(y\)};

  % Carga q
  \filldraw[red] (3,3) circle (2pt) node[right] {\(q\)};
  
  % Linhas tracejadas para indicar a e b
  \draw[dashed] (3,0) -- (3,3);
  \draw[dashed] (0,3) -- (3,3);

  % Setas para indicar a e b
  \draw[<->] (0,-0.5) -- (3,-0.5) node[midway, below] {\(a\)};
  \draw[<->] (-0.5,0) -- (-0.5,3) node[midway, left] {\(b\)};
\end{tikzpicture}
\end{center}

\textcolor{red}{\textbf{Solução:}}\\

Como os planos condutores estão aterrados, o potencial ao longo dos eixos \(x = 0\) e \(y = 0\) deve ser nulo. Para satisfazer essa condição, usaremos o \textbf{método das cargas imagem}.

A carga real \(q\) está localizada no ponto \((a, b)\), com \(a > 0\) e \(b > 0\), no primeiro quadrante. Para que o potencial seja nulo nos eixos coordenados, inserimos três cargas imagem:

\begin{itemize}
  \item Uma carga imagem \(-q\) em \((-a, b)\), que anula o potencial no plano \(x = 0\),
  \item Uma carga imagem \(-q\) em \((a, -b)\), que anula o potencial no plano \(y = 0\),
  \item Uma carga imagem \(+q\) em \((-a, -b)\), que garante simultaneamente que o potencial seja zero nos dois planos.
\end{itemize}

\bigskip

\begin{center}
\begin{tikzpicture}[scale=1.0]
  % Eixos
  \draw[->, thick] (-4,0) -- (4,0) node[below right] {\(x\)};
  \draw[->, thick] (0,-4) -- (0,4) node[above left] {\(y\)};

  % Carga real q
  \filldraw[red] (3,3) circle (2pt) node[above right] {\(q\)};

  % Carga imagem 1: (-q) em (-3,3)
  \filldraw[blue] (-3,3) circle (2pt) node[above left] {\(-q\)};

  % Carga imagem 2: (-q) em (3,-3)
  \filldraw[blue] (3,-3) circle (2pt) node[below right] {\(-q\)};

  % Carga imagem 3: (+q) em (-3,-3)
  \filldraw[blue] (-3,-3) circle (2pt) node[below left] {\(q\)};

  % Linhas tracejadas para indicar a e b
  \draw[dashed] (3,0) -- (3,3);
  \draw[dashed] (0,3) -- (3,3);

  % Setas para indicar a e b
  \draw[<->] (0,-0.8) -- (3,-0.8) node[midway, below] {\(a\)};
  \draw[<->] (-0.8,0) -- (-0.8,3) node[midway, left] {\(b\)};
\end{tikzpicture}
\end{center}

\subsection*{Cálculo da força}

A força total sobre a carga real \(q\) é a soma das forças de Coulomb exercidas pelas cargas imagem.

\subsubsection*{Força devido a \(-q\) em \((-a, b)\):}
\begin{equation}
\vec{F}_1 = \frac{1}{4\pi\varepsilon_0} \cdot \frac{q(-q)}{(2a)^2} \hat{i} = -\frac{q^2}{16\pi\varepsilon_0 a^2} \hat{i}
\end{equation}

\subsubsection*{Força devido a \(-q\) em \((a, -b)\):}
\begin{equation}
\vec{F}_2 = \frac{1}{4\pi\varepsilon_0} \cdot \frac{q(-q)}{(2b)^2} \hat{j} = -\frac{q^2}{16\pi\varepsilon_0 b^2} \hat{j}
\end{equation}

\subsubsection*{Força devido a \(+q\) em \((-a, -b)\):}

A distância até a carga real é \(2\sqrt{a^2 + b^2}\). O vetor deslocamento é \((2a, 2b)\), portanto:

\begin{equation}
\vec{F}_3 = \frac{1}{4\pi\varepsilon_0} \cdot \frac{q^2}{(2\sqrt{a^2 + b^2})^2} \cdot \frac{(2a, 2b)}{2\sqrt{a^2 + b^2}} 
= \frac{q^2}{16\pi\varepsilon_0 (a^2 + b^2)^{3/2}} (a\hat{i} + b\hat{j})
\end{equation}

\subsection*{Resultado final}

Somando os três termos:

\begin{equation}
\vec{F} = \vec{F}_1 + \vec{F}_2 + \vec{F}_3
\end{equation}

\begin{equation}
\boxed{
\vec{F} = \frac{q^2}{16\pi \varepsilon_0} \left[
\left( -\frac{1}{a^2} + \frac{a}{(a^2 + b^2)^{3/2}} \right) \hat{i} +
\left( -\frac{1}{b^2} + \frac{b}{(a^2 + b^2)^{3/2}} \right) \hat{j}
\right]
}
\end{equation}

\end{flushleft}


\begin{flushleft}
\textbf{\textcolor{blue}{\Large Problema 2}}\\

Uma distribuição de carga elétrica produz o campo elétrico
\begin{equation}
\mathbf{E} = c \left( 1 - e^{-\alpha r} \right) \frac{\hat{\mathbf{r}}}{r^2}
\end{equation}
onde \(c\) e \(\alpha\) são constantes. Encontre a carga total dentro do raio \(r = \frac{1}{\alpha}\).

\textcolor{red}{\textbf{Solução:}}\\

Utilizaremos o \textbf{teorema de Gauss}, que relaciona o fluxo do campo elétrico com a carga total no interior de uma superfície fechada:

\begin{equation}
\oint_{\text{superfície}} \mathbf{E} \cdot d\mathbf{A} = \frac{Q_{\text{int}}}{\varepsilon_0}
\end{equation}

Como o campo é radial e depende apenas de \(r\), escolhemos uma superfície gaussiana esférica de raio \(r = \frac{1}{\alpha}\). O campo elétrico sobre essa superfície tem módulo:
\begin{equation}
|\mathbf{E}(r)| = c \left( 1 - e^{-\alpha r} \right) \frac{1}{r^2}
\end{equation}

O vetor de área é \(d\mathbf{A} = \hat{\mathbf{r}}\, r^2 \sin\theta\, d\theta\, d\phi\), portanto:
\begin{equation}
\Phi_E = \oint \mathbf{E} \cdot d\mathbf{A} = |\mathbf{E}(r)| \cdot 4\pi r^2
\end{equation}

Substituindo:
\begin{align}
\Phi_E &= c \left(1 - e^{-\alpha r} \right) \cdot \frac{1}{r^2} \cdot 4\pi r^2 \\
&= 4\pi c \left(1 - e^{-\alpha r} \right)
\end{align}

Aplicando o teorema de Gauss:
\begin{equation}
Q_{\text{int}} = \varepsilon_0 \Phi_E = 4\pi \varepsilon_0 c \left(1 - e^{-\alpha r} \right)
\end{equation}

Finalmente, substituímos \(r = \frac{1}{\alpha}\):
\begin{equation}
Q = 4\pi \varepsilon_0 c \left(1 - e^{-1} \right)
\end{equation}

\section*{Resposta final}

\begin{equation}
\boxed{Q = 4\pi \varepsilon_0 c \left(1 - \frac{1}{e} \right)}
\end{equation}

\end{flushleft}

\begin{flushleft}
\textbf{\textcolor{blue}{\Large Problema 3}}\\

Uma haste fina e não condutora de comprimento \( l \) carrega uma carga 
\( Q \) uniformemente distribuída e está orientada conforme mostrado na figura:

\begin{center}
\begin{tikzpicture}[scale=1.5]
  % Eixos coordenados
  \draw[->, thick] (0,0) -- (1.8,0) node[below] {\(y\)};
  \draw[->, thick] (0,0) -- (0,2.2) node[above] {\(z\)};
  \draw[->, thick] (0,0) -- (-1.5,-1.0) node[left] {\(x\)};

  % Haste carregada (em vermelho)
  \draw[very thick, red] (0,-1) -- (0,1);
  \node[left] at (0,0) {\(Q\)};

  % Parte não carregada dos eixos z
  \draw[thick] (0,-1.5) -- (0,-1); % abaixo da haste
  \draw[thick] (0,1) -- (0,1.5);   % acima da haste

  % Marcas para +l/2 e -l/2
  \node[right] at (0,1) {\(\frac{l}{2}\)};
  \node[right] at (0,-1) {\(-\frac{l}{2}\)};
\end{tikzpicture}
\end{center}

\begin{enumerate}
    \item[(a)] Determine o potencial \( V \) devido à haste carregada para qualquer 
    ponto sobre o eixo \( z \), com \( z > l/2 \).

    \item[(b)] Encontre \( V(r, \theta, \varphi) \) para todos os pontos com \( |\mathbf{r}| > l/2 \), 
    onde \( r, \theta, \varphi \) são as coordenadas esféricas usuais.
\end{enumerate}

\textbf{Sugestão para a parte b:} A solução geral da equação de Laplace em coordenadas esféricas 
com simetria azimutal é dada por
\begin{equation}
V(r, \theta) = \sum_{l=0}^{\infty} \left( A_l r^l + \frac{B_l}{r^{l+1}} \right) P_l(\cos\theta)
\qquad \text{Griffiths, 3.65}
\end{equation}

\textcolor{red}{\textbf{Solução:}}\\

% Solution Here
\end{flushleft}

\begin{flushleft}
\textbf{\textcolor{blue}{\Large Problema 4}}\\

Considere uma esfera de raio \( a \) contendo uma densidade de carga uniforme \( \rho \) no 
seu interior, e sem carga no exterior. Deseja-se determinar o potencial eletrostático \( V(r) \) 
e o campo elétrico \( \mathbf{E}(r) \) em todo o espaço, assumindo que \( V \rightarrow 0 \) 
quando \( r \rightarrow \infty \). 

Determine o campo elétrico dentro e fora da esfera. Resolva a equação de Poisson para dentro 
e fora da esfera.

\textbf{Obs:} esse problema foi resolvido muitas vezes desde Física 3 por meio da lei 
de Gauss na formulação integral.

\textcolor{red}{\textbf{Solução:}}\\

% Solution Here
\end{flushleft}

\begin{flushleft}
\textbf{\textcolor{blue}{\Large Problema 5}}\\

Considere um tubo retangular de dimensões \( 0 \leq x \leq b \) e \( 0 \leq y \leq a \), infinito na direção \( z \).  
As fronteiras em \( x = 0 \), \( x = b \) e \( y = a \) estão mantidas a potencial nulo (\( V = 0 \)), enquanto  
a fronteira em \( y = 0 \) está mantida a um potencial constante \( V_0 \).  
Determinar o potencial eletrostático \( V(x, y) \) dentro do tubo.

\textcolor{red}{\textbf{Solução:}}\\

% Solution Here
\end{flushleft}

\begin{flushleft}
\textbf{\textcolor{blue}{\Large Problema 6}}\\

Em um dispositivo unidimensional, a densidade volumétrica de carga é dada por
\begin{equation}
\rho_v(x) = \rho_0 \frac{x}{a}
\end{equation}

\noindent
Sabendo que o campo elétrico \( E = 0 \) em \( x = 0 \) e o potencial \( V = 0 \) em \( x = a \), 
determinar as expressões para \( V(x) \) e \( \mathbf{E}(x) \).

\textcolor{red}{\textbf{Solução:}}\\

% Solution Here
\end{flushleft}

\begin{flushleft}
\textbf{\textcolor{blue}{\Large Problema 7}}\\

Considere duas cargas pontuais iguais e opostas, \( +q \) e \( -q \), localizadas nos vetores 
de posição \( \mathbf{r}_+ \) e \( \mathbf{r}_- \), conforme mostra a figura. Mostre que, em geral, 
o termo de quadrupolo é diferente de zero. Mostre que, para um dipolo “puro” na origem, o termo de 
quadrupolo se anula.

\begin{center}
\begin{tikzpicture}[scale=2.5, >=stealth]

  % Ponto de origem
  \coordinate (O) at (0,0);

  % Coordenadas dos vetores
  \coordinate (rp) at (1,1.2); % r_+
  \coordinate (rm) at (1.5,0.4); % r_-

  % Vetores
  \draw[->, thick] (O) -- (rp) node[midway, left] {$\mathbf{r}_+$};
  \draw[->, thick] (O) -- (rm) node[midway, below right] {$\mathbf{r}_-$};

  % Cargas
  \filldraw[red] (rp) circle (0.5pt) node[above right] {\small $+q$};
  \filldraw[red] (rm) circle (0.5pt) node[right] {\small $-q$};

  % Linha entre as cargas (opcional)
  \draw[thick] (rp) -- (rm);

  % Origem
  \node at (-0.1,-0.05) {\small $O$};

\end{tikzpicture}
\end{center}

\textcolor{red}{\textbf{Solução:}}\\

% Solution Here
\end{flushleft}

\begin{flushleft}
\textbf{\textcolor{blue}{\Large Problema 8}}\\

\textcolor{red}{\textbf{Solução:}}\\

% Solution Here
\end{flushleft}

\begin{flushleft}
\textbf{\textcolor{blue}{\Large Problema 9}}\\

\textcolor{red}{\textbf{Solução:}}\\

% Solution Here
\end{flushleft}

%%%%%%%% Bibliography 
% Os comandos para incluir as referências bibliográficas
%\printingbibliography

\end{document}
