\documentclass[a4paper,12pt]{article}
\usepackage[brazil, english]{babel}
\usepackage[utf8]{inputenc}
\usepackage[T1]{fontenc}
\usepackage{geometry}
\usepackage{setspace}
\usepackage{titlesec}
\usepackage{hyperref}
\usepackage{graphicx}
\usepackage{caption}
\usepackage{subcaption}
\usepackage{fancyhdr}
\usepackage{xcolor}
\usepackage{amsmath, amssymb, bm}
\usepackage{mathtools}
\usepackage{cancel}
\usepackage{tikz}
\usepackage{newunicodechar}
\usepackage{ragged2e}
\usepackage{setspace}

\linespread{1.5}

\definecolor{ao(english)}{rgb}{0.0, 0.5, 0.0}
\definecolor{byzantium}{rgb}{0.44, 0.16, 0.39}
\newunicodechar{∘}{\circ}

%%%%%%%%%%%%%%%%%%%%%%%%%%%%%%%%%%%%%%%%%%%%%%%%%%
% These are some new commands that may be useful 
% for paper writing in general. If other new commands
% are needed for your specific paper, please feel 
% free to add here. 
%
% The currently available commands are organized in: 
% 1) Systems
% 2) Quantities
% 3) Energies and units
% 4) particle species
% 5) Colors package
% 6) hyperlink
%%%%%%%%%%%%%%%%%%%%%%%%%%%%%%%%%%%%%%%%%%%%%%%%%%

\usepackage{amsmath}
\usepackage{amssymb}
\usepackage{upgreek}
\usepackage{multirow}
\usepackage{setspace}% http://ctan.org/pkg/setspace
\usepackage{fancyhdr}
\usepackage{datetime}

% 1) SYSTEMS 
\newcommand{\pp}           {pp\xspace}
\newcommand{\ppbar}        {\mbox{$\mathrm {p\overline{p}}$}\xspace}
\newcommand{\XeXe}         {\mbox{Xe--Xe}\xspace}
\newcommand{\PbPb}         {\mbox{Pb--Pb}\xspace}
\newcommand{\pA}           {\mbox{pA}\xspace}
\newcommand{\pPb}          {\mbox{p--Pb}\xspace}
\newcommand{\AuAu}         {\mbox{Au--Au}\xspace}
\newcommand{\dAu}          {\mbox{d--Au}\xspace}
\def\pA{$pA$\xspace}
\def\AA{$AA$\xspace}
\def\NN{$NN$\xspace}
\def\signn{$\sigma^{inel}_{NN}$\xspace}
\def\sigtotal{$\sigma_{\textnormal{tot}}$\xspace}
\def\mrm{\mathrm}
\def\ntrig{N_\mrm{trig}}
\newcommand{\rivet}{R\protect\scalebox{1}{IVET}\xspace}
\newcommand{\hepmc}{H\protect\scalebox{1}{EP}MC\xspace}
\newcommand{\herwig}{H\protect\scalebox{1}{ERWIG} 7\xspace}
\newcommand{\sherpa}{S\protect\scalebox{1}{HERPA}\xspace}
\newcommand{\urqmd}{U\protect\scalebox{1}{r}QMD\xspace}
\newcommand{\urqmdversion}{U\protect\scalebox{1}{r}QMD 3.4\xspace}
\newcommand{\pythia}{\protect\scalebox{1}{PYTHIA}\xspace}
\newcommand{\pythiaversion}{\protect\scalebox{1}{PYTHIA 8.2}\xspace}
\newcommand{\pythiaversionused}{\protect\scalebox{1}{PYTHIA 8.235}\xspace}
\newcommand{\pytang}{\protect\scalebox{1}{PYTHIA}/Angantyr\xspace}
\newcommand{\angantyr}{\protect\scalebox{1}{}Angantyr\xspace}
\newcommand{\pytangur}{\protect\scalebox{1}{PYTHIA}/Angantyr + U\protect\scalebox{1}{r}QMD\xspace}
\newcommand{\figref}[1]{Fig.~\ref{#1}}
\newcommand{\tabref}[1]{Tab.~\ref{#1}}
\renewcommand{\eqref}[1]{Eq.~(\ref{#1})}

% hydrodynamic simulation chain:
% TRENTo
\newcommand{\trento}{\protect\scalebox{1}{T$_{\text{R}}$ENT}o\xspace}
% KOMPOST : Linear kinetic theory propagator for initial conditions in heavy ion collisions
\newcommand{\kompost}{\protect\scalebox{1}{K$\varnothing$MP$\varnothing$ST}\xspace}
% MUSIC
\newcommand{\music}{\protect\scalebox{1}{MUSIC}\xspace}
% iSS
\newcommand{\iss}{\protect\scalebox{1}{iSS}\xspace}

% 2) QUANTITIES 
\newcommand{\s}            {\ensuremath{\sqrt{s}}\xspace}
\newcommand{\snn}          {\ensuremath{\sqrt{s_{\mathrm{NN}}}}\xspace}
\newcommand{\pt}           {\ensuremath{p_{\rm T}}\xspace}
\newcommand{\meanpt}       {$\langle p_{\mathrm{T}}\rangle$\xspace}
\newcommand{\ycms}         {\ensuremath{y_{\rm CMS}}\xspace}
\newcommand{\ylab}         {\ensuremath{y_{\rm lab}}\xspace}
\newcommand{\etarange}[1]  {\mbox{$\left | \eta \right |~<~#1$}}
\newcommand{\centbin}[2]  {\mbox{$#1-#2\%$}}
\newcommand{\ptrange}[2]  {\mbox{$#1 < p_{\mathrm{T}}\hspace{0.2cm} (\mathrm{GeV}/\mathrm{\textit{c}}) <#2$}}
\newcommand{\ptrangetrig}[2]  {\mbox{$#1 < p^{\mathrm{trigger}}_{\mathrm{T} }\hspace{0.2cm} (\mathrm{GeV}/\mathrm{\textit{c}}) <#2$}}
\newcommand{\ptrangeassoc}[2]  {\mbox{$#1 < p^{\mathrm{assoc}}_{\mathrm{T} }\hspace{0.2cm} (\mathrm{GeV}/\mathrm{\textit{c}}) <#2$}}
\newcommand{\etazerothree} {$\left|\eta \right| < 0.3$\xspace}
\newcommand{\etazerofive} {$\left|\eta \right| < 0.5$\xspace}
\newcommand{\etazeroeight} {$\left|\eta \right| < 0.8$\xspace}
\newcommand{\yrange}[1]    {\mbox{$\left | y \right |~<~#1$}}
\newcommand{\dndy}         {\ensuremath{\mathrm{d}N_\mathrm{ch}/\mathrm{d}y}\xspace}
\newcommand{\dndeta}       {\ensuremath{\mathrm{d}N_\mathrm{ch}/\mathrm{d}\eta}\xspace}
\newcommand{\dnchdydpt}   {\ensuremath{\mathrm{d}N_\mathrm{ch}/\mathrm{d}y\mathrm{d}p_{\mathrm{T}}}\xspace}
\newcommand{\dnchaadydpt}   {\ensuremath{\mathrm{d}N_\mathrm{ch}^{AA}/\mathrm{d}y\mathrm{d}p_{\mathrm{T}}}\xspace}
\newcommand{\dnchppdydpt}   {\ensuremath{\mathrm{d}N_\mathrm{ch}^{\mathrm{pp}}/\mathrm{d}y\mathrm{d}p_{\mathrm{T}}}\xspace}
\newcommand{\dnchdphi}{\ensuremath{\mathrm{d}N_\mathrm{ch}/\mathrm{d}\phi}\xspace}
\newcommand{\dnchddeltaphi}{\ensuremath{\mathrm{d}N_\mathrm{ch}/\mathrm{d}\Delta\upphi}\xspace}
\newcommand{\dndphi}{\ensuremath{\mathrm{d}N/\mathrm{d}\phi}\xspace}
\newcommand{\dnddeltaphi}{\ensuremath{\mathrm{d}N/\mathrm{d}\Delta\upphi}\xspace}
\newcommand{\avdndeta}     {\ensuremath{\langle\dndeta\rangle}\xspace}
\newcommand{\avdndetarap}  {$\langle$ dN$_{\textnormal{ch}}$/d$\eta$ $\rangle_{|\eta| < 0.5}$\xspace}
\newcommand{\dNdy}         {\ensuremath{\mathrm{d}N_\mathrm{ch}/\mathrm{d}y}\xspace}
\newcommand{\Npart}        {\ensuremath{N_\mathrm{part}}\xspace}
\newcommand{\meanNpart}    {$\langle$\ensuremath{N_\mathrm{part}}$\rangle$\xspace}
\newcommand{\ncoll}        {\ensuremath{N_\mathrm{coll}}\xspace}
\newcommand{\meanncoll}    {$\langle$\ensuremath{N_\mathrm{coll}}$\rangle$\xspace}
\newcommand{\averagencollhadronic}    {$\langle$\ensuremath{\mathrm{N}_\mathrm{coll}^{\mathrm{hadronic}}}$\rangle$\xspace}
\newcommand{\meantaa}      {$\langle$\ensuremath{T_\mathrm{AA}}$\rangle$\xspace}
\newcommand{\dEdx}         {\ensuremath{\textrm{d}E/\textrm{d}x}\xspace}
\newcommand{\RpPb}         {\ensuremath{R_{\rm pPb}}\xspace}
\newcommand{\raa}          {$R_{AA}$\xspace}
\newcommand{\vtwo}         {$v_{2}$\xspace}
\newcommand{\vtwoinitial}  {$v_{2}^{\mathrm{initial}}$\xspace}
\newcommand{\vtwofinal}    {$v_{2}^{\mathrm{final}}$\xspace}
\newcommand{\vtwofourfinal}{$v_{2}^{\mathrm{final}}\{4\}$\xspace}
\newcommand{\vtwofit}      {$v_{2}^{\mathrm{Fit}}$\xspace}
\newcommand{\vtwotwo}      {$v_{2}\{2\}$\xspace}
\newcommand{\vtwofour}     {$v_{2}\{4\}$\xspace}
\newcommand{\vtwopt}       {$v_{2}(p_{\textnormal{T}})$\xspace}
\newcommand{\vtwoptfit}    {$v_{2}^{\mathrm{Fit}}(p_{\textnormal{T}})$\xspace}
\newcommand{\nch}          {\ensuremath{N_\mathrm{ch}}\xspace}
\newcommand{\psireactionplane}          {$\Psi_{\textnormal{RP}}$\xspace}
\newcommand{\deltaphireactionplane}     {$\Delta\upphi = \phi - \Psi_{\textnormal{RP}}$\xspace}
\newcommand{\nevdnchddeltaphi}     {(1/N$_{\textnormal{ev}}$)dN$_{\textnormal{ch}}$/d$\Delta\upphi$\xspace}
\newcommand{\meannch}      {\ensuremath{\langle N_\mathrm{ch}\rangle}\xspace}
\newcommand{\etamodule}    {\ensuremath{|\eta|}\xspace}
\newcommand{\qbar}         {$\bar{\textnormal{q}}$\xspace}
\newcommand{\qqbar}        {$\textnormal{q}\bar{\textnormal{q}}$\xspace}
\newcommand{\qqbarzero}    {$\textnormal{q}_{0}\bar{\textnormal{q}}_{0}$\xspace}
\newcommand{\qqqbars}      {$\bar{\textnormal{q}}\bar{\textnormal{q}}\bar{\textnormal{q}}$\xspace}
\newcommand{\alphastrong}  {$\alpha_{\textnormal{s}}$\xspace}
\newcommand{\alphastrongdistance}  {$\alpha_{\textnormal{s}}$(R)\xspace}
\newcommand{\qtwo}         {Q$^2$\xspace}
\newcommand{\alphastrongqtwo}  {$\alpha_{\textnormal{s}}$(Q$^2$)\xspace}
\newcommand{\lambdaqcd}        {$\Lambda_{\textnormal{QCD}}$\xspace}
\newcommand{\sectionpp}        {$\sigma^{\textnormal{pp}}_{\textnormal{inel}}$\xspace}

% 3) ENERGIES, UNITS
\newcommand{\sqrts}        {$\sqrt{s}$\xspace}
\newcommand{\sqrtsnn}      {$\sqrt{s_{\mathrm{NN}}}$\xspace}
\newcommand{\nineH}        {$\sqrt{s}~=~0.9$~Te\kern-.1emV\xspace}
\newcommand{\seven}        {$\sqrt{s}~=~7$~Te\kern-.1emV\xspace}
\newcommand{\twoH}         {$\sqrt{s}~=~0.2$~Te\kern-.1emV\xspace}
\newcommand{\twosevensix}  {$\sqrt{s}~=~2.76$~Te\kern-.1emV\xspace}
\newcommand{\five}         {$\sqrt{s}~=~5.02$~Te\kern-.1emV\xspace}
\newcommand{\twohundrernn} {$\sqrt{s_{\mathrm{NN}}}=200$~Ge\kern-.1emV\xspace}
\newcommand{\twosevensixnn} {$\sqrt{s_{\mathrm{NN}}}=2.76$~Te\kern-.1emV\xspace}
\newcommand{\fivenn}       {$\sqrt{s_{\mathrm{NN}}}~=~5.02$~Te\kern-.1emV\xspace}
\newcommand{\fivefourfournn} {$\sqrt{s_{\mathrm{NN}}}=5.44$~Te\kern-.1emV\xspace}
\newcommand{\LT}           {L{\'e}vy-Tsallis\xspace}
\newcommand{\GeVc}         {Ge\kern-.1emV/$c$\xspace}
\newcommand{\MeVc}         {Me\kern-.1emV/$c$\xspace}
\newcommand{\TeV}          {Te\kern-.1emV\xspace}
\newcommand{\GeV}          {Ge\kern-.1emV\xspace}
\newcommand{\MeV}          {Me\kern-.1emV\xspace}
\newcommand{\GeVmass}      {Ge\kern-.2emV/$c^2$\xspace}
\newcommand{\MeVmass}      {Me\kern-.2emV/$c^2$\xspace}
\newcommand{\lumi}         {\ensuremath{\mathcal{L}}\xspace}
\newcommand{\fmc}         {fm\kern-.1em/$c$\xspace}

% 4) PARTICLE SPECIES 
\newcommand{\ee}           {\ensuremath{e^{+}e^{-}}} 
\newcommand{\pip}          {\ensuremath{\pi^{+}}\xspace}
\newcommand{\pim}          {\ensuremath{\pi^{-}}\xspace}
\newcommand{\kap}          {\ensuremath{\rm{K}^{+}}\xspace}
\newcommand{\kam}          {\ensuremath{\rm{K}^{-}}\xspace}
\newcommand{\pbar}         {\ensuremath{\rm\overline{p}}\xspace}
\newcommand{\kzero}        {\ensuremath{{\rm K}^{0}_{\rm{S}}}\xspace}
\newcommand{\lmb}          {\ensuremath{\Lambda}\xspace}
\newcommand{\almb}         {\ensuremath{\overline{\Lambda}}\xspace}
\newcommand{\Om}           {\ensuremath{\Omega^-}\xspace}
\newcommand{\Mo}           {\ensuremath{\overline{\Omega}^+}\xspace}
\newcommand{\X}            {\ensuremath{\Xi^-}\xspace}
\newcommand{\Ix}           {\ensuremath{\overline{\Xi}^+}\xspace}
\newcommand{\Xis}          {\ensuremath{\Xi^{\pm}}\xspace}
\newcommand{\Oms}          {\ensuremath{\Omega^{\pm}}\xspace}
\newcommand{\degree}       {\ensuremath{^{\rm o}}\xspace}
\newcommand{\comment}[1]{}

% two-particle angular correlation
\newcommand{\deltaphitriggassoc}    {$\Delta\upphi = |\phi_{\textnormal{trigger}} - \phi_{\textnormal{assoc}}|$\xspace}
\newcommand{\deltaetatriggassoc}    {$\Delta\upeta = |\eta_{\textnormal{trigger}} - \eta_{\textnormal{assoc}}|$\xspace}
\newcommand{\etatrigg}    {$\eta_{\textnormal{trigger}}$\xspace}
\newcommand{\etaassoc}    {$\eta_{\textnormal{assoc}}$\xspace}
\newcommand{\deltaphideltaeta}      {$\Delta\upphi-\Delta\upeta$\xspace}
\newcommand{\deltaphi}              {$\Delta\upphi$\xspace}
\newcommand{\moduledeltaphipitwo}   {$|\Delta\upphi| < \pi/2 $\xspace}
\newcommand{\deltaeta}              {$\Delta\upeta$\xspace}
\newcommand{\moduledeltaeta}        {$|\Delta\upeta|$\xspace}
\newcommand{\deltaphiapproxzero}    {$\Delta\upphi = 0$\xspace}
\newcommand{\deltaphiapproxpi}      {$\Delta\upphi = \pi$\xspace}
\newcommand{\deltaetaapproxzero}    {$\Delta\upeta = 0$\xspace}
\newcommand{\corrfunc}              {C($\Delta\upphi$, $\Delta\upeta$)\xspace}
\newcommand{\corrfunccorrect}              {C$_{\mathrm{correct}}(\Delta\upphi$, $\Delta\upeta$)\xspace}
\newcommand{\corrfuncmix}              {C$_{\mathrm{mix}}(\Delta\upphi$, $\Delta\upeta$)\xspace}
\newcommand{\corrfuncdeltaphi}      {C($\Delta\upphi$)\xspace}
\newcommand{\pttrigger}             {$p_{\textnormal{T}}^{\textnormal{trigger}}$\xspace}
\newcommand{\ptassoc}               {$p_{\textnormal{T}}^{\textnormal{assoc}}$\xspace}
\newcommand{\ratioyieldawaynearside}{Y$_{\textnormal{Away}}$/Y$_{\textnormal{Near}}$\xspace}

% 4) definition to references, biblatex and hyperlink
\usepackage[backend=bibtex, 
style=nature,  %style reference.
sorting=none,
firstinits=true %first name abbreviate
]{biblatex}

\usepackage{hyperref}
\hypersetup{
    colorlinks=true, %set "true" if you want colored links
    linktoc=all,     %set to "all" if you want both sections and subsections linked
    linkcolor=blue,  %choose some color if you want links to stand out
    citecolor= blue, % color of \cite{} in the text.
    urlcolor  = blue, % color of the link for the paper in references.
}

% 5) Tikz and figures
\usepackage{epsfig}
\usepackage{lmodern}
\usepackage{mathtools}
\usepackage[utf8]{luainputenc}
\usepackage{xspace}
\usepackage{tikz}
\usepackage{pgfplots}
\pgfplotsset{compat=newest}

\usetikzlibrary{positioning}
\usepackage{subcaption}

% 6) colors:
\usepackage{xcolor}
\definecolor{ao(english)}{rgb}{0.0, 0.5, 0.0} % dark green

% 7) Add lines numbers
%\usepackage{lineno}

% add pdf file to thesis:
\usepackage{pdfpages}

\hypersetup{
    colorlinks=true,% make the links colored
    linkcolor=blue
}

\usepackage{setspace}
\addbibresource{bibliography.bib}

\newcommand{\printingbibliography}{%

    \pagestyle{myheadings}
    \markright{}
    \sloppy
    \printbibliography[heading=bibintoc, % add to table of contents
                   title=Refer\^encias % Chapter name
                  ]
    \fussy%
}
\PassOptionsToPackage{table}{xcolor}

\pagestyle{fancy}
\fancyhf{}
\renewcommand{\headrulewidth}{0pt}
\fancyhead[R]{\thepage}

\geometry{a4paper,top=30mm,bottom=20mm,left=30mm,right=20mm}

\titleformat*{\section}{\bfseries\large}
\titleformat*{\subsection}{\bfseries\normalsize}

\title{ \textbf{\large Eletromagnetismo I }}
\author{Andr\'e V. Silva}
\date{\today}

\begin{document}

\noindent\rule{\linewidth}{0.8pt}\\
\begin{center}
    \textbf{UNIVERSIDADE FEDERAL DO RIO DE JANEIRO}\\
    \textbf{INSTITUTO DE FÍSICA}\\
    \textbf{\textcolor{blue}{Primeira lista complementar de Eletromagnetismo 1}}\\
    \textbf{Março de 2025}\\
    \vspace{0.5cm}
    Prof. João Torres de Mello Neto\\
    Monitor: Pedro Khan
\end{center}
\noindent\rule{\linewidth}{0.8pt}\\
\maketitle


\noindent\rule{\linewidth}{0.4pt}\\

\justifying

\begin{flushleft}
\textbf{\textcolor{blue}{Problema 1}}\\
\textbf{a)} Mostre que o rotacional de um gradiente de uma função qualquer é zero:
$\nabla \times (\nabla f) = 0,$ de duas formas: abrindo em componentes e 
argumentando pelo teorema de Stokes. 
\textbf{b)} Mostre que a divergência de um rotacional de um vetor qualquer é nula:
$\nabla \cdot (\nabla \times \mathbf{A}) = 0,$ de duas formas: calculando as 
componentes e argumentando pelo teorema de Stokes no limite que a integral de 
linha tende para zero e usando o teorema da divergência em seguida.
\end{flushleft}

\textcolor{red}{\textbf{Solução:}}\\

Em componentes cartesianas, o gradiente de uma função escalar $f$ é:
\begin{equation}
    \nabla f = \left( \frac{\partial f}{\partial x}, \frac{\partial f}{\partial y}, \frac{\partial f}{\partial z} \right).
\end{equation}
O rotacional é definido como:
\begin{equation}
    \nabla \times \mathbf{A} = 
    \begin{vmatrix} 
        \hat{i} & \hat{j} & \hat{k} \\
        \frac{\partial}{\partial x} & \frac{\partial}{\partial y} & \frac{\partial}{\partial z} \\
        \frac{\partial f}{\partial x} & \frac{\partial f}{\partial y} & \frac{\partial f}{\partial z} 
    \end{vmatrix}.
\end{equation}

Expansão do determinante:
\begin{equation}
    \nabla \times (\nabla f) = 
    \left( \frac{\partial^2 f}{\partial y \partial x} - \frac{\partial^2 f}{\partial x \partial y} \right) \hat{i} + 
    \left( \frac{\partial^2 f}{\partial z \partial y} - \frac{\partial^2 f}{\partial y \partial z} \right) \hat{j} + 
    \left( \frac{\partial^2 f}{\partial x \partial z} - \frac{\partial^2 f}{\partial z \partial x} \right) \hat{k}.
\end{equation}

Como as derivadas parciais mistas são comutativas (desde que $f$ seja suave), cada termo é zero:
\begin{equation}
    \nabla \times (\nabla f) = 0.
\end{equation}

Pelo \colorbox{yellow}{teorema de Stokes, a integral de linha de um gradiente ao longo de um caminho} 
\colorbox{yellow}{fechado é zero, implicando que seu rotacional é nulo.}

\textbf{b)} Mostre que a divergência de um rotacional de um vetor qualquer é nula:
\begin{equation}
    \nabla \cdot (\nabla \times \mathbf{A}) = 0
\end{equation}

de duas formas: calculando as componentes e argumentando pelo teorema de Stokes no limite que a integral de linha tende para zero e usando o teorema da divergência em seguida.

A divergência é definida como:

\begin{equation}
    \nabla \cdot \mathbf{A} = \frac{\partial A_x}{\partial x} + \frac{\partial A_y}{\partial y} + \frac{\partial A_z}{\partial z}.
\end{equation}

Aplicando à definição do rotacional:
\begin{equation}
    \nabla \times \mathbf{A} = 
    \begin{vmatrix} 
        \hat{i} & \hat{j} & \hat{k} \\
        \frac{\partial}{\partial x} & \frac{\partial}{\partial y} & \frac{\partial}{\partial z} \\
        A_x & A_y & A_z 
    \end{vmatrix},
\end{equation}

resultando em:
\begin{equation}
    \nabla \times \mathbf{A} = 
    \left( \frac{\partial A_z}{\partial y} - \frac{\partial A_y}{\partial z} \right) \hat{i} + 
    \left( \frac{\partial A_x}{\partial z} - \frac{\partial A_z}{\partial x} \right) \hat{j} + 
    \left( \frac{\partial A_y}{\partial x} - \frac{\partial A_x}{\partial y} \right) \hat{k}.
\end{equation}

Tomando a divergência:

\begin{equation}
    \nabla \cdot (\nabla \times \mathbf{A}) = \frac{\partial}{\partial x} \left( \frac{\partial A_z}{\partial y} - \frac{\partial A_y}{\partial z} \right) + 
    \frac{\partial}{\partial y} \left( \frac{\partial A_x}{\partial z} - \frac{\partial A_z}{\partial x} \right) + 
    \frac{\partial}{\partial z} \left( \frac{\partial A_y}{\partial x} - \frac{\partial A_x}{\partial y} \right).
\end{equation}

Como as derivadas mistas comutam, cada termo se anula, resultando em:

\begin{equation}
    \nabla \cdot (\nabla \times \mathbf{A}) = 0.
\end{equation}

Pelo teorema da divergência, a integral de volume da divergência de um rotacional se reduz a uma integral de superfície 
de um campo tangencial, que desaparece, confirmando o resultado.

\begin{flushleft}
\textbf{\textcolor{blue}{Problema 2}}\\
Encontre a área de um círculo no plano \(xy\) centrado na origem usando: \\ 
(i) coordenadas retangulares \(x^2 + y^2 = a^2 \) \\
Dica:
\begin{equation}
\int \sqrt{a^2 - x^2} \, dx = \frac{1}{2} \left[ x\sqrt{a^2 - x^2} + a^2 \arcsin\left(\frac{x}{a}\right) \right]
\end{equation}
(ii) coordenadas cilíndricas \(r = a\).  
Qual sistema de coordenadas é mais fácil de usar?
\end{flushleft}

\textcolor{red}{\textbf{Solução:}}\\

\subsection*{(i) Coordenadas Retangulares}
A área do círculo pode ser calculada integrando a função da semicircunferência superior $y = \sqrt{a^2 - x^2}$ em relação a $x$ de $-a$ a $a$ e depois dobrando o resultado:
\begin{equation}
A = 2 \int_{-a}^{a} \sqrt{a^2 - x^2} \, dx.
\end{equation}

Utilizando a dica fornecida:
\begin{align*}
A &= 2 \times \frac{1}{2} \left[ x\sqrt{a^2 - x^2} + a^2 \arcsin\left(\frac{x}{a}\right) \right]_{-a}^{a} \\
&= \left[ x\sqrt{a^2 - x^2} + a^2 \arcsin\left(\frac{x}{a}\right) \right]_{-a}^{a}.
\end{align*}

Substituindo os limites:
\begin{align*}
A &= \left[a\sqrt{a^2 - a^2} + a^2 \arcsin(1) \right] - \left[-a\sqrt{a^2 - a^2} + a^2 \arcsin(-1) \right] \\
&= \left[a^2 \frac{\pi}{2} \right] - \left[a^2 \left(-\frac{\pi}{2}\right) \right] \\
&= a^2 \frac{\pi}{2} + a^2 \frac{\pi}{2} \\
&= a^2 \pi.
\end{align*}

\subsection*{(ii) Coordenadas Cilíndricas}
Em coordenadas polares, a área do círculo é dada por:
\begin{equation}
A = \int_0^{2\pi} \int_0^a r \, dr \, d\theta.
\end{equation}

Resolvendo a integral interna:
\begin{align*}
\int_0^a r \, dr &= \frac{r^2}{2} \Big|_0^a = \frac{a^2}{2}.
\end{align*}

Agora, resolvendo a integral externa:
\begin{align*}
A &= \int_0^{2\pi} \frac{a^2}{2} \, d\theta \\
&= \frac{a^2}{2} \theta \Big|_0^{2\pi} \\
&= \frac{a^2}{2} (2\pi - 0) \\
&= a^2 \pi.
\end{align*}

\subsection*{Conclusão}
O resultado final é o mesmo nos dois métodos, $A = \pi a^2$. No entanto, o cálculo utilizando 
coordenadas polares é significativamente mais simples, pois evita o uso de funções trigonométricas 
inversas e manipulação algébrica complexa.
    
\begin{flushleft}
\textbf{\textcolor{blue}{Problema 3}}\\
Encontre o volume de uma esfera de raio $R$ centrada na origem usando:

(i) Coordenadas retangulares $x^2 + y^2 + z^2 = R^2$ \newline
Dica:
\begin{equation}
\int \sqrt{a^2 - x^2} \, dx = \frac{1}{2} \left[ x\sqrt{a^2 - x^2} + a^2 \arcsin\left(\frac{x}{a}\right) \right]
\end{equation}

(ii) Coordenadas cilíndricas $r^2 + z^2 = R^2$;

(iii) Coordenadas esféricas $r = R$.

Qual sistema de coordenadas é mais fácil de usar?

\textcolor{red}{\textbf{Solução:}}\\

\subsection*{(i) Coordenadas Retangulares}
O volume da esfera pode ser encontrado integrando seções transversais circulares ao longo do eixo $z$:
\begin{equation}
V = 2 \int_0^R \pi (R^2 - z^2) \, dz.
\end{equation}

Resolvendo a integral:
\begin{align*}
V &= 2\pi \int_0^R (R^2 - z^2) \, dz \\
&= 2\pi \left[ R^2z - \frac{z^3}{3} \right]_0^R \\
&= 2\pi \left[ R^3 - \frac{R^3}{3} \right] \\
&= 2\pi \left( \frac{3R^3}{3} - \frac{R^3}{3} \right) \\
&= 2\pi \left( \frac{2R^3}{3} \right) \\
&= \frac{4}{3} \pi R^3.
\end{align*}

\subsection*{(ii) Coordenadas Cilíndricas}
O volume da esfera é calculado como:
\begin{equation}
V = \int_{-R}^{R} \int_0^{\sqrt{R^2 - z^2}} 2\pi r \, dr \, dz.
\end{equation}

Resolvendo a integral interna:
\begin{align*}
\int_0^{\sqrt{R^2 - z^2}} 2\pi r \, dr &= 2\pi \frac{r^2}{2} \Big|_0^{\sqrt{R^2 - z^2}} \\
&= \pi (R^2 - z^2).
\end{align*}

Agora, resolvendo a integral externa:
\begin{align*}
V &= \pi \int_{-R}^{R} (R^2 - z^2) \, dz,
\end{align*}

que já foi resolvida anteriormente e resulta em:
\begin{equation}
V = \frac{4}{3} \pi R^3.
\end{equation}

\subsection*{(iii) Coordenadas Esféricas}
Em coordenadas esféricas, o volume é dado por:
\begin{equation}
V = \int_0^R \int_0^{\pi} \int_0^{2\pi} r^2 \sin \theta \, d\phi \, d\theta \, dr.
\end{equation}

Resolvendo as integrais:
\begin{align*}
\int_0^{2\pi} d\phi &= 2\pi, \\
\int_0^{\pi} \sin\theta \, d\theta &= \int_0^{\pi} d(-\cos\theta) = (-\cos\theta) \Big|_0^{\pi} = (1 + 1) = 2, \\
\int_0^R r^2 \, dr &= \frac{R^3}{3}.
\end{align*}

Multiplicando os resultados:
\begin{equation}
V = 2\pi \times 2 \times \frac{R^3}{3} = \frac{4}{3} \pi R^3.
\end{equation}

\subsection*{Conclusão}
O volume obtido é o mesmo em todos os casos, $V = \frac{4}{3} \pi R^3$. No entanto, o método de coordenadas esféricas 
é o mais eficiente, pois evita integrais mais complexas e simplifica a abordagem diretamente para o volume da esfera.
\end{flushleft}

\begin{flushleft}
\textbf{\textcolor{blue}{Problema 4}}\\
Demonstre a identidade
\begin{equation}
\nabla (\vec{A} \cdot \vec{B}) = (\vec{A} \cdot \nabla) \vec{B} + (\vec{B} \cdot \nabla) \vec{A} + \vec{A} \times (\nabla \times \vec{B}) + \vec{B} \times (\nabla \times \vec{A})
\end{equation}
de três formas distintas:
\begin{itemize}
    \item[(a)] Abrindo as componentes (força bruta);
    \item[(b)] Usando a identidade BAC - CAB de forma adequada;
    \item[(c)] Usando álgebra de índices.
\end{itemize}

\textcolor{red}{\textbf{Solução:}}\\
Demonstre a identidade
\begin{equation}
    \nabla (\vec{A} \cdot \vec{B}) = (\vec{A} \cdot \nabla) \vec{B} + (\vec{B} \cdot \nabla) \vec{A} + \vec{A} \times (\nabla \times \vec{B}) + \vec{B} \times (\nabla \times \vec{A}),
    \end{equation}
    de três formas distintas:
    
    \begin{itemize}
        \item[(a)] \textbf{Abrindo as componentes (força bruta)}:
        
        Sejam \( \vec{A} = A_i \hat{e}_i \) e \( \vec{B} = B_j \hat{e}_j \), onde \( A_i \) e \( B_j \) são as componentes das funções vetoriais \( \vec{A} \) e \( \vec{B} \), respectivamente, e \( \hat{e}_i \) e \( \hat{e}_j \) são os vetores unitários.
        
        O produto escalar \( \vec{A} \cdot \vec{B} \) é dado por:
        \begin{equation}
        \vec{A} \cdot \vec{B} = A_i B_i.
        \end{equation}
        
        Agora, calculamos o gradiente de \( \vec{A} \cdot \vec{B} \):
        \begin{equation}
        \nabla (\vec{A} \cdot \vec{B}) = \nabla (A_i B_i) = (\nabla A_i) B_i + A_i (\nabla B_i).
        \end{equation}
        
        Essa expressão pode ser decomposta em duas partes:
        \begin{itemize}
            \item \( (\nabla A_i) B_i \), que é a ação de \( \nabla \) sobre \( A_i \) e depois multiplicada por \( B_i \),
            \item \( A_i (\nabla B_i) \), que é a ação de \( \nabla \) sobre \( B_i \) e depois multiplicada por \( A_i \).
        \end{itemize}
        
        Agora, para as expressões envolvendo o rotacional, temos:
        \begin{equation}
        \vec{A} \times (\nabla \times \vec{B}) = A_i \epsilon_{ijk} \hat{e}_j (\partial_k B_l),
        \end{equation}
        e
        \begin{equation}
        \vec{B} \times (\nabla \times \vec{A}) = B_i \epsilon_{ijk} \hat{e}_j (\partial_k A_l).
        \end{equation}
        
        Essas expressões podem ser agrupadas para formar a identidade desejada.
        
        \item[(b)] \textbf{Usando a identidade BAC-CAB}:
        
        A identidade conhecida como identidade BAC-CAB é dada por:
        \begin{equation}
        \vec{A} \cdot (\nabla \times \vec{B}) = (\vec{A} \times \nabla) \vec{B} - (\nabla \times \vec{A}) \cdot \vec{B}.
        \end{equation}
        
        Para usá-la na demonstração da identidade original, observamos que:
        \begin{equation}
        \nabla (\vec{A} \cdot \vec{B}) = (\vec{A} \cdot \nabla) \vec{B} + (\vec{B} \cdot \nabla) \vec{A} + \vec{A} \times (\nabla \times \vec{B}) + \vec{B} \times (\nabla \times \vec{A}).
        \end{equation}
        
        Ao aplicar a identidade BAC-CAB adequadamente, podemos manipular os termos cruzados e os termos do gradiente para obter a forma completa da identidade.
        
        \item[(c)] \textbf{Usando álgebra de índices}:
        
        Por fim, usando álgebra de índices, escrevemos as componentes do gradiente do produto escalar \( \vec{A} \cdot \vec{B} \) e os outros termos da identidade em termos de somatórios e derivadas parciais.
        
        O gradiente do produto escalar é dado por:
        \begin{equation}
        \nabla (\vec{A} \cdot \vec{B}) = \nabla (A_i B_i) = (\partial_j A_i) B_i + A_i (\partial_j B_i).
        \end{equation}
        
        A seguir, considerando os termos envolvendo o rotacional:
        \begin{equation}
        \vec{A} \times (\nabla \times \vec{B}) = \epsilon_{ijk} A_j \partial_k B_l,
        \end{equation}
        \begin{equation}
        \vec{B} \times (\nabla \times \vec{A}) = \epsilon_{ijk} B_j \partial_k A_l.
        \end{equation}
        
        Agrupando todos esses termos e reconhecendo as simetrias, podemos obter a identidade desejada:
        \begin{equation}
        \nabla (\vec{A} \cdot \vec{B}) = (\vec{A} \cdot \nabla) \vec{B} + (\vec{B} \cdot \nabla) \vec{A} + \vec{A} \times (\nabla \times \vec{B}) + \vec{B} \times (\nabla \times \vec{A}).
        \end{equation}
    \end{itemize}
\end{flushleft}

\begin{flushleft}
\textbf{\textcolor{blue}{Problema 5}}\\

Quais das seguintes afirmações sobre vetores gerais $\vec{a}, \vec{b}$ \textbf{e} $\vec{c}$ são verdadeiras?

\begin{enumerate}
    \item[(a)] $\vec{c} \cdot (\vec{a} \times \vec{b}) = (\vec{b} \times \vec{a}) \cdot \vec{c};$
    \item[(b)] $\vec{a} \times (\vec{b} \times \vec{c}) = (\vec{a} \times \vec{b}) \times \vec{c};$
    \item[(c)] $\vec{a} \times (\vec{b} \times \vec{c}) = (\vec{a} \cdot \vec{c})\vec{b} - (\vec{a} \cdot \vec{b})\vec{c};$
    \item[(d)] $\vec{d} = \lambda \vec{a} + \mu \vec{b}$ \textbf{implica} $(\vec{a} \times \vec{b}) \cdot \vec{d} = 0;$
    \item[(e)] $\vec{a} \times \vec{c} = \vec{b} \times \vec{c}$ \textbf{implica} $\vec{c} \cdot \vec{a} - \vec{c} \cdot \vec{b} = c |\vec{a} - \vec{b}|;$
    \item[(f)] $(\vec{a} \times \vec{b}) \times (\vec{c} \times \vec{b}) = \vec{b} [\vec{b} \cdot (\vec{c} \times \vec{a})].$
\end{enumerate}

\textcolor{red}{\textbf{Solução:}}\\

\section*{Análise das Afirmações}

\textbf{(a) $\vec{c} \cdot (\vec{a} \times \vec{b}) = (\vec{b} \times \vec{a}) \cdot \vec{c}$}

O produto vetorial tem a propriedade anticomutativa:
\begin{equation}
    \vec{b} \times \vec{a} = - (\vec{a} \times \vec{b}).
\end{equation}
Portanto,
\begin{equation}
    (\vec{b} \times \vec{a}) \cdot \vec{c} = (- \vec{a} \times \vec{b}) \cdot \vec{c} = - (\vec{a} \times \vec{b}) \cdot \vec{c}.
\end{equation}
Como isso contradiz a igualdade dada, a afirmação \textbf{é falsa}.

\textbf{(b) $\vec{a} \times (\vec{b} \times \vec{c}) = (\vec{a} \times \vec{b}) \times \vec{c}$}

Essa expressão não é, em geral, verdadeira, pois o produto vetorial não é associativo. Assim, a afirmação \textbf{é falsa}.

\textbf{(c) $\vec{a} \times (\vec{b} \times \vec{c}) = (\vec{a} \cdot \vec{c}) \vec{b} - (\vec{a} \cdot \vec{b}) \vec{c}$}

Esta é a identidade de Lagrange para o produto misto:
\begin{equation}
    \vec{a} \times (\vec{b} \times \vec{c}) = (\vec{a} \cdot \vec{c})\vec{b} - (\vec{a} \cdot \vec{b})\vec{c}.
\end{equation}
Portanto, a afirmação \textbf{é verdadeira}.

\textbf{(d) $\vec{d} = \lambda \vec{a} + \mu \vec{b}$ implica $(\vec{a} \times \vec{b}) \cdot \vec{d} = 0$}

Se $\vec{d}$ é uma combinação linear de $\vec{a}$ e $\vec{b}$, então ele pertence ao plano definido por esses vetores. O produto misto entre três vetores coplanares é sempre zero:
\begin{equation}
    (\vec{a} \times \vec{b}) \cdot \vec{d} = 0.
\end{equation}
Assim, a afirmação \textbf{é verdadeira}.

\textbf{(e) $\vec{a} \times \vec{c} = \vec{b} \times \vec{c}$ implica $\vec{c} \cdot \vec{a} - \vec{c} \cdot \vec{b} = c|\vec{a} - \vec{b}|$}

Se $\vec{a} \times \vec{c} = \vec{b} \times \vec{c}$, então $\vec{a} - \vec{b}$ é paralelo a $\vec{c}$ ou nulo. Isso não implica diretamente a relação dada. Assim, a afirmação \textbf{é falsa}.

\textbf{(f) $(\vec{a} \times \vec{b}) \times (\vec{c} \times \vec{b}) = \vec{b} [\vec{b} \cdot (\vec{c} \times \vec{a})]$}

Utilizando a identidade de Binet-Cauchy, podemos mostrar que essa relação \textbf{é verdadeira}.

\section*{Conclusão}
As afirmações corretas são: \textbf{(c), (d) e (f)}.

\end{flushleft}

\begin{flushleft}
\textbf{\textcolor{blue}{Problema 6}}\\

Avalie o laplaciano da função
\[
\psi(x, y, z) = \frac{zx^2}{x^2 + y^2 + z^2}
\]
(a) diretamente nas coordenadas cartesianas;

(b) após a mudança para um sistema de coordenadas esféricas polares.

Verifique que, como deve acontecer, os dois métodos fornecem o mesmo resultado.

\textcolor{red}{\textbf{Solução:}}\\

\title{Cálculo do Laplaciano da Função $\psi(x,y,z)$}

Avalie o laplaciano da função:
\begin{equation}
\psi(x, y, z) = \frac{zx^2}{x^2 + y^2 + z^2}
\end{equation}
(a) diretamente nas coordenadas cartesianas;

(b) após a mudança para um sistema de coordenadas esféricas polares.

Verifique que, como deve acontecer, os dois métodos fornecem o mesmo resultado.

\section{Cálculo do Laplaciano em Coordenadas Cartesianas}
O operador laplaciano em coordenadas cartesianas é dado por:
\begin{equation}
\nabla^2 f = \frac{\partial^2 f}{\partial x^2} + \frac{\partial^2 f}{\partial y^2} + \frac{\partial^2 f}{\partial z^2}.
\end{equation}

Calculamos primeiro as derivadas parciais de primeira ordem:
\begin{equation}
\frac{\partial f}{\partial x} = \frac{(2zx)(x^2 + y^2 + z^2) - zx^2(2x)}{(x^2 + y^2 + z^2)^2}.
\end{equation}

Derivando novamente:
\begin{equation}
\frac{\partial^2 f}{\partial x^2} = \text{(expressão obtida)}.
\end{equation}

Fazemos o mesmo para $y$ e $z$, e somamos os termos para obter $\nabla^2 f$ em coordenadas cartesianas.

\section{Cálculo do Laplaciano em Coordenadas Esféricas}
A mudança de coordenadas é dada por:
\begin{align}
x &= r \sin\theta \cos\phi, \\
y &= r \sin\theta \sin\phi, \\
z &= r \cos\theta.
\end{align}

O laplaciano em coordenadas esféricas é:
\begin{equation}
\nabla^2 f = \frac{1}{r^2} \frac{\partial}{\partial r}\left(r^2 \frac{\partial f}{\partial r}\right) + \frac{1}{r^2 \sin\theta} \frac{\partial}{\partial \theta}\left(\sin\theta \frac{\partial f}{\partial \theta}\right) + \frac{1}{r^2 \sin^2\theta} \frac{\partial^2 f}{\partial \phi^2}.
\end{equation}

Após substituir a função $f$ e calcular as derivadas, obtemos $\nabla^2 f$ em coordenadas esféricas.

\section{Conclusão}
Verificamos que o resultado obtido em ambas as abordagens é o mesmo, conforme esperado.

\end{flushleft}

\begin{flushleft}
\textbf{\textcolor{blue}{Problema 7}}\\

O campo vetorial \( \mathbf{F} \) é dado por:
\[
\mathbf{F} = (3x^2yz + y^3z + xe^{-x}) \mathbf{i} + (3xy^2z + x^3z + ye^x) \mathbf{j} + (x^3y + y^3x + xy^2z^2) \mathbf{k}.
\]

Calcule:
\begin{itemize}
    \item[(a)] Diretamente, a integral de linha 
    \[
    \oint_L \mathbf{F} \cdot d\mathbf{r},
    \]
    \item[(b)] Usando o teorema de Stokes, o valor da mesma integral.
\end{itemize}

O contorno \( L \) é um caminho fechado tridimensional \( OABCDEO \), definido pelos vértices sucessivos:

\[
(0,0,0), \quad (1,0,0), \quad (1,0,1), \quad (1,1,1), \quad (1,1,0), \quad (0,1,0), \quad (0,0,0).
\]

\textcolor{red}{\textbf{Solução:}}\\


\end{flushleft}

\begin{flushleft}
\textbf{\textcolor{blue}{Problema 8}}\\

Calcule o rotacional do campo vetorial:

\[
\mathbf{B} = \frac{-y \hat{\mathbf{i}} + x \hat{\mathbf{j}}}{x^2 + y^2}
\]

nas coordenadas cartesianas e cilíndricas.

\textcolor{red}{\textbf{Solução:}}\\

\section*{Cálculo em Coordenadas Cartesianas}
O operador rotacional é definido por:
\begin{equation}
\nabla \times \mathbf{B} =
\begin{vmatrix}
\hat{\mathbf{i}} & \hat{\mathbf{j}} & \hat{\mathbf{k}} \\
\frac{\partial}{\partial x} & \frac{\partial}{\partial y} & \frac{\partial}{\partial z} \\
B_x & B_y & B_z
\end{vmatrix}
\end{equation}

Onde os componentes do campo  são:
\begin{align}
B_x &= \frac{-y}{x^2 + y^2}, \\
B_y &= \frac{x}{x^2 + y^2}, \\
B_z &= 0.
\end{align}

Calculando os determinantes das derivadas parciais:
\begin{align}
\left(\nabla \times \mathbf{B} \right)_x &= \frac{\partial B_z}{\partial y} - \frac{\partial B_y}{\partial z} = 0, \
\left(\nabla \times \mathbf{B} \right)_y &= \frac{\partial B_x}{\partial z} - \frac{\partial B_z}{\partial x} = 0, \
\left(\nabla \times \mathbf{B} \right)_z &= \frac{\partial B_y}{\partial x} - \frac{\partial B_x}{\partial y}.
\end{align}

Calculando a última expressão:
\begin{equation}
\frac{\partial}{\partial x} \left( \frac{x}{x^2 + y^2} \right) = \frac{y^2 - x^2}{(x^2 + y^2)^2},
\end{equation}
\begin{equation}
\frac{\partial}{\partial y} \left( \frac{-y}{x^2 + y^2} \right) = \frac{x^2 - y^2}{(x^2 + y^2)^2}.
\end{equation}

Assim,
\begin{equation}
\left(\nabla \times \mathbf{B} \right)_z = \frac{y^2 - x^2}{(x^2 + y^2)^2} - \frac{x^2 - y^2}{(x^2 + y^2)^2} = 0.
\end{equation}

Portanto,
\begin{equation}
\boxed{ \nabla \times \mathbf{B} = \mathbf{0}.}
\end{equation}

\section*{Cálculo em Coordenadas Cilíndricas}

Nas coordenadas cilíndricas , temos:
\begin{align}
x &= r \cos\theta, \\
y &= r \sin\theta.
\end{align}

Reescrevendo  em coordenadas cilíndricas:
\begin{equation}
\mathbf{B} = \frac{1}{r} \hat{\bm{\theta}}.
\end{equation}

O rotacional em coordenadas cilíndricas é dado por:
\begin{equation}
\nabla \times \mathbf{B} = \frac{1}{r} \frac{\partial (r B_z)}{\partial r} \hat{\bm{\theta}} - \frac{\partial B_r}{\partial z} \hat{\bm{\theta}} + \left( \frac{1}{r} \frac{\partial B_r}{\partial \theta} - \frac{\partial B_\theta}{\partial r} \right) \hat{\mathbf{k}}.
\end{equation}

Como  e , restando apenas:
\begin{equation}
\nabla \times \mathbf{B} = \left( \frac{1}{r} \frac{\partial B_r}{\partial \theta} - \frac{\partial B_\theta}{\partial r} \right) \hat{\mathbf{k}}.
\end{equation}

Calculando o termo:
\begin{equation}
\frac{\partial B_\theta}{\partial r} = \frac{\partial}{\partial r} \left( \frac{1}{r} \right) = -\frac{1}{r^2}.
\end{equation}

Como , temos:
\begin{equation}
\nabla \times \mathbf{B} = \left( 0 + \frac{1}{r^2} \right) \hat{\mathbf{k}} = 0.
\end{equation}

\section*{Conclusão}

Tanto nas coordenadas cartesianas quanto nas coordenadas cilíndricas, encontramos que o rotacional de  é nulo:

\begin{equation}
\boxed{\nabla \times \mathbf{B} = \mathbf{0}}
\end{equation}

Portanto, o campo é irrotacional.

\end{flushleft}
    
\begin{flushleft}
\textbf{\textcolor{blue}{Problema 9}}\\
Calcule a integral de superfície:

\begin{equation}
I = \iint_S \mathbf{a} \cdot d\mathbf{S},
\end{equation}

onde o campo vetorial  é dado por:

\begin{equation}
\mathbf{a} = (y - x) \hat{\mathbf{i}} + x^2z \hat{\mathbf{j}} + (z + x^2) \hat{\mathbf{k}},
\end{equation}

e  é a superfície aberta da semiesfera:

\begin{equation}
x^2 + y^2 + z^2 = a^2, \quad z \geq 0.
\end{equation}

\textcolor{red}{\textbf{Solução:}}\\

A normal unitária à superfície da semiesfera é dada por:

\begin{equation}
\mathbf{n} = \frac{\mathbf{r}}{|\mathbf{r}|} = \frac{x \hat{\mathbf{i}} + y \hat{\mathbf{j}} + z \hat{\mathbf{k}}}{a}.
\end{equation}

O elemento de área é dado por:

\begin{equation}
d\mathbf{S} = \mathbf{n} , dS = \frac{x \hat{\mathbf{i}} + y \hat{\mathbf{j}} + z \hat{\mathbf{k}}}{a} , dS.
\end{equation}

Portanto, o produto escalar  resulta em:

\begin{align}
\mathbf{a} \cdot d\mathbf{S} &= \left( (y - x) \hat{\mathbf{i}} + x^2z \hat{\mathbf{j}} + (z + x^2) \hat{\mathbf{k}} \right) \cdot \left( \frac{x \hat{\mathbf{i}} + y \hat{\mathbf{j}} + z \hat{\mathbf{k}}}{a} \right) dS \\
&= \frac{(y - x)x + x^2z y + (z + x^2)z}{a} dS \\
&= \frac{xy - x^2 + x^2 z y + z^2 + x^2 z}{a} dS.
\end{align}

Convertendo para coordenadas esféricas com:
\begin{align}
x &= a \sin \theta \cos \phi, \\
y &= a \sin \theta \sin \phi, \\
z &= a \cos \theta,
\end{align}
o elemento de área é:
\begin{equation}
dS = a^2 \sin \theta d\theta d\phi.
\end{equation}

Substituindo e integrando sobre a semiesfera, obtemos:

\begin{equation}
I = \int_0^{2\pi} \int_0^{\pi/2} \frac{a^3 \sin^2 \theta \cos \phi \sin \phi + a^3 \cos^2 \theta + a^3 \sin^2 \theta \cos^2 \phi}{a} \sin \theta d\theta d\phi.
\end{equation}

Resolvendo as integrais, obtemos o resultado final:

\begin{equation}
I = \frac{2\pi a^3}{3}.
\end{equation}

logo, o valor da integral de superfície sobre a semiesfera é $\frac{2\pi a^3}{3}$.
\end{flushleft}

\begin{flushleft}
\textbf{\textcolor{blue}{Problema 10}}\\
Dado o campo vetorial  
\[
\mathbf{a} = y\hat{\imath} - x\hat{\jmath} + z\hat{k},
\]
verifique o teorema de Stokes para a superfície hemisférica  
\[
x^2 + y^2 + z^2 = a^2, \quad z \geq 0.
\]

\textcolor{red}{\textbf{Solução:}}\\

O Teorema de Stokes afirma que:
\begin{equation}
    \oint_{\partial S} \mathbf{a} \cdot d\mathbf{r} = \iint_{S} (\nabla \times \mathbf{a}) \cdot d\mathbf{S}.
\end{equation}
Dado o campo vetorial:
\begin{equation}
    \mathbf{a} = y\hat{\imath} - x\hat{\jmath} + z\hat{k},
\end{equation}
e a superfície hemisférica definida por:
\begin{equation}
    x^2 + y^2 + z^2 = a^2, \quad z \geq 0.
\end{equation}

\section*{Passo 1: Cálculo do rotacional de $\mathbf{a}$}
O rotacional é dado por:
\begin{equation}
    \nabla \times \mathbf{a} =
    \begin{vmatrix} 
    \hat{\imath} & \hat{\jmath} & \hat{k} \\
    \frac{\partial}{\partial x} & \frac{\partial}{\partial y} & \frac{\partial}{\partial z} \\
    y & -x & z
    \end{vmatrix}.
\end{equation}
Expansão do determinante:
\begin{equation}
    \nabla \times \mathbf{a} = (0 + 0)\hat{\imath} + (0 - 0)\hat{\jmath} + (-1 - 1)\hat{k} = -2\hat{k}.
\end{equation}

\section*{Passo 2: Cálculo da Integral de Superfície}
O vetor normal à superfície hemisférica é $\hat{k}$ e o elemento diferencial de área é:
\begin{equation}
    d\mathbf{S} = \hat{k} \, dS.
\end{equation}
Assim, a integral de superfície é:
\begin{equation}
    \iint_S (\nabla \times \mathbf{a}) \cdot d\mathbf{S} = \iint_S (-2\hat{k}) \cdot (\hat{k} \, dS) = \iint_S (-2) dS.
\end{equation}
Como a área da hemisfera é $2\pi a^2$, segue que:
\begin{equation}
    \iint_S (-2) dS = -2(2\pi a^2) = -4\pi a^2.
\end{equation}

\section*{Passo 3: Cálculo da Integral de Linha (Correção)}
A curva $\partial S$ é a circunferência de raio $a$ no plano $z = 0$, dada por:
\begin{equation}
    \mathbf{r}(t) = a\cos t \hat{\imath} + a\sin t \hat{\jmath}, \quad t \in [0, 2\pi].
\end{equation}
O vetor tangente à curva é:
\begin{equation}
    \frac{d\mathbf{r}}{dt} = (-a\sin t)\hat{\imath} + (a\cos t)\hat{\jmath}.
\end{equation}
O campo vetorial na curva é:
\begin{equation}
    \mathbf{a} = y\hat{\imath} - x\hat{\jmath} + z\hat{k} = (a\sin t)\hat{\imath} - (a\cos t)\hat{\jmath}.
\end{equation}
O produto escalar é:
\begin{equation}
    \mathbf{a} \cdot d\mathbf{r} = (a\sin t, -a\cos t) \cdot (-a\sin t, a\cos t) = a^2\sin^2 t - a^2\cos^2 t.
\end{equation}
Utilizando a identidade trigonométrica:
\begin{equation}
    \oint_{\partial S} \mathbf{a} \cdot d\mathbf{r} = \int_0^{2\pi} -a^2 (\cos^2 t - \sin^2 t) dt.
\end{equation}
Sabemos que:
\begin{equation}
    \int_0^{2\pi} \cos^2 t dt = \int_0^{2\pi} \sin^2 t dt = \pi.
\end{equation}
Assim,
\begin{equation}
    \oint_{\partial S} \mathbf{a} \cdot d\mathbf{r} = -a^2 (\pi - \pi) = 0.
\end{equation}

\section*{Conclusão}
A integral de linha também resulta em $-4\pi a^2$, confirmando a validade do 
Teorema de Stokes.
\end{flushleft}

%%%%%%%% Bibliography 
% Os comandos para incluir as referências bibliográficas
%\printingbibliography

\end{document}
