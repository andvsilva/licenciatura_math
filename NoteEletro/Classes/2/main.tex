\documentclass[a4paper,12pt]{article}
\usepackage[brazil, english]{babel}
\usepackage[utf8]{inputenc}
\usepackage[T1]{fontenc}
\usepackage{geometry}
\usepackage{setspace}
\usepackage{titlesec}
\usepackage{hyperref}
\usepackage{graphicx}
\usepackage{caption}
\usepackage{subcaption}
\usepackage{fancyhdr}
\usepackage{xcolor}
\usepackage{amsmath, amssymb, bm}
\usepackage{mathtools}
\usepackage{cancel}
\usepackage{tikz}
\usepackage{newunicodechar}
\usepackage{ragged2e}

\definecolor{ao(english)}{rgb}{0.0, 0.5, 0.0}
\definecolor{byzantium}{rgb}{0.44, 0.16, 0.39}
\newunicodechar{∘}{\circ}

%%%%%%%%%%%%%%%%%%%%%%%%%%%%%%%%%%%%%%%%%%%%%%%%%%
% These are some new commands that may be useful 
% for paper writing in general. If other new commands
% are needed for your specific paper, please feel 
% free to add here. 
%
% The currently available commands are organized in: 
% 1) Systems
% 2) Quantities
% 3) Energies and units
% 4) particle species
% 5) Colors package
% 6) hyperlink
%%%%%%%%%%%%%%%%%%%%%%%%%%%%%%%%%%%%%%%%%%%%%%%%%%

\usepackage{amsmath}
\usepackage{amssymb}
\usepackage{upgreek}
\usepackage{multirow}
\usepackage{setspace}% http://ctan.org/pkg/setspace
\usepackage{fancyhdr}
\usepackage{datetime}

% 1) SYSTEMS
\newcommand{\btc}               {\textbf{BTC}}
\newcommand{\btcspace}          {\textbf{BTC} }
\newcommand{\pow}               {\textbf{PoW}}

% 4) definition to references, biblatex and hyperlink
\usepackage[backend=bibtex, 
style=nature,  %style reference.
sorting=none,
firstinits=true %first name abbreviate
]{biblatex}

\usepackage{hyperref}
\hypersetup{
    colorlinks=true, %set "true" if you want colored links
    linktoc=all,     %set to "all" if you want both sections and subsections linked
    linkcolor=blue,  %choose some color if you want links to stand out
    citecolor= blue, % color of \cite{} in the text.
    urlcolor  = blue, % color of the link for the paper in references.
}

% 5) Tikz and figures
\usepackage{epsfig}
\usepackage{lmodern}
\usepackage{mathtools}
\usepackage[utf8]{luainputenc}
\usepackage{xspace}
\usepackage{tikz}
\usepackage{pgfplots}
\pgfplotsset{compat=newest}

\usetikzlibrary{positioning}
\usepackage{subcaption}

% 6) colors:
\usepackage{xcolor}
\definecolor{ao(english)}{rgb}{0.0, 0.5, 0.0} % dark green

% 7) Add lines numbers
%\usepackage{lineno}

% add pdf file to thesis:
\usepackage{pdfpages}

\hypersetup{
    colorlinks=true,% make the links colored
    linkcolor=blue
}

\usepackage{setspace}
\addbibresource{bibliography.bib}

\newcommand{\printingbibliography}{%

    \pagestyle{myheadings}
    \markright{}
    \sloppy
    \printbibliography[heading=bibintoc, % add to table of contents
                   title=Refer\^encias % Chapter name
                  ]
    \fussy%
}
\PassOptionsToPackage{table}{xcolor}

\pagestyle{fancy}
\fancyhf{}
\renewcommand{\headrulewidth}{0pt}
\fancyhead[R]{\thepage}

\geometry{a4paper,top=30mm,bottom=20mm,left=30mm,right=20mm}

\titleformat*{\section}{\bfseries\large}
\titleformat*{\subsection}{\bfseries\normalsize}

\title{ \textbf{\large Eletromagnetismo I }}
\author{Andr\'e V. Silva}
\date{\today}

\begin{document}

\noindent\rule{\linewidth}{0.8pt}\\
\begin{center}
    \textbf{UNIVERSIDADE FEDERAL DO RIO DE JANEIRO}\\
    \textbf{INSTITUTO DE FÍSICA}\\
    \textbf{\textcolor{blue}{Primeira lista complementar de Eletromagnetismo 1}}\\
    \textbf{Março de 2025}\\
    \vspace{0.5cm}
    Prof. João Torres de Mello Neto\\
    Monitor: Pedro Khan
\end{center}
\noindent\rule{\linewidth}{0.8pt}\\
\maketitle


\noindent\rule{\linewidth}{0.4pt}\\

\justifying

\begin{flushleft}
\textbf{\textcolor{blue}{Problema 1}}\\
\textbf{a)} Mostre que o rotacional de um gradiente de uma função qualquer é zero:
$\nabla \times (\nabla f) = 0,$ de duas formas: abrindo em componentes e 
argumentando pelo teorema de Stokes. 
\textbf{b)} Mostre que a divergência de um rotacional de um vetor qualquer é nula:
$\nabla \cdot (\nabla \times \mathbf{A}) = 0,$ de duas formas: calculando as 
componentes e argumentando pelo teorema de Stokes no limite que a integral de 
linha tende para zero e usando o teorema da divergência em seguida.
\end{flushleft}

\textcolor{red}{\textbf{Solução:}}\\

Em componentes cartesianas, o gradiente de uma função escalar $f$ é:
\begin{equation}
    \nabla f = \left( \frac{\partial f}{\partial x}, \frac{\partial f}{\partial y}, \frac{\partial f}{\partial z} \right).
\end{equation}
O rotacional é definido como:
\begin{equation}
    \nabla \times \mathbf{A} = 
    \begin{vmatrix} 
        \hat{i} & \hat{j} & \hat{k} \\
        \frac{\partial}{\partial x} & \frac{\partial}{\partial y} & \frac{\partial}{\partial z} \\
        \frac{\partial f}{\partial x} & \frac{\partial f}{\partial y} & \frac{\partial f}{\partial z} 
    \end{vmatrix}.
\end{equation}

Expansão do determinante:
\begin{equation}
    \nabla \times (\nabla f) = 
    \left( \frac{\partial^2 f}{\partial y \partial x} - \frac{\partial^2 f}{\partial x \partial y} \right) \hat{i} + 
    \left( \frac{\partial^2 f}{\partial z \partial y} - \frac{\partial^2 f}{\partial y \partial z} \right) \hat{j} + 
    \left( \frac{\partial^2 f}{\partial x \partial z} - \frac{\partial^2 f}{\partial z \partial x} \right) \hat{k}.
\end{equation}

Como as derivadas parciais mistas são comutativas (desde que $f$ seja suave), cada termo é zero:
\begin{equation}
    \nabla \times (\nabla f) = 0.
\end{equation}

Pelo teorema de Stokes, a integral de linha de um gradiente ao longo de um caminho fechado é zero, implicando que seu rotacional é nulo.

\textbf{b)} Mostre que a divergência de um rotacional de um vetor qualquer é nula:
\begin{equation}
    \nabla \cdot (\nabla \times \mathbf{A}) = 0
\end{equation}

de duas formas: calculando as componentes e argumentando pelo teorema de Stokes no limite que a integral de linha tende para zero e usando o teorema da divergência em seguida.

A divergência é definida como:

\begin{equation}
    \nabla \cdot \mathbf{A} = \frac{\partial A_x}{\partial x} + \frac{\partial A_y}{\partial y} + \frac{\partial A_z}{\partial z}.
\end{equation}

Aplicando à definição do rotacional:
\begin{equation}
    \nabla \times \mathbf{A} = 
    \begin{vmatrix} 
        \hat{i} & \hat{j} & \hat{k} \\
        \frac{\partial}{\partial x} & \frac{\partial}{\partial y} & \frac{\partial}{\partial z} \\
        A_x & A_y & A_z 
    \end{vmatrix},
\end{equation}

resultando em:
\begin{equation}
    \nabla \times \mathbf{A} = 
    \left( \frac{\partial A_z}{\partial y} - \frac{\partial A_y}{\partial z} \right) \hat{i} + 
    \left( \frac{\partial A_x}{\partial z} - \frac{\partial A_z}{\partial x} \right) \hat{j} + 
    \left( \frac{\partial A_y}{\partial x} - \frac{\partial A_x}{\partial y} \right) \hat{k}.
\end{equation}

Tomando a divergência:

\begin{equation}
    \nabla \cdot (\nabla \times \mathbf{A}) = \frac{\partial}{\partial x} \left( \frac{\partial A_z}{\partial y} - \frac{\partial A_y}{\partial z} \right) + 
    \frac{\partial}{\partial y} \left( \frac{\partial A_x}{\partial z} - \frac{\partial A_z}{\partial x} \right) + 
    \frac{\partial}{\partial z} \left( \frac{\partial A_y}{\partial x} - \frac{\partial A_x}{\partial y} \right).
\end{equation}

Como as derivadas mistas comutam, cada termo se anula, resultando em:

\begin{equation}
    \nabla \cdot (\nabla \times \mathbf{A}) = 0.
\end{equation}

Pelo teorema da divergência, a integral de volume da divergência de um rotacional se reduz a uma integral de superfície de um campo tangencial, que desaparece, confirmando o resultado.



%%%%%%%% Bibliography 
% Os comandos para incluir as referências bibliográficas
%\printingbibliography

\end{document}
