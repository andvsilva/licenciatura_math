\documentclass[a4paper,12pt]{article}
\usepackage[brazil, english]{babel}
\usepackage[utf8]{inputenc}
\usepackage[T1]{fontenc}
\usepackage{geometry}
\usepackage{setspace}
\usepackage{titlesec}
\usepackage{hyperref}
\usepackage{graphicx}
\usepackage{caption}
\usepackage{subcaption}
\usepackage{fancyhdr}
\setlength{\headheight}{15pt}
\addtolength{\topmargin}{-2.5pt}
\usepackage{xcolor}
\usepackage{amsmath, amssymb, bm}
\usepackage{mathtools}
\usepackage{cancel}
\usepackage{tikz}
\usepackage{newunicodechar}
\usepackage{ragged2e}
\usepackage{setspace}
\usepackage{tikz-3dplot} % Necessário para coordenadas 3D
\usetikzlibrary{intersections}
\usepackage{siunitx}
\usetikzlibrary{3d, arrows.meta}

\usepackage{color}
\definecolor{myblue}{rgb}{.8, .8, 1}

\usepackage{amsmath}
\usepackage{empheq}

\newlength\mytemplen
\newsavebox\mytempbox

\makeatletter
\newcommand\mybluebox{%
    \@ifnextchar[%]
       {\@mybluebox}%
       {\@mybluebox[0pt]}}

\def\@mybluebox[#1]{%
    \@ifnextchar[%]
       {\@@mybluebox[#1]}%
       {\@@mybluebox[#1][0pt]}}

\def\@@mybluebox[#1][#2]#3{
    \sbox\mytempbox{#3}%
    \mytemplen\ht\mytempbox
    \advance\mytemplen #1\relax
    \ht\mytempbox\mytemplen
    \mytemplen\dp\mytempbox
    \advance\mytemplen #2\relax
    \dp\mytempbox\mytemplen
    \colorbox{myblue}{\hspace{1em}\usebox{\mytempbox}\hspace{1em}}}
\makeatother

\usepackage[most]{tcolorbox}

\newtcbox{\mymath}[1][]{%
    nobeforeafter, math upper, tcbox raise base,
    enhanced, colframe=blue!30!black,
    colback=blue!30, boxrule=1pt,
    #1}

\tcbset{
    highlight math style={
        enhanced,
        colframe=red!60!black,
        colback=yellow!50,
        arc=4pt,
        boxrule=1pt,
        drop fuzzy shadow
    }
    }

\usepackage{physics}
\usepackage{pgfplots}
\pgfplotsset{compat=1.17}

\linespread{1.5}

\definecolor{ao(english)}{rgb}{0.0, 0.5, 0.0}
\definecolor{byzantium}{rgb}{0.44, 0.16, 0.39}
\newunicodechar{∘}{\circ}

%%%%%%%%%%%%%%%%%%%%%%%%%%%%%%%%%%%%%%%%%%%%%%%%%%
% These are some new commands that may be useful 
% for paper writing in general. If other new commands
% are needed for your specific paper, please feel 
% free to add here. 
%
% The currently available commands are organized in: 
% 1) Systems
% 2) Quantities
% 3) Energies and units
% 4) particle species
% 5) Colors package
% 6) hyperlink
%%%%%%%%%%%%%%%%%%%%%%%%%%%%%%%%%%%%%%%%%%%%%%%%%%

\usepackage{amsmath}
\usepackage{amssymb}
\usepackage{upgreek}
\usepackage{multirow}
\usepackage{setspace}% http://ctan.org/pkg/setspace
\usepackage{fancyhdr}
\usepackage{datetime}

% 1) SYSTEMS
\newcommand{\btc}               {\textbf{BTC}}
\newcommand{\btcspace}          {\textbf{BTC} }
\newcommand{\pow}               {\textbf{PoW}}

% 4) definition to references, biblatex and hyperlink
\usepackage[backend=bibtex, 
style=nature,  %style reference.
sorting=none,
firstinits=true %first name abbreviate
]{biblatex}

\usepackage{hyperref}
\hypersetup{
    colorlinks=true, %set "true" if you want colored links
    linktoc=all,     %set to "all" if you want both sections and subsections linked
    linkcolor=blue,  %choose some color if you want links to stand out
    citecolor= blue, % color of \cite{} in the text.
    urlcolor  = blue, % color of the link for the paper in references.
}

% 5) Tikz and figures
\usepackage{epsfig}
\usepackage{lmodern}
\usepackage{mathtools}
\usepackage[utf8]{luainputenc}
\usepackage{xspace}
\usepackage{tikz}
\usepackage{pgfplots}
\pgfplotsset{compat=newest}

\usetikzlibrary{positioning}
\usepackage{subcaption}

% 6) colors:
\usepackage{xcolor}
\definecolor{ao(english)}{rgb}{0.0, 0.5, 0.0} % dark green

% 7) Add lines numbers
%\usepackage{lineno}

% add pdf file to thesis:
\usepackage{pdfpages}

\hypersetup{
    colorlinks=true,% make the links colored
    linkcolor=blue
}

\usepackage{setspace}
\addbibresource{bibliography.bib}

\newcommand{\printingbibliography}{%

    \pagestyle{myheadings}
    \markright{}
    \sloppy
    \printbibliography[heading=bibintoc, % add to table of contents
                   title=Refer\^encias % Chapter name
                  ]
    \fussy%
}
\PassOptionsToPackage{table}{xcolor}

\pagestyle{fancy}
\fancyhf{}
\renewcommand{\headrulewidth}{0pt}
\fancyhead[R]{\thepage}

\geometry{a4paper,top=30mm,bottom=20mm,left=30mm,right=20mm}

\titleformat*{\section}{\bfseries\large}
\titleformat*{\subsection}{\bfseries\normalsize}

\title{ \textbf{\large Eletromagnetismo I }}
\author{Andr\'e V. Silva}
\date{\today}

\begin{document}

\noindent\rule{\linewidth}{0.8pt}\\
\begin{center}
    \textbf{UNIVERSIDADE FEDERAL DO RIO DE JANEIRO}\\
    \textbf{INSTITUTO DE FÍSICA}\\
    \textbf{\textcolor{blue}{Segunda lista complementar de Eletromagnetismo 1}}\\
    \textbf{Abril de 2025}\\
    \vspace{0.5cm}
    Prof. João Torres de Mello Neto\\
    Monitor: Pedro Khan
\end{center}
\noindent\rule{\linewidth}{0.8pt}\\
\maketitle


\noindent\rule{\linewidth}{0.4pt}\\

\justifying

\begin{flushleft}
\textbf{\textcolor{blue}{\Large Problema 1}}\\
Uma esfera inicialmente carregada com uma carga total Q e colocada em
contato moment\^aneo com uma esfera id\^entica inicialmente descarregada. \\
\textbf{a)} Qual \'e a carga em cada esfera ap\'os o contato?\\
\textbf{b)} Esse processo \'e repetido com N esferas identicas inicialmente descarregadas. 
Qual \'e a carga em cada uma das N + 1 esferas, incluindo a esfera que originalmente possuia a carga?\\
\textbf{c)} Qual \'e a carga total no sistema ap\'os N contatos?

\textcolor{red}{\textbf{Solução:}}\\
\end{flushleft}

\textbf{a)} Quando duas esferas idênticas entram em contato, a carga total se redistribui igualmente 
entre elas. Assim, a carga em cada esfera será:
\begin{equation}
q = \frac{Q}{2}
\end{equation}

\textbf{b)} O processo se repete: a esfera originalmente carregada (agora com carga \(\frac{Q}{2}\)) 
entra em contato com uma nova esfera descarregada, dividindo novamente sua carga por dois. 

Após cada contato, a carga da esfera carregada será dividida pela metade. Assim, após \( N \) contatos, 
a carga da esfera original será:

\tcbset{highlight math style={boxsep=5mm,colback=blue!30!red!30!white}}

\begin{equation}
    \tcbhighmath[boxrule=1pt,arc=1pt,colback=green!10!white,colframe=black,
      drop fuzzy shadow=red]{ q_N = \frac{Q}{2^N} }
\end{equation}

Cada nova esfera tocada recebe metade da carga da esfera carregada no momento do contato. Portanto:
\begin{itemize}
    \item 1ª esfera tocada: \( \frac{Q}{2} \)
    \item 2ª esfera tocada: \( \frac{Q}{4} \)
    \item 3ª esfera tocada: \( \frac{Q}{8} \)
    \item \(\vdots\)
    \item \( N \)-ésima esfera tocada: \( \frac{Q}{2^N} \)
    \item original: \( \frac{Q}{2^N} \)
\end{itemize}

\textbf{c)}
A carga total do sistema após os \( N \) contatos será a soma das cargas de todas as esferas:

\begin{equation}
Q_{\text{total}} = \left( \frac{Q}{2} + \frac{Q}{4} + \frac{Q}{8} + \cdots + \frac{Q}{2^N} \right) + \frac{Q}{2^N}
\end{equation}

O somatório \(\frac{Q}{2} + \frac{Q}{4} + \frac{Q}{8} + \cdots + \frac{Q}{2^N}\) é uma progressão geométrica 
de razão \( r = \frac{1}{2} \).

A soma dos \( N \) primeiros termos é:

\begin{equation}
S = \frac{\frac{Q}{2} \left(1 - \left( \frac{1}{2} \right)^N\right)}{1 - \frac{1}{2}}
= Q\left(1 - \left( \frac{1}{2} \right)^N\right)
\end{equation}

Somando com a carga restante na esfera original:

\begin{equation}
Q_{\text{total}} = Q\left(1 - \left( \frac{1}{2} \right)^N\right) + \frac{Q}{2^N}
\end{equation}

\begin{equation}
    \tcbhighmath[boxrule=1pt,arc=1pt,colback=green!10!white,colframe=black,
      drop fuzzy shadow=red]{Q_{\text{total}} = Q}
\end{equation}

Portanto, \colorbox{yellow}{a carga total do sistema permanece constante e igual a \( Q \)}, respeitando a conservação
da carga elétrica.


\newpage 

\begin{flushleft}
\textbf{\textcolor{blue}{\Large Problema 2}}\\
Uma placa infinita nos eixos \(x\) e \(y\) possui a seguinte distribuição superficial de carga:
\begin{equation}
    \tcbhighmath[boxrule=1pt,arc=1pt,colback=blue!10!white,colframe=black,
      drop fuzzy shadow=red]{\sigma(x,y) = \frac{\sigma_0 e^{-|x|/a}}{1 + (y/b)^2}}
\end{equation}
onde \(a\) e \(b\) são constantes. \\

\sloppy
\begin{center}
    \begin{tikzpicture}
    \begin{axis}[
        view={60}{30},
        xlabel={$x$},
        ylabel={$y$},
        zlabel={$\sigma(x,y)$},
        domain=-5:5,
        y domain=-3:3,
        samples=40,
        samples y=40,
        mesh/ordering=y varies,
        colormap/cool,
        z buffer=sort,
        title={$\boxed{ \sigma(x,y) = \dfrac{\sigma_0 e^{-|x|/a}}{1 + (y/b)^2} }$},
    ]
    \addplot3 [
        mesh,
        shader=interp,
    ]
    {exp(-abs(x)/1.5)/(1 + (y/1.2)^2)};
    \end{axis}
    \end{tikzpicture}
    \end{center}

\textcolor{red}{\textbf{Solução:}}\\
\end{flushleft}



A carga total \(Q\) na placa é dada pela integral da densidade superficial de carga \(\sigma(x, y)\) sobre toda a área da placa:
\begin{equation}
Q = \int_{-\infty}^{\infty} \int_{-\infty}^{\infty} \sigma(x, y) \, dx \, dy
\end{equation}

Substituindo a expressão de \(\sigma(x, y)\), temos:
\begin{equation}
Q = \int_{-\infty}^{\infty} \int_{-\infty}^{\infty} \frac{\sigma_0 e^{-|x|/a}}{1 + (y/b)^2} \, dx \, dy
\end{equation}

\section*{Passo 1: Integral sobre \(x\)}

Calculamos a integral sobre \(x\):
\begin{equation}
\int_{-\infty}^{\infty} e^{-|x|/a} \, dx
\end{equation}
Dividindo a integral em duas partes (por causa da função \(|x|\)), temos:
\begin{equation}
\int_{-\infty}^{\infty} e^{-|x|/a} \, dx = 2 \int_0^{\infty} e^{-x/a} \, dx
\end{equation}
A integral da exponencial é dada por:
\begin{equation}
\int_0^{\infty} e^{-x/a} \, dx = a
\end{equation}
Portanto, temos:
\begin{equation}
\int_{-\infty}^{\infty} e^{-|x|/a} \, dx = 2a
\end{equation}

\section*{Passo 2: Integral sobre \(y\)}

Agora, calculamos a integral sobre \(y\):
\begin{equation}
\int_{-\infty}^{\infty} \frac{1}{1 + (y/b)^2} \, dy
\end{equation}
Essa é uma integral padrão, conhecida como a integral de Cauchy, que resulta em:
\begin{equation}
\int_{-\infty}^{\infty} \frac{1}{1 + (y/b)^2} \, dy = \pi b
\end{equation}

\section*{Passo 3: Cálculo da carga total}

Agora que temos as integrais sobre \(x\) e \(y\), podemos calcular a carga total:
\begin{equation}
Q = \sigma_0 \cdot 2a \cdot \pi b
\end{equation}
Portanto, \colorbox{yellow}{a carga total na placa é}:

\begin{equation}
\tcbhighmath[boxrule=1pt,arc=1pt,colback=blue!10!white,colframe=black,
drop fuzzy shadow=red]{Q = \sigma_0 2\pi a b }
\end{equation}

\begin{flushleft}
\textbf{\textcolor{blue}{\Large Problema 3}}\\
Considere uma linha de carga com densidade linear uniforme \( \lambda_0 \), de comprimento 
total \( 2L \), centrada no eixo \( z \). Calcule o potencial elétrico em um ponto de campo 
localizado a uma distância \( r \) do eixo \( z \) (por exemplo, no plano \( xy \)) e a uma 
altura \( z \). Calcule o campo elétrico no mesmo ponto a partir do potencial. Calcule os limites 
quando \( L \gg r \) e calcule também o limite quando \( r \gg L \).

\textcolor{red}{\textbf{Solução:}}\\
\end{flushleft}

Considere uma linha de carga com densidade linear uniforme \( \lambda_0 \), de comprimento total \( 2L \), 
centrada no eixo \( z \).

\begin{figure}[h!]
\begin{center}
    \begin{tikzpicture}[scale=1.2]
    
    % Eixo z
    \draw[->] (0,-3) -- (0,3) node[above] {$z$};
    
    % Linha de carga de -L a L
    \draw[very thick, red] (0,-2) -- (0,2);
    \node[left] at (0,2) {$L$};
    \node[left] at (0,-2) {$-L$};
    
    % Ponto P
    \filldraw (2,1) circle (1.5pt) node[below] {};
    \node[left] at (3,0.7) {$P(r,z)$};
    \draw[->, red] (2,1) -- (2.7,1) node[right] {$E_{r}$};
    
    % Distância r
    \draw[dashed] (-0.2,1) -- (2,1);
    \draw[<->] (0,1.2) -- (2,1.2);
    \node[above] at (1,1.1) {$r$};
    
    % Distância z
    \draw[dashed] (0,0) -- (0,1);
    \draw[<->] (-0.2,0) -- (-0.2,1);
    \node[left] at (-0.2,0.5) {$z$};
    
    % Elemento dz'
    \draw[fill=blue] (0,0.5) circle (1pt);
    \node[left] at (0.7,0.4) {$z'$};
    \draw[<->] (0.2,0) -- (0.2,0.5);
    \draw[dashed] (0,0.5) -- (0.4,0.5);
    \draw[->, blue!80] (0,0.5) -- (2,1);
    
    % Indicação do eixo x para orientar
    \draw[->] (0,0) -- (2.5,0) node[right] {$x$};
    
    % Indicação do eixo y para orientar (para fora do plano)
    \draw[->] (0,0) -- (-0.7,-0.7) node[below] {$y$};
    
\end{tikzpicture}
\end{center}
\caption{Linha de carga com densidade linear uniforme \( \lambda_0 \), de comprimento total \( 2L \), 
centrada no eixo \( z \).}
\end{figure}

\subsection*{Potencial Elétrico}

Um elemento infinitesimal de carga é dado por:
\begin{equation}
\textcolor{blue!80}{dq} = \lambda_0 \, dz'
\end{equation}
O potencial devido a esse elemento no ponto \( (r, z) \) é:
\begin{equation}
dV = \frac{1}{4\pi\varepsilon_0} \frac{dq}{\sqrt{r^2 + (z - z')^2}}
\end{equation}
Substituindo \( dq \):
\begin{equation}
dV = \frac{\lambda_0}{4\pi\varepsilon_0} \frac{dz'}{\sqrt{r^2 + (z - z')^2}}
\end{equation}
O potencial total é a integral de \( dV \) de \( z' = -L \) até \( z' = L \):
\begin{equation}
V(r,z) = \frac{\lambda_0}{4\pi\varepsilon_0} \int_{-L}^{L} \frac{dz'}{\sqrt{r^2 + (z - z')^2}}
\end{equation}

Fazendo a substituição \( u = z - z' \), com \( du = -dz' \), temos:
\begin{equation}
V(r,z) = \frac{\lambda_0}{4\pi\varepsilon_0} \int_{z+L}^{z-L} \frac{-du}{\sqrt{r^2 + u^2}} = \frac{\lambda_0}{4\pi\varepsilon_0} \int_{z-L}^{z+L} \frac{du}{\sqrt{r^2 + u^2}}
\end{equation}

Integrando:
\begin{equation}
\int \frac{du}{\sqrt{r^2 + u^2}} = \ln\left(u + \sqrt{r^2 + u^2}\right) + C
\end{equation}

Aplicando os limites:
\begin{equation}
V(r,z) = \frac{\lambda_0}{4\pi\varepsilon_0} \left[ \ln\left(z+L + \sqrt{r^2 + (z+L)^2}\right) - \ln\left(z-L + \sqrt{r^2 + (z-L)^2}\right) \right]
\end{equation}
\begin{equation}
\boxed{
V(r,z) = \frac{\lambda_0}{4\pi\varepsilon_0} \ln\left( \frac{z+L + \sqrt{r^2 + (z+L)^2}}{z-L + \sqrt{r^2 + (z-L)^2}} \right)
}
\end{equation}

\subsection*{Campo Elétrico}

O campo elétrico é dado por:
\begin{equation}
\vec{E} = -\nabla V
\end{equation}
Em coordenadas cilíndricas (\( r, \theta, z \)) e considerando a simetria do problema:
\begin{equation}
E_r = -\frac{\partial V}{\partial r}, \quad E_z = -\frac{\partial V}{\partial z}, \quad E_\theta = 0
\end{equation}

\subsection*{Limites}

\subsubsection*{1. Quando \( L \gg r \)}

Neste caso, a linha de carga se comporta como um fio infinito. Aproximadamente:

Considere o potencial elétrico gerado por uma linha de carga de densidade linear constante $\lambda_0$, com comprimento total $2L$, centrada na origem ao longo do eixo $z$:

\begin{equation}
V(r,z) = \frac{\lambda_0}{4\pi \varepsilon_0} \ln \left( \frac{z + L + \sqrt{r^2 + (z + L)^2}}{z - L + \sqrt{r^2 + (z - L)^2}} \right)
\end{equation}

Para obter uma aproximação no caso em que o ponto de observação está no plano médio da linha de carga (\textit{i.e.}, $z = 0$), temos:

\begin{equation}
V(r) = \frac{\lambda_0}{4\pi \varepsilon_0} \ln \left( \frac{L + \sqrt{r^2 + L^2}}{-L + \sqrt{r^2 + L^2}} \right)
\end{equation}

Note que:
\begin{equation}
\frac{L + \sqrt{r^2 + L^2}}{-L + \sqrt{r^2 + L^2}} = \frac{\sqrt{r^2 + L^2} + L}{\sqrt{r^2 + L^2} - L}
\end{equation}

Logo, temos:
\begin{equation}
\boxed{V(r) = \frac{\lambda_0}{4\pi \varepsilon_0} \ln \left( \frac{\sqrt{r^2 + L^2} + L}{\sqrt{r^2 + L^2} - L} \right)}
\end{equation}

\begin{equation}
V(r) = \frac{\lambda_0}{4\pi \varepsilon_0} \ln \left( \frac{1 + \frac{L}{\sqrt{r^2 + L^2}}}{1 - \frac{L}{\sqrt{r^2 + L^2}}} \right)
\end{equation}

Utilizando a identidade:
\begin{equation}
\ln \left( \frac{1 + x}{1 - x} \right) = 2 \tanh^{-1}(x)
\end{equation}
obtemos:
\begin{equation}
V(r) = \frac{\lambda_0}{2\pi \varepsilon_0} \tanh^{-1} \left( \frac{L}{\sqrt{r^2 + L^2}} \right)
\end{equation}

Agora, assumimos que $L \gg r$. Expandimos a raiz utilizando binômio de Newton:

\begin{equation}
\sqrt{r^2 + L^2} = L \sqrt{1 + \frac{r^2}{L^2}} \approx L \left( 1 + \frac{r^2}{2L^2} \right)
\end{equation}

Então:

\begin{equation}
\frac{L}{\sqrt{r^2 + L^2}} \approx \frac{1}{1 + \frac{r^2}{2L^2}} \approx 1 - \frac{r^2}{2L^2}
\end{equation}

E, portanto:

\begin{equation}
\tanh^{-1} \left( \frac{L}{\sqrt{r^2 + L^2}} \right) \approx \tanh^{-1} \left( 1 - \frac{r^2}{2L^2} \right)
\end{equation}

Sabemos que para $x \approx 1$, vale:

\begin{equation}
\tanh^{-1}(x) \approx \ln \left( \frac{2}{1 - x} \right)
\end{equation}

Logo:

\begin{equation}
\tanh^{-1} \left( 1 - \frac{r^2}{2L^2} \right) \approx \ln \left( \frac{4L^2}{r^2} \right) = 2 \ln \left( \frac{2L}{r} \right)
\end{equation}

Substituindo no potencial:

\begin{equation}
V(r) \approx \frac{\lambda_0}{2\pi \varepsilon_0} \ln \left( \frac{2L}{r} \right)
\end{equation}

ou seja, quando $L \gg r$, o potencial elétrico no plano médio da linha de carga se aproxima de 
um fio carregado com densidade linear $\lambda_0$ e comprimento muito grande, o que nos leva 
ao caso idealizado de um \textbf{fio infinito}.

\begin{equation}
    \tcbhighmath[boxrule=1pt,arc=0.5pt,colback=red!10!white,colframe=black,
      drop fuzzy shadow=red]{
V(r) \sim \frac{\lambda_0}{2\pi\varepsilon_0} \ln\left(\frac{2L}{r}\right)
}
\end{equation}

\begin{equation}
    \tcbhighmath[boxrule=1pt,arc=0.5pt,colback=red!10!white,colframe=black,
      drop fuzzy shadow=red]{
E_r \sim \frac{\lambda_0}{2\pi\varepsilon_0 r}, \quad E_z \sim 0
}
\end{equation}

\subsubsection*{2. Quando \( r \gg L \)}

Aqui, o sistema se comporta como uma carga pontual de carga total \( Q = 2L\lambda_0 \). Portanto:
\begin{equation}
    \tcbhighmath[boxrule=1pt,arc=0.5pt,colback=red!10!white,colframe=black,
      drop fuzzy shadow=red]{
V(r) \sim \frac{Q}{4\pi\varepsilon_0 r} = \frac{2L\lambda_0}{4\pi\varepsilon_0 r}
}
\end{equation}

\begin{equation}
    \tcbhighmath[boxrule=1pt,arc=0.5pt,colback=red!10!white,colframe=black,
      drop fuzzy shadow=red]{
\vec{E} \sim \frac{Q}{4\pi\varepsilon_0 r^2} \hat{r}
}
\end{equation}

\begin{flushleft}
\textbf{\textcolor{blue}{\Large Problema 4}}\\

Considere um elétron em um átomo de hidrogênio a uma distância de \( 0{,}53 \times 10^{-10} \, \text{m} \) do próton. 
Sabendo que o próton tem carga \( +e \) e o elétron \( -e \), 

\begin{figure}[h]
    \centering
    \begin{tikzpicture}[scale=3]
    
    % Desenhar o círculo da órbita
    \draw[dashed, blue] (0,0) circle (1);
    
    % Próton no centro
    \filldraw[red] (0,0) circle (0.05);
    \node at (0.2,0.15) {\( (+e)\)};
    
    % Elétron na órbita
    \filldraw[blue] (1,0) circle (0.05);
    \node at (1.2,0.1) {\((-e)\)};
    
    % Vetor de força ou campo
    \draw[->, thick] (1,0) -- (0.5,0);
    \node at (0.75,0.1) {\(\vec{F}\)};
    
    % Distância
    \draw[<->] (0,1.2) -- (1,1.2);
    \node at (0.7,1.35) {\( r = 0{,}53 \times 10^{-10} \, \text{m} \)};
    \draw[->, dashed] (0,0) -- (0,1.2);
    \draw[->, dashed] (1,0) -- (1,1.2);
    
    \end{tikzpicture}
    \caption{Esquema do elétron orbitando o próton em um átomo de hidrogênio.}
\end{figure}


resolva:

\begin{enumerate}
    \item[a)] Calcule a energia potencial eletrostática do elétron em eV.
    \item[b)] Sabendo que a velocidade do elétron é \( v = 2{,}189 \times 10^6 \, \text{m/s} \), 
    calcule a energia total do elétron no átomo de hidrogênio em eV.
\end{enumerate}

\textcolor{red}{\textbf{Solução:}}\\

\subsection*{Letra (a): Energia Potencial Eletrostática}

A fórmula da energia potencial eletrostática \( U \) entre duas cargas \( q_1 \) e \( q_2 \) separadas por uma distância \( r \) é dada por:

\begin{equation}
    \tcbhighmath[boxrule=1pt,arc=0.5pt,colback=blue!10!white,colframe=black,
      drop fuzzy shadow=red]{U = \frac{k \cdot q_1 \cdot q_2}{r}}
\end{equation}

onde:
\begin{equation}
k = 8,99 \times 10^9 \, \text{N} \cdot \text{m}^2 / \text{C}^2 \quad \text{(constante eletrostática)},
\end{equation}
\begin{equation}
q_1 = e = 1,6 \times 10^{-19} \, \text{C} \quad \text{(carga do próton)},
\end{equation}
\begin{equation}
q_2 = -e = -1,6 \times 10^{-19} \, \text{C} \quad \text{(carga do elétron)},
\end{equation}
\begin{equation}
r = 0,53 \times 10^{-10} \, \text{m} \quad \text{(distância entre as cargas)}.
\end{equation}

Substituindo os valores na fórmula:

\begin{equation}
U = \frac{(8,99 \times 10^9) \cdot (1,6 \times 10^{-19}) \cdot (-1,6 \times 10^{-19})}{0,53 \times 10^{-10}}
\end{equation}

Calculando:

\begin{equation}
U \approx \frac{(8,99 \times 10^9) \cdot (-2,56 \times 10^{-38})}{0,53 \times 10^{-10}} \approx -4,32 \times 10^{-18} \, \text{J}
\end{equation}

Convertendo para eV, usando \( 1 \, \text{eV} = 1,602 \times 10^{-19} \, \text{J} \):

\begin{equation}
U \approx \frac{-4,32 \times 10^{-18}}{1,602 \times 10^{-19}} \approx -27 \, \text{eV}
\end{equation}

Portanto, \colorbox{yellow}{a energia energia potencial eletrostática é}:

\begin{equation}
\boxed{ U \approx -27 \, \text{eV}}
\end{equation}

\subsection*{Letra (b): Energia Total do Elétron}

A energia total \( E \) do elétron é a soma da energia cinética \( E_{\text{cinet}} \) e da energia potencial \( U \). A energia cinética é dada por:

\begin{equation}
    \tcbhighmath[boxrule=1pt,arc=0.5pt,colback=blue!10!white,colframe=black,
      drop fuzzy shadow=red]{E_{\text{cinet}} = \frac{1}{2} m v^2}
\end{equation}

onde:
\begin{equation}
m = 9,11 \times 10^{-31} \, \text{kg} \quad \text{(massa do elétron)},
\end{equation}
\begin{equation}
v = 2,189 \times 10^6 \, \text{m/s} \quad \text{(velocidade do elétron)}.
\end{equation}

Substituindo os valores:

\begin{equation}
E_{\text{cinet}} = \frac{1}{2} \cdot (9,11 \times 10^{-31}) \cdot (2,189 \times 10^6)^2
\end{equation}

Calculando:

\begin{equation}
E_{\text{cinet}} \approx \frac{1}{2} \cdot 9,11 \times 10^{-31} \cdot 4,79 \times 10^{12} \approx 2,18 \times 10^{-18} \, \text{J}
\end{equation}

Convertendo para eV:

\begin{equation}
E_{\text{cinet}} \approx \frac{2,18 \times 10^{-18}}{1,602 \times 10^{-19}} \approx 13,6 \, \text{eV}
\end{equation}

Agora, \colorbox{yellow}{a energia total do elétron é a soma da energia cinética e da energia potencial}:

\begin{equation}
E = E_{\text{cinet}} + U
\end{equation}

\begin{equation}
E = 13,6 \, \text{eV} + (-27 \, \text{eV}) \approx -13,6 \, \text{eV}
\end{equation}

Portanto, \colorbox{yellow}{a energia total do elétron no átomo de hidrogênio é}:

\begin{equation}
    \tcbhighmath[boxrule=1pt,arc=0.5pt,colback=green!10!white,colframe=black,
      drop fuzzy shadow=red]{E \approx -13,6 \, \text{eV}}
\end{equation}
\end{flushleft}

\textbf{\large Interpretação física}

\begin{itemize}
    \item O valor de –13,6 eV para a energia total do elétron no estado fundamental 
    do átomo de hidrogênio vem do modelo atômico de Bohr.
    \item O sinal negativo indica que o elétron está ligado ao núcleo — é necessária 
    energia positiva para removê-lo (ionização).
\end{itemize}

\begin{flushleft}
\textbf{\textcolor{blue}{\Large Problema 5}}\\

Imagine que a Terra tenha densidade uniforme e que um túnel seja escavado ao longo de um diâmetro.

\begin{enumerate}
    \item[a)] Se um objeto for solto no túnel, mostre que ele oscilaria com um período \( P \) igual ao 
    período de um satélite em órbita na superfície da Terra.
    \item[b)] Calcule \( P \).
\end{enumerate}

\begin{center}
\begin{tikzpicture}[scale=2]

    % Terra (círculo)
    \draw[fill=blue!10] (0,0) circle (1);
  
    % Túnel (linha pelo diâmetro)
    \draw[very thick, red] (-1,0) -- (1,0);
  
    % Centro da Terra
    \fill (0,0) circle (0.02);
    \node[below right] at (0,0) {Centro};
  
    % Objeto dentro do túnel
    \filldraw[black] (-0.5,0) circle (0.03);
    \node[above] at (-0.5,0) {Objeto};
  
    % Raio da Terra
    \draw[->, thick] (0,0) -- (0.707,0.707);
    \node[right] at (0.707,0.707) {$R$};
  
    % Eixos
    \draw[dashed] (0,-1.2) -- (0,1.2);
    \node at (0,1.3) {Eixo vertical};
  
    % Legenda
    \node[below] at (0,-1.2) {Túnel escavado ao longo do diâmetro da Terra};
  
\end{tikzpicture}
\end{center}

\textcolor{red}{\textbf{Solução:}}\\

\textbf{(a) Movimento do objeto no túnel}

A força gravitacional sentida a uma distância \( r \) do centro é devida apenas à massa dentro da esfera de raio \( r \), e é dada por:

\begin{equation}
\boxed{M_{\text{interna}} = M \left( \frac{r^3}{R^3} \right)}
\end{equation}

Assim, a força gravitacional é:

\begin{equation}
F = - G \frac{M_{\text{interna}} m}{r^2} = -G \frac{M \left( \frac{r^3}{R^3} \right) m}{r^2}
\end{equation}

Simplificando:

\begin{equation}
\boxed{F = -G \frac{M m}{R^3} r}
\end{equation}

Esta força é proporcional a \( r \) e dirigida para o centro, característica típica de um \textit{movimento harmônico simples} (MHS).

A equação do movimento é:

\begin{equation}
\boxed{
m \frac{d^2r}{dt^2} = -G \frac{M m}{R^3} r
}
\end{equation}

Dividindo por \( m \):

\begin{equation}
\frac{d^2r}{dt^2} = -\left( G \frac{M}{R^3} \right) r
\end{equation}

Portanto, a frequência angular \( \omega \) do movimento é:

\begin{equation}
\omega^2 = G \frac{M}{R^3}
\end{equation}

e o período \( P \) é:

\begin{equation}
P = \frac{2\pi}{\omega} = 2\pi \sqrt{\frac{R^3}{G M}}
\end{equation}

\vspace{0.5cm}

Para um satélite em órbita na superfície da Terra, o período também é dado por:

\begin{equation}
P = 2\pi \sqrt{\frac{r^3}{G M}}
\end{equation}

com \( r = R \), portanto:

\begin{equation}
P = 2\pi \sqrt{\frac{R^3}{G M}}
\end{equation}

Assim, o período do objeto no túnel é \textbf{igual} ao período do satélite em órbita rasante.

\vspace{0.5cm}

\textbf{(b) Cálculo do período}

Sabemos que:

\begin{equation}
g = \frac{G M}{R^2}
\quad \Rightarrow \quad G M = g R^2
\end{equation}

Substituindo:

\begin{equation}
P = 2\pi \sqrt{ \frac{R^3}{g R^2} } = 2\pi \sqrt{ \frac{R}{g} }
\end{equation}

Substituindo os valores:

\begin{equation}
R = 6{,}37 \times 10^6 \, \text{m}, \quad g = 9{,}8 \, \text{m/s}^2
\end{equation}

\begin{equation}
P = 2\pi \sqrt{ \frac{6{,}37 \times 10^6}{9{,}8} }
\end{equation}

\begin{equation}
P = 2\pi \sqrt{650000}
\end{equation}

\begin{equation}
P = 2\pi \times 806{,}2
\end{equation}

\begin{equation}
P \approx 5065 \, \text{segundos}
\end{equation}

Convertendo para minutos:

\begin{equation}
P \approx \frac{5065}{60} \approx 84{,}4 \, \text{minutos}
\end{equation}

\vspace{0.5cm}

\textbf{Resposta final:}

\begin{equation}
    \tcbhighmath[boxrule=1pt,arc=0.5pt,colback=green!10!white,colframe=black,
      drop fuzzy shadow=red]{P \approx 84{,}4 \, \text{minutos} = 1{,}406 \, \text{horas}}
\end{equation}

\end{flushleft}



\begin{flushleft}
\textbf{\textcolor{blue}{\Large Problema 6}}\\

Dois cilindros condutores longos e concêntricos são isolados entre si e carregados. Longe das extremidades, 
o cilindro interno possui densidade de carga linear \( +\lambda_1 \), e o cilindro externo possui densidade de 
carga linear \( +\lambda_2 \).

O cilindro interno apresenta raios interno \( r_1 \) e externo \( r_2 \), enquanto o cilindro externo apresenta raios interno \( r_3 \) e externo \( r_4 \).

\begin{center}
    \begin{tikzpicture}[scale=1.5]
    
      % Cores dos cilindros
      \fill[red!30] (0,0) circle (2.0);   % cilindro externo completo (r4)
      \fill[white] (0,0) circle (1.5);   % interior do cilindro externo (r3)

      \fill[blue!20] (0,0) circle (0.8);  % cilindro interno completo (r2)
      \fill[white] (0,0) circle (0.4);  % interior do cilindro interno (r1)
    
      % Linhas de contorno
      \draw[thick] (0,0) circle (0.4);   % r1
      \draw[thick] (0,0) circle (0.8);   % r2
      \draw[thick] (0,0) circle (1.5);   % r3
      \draw[thick] (0,0) circle (2.0);   % r4

      % superficie de gauss
      \draw[dashed, red] (0,0) circle (1.1); %S1
      \draw[dashed, red] (0,0) circle (2.2); %S2

      \filldraw[black] (0,0) circle (0.03);
    
      % Raio r1
      \draw[->] (0,0) -- (0.4,0) node[midway, below] {$r_1$};
    
      % Raio r2
      \draw[->] (0,0) -- (0.56,0.56) node[midway, above] {$r_2$};
    
      % Raio r3
      \draw[->] (0,0) -- (-1.07,-1.07) node[midway, above] {$r_3$};
    
      % Raio r4
      \draw[->] (0,0) -- (-1.95,0.5) node[midway, above] {$r_4$};
    
      % Cargas
      \node at (0.6,0.2) {$\lambda_1$};
      \node at (1.6,0.9) {$\lambda_2$};
    
      % Eixo vertical central (opcional)
      %\draw[dashed] (0,-2.2) -- (0,2.2);  
    
    \end{tikzpicture}
\end{center}

\begin{enumerate}
    \item[(a)] Encontre o campo elétrico \( E(r) \):
    \begin{enumerate}
        \item[(1)] Em um ponto próximo ao meio dos cilindros (desprezando efeitos de borda).
        \item[(2)] Logo fora do cilindro externo.
    \end{enumerate}

    \item[(b)] Encontre a diferença de potencial $\Delta\phi$ entre os dois cilindros.

    \item[(c)] Descreva qualitativamente as mudanças nos campos elétricos e nos potenciais se:
    \begin{enumerate}
        \item[(1)] O raio interno \( r_1 \) do cilindro interno for diminuído.
        \item[(2)] O raio externo \( r_2 \) do cilindro interno for aumentado.
        \item[(3)] A seção transversal externa do cilindro interno for transformada em
                   quadrado de lado \( 2r_2 \) (assumindo \( 2r_2 < r_3 \) ).
    \end{enumerate}
\end{enumerate}

\textcolor{red}{\textbf{Solução:}}\\

Dois cilindros condutores longos e concêntricos possuem densidades lineares de carga $\lambda_1$ (interno) e $\lambda_2$ (externo). Vamos resolver as questões:

\begin{itemize}
    \item[(a)] Encontrar $E(r)$:

    \begin{itemize}
        \item[(1)] \textbf{Em um ponto próximo ao meio (entre $r_2 < r < r_3$)}:

        Aplicando a Lei de Gauss:

        \begin{equation}
        \oint \vec{E} \cdot d\vec{A} = \frac{Q_{\text{interna}}}{\varepsilon_0}
        \end{equation}

        Como o sistema é cilíndrico:

        \begin{equation}
        E(2\pi r L) = \frac{\lambda_{\text{enc}} L}{\varepsilon_0}
        \end{equation}
        
        Logo:

        \begin{equation}
        E(r) = \frac{\lambda_{\text{enc}}}{2\pi\varepsilon_0 r}
        \end{equation}

        Onde $\lambda_{\text{enc}}$ é a carga linear total até o raio $r$.

        Para $r_2 < r < r_3$, somente o cilindro interno contribui, logo:

        \begin{equation}
        \lambda_{\text{enc}} = \lambda_1
        \end{equation}

        Assim:

        \begin{equation}
        E(r) = \frac{\lambda_1}{2\pi\varepsilon_0 r}
        \end{equation}

        \item[(2)] \textbf{Logo fora do cilindro externo ($r > r_4$)}:

        Agora, as cargas dos dois cilindros contribuem:

        \begin{equation}
        \lambda_{\text{enc}} = \lambda_1 + \lambda_2
        \end{equation}

        Portanto:

        \begin{equation}
        E(r) = \frac{\lambda_1 + \lambda_2}{2\pi\varepsilon_0 r}
        \end{equation}
    \end{itemize}

    \item[(b)] Encontrar a diferença de potencial $\Delta\phi$ entre os dois cilindros:

    O potencial $\phi(r)$ é dado por:

    \begin{equation}
    \phi(r) = -\int E(r) \, dr
    \end{equation}

    Logo, a diferença de potencial entre $r_2$ e $r_3$ é:

    \begin{equation}
    \Delta \phi = \phi(r_3) - \phi(r_2)
    \end{equation}

    Integrando:

    \begin{equation}
    \Delta \phi = -\int_{r_2}^{r_3} \frac{\lambda_1}{2\pi\varepsilon_0 r} \, dr
    \end{equation}

    \begin{equation}
    \Delta \phi = -\frac{\lambda_1}{2\pi\varepsilon_0} \ln\left(\frac{r_3}{r_2}\right)
    \end{equation}

    \item[(c)] Descrever qualitativamente:

    \begin{itemize}
        \item[(1)] Se $r_1$ for diminuído:

        A distribuição de carga no cilindro interno se concentra mais, mas, fora dele ($r > r_2$), o campo não muda, pois depende apenas da carga linear total $\lambda_1$.

        \item[(2)] Se $r_2$ for aumentado:

        O cilindro interno se torna mais largo. A distância até o cilindro externo diminui, o que pode alterar a diferença de potencial $\Delta \phi$ (diminuindo a magnitude do potencial).

        \item[(3)] Se a seção transversal externa do cilindro interno for transformada em um quadrado de lado $2r_2$:

        A simetria cilíndrica se perde. Assim, o campo elétrico deixa de ser puramente radial e passa a variar conforme a direção, especialmente próximo às bordas do quadrado. No entanto, se $\sqrt{2} r_2 < r_3$, ainda há uma região entre o quadrado e o cilindro externo onde o campo pode ser aproximadamente radial.
    \end{itemize}

\end{itemize}

\end{flushleft}

\begin{flushleft}
\textbf{\textcolor{blue}{\Large Problema 7}}\\
A partícula 1 tem massa \( m_1 = 3{,}6 \times 10^{-6} \, \text{kg} \), enquanto a partícula 2 tem 
massa \( m_2 = 6{,}2 \times 10^{-6} \, \text{kg} \). Ambas possuem a mesma carga elétrica. As partículas 
estão inicialmente em repouso, e o sistema de duas partículas possui uma energia potencial elétrica 
inicial de \( 0{,}150 \, \text{J} \).

As partículas são então liberadas e se repelem devido à força elétrica. Efeitos gravitacionais 
são desprezados, e nenhuma outra força atua sobre as partículas. Em um instante após a liberação, a 
velocidade da partícula 1 é medida como \( v_1 = 170 \, \text{m/s} \).

\begin{center}
    \begin{tikzpicture}[scale=1.1]
    
      % --- Situação Inicial ---
      \node at (-3.2,1.5) {\textbf{Antes da liberação}};
      
      % Partícula 1
      \fill[blue!30] (-4,0) circle (0.3);
      \node at (-4,0) {\scriptsize $m_1$};
      \node at (-4,-0.5) {\scriptsize $+q$};
    
      % Partícula 2
      \fill[red!30] (-2,0) circle (0.35);
      \node at (-2,0) {\scriptsize $m_2$};
      \node at (-2,-0.5) {\scriptsize $+q$};
    
      % Linha entre as partículas
      \draw[thick] (-3.7,0) -- (-2.35,0);
      \node at (-3,0.3) {\scriptsize $U_i = 0{,}150\,\text{J}$};
      \node at (-3,-1) {\scriptsize sistema em repouso};
    
      % --- Situação Após a liberação ---
      \node at (2.8,1.5) {\textbf{Após a liberação}};
    
      % Partícula 1 indo para a esquerda
      \fill[blue!30] (1,0) circle (0.3);
      \node at (1,0) {\scriptsize $m_1$};
      \node at (1,-0.5) {\scriptsize $+q$};
      \draw[->, thick] (0.7,0) -- (0,0);
      \node at (0.05,0.3) {\scriptsize $v_1 = 170\,\text{m/s}$};
    
      % Partícula 2 indo para a direita
      \fill[red!30] (4,0) circle (0.35);
      \node at (4,0) {\scriptsize $m_2$};
      \node at (4,-0.5) {\scriptsize $+q$};
      \draw[->, thick] (4.35,0) -- (5,0);
      \node at (4.5,0.3) {\scriptsize $v_2$};
    
      % Linha pontilhada entre as posições finais
      \draw[dashed] (1.3,0) -- (3.65,0);
    
      % Notas
      \node at (2.5,-1) {\scriptsize $E_c = \frac{1}{2}m_1v_1^2 + \frac{1}{2}m_2v_2^2$};
      \node at (2.5,-1.5) {\scriptsize $p = m_1v_1 = m_2v_2$};
    
    \end{tikzpicture}
\end{center}

\begin{enumerate}
    \item[(a)] Qual é a energia potencial elétrica do sistema de duas partículas nesse instante?
    \item[(b)] Que tipos de energia o sistema tinha inicialmente?
    \item[(c)] Que tipos de energia o sistema tem no instante posterior?
    \item[(d)] O princípio da conservação de energia se aplica? Justifique.
    \item[(e)] O princípio da conservação do momento linear se aplica? Justifique.
\end{enumerate}

\textcolor{red}{\textbf{Solução:}}\\

\section*{Dados do problema}

\begin{itemize}
    \item Massa da partícula 1: \( m_1 = 3{,}6 \times 10^{-6} \, \text{kg} \)
    \item Massa da partícula 2: \( m_2 = 6{,}2 \times 10^{-6} \, \text{kg} \)
    \item Velocidade da partícula 1: \( v_1 = 170 \, \text{m/s} \)
    \item Energia potencial inicial: \( U_i = 0{,}150 \, \text{J} \)
\end{itemize}

Efeitos gravitacionais são desprezados, e apenas forças elétricas atuam.

\section*{(a) Energia potencial elétrica do sistema nesse instante}

Pela conservação da energia:

\begin{equation}
E_{\text{inicial}} = E_{\text{final}}
\end{equation}
\begin{equation}
U_i = U_f + K
\end{equation}

Onde \( K \) é a energia cinética total:

\begin{equation}
K = \frac{1}{2} m_1 v_1^2 + \frac{1}{2} m_2 v_2^2
\end{equation}

Utilizando a conservação do momento linear:

\begin{equation}
m_1 v_1 + m_2 v_2 = 0
\quad \Rightarrow \quad
v_2 = -\frac{m_1}{m_2} v_1
\end{equation}

Substituindo os valores:

\begin{equation}
v_2 = -\frac{3{,}6 \times 10^{-6}}{6{,}2 \times 10^{-6}} \times 170
\quad \Rightarrow \quad
v_2 \approx -98{,}7 \, \text{m/s}
\end{equation}

Calculando a energia cinética:

\begin{equation}
K = \frac{1}{2} (3{,}6 \times 10^{-6}) (170)^2 + \frac{1}{2} (6{,}2 \times 10^{-6}) (98{,}7)^2
\end{equation}
\begin{equation}
K = (1{,}8 \times 10^{-6})(28900) + (3{,}1 \times 10^{-6})(9746{,}69)
\end{equation}
\begin{equation}
K = 0{,}05202 + 0{,}03031 = 0{,}08233 \, \text{J}
\end{equation}

Agora, determinamos \( U_f \):

\begin{equation}
U_f = U_i - K
\end{equation}
\begin{equation}
U_f = 0{,}150 - 0{,}08233
\end{equation}
\begin{equation}
\boxed{U_f = 0{,}0677 \, \text{J}}
\end{equation}

\section*{(b) Tipos de energia inicialmente}

Inicialmente, as partículas estavam em repouso, logo:

\begin{equation}
\text{Energia inicial} = \text{energia potencial elétrica}
\end{equation}

\colorbox{green!20}{\textbf{Resposta:} Apenas energia potencial elétrica.}

\section*{(c) Tipos de energia no instante posterior}

Após serem liberadas, as partículas possuem:

\begin{equation}
\text{Energia posterior} = \text{energia cinética} + \text{energia potencial elétrica}
\end{equation}

\colorbox{green!20}{\textbf{Resposta:} Energia cinética e energia potencial elétrica.}

\section*{(d) Conservação da energia}

Sim, a conservação da energia se aplica, pois:

\begin{itemize}
    \item O sistema é isolado (sem forças externas realizando trabalho).
    \item Apenas forças internas (elétricas) atuam.
\end{itemize}

\colorbox{green!20}{\textbf{Resposta:} O princípio da conservação da energia se aplica.}

\section*{(e) Conservação do momento linear}

Sim, a conservação do momento linear se aplica, pois:

\begin{itemize}
    \item O sistema está isolado (sem forças externas).
\end{itemize}

\colorbox{green!20}{\textbf{Resposta:} O princípio da conservação do momento linear se aplica.}
\end{flushleft}

\begin{flushleft}
\textbf{\textcolor{blue}{\Large Problema 8}}\\

Duas cargas puntiformes idênticas \( q_A = q_B = +2{,}4 \times 10^{-9} \, \text{C} \) estão 
fixas no espaço e separadas por \( 0{,}50 \, \text{m} \). Determine o campo elétrico e o potencial 
elétrico no ponto médio da linha entre as cargas \( q_A \) e \( q_B \).

\begin{enumerate}
    \item[(a)] Quais são as direções das contribuições individuais do campo elétrico de \( q_A \) e \( q_B \) no ponto médio (PM)?
    \item[(b)] O campo elétrico líquido no ponto médio tem módulo maior, menor ou igual a zero?
    \item[(c)] O potencial elétrico total no ponto médio é positivo, negativo ou zero?
    \item[(d)] O potencial elétrico total tem direção associada?
    \item[(e)] Calcule o valor do campo e do potencial elétrico no ponto médio.
\end{enumerate}

\begin{center}
    \begin{tikzpicture}[scale=1.5, >=stealth]
    
      % Cargas
      \draw[fill=red!50] (0,0) circle (2pt) node[above=6pt] {$q_A$};
      \draw[fill=red!50] (4,0) circle (2pt) node[above=6pt] {$q_B$};
    
      % Linha entre as cargas
      \draw[thick] (0,0) -- (4,0);
    
      % Ponto médio
      \draw[fill=blue!60] (2,0) circle (2pt) node[above=6pt] {PM};
    
      % Campo elétrico no ponto médio
      \draw[->, thick, blue] (2,0) -- (2.7,0) node[above right] {$\vec{E}_A$};
      \draw[->, thick, blue] (2,0) -- (1.3,0) node[above left] {$\vec{E}_B$};
    
      % Distâncias
      \draw[<->] (0,-0.5) -- (2,-0.5) node[midway, below] {\SI{0.25}{m}};
      \draw[<->] (2,-0.5) -- (4,-0.5) node[midway, below] {\SI{0.25}{m}};
      \draw[<->, thick] (0,-1.0) -- (4,-1.0) node[midway, below] {\SI{0.50}{m}};

      \draw[dashed] (0,0) -- (0,-1.2);
      \draw[dashed] (2,0) -- (2,-0.6);
      \draw[dashed] (4,0) -- (4,-1.2);
    
    \end{tikzpicture}
\end{center}

\textcolor{red}{\textbf{Solução:}}\\

\begin{enumerate}
    \item \textbf{Quais são as direções das contribuições individuais do campo elétrico de \( q_A \) e \( q_B \) no ponto médio?}
    
    Como as duas cargas são positivas, o campo elétrico gerado por cada uma delas no ponto médio aponta para fora da carga, ou seja:
    \begin{itemize}
        \item O campo de \( q_A \) aponta para a direita.
        \item O campo de \( q_B \) aponta para a esquerda.
    \end{itemize}
    
    \item \textbf{O campo elétrico líquido no ponto médio tem módulo maior, menor ou igual a zero?}
    
    Como os campos têm mesma intensidade e direções opostas, eles se anulam.
    \begin{equation}
    \boxed{E_{\text{total}} = 0}
    \end{equation}
    
    \item \textbf{O potencial elétrico total no ponto médio é positivo, negativo ou zero?}
    
    O potencial elétrico é escalar e soma-se algébrica e positivamente no ponto médio, já que ambas as cargas são positivas.
    \begin{equation}
    \boxed{V_{\text{total}} > 0}
    \end{equation}
    
    \item \textbf{O potencial elétrico total tem direção associada?}
    
    Não. O potencial elétrico é uma grandeza escalar, portanto:
    \begin{equation}
    \boxed{\text{O potencial não possui direção associada.}}
    \end{equation}
    
    \item \textbf{Calcule o valor do campo e do potencial elétrico no ponto médio.}
    
    Distância de cada carga ao ponto médio: \( r = \frac{0{,}50}{2} = 0{,}25 \, \text{m} \)

    \begin{itemize}
        \item \textbf{Campo elétrico (contribuição de uma carga):}
        \begin{equation}
        E = k \frac{q}{r^2} = 9{,}0 \times 10^9 \cdot \frac{2{,}4 \times 10^{-9}}{(0{,}25)^2} 
        = \frac{21{,}6}{0{,}0625} = 345{,}6 \, \text{N/C}
        \end{equation}
        Como os campos se anulam:
        \begin{equation}
        \boxed{E_{\text{total}} = 0 \, \text{N/C}}
        \end{equation}
        
        \item \textbf{Potencial elétrico total:}
        \begin{equation}
        V = 2 \cdot k \frac{q}{r} = 2 \cdot 9{,}0 \times 10^9 \cdot \frac{2{,}4 \times 10^{-9}}{0{,}25}
        = 2 \cdot \frac{21{,}6}{0{,}25} = 172{,}8 \, \text{V}
        \end{equation}
        \begin{equation}
        \boxed{V_{\text{total}} = 172{,}8 \, \text{V}}
        \end{equation}
    \end{itemize}
\end{enumerate}
\end{flushleft}

\begin{flushleft}
\textbf{\textcolor{blue}{\Large Problema 9}}\\

Uma região esférica de raio \( R \) está preenchida com carga de tal forma que o campo elétrico no 
interior da região é dado por:
\begin{equation}
\mathbf{E} = \frac{E_0}{R^2} \, \mathbf{r}
\end{equation}
onde \( \mathbf{r} \) é o vetor posição a partir do centro da esfera, e \( E_0 \) é uma constante.\\

\colorbox{red!15}{Determine a densidade de carga na região.}

\begin{center}
    \begin{tikzpicture}[scale=2.5, >=Latex]

        % Esfera vista em 2D
        \shade[ball color=blue!10!white, opacity=0.3] (0,0) circle (1);
        \draw[thick] (0,0) circle (1);
      
        % Centro da esfera
        \fill (0,0) circle (0.5pt) node[below left] {$O$};
      
        % Vetores posição r e campo E (mesma direção)
        \foreach \angle in {30, 70, 110, 150, 210, 250, 290, 330} {
          \draw[->, red, thick] (0,0) -- ({0.9*cos(\angle)}, {0.9*sin(\angle)})
            node[pos=0.9, anchor=south east] {\scriptsize $\vec{E}$};
        }
      
        % Vetor posição r em destaque
        \draw[->, thick, blue] (0,0) -- (0.7,0.4)
          node[midway, above right] {\footnotesize $\vec{r}$};
      
        % Raio da esfera
        \draw[dashed] (0,0) -- (1,0) node[midway, below] {\footnotesize $R$};
      
        % Título
        \node at (0, -1.2) {Região esférica com $\vec{E} = \dfrac{E_0}{R^2} \, \vec{r}$};
      
      \end{tikzpicture}
\end{center}

\textcolor{red}{\textbf{Solução:}}\\

Uma região esférica de raio \( R \) está preenchida com carga de tal forma que o campo elétrico no interior da região é dado por:

\begin{equation}
\mathbf{E} = \frac{E_0}{R^2} \, \mathbf{r}
\end{equation}

onde \( \mathbf{r} \) é o vetor posição a partir do centro da esfera, e \( E_0 \) é uma constante. Determine a densidade de carga na região.

\section*{Solução}

Para determinar a densidade de carga, usamos a Lei de Gauss na forma diferencial:

\begin{equation}
\boxed{\nabla \cdot \mathbf{E} = \frac{\rho}{\varepsilon_0}}
\end{equation}

onde \( \rho \) é a densidade de carga e \( \varepsilon_0 \) é a permissividade do vácuo.

O campo elétrico no interior da esfera é dado por:

\begin{equation}
\mathbf{E} = \frac{E_0}{R^2} \, \mathbf{r}
\end{equation}

Agora, vamos calcular o divergente do campo elétrico.

\subsection*{Cálculo do Divergente de \( \mathbf{E} \)}

O campo elétrico tem a forma:

\begin{equation}
\mathbf{E} = \frac{E_0}{R^2} (x \hat{i} + y \hat{j} + z \hat{k})
\end{equation}

Onde \( \mathbf{r} = x \hat{i} + y \hat{j} + z \hat{k} \) é o vetor posição.

O divergente em coordenadas cartesianas é dado por:

\begin{equation}
\nabla \cdot \mathbf{E} = \frac{\partial E_x}{\partial x} + \frac{\partial E_y}{\partial y} + \frac{\partial E_z}{\partial z}
\end{equation}

Com:

\begin{equation}
E_x = \frac{E_0}{R^2} x, \quad E_y = \frac{E_0}{R^2} y, \quad E_z = \frac{E_0}{R^2} z
\end{equation}

Calculando as derivadas parciais:

\begin{equation}
\frac{\partial E_x}{\partial x} = \frac{E_0}{R^2}, \quad \frac{\partial E_y}{\partial y} = \frac{E_0}{R^2}, \quad \frac{\partial E_z}{\partial z} = \frac{E_0}{R^2}
\end{equation}

Somando as derivadas:

\begin{equation}
\nabla \cdot \mathbf{E} = 3 \times \frac{E_0}{R^2} = \frac{3 E_0}{R^2}
\end{equation}

\subsection*{Aplicando a Lei de Gauss}

De acordo com a Lei de Gauss, temos:

\begin{equation}
\nabla \cdot \mathbf{E} = \frac{\rho}{\varepsilon_0}
\end{equation}

Substituindo o valor do divergente:

\begin{equation}
\frac{3 E_0}{R^2} = \frac{\rho}{\varepsilon_0}
\end{equation}

Isolando \( \rho \):

\begin{equation}
\rho = \varepsilon_0 \frac{3 E_0}{R^2}
\end{equation}

\section*{Resultado Final}

A densidade de carga na região é:

\begin{equation}
\boxed{\rho = \frac{3 \varepsilon_0 E_0}{R^2}}
\end{equation}

\end{flushleft}

\begin{flushleft}
\textbf{\textcolor{blue}{\Large Problema 10}}\\

Determine o campo elétrico \( \vec{E} \) e a densidade volumétrica de carga \( \rho \) para as 
seguintes distribuições de potencial elétrico:

\begin{itemize}
    \item[(a)] \( V = A x^2 \)
    \item[(b)] \( V = A x y z \)
\end{itemize}

\textcolor{red}{\textbf{Solução:}}\\

\section*{(a) \( V = A x^2 \)}

\textbf{1. Calcular o campo elétrico \( \vec{E} \)}

O campo elétrico é dado por:

\begin{equation}
\boxed{\vec{E} = -\nabla V}
\end{equation}

Para \( V = A x^2 \), calculamos \underline{o gradiente em coordenadas cartesianas}. Como \(V\) depende apenas de \(x\), temos:

\begin{equation}
\nabla V = \frac{\partial V}{\partial x} \hat{i} + \frac{\partial V}{\partial y} \hat{j} + \frac{\partial V}{\partial z} \hat{k}
\end{equation}

Logo:

\begin{equation}
\frac{\partial V}{\partial x} = 2 A x, \quad \frac{\partial V}{\partial y} = 0, \quad \frac{\partial V}{\partial z} = 0
\end{equation}

Portanto, o gradiente de \(V\) é:

\begin{equation}
\nabla V = 2 A x \hat{i}
\end{equation}

O campo elétrico será:

\begin{equation}
\vec{E} = -\nabla V = - 2 A x \hat{i}
\end{equation}

\textbf{2. Calcular a densidade de carga \( \rho \)}

Agora, aplicamos a Lei de Gauss:

\begin{equation}
\boxed{\nabla \cdot \vec{E} = \frac{\rho}{\varepsilon_0}}
\end{equation}

Primeiro, calculamos o divergente do campo elétrico \( \vec{E} \):

\begin{equation}
\nabla \cdot \vec{E} = \frac{\partial}{\partial x} (-2 A x) + 0 + 0 = -2 A
\end{equation}

Portanto:

\begin{equation}
\frac{\rho}{\varepsilon_0} = -2 A
\end{equation}

Logo, a densidade de carga é:

\begin{equation}
\boxed{\rho = -2 A \varepsilon_0}
\end{equation}

\section*{(b) \( V = A x y z \)}

\textbf{1. Calcular o campo elétrico \( \vec{E} \)}

O campo elétrico é dado por:

\begin{equation}
\boxed{\vec{E} = -\nabla V}
\end{equation}

Para \( V = A x y z \), calculamos o gradiente:

\begin{equation}
\nabla V = \frac{\partial V}{\partial x} \hat{i} + \frac{\partial V}{\partial y} \hat{j} + \frac{\partial V}{\partial z} \hat{k}
\end{equation}

Logo:

\begin{equation}
\frac{\partial V}{\partial x} = A y z, \quad \frac{\partial V}{\partial y} = A x z, \quad \frac{\partial V}{\partial z} = A x y
\end{equation}

Portanto, o gradiente de \( V \) é:

\begin{equation}
\nabla V = A y z \hat{i} + A x z \hat{j} + A x y \hat{k}
\end{equation}

O campo elétrico será:

\begin{equation}
\vec{E} = -\nabla V = - A y z \hat{i} - A x z \hat{j} - A x y \hat{k}
\end{equation}

\textbf{2. Calcular a densidade de carga \( \rho \)}

Agora, aplicamos a Lei de Gauss:

\begin{equation}
\boxed{\nabla \cdot \vec{E} = \frac{\rho}{\varepsilon_0}}
\end{equation}

Calculando o divergente de \( \vec{E} \):

\begin{equation}
\nabla \cdot \vec{E} = \frac{\partial}{\partial x} (- A y z) + \frac{\partial}{\partial y} (- A x z) + \frac{\partial}{\partial z} (- A x y)
\end{equation}

\begin{equation}
\nabla \cdot \vec{E} = 0 + 0 + 0 = 0
\end{equation}

Logo, a densidade de carga é:

\begin{equation}
\boxed{\rho = 0}
\end{equation}

\section*{Resultado Final}

Para o potencial \( V = A x^2 \):
\begin{itemize}
\item Campo elétrico: \( \vec{E} = - 2 A x \hat{i} \)
\item Densidade de carga: \( \rho = - 2 A \varepsilon_0 \)
\end{itemize}

Para o potencial \( V = A x y z \):

\begin{itemize}
\item Campo elétrico: \( \vec{E} = - A y z \hat{i} - A x z \hat{j} - A x y \hat{k} \)
\item Densidade de carga: \( \rho = 0 \)
\end{itemize}

\end{flushleft}


\begin{flushleft}
\textbf{\textcolor{blue}{\Large Problema 11}}\\

Quais dos seguintes vetores podem ser um campo elétrico? Se forem, qual é a densidade 
volumétrica de carga associada?

\begin{itemize}
    \item[(a)] \( \vec{E} = a x^2 y^2 \, \hat{x} \)
    \item[(b)] \( \vec{E} = a ( \hat{r} \cos \theta - \hat{\theta} \sin \theta ) \)
\end{itemize}

\textcolor{red}{\textbf{Solução:}}\\

\subsection*{(a) \( \vec{E} = a x^2 y^2 \, \hat{x} \)}

Para verificar se o vetor \( \vec{E} = a x^2 y^2 \hat{x} \) pode ser um campo elétrico, devemos calcular seu divergente.

Em coordenadas cartesianas, o divergente de \( \vec{E} \) é dado por:

\begin{equation}
\nabla \cdot \vec{E} = \frac{\partial}{\partial x}(a x^2 y^2) + \frac{\partial}{\partial y}(0) + \frac{\partial}{\partial z}(0)
\end{equation}

Como \( \vec{E} \) não depende de \( y \) nem de \( z \), temos:

\begin{equation}
\nabla \cdot \vec{E} = \frac{\partial}{\partial x}(a x^2 y^2) = 2a x y^2
\end{equation}

Portanto, a densidade de carga associada é:

\begin{equation}
\rho = \varepsilon_0 \nabla \cdot \vec{E} = \varepsilon_0 (2a x y^2)
\end{equation}

Este resultado indica que o campo elétrico pode ser gerado por uma distribuição de carga 
não uniforme, logo, \( \vec{E} = a x^2 y^2 \hat{x} \) pode representar um campo elétrico.

\subsection*{(b) \( \vec{E} = a ( \hat{r} \cos \theta - \hat{\theta} \sin \theta ) \)}

Agora, temos o campo elétrico \( \vec{E} = a ( \hat{r} \cos \theta - \hat{\theta} \sin \theta ) \) em 
coordenadas cilíndricas. Para verificar se esse vetor pode ser um campo elétrico, devemos calcular o seu divergente.

Em coordenadas cilíndricas, o divergente de um vetor \( \vec{E} = E_r \hat{r} + E_\theta \hat{\theta} \) é dado por:

\begin{equation}
\nabla \cdot \vec{E} = \frac{1}{r} \frac{\partial}{\partial r}(r E_r) + \frac{1}{r} \frac{\partial}{\partial \theta}(E_\theta) + \frac{\partial}{\partial z}(E_z)
\end{equation}

Para o campo \( \vec{E} = a ( \hat{r} \cos \theta - \hat{\theta} \sin \theta ) \), temos:

\begin{equation}
E_r = a \cos \theta, \quad E_\theta = -a \sin \theta, \quad E_z = 0
\end{equation}

Logo, o divergente de \( \vec{E} \) em coordenadas cilíndricas é:

\begin{equation}
\nabla \cdot \vec{E} = \frac{1}{r} \frac{\partial}{\partial r}(r a \cos \theta) + \frac{1}{r} \frac{\partial}{\partial \theta}(-a \sin \theta)
\end{equation}

A primeira derivada em \( r \) é zero, pois \( E_r \) não depende de \( r \). A segunda derivada em \( \theta \) é:

\begin{equation}
\frac{\partial}{\partial \theta}(-a \sin \theta) = -a \cos \theta
\end{equation}

Portanto:

\begin{equation}
\nabla \cdot \vec{E} = \frac{-a \cos \theta}{r}
\end{equation}

Então, a densidade de carga associada é:

\begin{equation}
\rho = \varepsilon_0 \nabla \cdot \vec{E} = -\frac{a \varepsilon_0 \cos \theta}{r}
\end{equation}

Este resultado indica que o campo elétrico pode ser gerado por uma distribuição de carga que depende 
de \( \theta \) e \( r \). Logo, \( \vec{E} = a ( \hat{r} \cos \theta - \hat{\theta} \sin \theta ) \) também 
pode representar um campo elétrico.

\section*{Resultado Final}

\begin{itemize}
\item Para o campo \( \vec{E} = a x^2 y^2 \hat{x} \), o divergente é \( \nabla \cdot \vec{E} = 2a x y^2 \), e a 
densidade de carga associada é \(\boxed{ \rho = \varepsilon_0 (2a x y^2) .}\)

\item Para o campo \( \vec{E} = a ( \hat{r} \cos \theta - \hat{\theta} \sin \theta ) \), o divergente 
é $$ \nabla \cdot \vec{E} = \frac{-a \cos \theta}{r},$$ e a densidade de carga associada 
é $$\boxed{ \rho = -\frac{a \varepsilon_0 \cos \theta}{r}.}$$
\end{itemize}

\end{flushleft}



%%%%%%%% Bibliography 
% Os comandos para incluir as referências bibliográficas
%\printingbibliography

\end{document}
