\documentclass[a4paper,12pt]{article}
\usepackage[brazil, english]{babel}
\usepackage[utf8]{inputenc}
\usepackage[T1]{fontenc}
\usepackage{geometry}
\usepackage{setspace}
\usepackage{titlesec}
\usepackage{hyperref}
\usepackage{graphicx}
\usepackage{caption}
\usepackage{subcaption}
\usepackage{fancyhdr}
\usepackage{xcolor}
\usepackage{amsmath, amssymb, bm}
\usepackage{mathtools}
\usepackage{cancel}
\usepackage{tikz}
\usepackage{newunicodechar}
\usepackage{ragged2e}
\usepackage{setspace}
\usepackage{tikz-3dplot} % Necessário para coordenadas 3D
\usetikzlibrary{intersections}

\linespread{1.5}

\definecolor{ao(english)}{rgb}{0.0, 0.5, 0.0}
\definecolor{byzantium}{rgb}{0.44, 0.16, 0.39}
\newunicodechar{∘}{\circ}

%%%%%%%%%%%%%%%%%%%%%%%%%%%%%%%%%%%%%%%%%%%%%%%%%%
% These are some new commands that may be useful 
% for paper writing in general. If other new commands
% are needed for your specific paper, please feel 
% free to add here. 
%
% The currently available commands are organized in: 
% 1) Systems
% 2) Quantities
% 3) Energies and units
% 4) particle species
% 5) Colors package
% 6) hyperlink
%%%%%%%%%%%%%%%%%%%%%%%%%%%%%%%%%%%%%%%%%%%%%%%%%%

\usepackage{amsmath}
\usepackage{amssymb}
\usepackage{upgreek}
\usepackage{multirow}
\usepackage{setspace}% http://ctan.org/pkg/setspace
\usepackage{fancyhdr}
\usepackage{datetime}

% 1) SYSTEMS 
\newcommand{\pp}           {pp\xspace}
\newcommand{\ppbar}        {\mbox{$\mathrm {p\overline{p}}$}\xspace}
\newcommand{\XeXe}         {\mbox{Xe--Xe}\xspace}
\newcommand{\PbPb}         {\mbox{Pb--Pb}\xspace}
\newcommand{\pA}           {\mbox{pA}\xspace}
\newcommand{\pPb}          {\mbox{p--Pb}\xspace}
\newcommand{\AuAu}         {\mbox{Au--Au}\xspace}
\newcommand{\dAu}          {\mbox{d--Au}\xspace}
\def\pA{$pA$\xspace}
\def\AA{$AA$\xspace}
\def\NN{$NN$\xspace}
\def\signn{$\sigma^{inel}_{NN}$\xspace}
\def\sigtotal{$\sigma_{\textnormal{tot}}$\xspace}
\def\mrm{\mathrm}
\def\ntrig{N_\mrm{trig}}
\newcommand{\rivet}{R\protect\scalebox{1}{IVET}\xspace}
\newcommand{\hepmc}{H\protect\scalebox{1}{EP}MC\xspace}
\newcommand{\herwig}{H\protect\scalebox{1}{ERWIG} 7\xspace}
\newcommand{\sherpa}{S\protect\scalebox{1}{HERPA}\xspace}
\newcommand{\urqmd}{U\protect\scalebox{1}{r}QMD\xspace}
\newcommand{\urqmdversion}{U\protect\scalebox{1}{r}QMD 3.4\xspace}
\newcommand{\pythia}{\protect\scalebox{1}{PYTHIA}\xspace}
\newcommand{\pythiaversion}{\protect\scalebox{1}{PYTHIA 8.2}\xspace}
\newcommand{\pythiaversionused}{\protect\scalebox{1}{PYTHIA 8.235}\xspace}
\newcommand{\pytang}{\protect\scalebox{1}{PYTHIA}/Angantyr\xspace}
\newcommand{\angantyr}{\protect\scalebox{1}{}Angantyr\xspace}
\newcommand{\pytangur}{\protect\scalebox{1}{PYTHIA}/Angantyr + U\protect\scalebox{1}{r}QMD\xspace}
\newcommand{\figref}[1]{Fig.~\ref{#1}}
\newcommand{\tabref}[1]{Tab.~\ref{#1}}
\renewcommand{\eqref}[1]{Eq.~(\ref{#1})}

% hydrodynamic simulation chain:
% TRENTo
\newcommand{\trento}{\protect\scalebox{1}{T$_{\text{R}}$ENT}o\xspace}
% KOMPOST : Linear kinetic theory propagator for initial conditions in heavy ion collisions
\newcommand{\kompost}{\protect\scalebox{1}{K$\varnothing$MP$\varnothing$ST}\xspace}
% MUSIC
\newcommand{\music}{\protect\scalebox{1}{MUSIC}\xspace}
% iSS
\newcommand{\iss}{\protect\scalebox{1}{iSS}\xspace}

% 2) QUANTITIES 
\newcommand{\s}            {\ensuremath{\sqrt{s}}\xspace}
\newcommand{\snn}          {\ensuremath{\sqrt{s_{\mathrm{NN}}}}\xspace}
\newcommand{\pt}           {\ensuremath{p_{\rm T}}\xspace}
\newcommand{\meanpt}       {$\langle p_{\mathrm{T}}\rangle$\xspace}
\newcommand{\ycms}         {\ensuremath{y_{\rm CMS}}\xspace}
\newcommand{\ylab}         {\ensuremath{y_{\rm lab}}\xspace}
\newcommand{\etarange}[1]  {\mbox{$\left | \eta \right |~<~#1$}}
\newcommand{\centbin}[2]  {\mbox{$#1-#2\%$}}
\newcommand{\ptrange}[2]  {\mbox{$#1 < p_{\mathrm{T}}\hspace{0.2cm} (\mathrm{GeV}/\mathrm{\textit{c}}) <#2$}}
\newcommand{\ptrangetrig}[2]  {\mbox{$#1 < p^{\mathrm{trigger}}_{\mathrm{T} }\hspace{0.2cm} (\mathrm{GeV}/\mathrm{\textit{c}}) <#2$}}
\newcommand{\ptrangeassoc}[2]  {\mbox{$#1 < p^{\mathrm{assoc}}_{\mathrm{T} }\hspace{0.2cm} (\mathrm{GeV}/\mathrm{\textit{c}}) <#2$}}
\newcommand{\etazerothree} {$\left|\eta \right| < 0.3$\xspace}
\newcommand{\etazerofive} {$\left|\eta \right| < 0.5$\xspace}
\newcommand{\etazeroeight} {$\left|\eta \right| < 0.8$\xspace}
\newcommand{\yrange}[1]    {\mbox{$\left | y \right |~<~#1$}}
\newcommand{\dndy}         {\ensuremath{\mathrm{d}N_\mathrm{ch}/\mathrm{d}y}\xspace}
\newcommand{\dndeta}       {\ensuremath{\mathrm{d}N_\mathrm{ch}/\mathrm{d}\eta}\xspace}
\newcommand{\dnchdydpt}   {\ensuremath{\mathrm{d}N_\mathrm{ch}/\mathrm{d}y\mathrm{d}p_{\mathrm{T}}}\xspace}
\newcommand{\dnchaadydpt}   {\ensuremath{\mathrm{d}N_\mathrm{ch}^{AA}/\mathrm{d}y\mathrm{d}p_{\mathrm{T}}}\xspace}
\newcommand{\dnchppdydpt}   {\ensuremath{\mathrm{d}N_\mathrm{ch}^{\mathrm{pp}}/\mathrm{d}y\mathrm{d}p_{\mathrm{T}}}\xspace}
\newcommand{\dnchdphi}{\ensuremath{\mathrm{d}N_\mathrm{ch}/\mathrm{d}\phi}\xspace}
\newcommand{\dnchddeltaphi}{\ensuremath{\mathrm{d}N_\mathrm{ch}/\mathrm{d}\Delta\upphi}\xspace}
\newcommand{\dndphi}{\ensuremath{\mathrm{d}N/\mathrm{d}\phi}\xspace}
\newcommand{\dnddeltaphi}{\ensuremath{\mathrm{d}N/\mathrm{d}\Delta\upphi}\xspace}
\newcommand{\avdndeta}     {\ensuremath{\langle\dndeta\rangle}\xspace}
\newcommand{\avdndetarap}  {$\langle$ dN$_{\textnormal{ch}}$/d$\eta$ $\rangle_{|\eta| < 0.5}$\xspace}
\newcommand{\dNdy}         {\ensuremath{\mathrm{d}N_\mathrm{ch}/\mathrm{d}y}\xspace}
\newcommand{\Npart}        {\ensuremath{N_\mathrm{part}}\xspace}
\newcommand{\meanNpart}    {$\langle$\ensuremath{N_\mathrm{part}}$\rangle$\xspace}
\newcommand{\ncoll}        {\ensuremath{N_\mathrm{coll}}\xspace}
\newcommand{\meanncoll}    {$\langle$\ensuremath{N_\mathrm{coll}}$\rangle$\xspace}
\newcommand{\averagencollhadronic}    {$\langle$\ensuremath{\mathrm{N}_\mathrm{coll}^{\mathrm{hadronic}}}$\rangle$\xspace}
\newcommand{\meantaa}      {$\langle$\ensuremath{T_\mathrm{AA}}$\rangle$\xspace}
\newcommand{\dEdx}         {\ensuremath{\textrm{d}E/\textrm{d}x}\xspace}
\newcommand{\RpPb}         {\ensuremath{R_{\rm pPb}}\xspace}
\newcommand{\raa}          {$R_{AA}$\xspace}
\newcommand{\vtwo}         {$v_{2}$\xspace}
\newcommand{\vtwoinitial}  {$v_{2}^{\mathrm{initial}}$\xspace}
\newcommand{\vtwofinal}    {$v_{2}^{\mathrm{final}}$\xspace}
\newcommand{\vtwofourfinal}{$v_{2}^{\mathrm{final}}\{4\}$\xspace}
\newcommand{\vtwofit}      {$v_{2}^{\mathrm{Fit}}$\xspace}
\newcommand{\vtwotwo}      {$v_{2}\{2\}$\xspace}
\newcommand{\vtwofour}     {$v_{2}\{4\}$\xspace}
\newcommand{\vtwopt}       {$v_{2}(p_{\textnormal{T}})$\xspace}
\newcommand{\vtwoptfit}    {$v_{2}^{\mathrm{Fit}}(p_{\textnormal{T}})$\xspace}
\newcommand{\nch}          {\ensuremath{N_\mathrm{ch}}\xspace}
\newcommand{\psireactionplane}          {$\Psi_{\textnormal{RP}}$\xspace}
\newcommand{\deltaphireactionplane}     {$\Delta\upphi = \phi - \Psi_{\textnormal{RP}}$\xspace}
\newcommand{\nevdnchddeltaphi}     {(1/N$_{\textnormal{ev}}$)dN$_{\textnormal{ch}}$/d$\Delta\upphi$\xspace}
\newcommand{\meannch}      {\ensuremath{\langle N_\mathrm{ch}\rangle}\xspace}
\newcommand{\etamodule}    {\ensuremath{|\eta|}\xspace}
\newcommand{\qbar}         {$\bar{\textnormal{q}}$\xspace}
\newcommand{\qqbar}        {$\textnormal{q}\bar{\textnormal{q}}$\xspace}
\newcommand{\qqbarzero}    {$\textnormal{q}_{0}\bar{\textnormal{q}}_{0}$\xspace}
\newcommand{\qqqbars}      {$\bar{\textnormal{q}}\bar{\textnormal{q}}\bar{\textnormal{q}}$\xspace}
\newcommand{\alphastrong}  {$\alpha_{\textnormal{s}}$\xspace}
\newcommand{\alphastrongdistance}  {$\alpha_{\textnormal{s}}$(R)\xspace}
\newcommand{\qtwo}         {Q$^2$\xspace}
\newcommand{\alphastrongqtwo}  {$\alpha_{\textnormal{s}}$(Q$^2$)\xspace}
\newcommand{\lambdaqcd}        {$\Lambda_{\textnormal{QCD}}$\xspace}
\newcommand{\sectionpp}        {$\sigma^{\textnormal{pp}}_{\textnormal{inel}}$\xspace}

% 3) ENERGIES, UNITS
\newcommand{\sqrts}        {$\sqrt{s}$\xspace}
\newcommand{\sqrtsnn}      {$\sqrt{s_{\mathrm{NN}}}$\xspace}
\newcommand{\nineH}        {$\sqrt{s}~=~0.9$~Te\kern-.1emV\xspace}
\newcommand{\seven}        {$\sqrt{s}~=~7$~Te\kern-.1emV\xspace}
\newcommand{\twoH}         {$\sqrt{s}~=~0.2$~Te\kern-.1emV\xspace}
\newcommand{\twosevensix}  {$\sqrt{s}~=~2.76$~Te\kern-.1emV\xspace}
\newcommand{\five}         {$\sqrt{s}~=~5.02$~Te\kern-.1emV\xspace}
\newcommand{\twohundrernn} {$\sqrt{s_{\mathrm{NN}}}=200$~Ge\kern-.1emV\xspace}
\newcommand{\twosevensixnn} {$\sqrt{s_{\mathrm{NN}}}=2.76$~Te\kern-.1emV\xspace}
\newcommand{\fivenn}       {$\sqrt{s_{\mathrm{NN}}}~=~5.02$~Te\kern-.1emV\xspace}
\newcommand{\fivefourfournn} {$\sqrt{s_{\mathrm{NN}}}=5.44$~Te\kern-.1emV\xspace}
\newcommand{\LT}           {L{\'e}vy-Tsallis\xspace}
\newcommand{\GeVc}         {Ge\kern-.1emV/$c$\xspace}
\newcommand{\MeVc}         {Me\kern-.1emV/$c$\xspace}
\newcommand{\TeV}          {Te\kern-.1emV\xspace}
\newcommand{\GeV}          {Ge\kern-.1emV\xspace}
\newcommand{\MeV}          {Me\kern-.1emV\xspace}
\newcommand{\GeVmass}      {Ge\kern-.2emV/$c^2$\xspace}
\newcommand{\MeVmass}      {Me\kern-.2emV/$c^2$\xspace}
\newcommand{\lumi}         {\ensuremath{\mathcal{L}}\xspace}
\newcommand{\fmc}         {fm\kern-.1em/$c$\xspace}

% 4) PARTICLE SPECIES 
\newcommand{\ee}           {\ensuremath{e^{+}e^{-}}} 
\newcommand{\pip}          {\ensuremath{\pi^{+}}\xspace}
\newcommand{\pim}          {\ensuremath{\pi^{-}}\xspace}
\newcommand{\kap}          {\ensuremath{\rm{K}^{+}}\xspace}
\newcommand{\kam}          {\ensuremath{\rm{K}^{-}}\xspace}
\newcommand{\pbar}         {\ensuremath{\rm\overline{p}}\xspace}
\newcommand{\kzero}        {\ensuremath{{\rm K}^{0}_{\rm{S}}}\xspace}
\newcommand{\lmb}          {\ensuremath{\Lambda}\xspace}
\newcommand{\almb}         {\ensuremath{\overline{\Lambda}}\xspace}
\newcommand{\Om}           {\ensuremath{\Omega^-}\xspace}
\newcommand{\Mo}           {\ensuremath{\overline{\Omega}^+}\xspace}
\newcommand{\X}            {\ensuremath{\Xi^-}\xspace}
\newcommand{\Ix}           {\ensuremath{\overline{\Xi}^+}\xspace}
\newcommand{\Xis}          {\ensuremath{\Xi^{\pm}}\xspace}
\newcommand{\Oms}          {\ensuremath{\Omega^{\pm}}\xspace}
\newcommand{\degree}       {\ensuremath{^{\rm o}}\xspace}
\newcommand{\comment}[1]{}

% two-particle angular correlation
\newcommand{\deltaphitriggassoc}    {$\Delta\upphi = |\phi_{\textnormal{trigger}} - \phi_{\textnormal{assoc}}|$\xspace}
\newcommand{\deltaetatriggassoc}    {$\Delta\upeta = |\eta_{\textnormal{trigger}} - \eta_{\textnormal{assoc}}|$\xspace}
\newcommand{\etatrigg}    {$\eta_{\textnormal{trigger}}$\xspace}
\newcommand{\etaassoc}    {$\eta_{\textnormal{assoc}}$\xspace}
\newcommand{\deltaphideltaeta}      {$\Delta\upphi-\Delta\upeta$\xspace}
\newcommand{\deltaphi}              {$\Delta\upphi$\xspace}
\newcommand{\moduledeltaphipitwo}   {$|\Delta\upphi| < \pi/2 $\xspace}
\newcommand{\deltaeta}              {$\Delta\upeta$\xspace}
\newcommand{\moduledeltaeta}        {$|\Delta\upeta|$\xspace}
\newcommand{\deltaphiapproxzero}    {$\Delta\upphi = 0$\xspace}
\newcommand{\deltaphiapproxpi}      {$\Delta\upphi = \pi$\xspace}
\newcommand{\deltaetaapproxzero}    {$\Delta\upeta = 0$\xspace}
\newcommand{\corrfunc}              {C($\Delta\upphi$, $\Delta\upeta$)\xspace}
\newcommand{\corrfunccorrect}              {C$_{\mathrm{correct}}(\Delta\upphi$, $\Delta\upeta$)\xspace}
\newcommand{\corrfuncmix}              {C$_{\mathrm{mix}}(\Delta\upphi$, $\Delta\upeta$)\xspace}
\newcommand{\corrfuncdeltaphi}      {C($\Delta\upphi$)\xspace}
\newcommand{\pttrigger}             {$p_{\textnormal{T}}^{\textnormal{trigger}}$\xspace}
\newcommand{\ptassoc}               {$p_{\textnormal{T}}^{\textnormal{assoc}}$\xspace}
\newcommand{\ratioyieldawaynearside}{Y$_{\textnormal{Away}}$/Y$_{\textnormal{Near}}$\xspace}

% 4) definition to references, biblatex and hyperlink
\usepackage[backend=bibtex, 
style=nature,  %style reference.
sorting=none,
firstinits=true %first name abbreviate
]{biblatex}

\usepackage{hyperref}
\hypersetup{
    colorlinks=true, %set "true" if you want colored links
    linktoc=all,     %set to "all" if you want both sections and subsections linked
    linkcolor=blue,  %choose some color if you want links to stand out
    citecolor= blue, % color of \cite{} in the text.
    urlcolor  = blue, % color of the link for the paper in references.
}

% 5) Tikz and figures
\usepackage{epsfig}
\usepackage{lmodern}
\usepackage{mathtools}
\usepackage[utf8]{luainputenc}
\usepackage{xspace}
\usepackage{tikz}
\usepackage{pgfplots}
\pgfplotsset{compat=newest}

\usetikzlibrary{positioning}
\usepackage{subcaption}

% 6) colors:
\usepackage{xcolor}
\definecolor{ao(english)}{rgb}{0.0, 0.5, 0.0} % dark green

% 7) Add lines numbers
%\usepackage{lineno}

% add pdf file to thesis:
\usepackage{pdfpages}

\hypersetup{
    colorlinks=true,% make the links colored
    linkcolor=blue
}

\usepackage{setspace}
\addbibresource{bibliography.bib}

\newcommand{\printingbibliography}{%

    \pagestyle{myheadings}
    \markright{}
    \sloppy
    \printbibliography[heading=bibintoc, % add to table of contents
                   title=Refer\^encias % Chapter name
                  ]
    \fussy%
}
\PassOptionsToPackage{table}{xcolor}

\pagestyle{fancy}
\fancyhf{}
\renewcommand{\headrulewidth}{0pt}
\fancyhead[R]{\thepage}

\geometry{a4paper,top=30mm,bottom=20mm,left=30mm,right=20mm}

\titleformat*{\section}{\bfseries\large}
\titleformat*{\subsection}{\bfseries\normalsize}

\title{ \textbf{\large Eletromagnetismo I }}
\author{Andr\'e V. Silva}
\date{\today}

\begin{document}

\noindent\rule{\linewidth}{0.8pt}\\
\begin{center}
    \textbf{UNIVERSIDADE FEDERAL DO RIO DE JANEIRO}\\
    \textbf{INSTITUTO DE FÍSICA}\\
    \textbf{\textcolor{blue}{Segunda lista complementar de Eletromagnetismo 1}}\\
    \textbf{Abril de 2025}\\
    \vspace{0.5cm}
    Prof. João Torres de Mello Neto\\
    Monitor: Pedro Khan
\end{center}
\noindent\rule{\linewidth}{0.8pt}\\
\maketitle


\noindent\rule{\linewidth}{0.4pt}\\

\justifying

\begin{flushleft}
\textbf{\textcolor{blue}{\Large Problema 1}}\\
Uma esfera inicialmente carregada com uma carga total Q e colocada em
contato moment\^aneo com uma esfera id\^entica inicialmente descarregada. \\
\textbf{a)} Qual \'e a carga em cada esfera ap\'os o contato?\\
\textbf{b)} Esse processo \'e repetido com N esferas identicas inicialmente descarregadas. 
Qual \'e a carga em cada uma das N + 1 esferas, incluindo a esfera que originalmente possuia a carga?\\
\textbf{c)} Qual \'e a carga total no sistema ap\'os N contatos?

\textcolor{red}{\textbf{Solução:}}\\
\end{flushleft}

\textbf{a)} Quando duas esferas idênticas entram em contato, a carga total se redistribui igualmente 
entre elas. Assim, a carga em cada esfera será:
\begin{equation}
q = \frac{Q}{2}
\end{equation}

\textbf{b)} O processo se repete: a esfera originalmente carregada (agora com carga \(\frac{Q}{2}\)) 
entra em contato com uma nova esfera descarregada, dividindo novamente sua carga por dois. 

Após cada contato, a carga da esfera carregada será dividida pela metade. Assim, após \( N \) contatos, 
a carga da esfera original será:

\begin{equation}
q_N = \frac{Q}{2^N}
\end{equation}

Cada nova esfera tocada recebe metade da carga da esfera carregada no momento do contato. Portanto:
\begin{itemize}
    \item 1ª esfera tocada: \( \frac{Q}{2} \)
    \item 2ª esfera tocada: \( \frac{Q}{4} \)
    \item 3ª esfera tocada: \( \frac{Q}{8} \)
    \item \(\vdots\)
    \item \( N \)-ésima esfera tocada: \( \frac{Q}{2^N} \)
    \item original: \( \frac{Q}{2^N} \)
\end{itemize}

\textbf{c)}
A carga total do sistema após os \( N \) contatos será a soma das cargas de todas as esferas:

\begin{equation}
Q_{\text{total}} = \left( \frac{Q}{2} + \frac{Q}{4} + \frac{Q}{8} + \cdots + \frac{Q}{2^N} \right) + \frac{Q}{2^N}
\end{equation}

O somatório \(\frac{Q}{2} + \frac{Q}{4} + \frac{Q}{8} + \cdots + \frac{Q}{2^N}\) é uma progressão geométrica 
de razão \( r = \frac{1}{2} \).

A soma dos \( N \) primeiros termos é:

\begin{equation}
S = \frac{\frac{Q}{2} \left(1 - \left( \frac{1}{2} \right)^N\right)}{1 - \frac{1}{2}}
= Q\left(1 - \left( \frac{1}{2} \right)^N\right)
\end{equation}

Somando com a carga restante na esfera original:

\begin{equation}
Q_{\text{total}} = Q\left(1 - \left( \frac{1}{2} \right)^N\right) + \frac{Q}{2^N}
\end{equation}

\begin{equation}
Q_{\text{total}} = Q
\end{equation}

Portanto, a carga total do sistema permanece constante e igual a \( Q \), respeitando a 
da carga elétrica.


\newpage 

\begin{flushleft}
\textbf{\textcolor{blue}{\Large Problema 2}}\\
Uma placa infinita nos eixos \(x\) e \(y\) possui a seguinte distribuição superficial de carga:
\begin{equation}
\sigma(x,y) = \frac{\sigma_0 e^{-|x|/a}}{1 + (y/b)^2}
\end{equation}
onde \(a\) e \(b\) são constantes. \\

\textcolor{red}{\textbf{Solução:}}\\
\end{flushleft}



A carga total \(Q\) na placa é dada pela integral da densidade superficial de carga \(\sigma(x, y)\) sobre toda a área da placa:
\begin{equation}
Q = \int_{-\infty}^{\infty} \int_{-\infty}^{\infty} \sigma(x, y) \, dx \, dy
\end{equation}

Substituindo a expressão de \(\sigma(x, y)\), temos:
\begin{equation}
Q = \int_{-\infty}^{\infty} \int_{-\infty}^{\infty} \frac{\sigma_0 e^{-|x|/a}}{1 + (y/b)^2} \, dx \, dy
\end{equation}

\section*{Passo 1: Integral sobre \(x\)}

Calculamos a integral sobre \(x\):
\begin{equation}
\int_{-\infty}^{\infty} e^{-|x|/a} \, dx
\end{equation}
Dividindo a integral em duas partes (por causa da função \(|x|\)), temos:
\begin{equation}
\int_{-\infty}^{\infty} e^{-|x|/a} \, dx = 2 \int_0^{\infty} e^{-x/a} \, dx
\end{equation}
A integral da exponencial é dada por:
\begin{equation}
\int_0^{\infty} e^{-x/a} \, dx = a
\end{equation}
Portanto, temos:
\begin{equation}
\int_{-\infty}^{\infty} e^{-|x|/a} \, dx = 2a
\end{equation}

\section*{Passo 2: Integral sobre \(y\)}

Agora, calculamos a integral sobre \(y\):
\begin{equation}
\int_{-\infty}^{\infty} \frac{1}{1 + (y/b)^2} \, dy
\end{equation}
Essa é uma integral padrão, conhecida como a integral de Cauchy, que resulta em:
\begin{equation}
\int_{-\infty}^{\infty} \frac{1}{1 + (y/b)^2} \, dy = \pi b
\end{equation}

\section*{Passo 3: Cálculo da carga total}

Agora que temos as integrais sobre \(x\) e \(y\), podemos calcular a carga total:
\begin{equation}
Q = \sigma_0 \cdot 2a \cdot \pi b
\end{equation}
Portanto, a carga total na placa é:
\begin{equation}
\boxed{Q = \sigma_0 2\pi a b }
\end{equation}

\begin{flushleft}
\textbf{\textcolor{blue}{\Large Problema 3}}\\
Considere uma linha de carga com densidade linear uniforme \( \lambda_0 \), de comprimento 
total \( 2L \), centrada no eixo \( z \). Calcule o potencial elétrico em um ponto de campo 
localizado a uma distância \( r \) do eixo \( z \) (por exemplo, no plano \( xy \)) e a uma 
altura \( z \). Calcule o campo elétrico no mesmo ponto a partir do potencial. Calcule os limites 
quando \( L \gg r \) e calcule também o limite quando \( r \gg L \).

\textcolor{red}{\textbf{Solução:}}\\
\end{flushleft}

Considere uma linha de carga com densidade linear uniforme \( \lambda_0 \), de comprimento total \( 2L \), 
centrada no eixo \( z \).

\begin{figure}[h!]
\begin{center}
    \begin{tikzpicture}[scale=1.2]
    
    % Eixo z
    \draw[->] (0,-3) -- (0,3) node[above] {$z$};
    
    % Linha de carga de -L a L
    \draw[very thick, red] (0,-2) -- (0,2);
    \node[left] at (0,2) {$L$};
    \node[left] at (0,-2) {$-L$};
    
    % Ponto P
    \filldraw (2,1) circle (1.5pt) node[right] {$P(r,z)$};
    
    % Distância r
    \draw[dashed] (-0.2,1) -- (2,1);
    \draw[<->] (0,1.2) -- (2,1.2);
    \node[above] at (1,1.1) {$r$};
    
    % Distância z
    \draw[dashed] (0,0) -- (0,1);
    \draw[<->] (-0.2,0) -- (-0.2,1);
    \node[left] at (-0.2,0.5) {$z$};
    
    % Elemento dz'
    \draw[fill=black] (0,0.5) circle (1pt);
    \node[left] at (0.7,0.4) {$z'$};
    \draw[<->] (0.2,0) -- (0.2,0.5);
    \draw[dashed] (0,0.5) -- (0.4,0.5);
    
    % Indicação do eixo x para orientar
    \draw[->] (0,0) -- (2.5,0) node[right] {$x$};
    
    % Indicação do eixo y para orientar (para fora do plano)
    \draw[->] (0,0) -- (-0.7,-0.7) node[below] {$y$};
    
\end{tikzpicture}
\end{center}
\caption{Linha de carga com densidade linear uniforme \( \lambda_0 \), de comprimento total \( 2L \), 
centrada no eixo \( z \).}
\end{figure}

\subsection*{Potencial Elétrico}

Um elemento infinitesimal de carga é dado por:
\begin{equation}
dq = \lambda_0 \, dz'
\end{equation}
O potencial devido a esse elemento no ponto \( (r, z) \) é:
\begin{equation}
dV = \frac{1}{4\pi\varepsilon_0} \frac{dq}{\sqrt{r^2 + (z - z')^2}}
\end{equation}
Substituindo \( dq \):
\begin{equation}
dV = \frac{\lambda_0}{4\pi\varepsilon_0} \frac{dz'}{\sqrt{r^2 + (z - z')^2}}
\end{equation}
O potencial total é a integral de \( dV \) de \( z' = -L \) até \( z' = L \):
\begin{equation}
V(r,z) = \frac{\lambda_0}{4\pi\varepsilon_0} \int_{-L}^{L} \frac{dz'}{\sqrt{r^2 + (z - z')^2}}
\end{equation}

Fazendo a substituição \( u = z - z' \), com \( du = -dz' \), temos:
\begin{equation}
V(r,z) = \frac{\lambda_0}{4\pi\varepsilon_0} \int_{z+L}^{z-L} \frac{-du}{\sqrt{r^2 + u^2}} = \frac{\lambda_0}{4\pi\varepsilon_0} \int_{z-L}^{z+L} \frac{du}{\sqrt{r^2 + u^2}}
\end{equation}

Integrando:
\begin{equation}
\int \frac{du}{\sqrt{r^2 + u^2}} = \ln\left(u + \sqrt{r^2 + u^2}\right) + C
\end{equation}

Aplicando os limites:
\begin{equation}
V(r,z) = \frac{\lambda_0}{4\pi\varepsilon_0} \left[ \ln\left(z+L + \sqrt{r^2 + (z+L)^2}\right) - \ln\left(z-L + \sqrt{r^2 + (z-L)^2}\right) \right]
\end{equation}
\begin{equation}
V(r,z) = \frac{\lambda_0}{4\pi\varepsilon_0} \ln\left( \frac{z+L + \sqrt{r^2 + (z+L)^2}}{z-L + \sqrt{r^2 + (z-L)^2}} \right)
\end{equation}

\subsection*{Campo Elétrico}

O campo elétrico é dado por:
\begin{equation}
\vec{E} = -\nabla V
\end{equation}
Em coordenadas cilíndricas (\( r, \theta, z \)) e considerando a simetria do problema:
\begin{equation}
E_r = -\frac{\partial V}{\partial r}, \quad E_z = -\frac{\partial V}{\partial z}, \quad E_\theta = 0
\end{equation}

\subsection*{Limites}

\subsubsection*{1. Quando \( L \gg r \)}

Neste caso, a linha de carga se comporta como um fio infinito. Aproximadamente:
\begin{equation}
V(r) \sim \frac{\lambda_0}{2\pi\varepsilon_0} \ln\left(\frac{2L}{r}\right)
\end{equation}
\begin{equation}
E_r \sim \frac{\lambda_0}{2\pi\varepsilon_0 r}, \quad E_z \sim 0
\end{equation}

\subsubsection*{2. Quando \( r \gg L \)}

Aqui, o sistema se comporta como uma carga pontual de carga total \( Q = 2L\lambda_0 \). Portanto:
\begin{equation}
V(r) \sim \frac{Q}{4\pi\varepsilon_0 r} = \frac{2L\lambda_0}{4\pi\varepsilon_0 r}
\end{equation}
\begin{equation}
\vec{E} \sim \frac{Q}{4\pi\varepsilon_0 r^2} \hat{r}
\end{equation}

\begin{flushleft}
\textbf{\textcolor{blue}{\Large Problema 4}}\\

Considere um elétron em um átomo de hidrogênio a uma distância de \( 0{,}53 \times 10^{-10} \, \text{m} \) do próton. 
Sabendo que o próton tem carga \( +e \) e o elétron \( -e \), resolva:

\begin{enumerate}
    \item[a)] Calcule a energia potencial eletrostática do elétron em eV.
    \item[b)] Sabendo que a velocidade do elétron é \( v = 2{,}189 \times 10^6 \, \text{m/s} \), 
    calcule a energia total do elétron no átomo de hidrogênio em eV.
\end{enumerate}

\textcolor{red}{\textbf{Solução:}}\\

\subsection*{Letra (a): Energia Potencial Eletrostática}

A fórmula da energia potencial eletrostática \( U \) entre duas cargas \( q_1 \) e \( q_2 \) separadas por uma distância \( r \) é dada por:

\begin{equation}
U = \frac{k \cdot q_1 \cdot q_2}{r}
\end{equation}

onde:
\begin{equation}
k = 8,99 \times 10^9 \, \text{N} \cdot \text{m}^2 / \text{C}^2 \quad \text{(constante eletrostática)},
\end{equation}
\begin{equation}
q_1 = e = 1,6 \times 10^{-19} \, \text{C} \quad \text{(carga do próton)},
\end{equation}
\begin{equation}
q_2 = -e = -1,6 \times 10^{-19} \, \text{C} \quad \text{(carga do elétron)},
\end{equation}
\begin{equation}
r = 0,53 \times 10^{-10} \, \text{m} \quad \text{(distância entre as cargas)}.
\end{equation}

Substituindo os valores na fórmula:

\begin{equation}
U = \frac{(8,99 \times 10^9) \cdot (1,6 \times 10^{-19}) \cdot (-1,6 \times 10^{-19})}{0,53 \times 10^{-10}}
\end{equation}

Calculando:

\begin{equation}
U \approx \frac{(8,99 \times 10^9) \cdot (-2,56 \times 10^{-38})}{0,53 \times 10^{-10}} \approx -4,32 \times 10^{-18} \, \text{J}
\end{equation}

Convertendo para eV, usando \( 1 \, \text{eV} = 1,602 \times 10^{-19} \, \text{J} \):

\begin{equation}
U \approx \frac{-4,32 \times 10^{-18}}{1,602 \times 10^{-19}} \approx -27 \, \text{eV}
\end{equation}

Portanto, a energia potencial eletrostática é:

\begin{equation}
\boxed{ U \approx -27 \, \text{eV}}
\end{equation}

\subsection*{Letra (b): Energia Total do Elétron}

A energia total \( E \) do elétron é a soma da energia cinética \( E_{\text{cinet}} \) e da energia potencial \( U \). A energia cinética é dada por:

\begin{equation}
E_{\text{cinet}} = \frac{1}{2} m v^2
\end{equation}

onde:
\begin{equation}
m = 9,11 \times 10^{-31} \, \text{kg} \quad \text{(massa do elétron)},
\end{equation}
\begin{equation}
v = 2,189 \times 10^6 \, \text{m/s} \quad \text{(velocidade do elétron)}.
\end{equation}

Substituindo os valores:

\begin{equation}
E_{\text{cinet}} = \frac{1}{2} \cdot (9,11 \times 10^{-31}) \cdot (2,189 \times 10^6)^2
\end{equation}

Calculando:

\begin{equation}
E_{\text{cinet}} \approx \frac{1}{2} \cdot 9,11 \times 10^{-31} \cdot 4,79 \times 10^{12} \approx 2,18 \times 10^{-18} \, \text{J}
\end{equation}

Convertendo para eV:

\begin{equation}
E_{\text{cinet}} \approx \frac{2,18 \times 10^{-18}}{1,602 \times 10^{-19}} \approx 13,6 \, \text{eV}
\end{equation}

Agora, a energia total do elétron é a soma da energia cinética e da energia potencial:

\begin{equation}
E = E_{\text{cinet}} + U
\end{equation}

\begin{equation}
E = 13,6 \, \text{eV} + (-27 \, \text{eV}) \approx -13,6 \, \text{eV}
\end{equation}

Portanto, a energia total do elétron no átomo de hidrogênio é:

\begin{equation}
\boxed{E \approx -13,6 \, \text{eV}}
\end{equation}
\end{flushleft}

\begin{flushleft}
\textbf{\textcolor{blue}{\Large Problema 5}}\\

Imagine que a Terra tenha densidade uniforme e que um túnel seja escavado ao longo de um diâmetro.

\begin{enumerate}
    \item[a)] Se um objeto for solto no túnel, mostre que ele oscilaria com um período \( P \) igual ao 
    período de um satélite em órbita na superfície da Terra.
    \item[b)] Calcule \( P \).
\end{enumerate}

\textcolor{red}{\textbf{Solução:}}\\
\end{flushleft}

\begin{flushleft}
\textbf{\textcolor{blue}{\Large Problema 6}}\\

\textcolor{red}{\textbf{Solução:}}\\
\end{flushleft}

\begin{flushleft}
\textbf{\textcolor{blue}{\Large Problema 7}}\\

\textcolor{red}{\textbf{Solução:}}\\
\end{flushleft}

\begin{flushleft}
\textbf{\textcolor{blue}{\Large Problema 8}}\\

\textcolor{red}{\textbf{Solução:}}\\
\end{flushleft}

\begin{flushleft}
\textbf{\textcolor{blue}{\Large Problema 9}}\\

\textcolor{red}{\textbf{Solução:}}\\
\end{flushleft}

\begin{flushleft}
\textbf{\textcolor{blue}{\Large Problema 10}}\\

\textcolor{red}{\textbf{Solução:}}\\
\end{flushleft}


\begin{flushleft}
\textbf{\textcolor{blue}{\Large Problema 11}}\\

\textcolor{red}{\textbf{Solução:}}\\
\end{flushleft}



%%%%%%%% Bibliography 
% Os comandos para incluir as referências bibliográficas
%\printingbibliography

\end{document}
