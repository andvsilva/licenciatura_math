\documentclass[a4paper,12pt]{article}
\usepackage[brazil, english]{babel}
\usepackage[utf8]{inputenc}
\usepackage[T1]{fontenc}
\usepackage{geometry}
\usepackage{setspace}
\usepackage{titlesec}
\usepackage{hyperref}
\usepackage{graphicx}
\usepackage{caption}
\usepackage{subcaption}
\usepackage{fancyhdr}
\usepackage{xcolor}
\usepackage{amsmath}
\usepackage{mathtools}
\usepackage{cancel}


%%%%%%%%%%%%%%%%%%%%%%%%%%%%%%%%%%%%%%%%%%%%%%%%%%
% These are some new commands that may be useful 
% for paper writing in general. If other new commands
% are needed for your specific paper, please feel 
% free to add here. 
%
% The currently available commands are organized in: 
% 1) Systems
% 2) Quantities
% 3) Energies and units
% 4) particle species
% 5) Colors package
% 6) hyperlink
%%%%%%%%%%%%%%%%%%%%%%%%%%%%%%%%%%%%%%%%%%%%%%%%%%

\usepackage{amsmath}
\usepackage{amssymb}
\usepackage{upgreek}
\usepackage{multirow}
\usepackage{setspace}% http://ctan.org/pkg/setspace
\usepackage{fancyhdr}
\usepackage{datetime}

% 1) SYSTEMS
\newcommand{\btc}               {\textbf{BTC}}
\newcommand{\btcspace}          {\textbf{BTC} }
\newcommand{\pow}               {\textbf{PoW}}

% 4) definition to references, biblatex and hyperlink
\usepackage[backend=bibtex, 
style=nature,  %style reference.
sorting=none,
firstinits=true %first name abbreviate
]{biblatex}

\usepackage{hyperref}
\hypersetup{
    colorlinks=true, %set "true" if you want colored links
    linktoc=all,     %set to "all" if you want both sections and subsections linked
    linkcolor=blue,  %choose some color if you want links to stand out
    citecolor= blue, % color of \cite{} in the text.
    urlcolor  = blue, % color of the link for the paper in references.
}

% 5) Tikz and figures
\usepackage{epsfig}
\usepackage{lmodern}
\usepackage{mathtools}
\usepackage[utf8]{luainputenc}
\usepackage{xspace}
\usepackage{tikz}
\usepackage{pgfplots}
\pgfplotsset{compat=newest}

\usetikzlibrary{positioning}
\usepackage{subcaption}

% 6) colors:
\usepackage{xcolor}
\definecolor{ao(english)}{rgb}{0.0, 0.5, 0.0} % dark green

% 7) Add lines numbers
%\usepackage{lineno}

% add pdf file to thesis:
\usepackage{pdfpages}

\hypersetup{
    colorlinks=true,% make the links colored
    linkcolor=blue
}

\usepackage{setspace}
\addbibresource{bibliography.bib}

\newcommand{\printingbibliography}{%

    \pagestyle{myheadings}
    \markright{}
    \sloppy
    \printbibliography[heading=bibintoc, % add to table of contents
                   title=Refer\^encias % Chapter name
                  ]
    \fussy%
}
\PassOptionsToPackage{table}{xcolor}

\pagestyle{fancy}
\fancyhf{}
\renewcommand{\headrulewidth}{0pt}
\fancyhead[R]{\thepage}

\geometry{a4paper,top=30mm,bottom=20mm,left=30mm,right=20mm}

\titleformat*{\section}{\bfseries\large}
\titleformat*{\subsection}{\bfseries\normalsize}

\title{ \textbf{\large Eletromagnetismo I }}
\author{Andr\'e V. Silva}
\date{\today}

\begin{document}

\maketitle

\begin{center}
    \textbf{Prof. Bruno Moraes - 2024-2}\\
    \textbf{Guia de Estudo 1: Revisão Matemática}
    \end{center}
    
    * A numeração dos exercícios do Griffiths propostos correspondem à 4a edição em inglês.\\
    
    \begin{center}
    \colorbox{yellow}{\textbf{Resolu\c{c}\~ao de Exerc\'icios}}
    \end{center}
    
    \begin{enumerate}
    \item Griffiths \textbf{Seção 1.1} - 1.5
    \item Griffiths \textbf{Seção 1.2} - 1.13(*), 1.16(*), 1.19, 1.21, 1.22(a-b)
    \item Griffiths \textbf{Seção 1.4} - 1.38(*), 1.42
    \item Griffiths \textbf{Seção 1.5} - 1.44, 1.45, 1.46, 1.47, 1
    \item Griffiths \textbf{Seção 1.3} - 1.36.48, 1.49
    \item Griffiths \textbf{Seção 1.6} - 1.51, 1.52
    \item Problemas adicionais - 1.62(*), 1.63(*)
    \end{enumerate}

\section*{Problem 1.5 Griffiths - Resolu\c{c}\~ao}

O produto vetorial triplo: $\textrm{\textbf{A}}\times\textrm{\textbf{B}}\times\textrm{\textbf{C}}$
ser simplificado pela express\~ao \textbf{BAC} - \textbf{CAB}:

\begin{equation}
    \textbf{A}\times(\textbf{B}\times\textbf{C}) =  \textbf{B}(\textbf{A}\cdot\textbf{C}) - \textbf{C}(\textbf{A}\cdot\textbf{B})
\end{equation}

Com isso, podemos notar que:

\begin{equation}
    (\textbf{A}\times\textbf{B})\times\textbf{C} = -\textbf{C}\times(\textbf{A}\times\textbf{B}) = -\textbf{A}(\textbf{B}\cdot\textbf{C}) + \textbf{B}(\textbf{A}\cdot\textbf{C})
\end{equation}

Vamos provar \textbf{BAC} - \textbf{CAB} escrevendo explicitamente ambos os lados em termos de suas componentes:\\

Primeramente, definindo \textbf{A} $= (\textrm{A}_{x}, \textrm{A}_{y}, \textrm{A}_{z}$), \textbf{B} $= (\textrm{B}_{x}, \textrm{B}_{y}, \textrm{B}_{z}$) 
e \textbf{C} $= (\textrm{C}_{x}, \textrm{C}_{y}, \textrm{C}_{z}$)\\

Lado esquerdo da equa\c{c}\~ao (1) parte por parte:

\begin{equation}
    \mathbf{B} \times \mathbf{C} =
\begin{vmatrix}
\mathbf{i} & \mathbf{j} & \mathbf{k} \\
B_x & B_y & B_z \\
C_x & C_y & C_z
\end{vmatrix} =
(B_yC_z - B_zC_y)\mathbf{i} -
(B_zC_x - B_xC_z)\mathbf{j} +
(B_xC_y - B_yC_x)\mathbf{k}
\end{equation}


\begin{equation}
\mathbf{A} \times (\mathbf{B} \times \mathbf{C}) =
\begin{vmatrix}
\mathbf{i} & \mathbf{j} & \mathbf{k} \\
A_x & A_y & A_z \\
(B_yC_z - B_zC_y) & (B_zC_x - B_xC_z) & (B_xC_y - B_yC_x)
\end{vmatrix}
\end{equation}

\begin{equation}
    \begin{aligned}
    \mathbf{A} \times (\mathbf{B} \times \mathbf{C}) = \left[ A_y(B_xC_y - B_yC_x) - A_z(B_zC_x - B_xC_z) \right] \mathbf{i} \\ +
    \left[A_z(B_yC_z - B_zC_y) - A_x(B_xC_y - B_yC_x) \right] \mathbf{j}\\ +
    \left[ A_x(B_zC_x - B_xC_z) - A_y(B_yC_z - B_zC_y) \right] \mathbf{k}.
\end{aligned}
\end{equation}\\


A expressão \( \mathbf{B}(\mathbf{A} \cdot \mathbf{C}) \) em termos das componentes \( \mathbf{i}, \mathbf{j}, \mathbf{k} \) é dada por:

Seja \( \mathbf{B} = B_x \mathbf{i} + B_y \mathbf{j} + B_z \mathbf{k} \) e o produto escalar \( \mathbf{A} \cdot \mathbf{C} = A_x C_x + A_y C_y + A_z C_z \), então:

\begin{equation}
\mathbf{B} (\mathbf{A} \cdot \mathbf{C}) = (B_x \mathbf{i} + B_y \mathbf{j} + B_z \mathbf{k}) (A_x C_x + A_y C_y + A_z C_z)
\end{equation}

Ou, expandindo a expressão, temos:

\begin{equation}
    \begin{aligned}
\mathbf{B} (\mathbf{A} \cdot \mathbf{C}) = \left( B_x (A_x C_x + A_y C_y + A_z C_z) \right) \mathbf{i}\\
             + \left( B_y (A_x C_x + A_y C_y + A_z C_z) \right) \mathbf{j}\\
             + \left( B_z (A_x C_x + A_y C_y + A_z C_z) \right) \mathbf{k}
\end{aligned}
\end{equation}

A expressão \( \mathbf{C}(\mathbf{A} \cdot \mathbf{B}) \) em termos das componentes \( \mathbf{i}, \mathbf{j}, \mathbf{k} \) é dada por:

Seja \( \mathbf{C} = C_x \mathbf{i} + C_y \mathbf{j} + C_z \mathbf{k} \) e o produto escalar \( \mathbf{A} \cdot \mathbf{B} = A_x B_x + A_y B_y + A_z B_z \), então:

\begin{equation}
\mathbf{C} (\mathbf{A} \cdot \mathbf{B}) = (C_x \mathbf{i} + C_y \mathbf{j} + C_z \mathbf{k}) (A_x B_x + A_y B_y + A_z B_z)
\end{equation}

Ou, expandindo a expressão, temos:

\begin{equation}
    \begin{aligned}
\mathbf{C} (\mathbf{A} \cdot \mathbf{B}) = \left( C_x (A_x B_x + A_y B_y + A_z B_z) \right) \mathbf{i}\\
        + \left( C_y (A_x B_x + A_y B_y + A_z B_z) \right) \mathbf{j} \\
        + \left( C_z (A_x B_x + A_y B_y + A_z B_z) \right) \mathbf{k}
\end{aligned}
\end{equation}

Ent\~ao


\begin{equation}
    \begin{aligned}
\mathbf{B} (\mathbf{A} \cdot \mathbf{C}) - \mathbf{C} (\mathbf{A} \cdot \mathbf{B}) = & \, \Big( B_x (A_x C_x + A_y C_y + A_z C_z) - C_x (A_x B_x + A_y B_y + A_z B_z) \Big) \mathbf{i} \\
& + \Big( B_y (A_x C_x + A_y C_y + A_z C_z) - C_y (A_x B_x + A_y B_y + A_z B_z) \Big) \mathbf{j} \\
& + \Big( B_z (A_x C_x + A_y C_y + A_z C_z) - C_z (A_x B_x + A_y B_y + A_z B_z) \Big) \mathbf{k}.
    \end{aligned}
\end{equation}

\begin{equation}
    \begin{aligned}
\mathbf{B} (\mathbf{A} \cdot \mathbf{C}) - \mathbf{C} (\mathbf{A} \cdot \mathbf{B}) = & \, \Big( \cancel{A_x B_x C_x} + A_y B_x C_y + A_z B_x C_z -  \cancel{A_x B_x C_x}  - A_y C_x B_y - A_z C_xB_z \Big) \mathbf{i} \\
& + \Big(  A_x B_y C_x + \cancel{A_y B_y C_y} + A_z B_y C_z  - A_x C_y B_x - \cancel{A_y B_y C_y} - A_z C_y B_z \Big) \mathbf{j} \\
& + \Big(  A_x B_z C_x + A_y B_z C_y + \cancel{A_z B_z C_z} - A_x C_z B_x - A_y B_y C_z - \cancel{A_z B_z C_z} \Big) \mathbf{k}.
    \end{aligned}
\end{equation}

\begin{equation}
    \begin{aligned}
\mathbf{B} (\mathbf{A} \cdot \mathbf{C}) - \mathbf{C} (\mathbf{A} \cdot \mathbf{B}) = & \, \Big(A_y B_x C_y + A_z B_x C_z - A_y C_x B_y - A_z C_xB_z \Big) \mathbf{i} \\
& + \Big(  A_x B_y C_x + A_z B_y C_z  - A_x C_y B_x - A_z C_y B_z \Big) \mathbf{j} \\
& + \Big(  A_x B_z C_x + A_y B_z C_y - A_x C_z B_x - A_y B_y C_z \Big) \mathbf{k}.
    \end{aligned}
\end{equation}

\begin{equation}
\boxed{
    \begin{aligned}
\mathbf{B} (\mathbf{A} \cdot \mathbf{C}) - \mathbf{C} (\mathbf{A} \cdot \mathbf{B}) = & \, \Big[A_y (B_x C_y -B_y C_x ) - A_z (B_z C_x - B_x C_z) \Big] \mathbf{i} \\
& + \Big[ A_z (B_y C_z - C_y B_z) - A_x (B_x C_y  - B_y C_x)     \Big] \mathbf{j} \\
& + \Big[  A_x(B_z C_x - B_x C_z)  - A_y (B_y C_z - B_z C_y)   \Big] \mathbf{k}.
    \end{aligned}
}
\end{equation}


\begin{equation}
    \boxed{
    \begin{aligned}
    \mathbf{A} \times (\mathbf{B} \times \mathbf{C}) = \Big[ A_y(B_xC_y - B_yC_x) - A_z(B_zC_x - B_xC_z) \Big] \mathbf{i} \\ +
    \Big[A_z(B_yC_z - B_zC_y) - A_x(B_xC_y - B_yC_x) \Big] \mathbf{j}\\ +
    \Big[ A_x(B_zC_x - B_xC_z) - A_y(B_yC_z - B_zC_y) \Big] \mathbf{k}.
\end{aligned}
    }
\end{equation}\\

%%%%%%%% Bibliography 
% Os comandos para incluir as referências bibliográficas
%\printingbibliography

\end{document}
