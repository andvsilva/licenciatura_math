\documentclass[a4paper,12pt]{article}
\usepackage[brazil, english]{babel}
\usepackage[utf8]{inputenc}
\usepackage[T1]{fontenc}
\usepackage{geometry}
\usepackage{setspace}
\usepackage{titlesec}
\usepackage{hyperref}
\usepackage{graphicx}
\usepackage{caption}
\usepackage{subcaption}
\usepackage{fancyhdr}
\setlength{\headheight}{15pt}
\addtolength{\topmargin}{-2.5pt}
\usepackage{xcolor}
\usepackage{amsmath, amssymb, bm}
\usepackage{mathtools}
\usepackage{cancel}
\usepackage{tikz}
\usepackage{newunicodechar}
\usepackage{ragged2e}
\usepackage{setspace}
\usepackage{tikz-3dplot} % Necessário para coordenadas 3D
\usetikzlibrary{intersections}
\usepackage{siunitx}
\usetikzlibrary{3d, arrows.meta}
\usepackage{booktabs}


\usepackage{color}
\definecolor{myblue}{rgb}{.8, .8, 1}

\definecolor{ao(english)}{rgb}{0.0, 0.5, 0.0}

\usepackage{amsmath}
\usepackage{empheq}

\newlength\mytemplen
\newsavebox\mytempbox

\makeatletter
\newcommand\mybluebox{%
    \@ifnextchar[%]
       {\@mybluebox}%
       {\@mybluebox[0pt]}}

\def\@mybluebox[#1]{%
    \@ifnextchar[%]
       {\@@mybluebox[#1]}%
       {\@@mybluebox[#1][0pt]}}

\def\@@mybluebox[#1][#2]#3{
    \sbox\mytempbox{#3}%
    \mytemplen\ht\mytempbox
    \advance\mytemplen #1\relax
    \ht\mytempbox\mytemplen
    \mytemplen\dp\mytempbox
    \advance\mytemplen #2\relax
    \dp\mytempbox\mytemplen
    \colorbox{myblue}{\hspace{1em}\usebox{\mytempbox}\hspace{1em}}}
\makeatother

\usepackage[most]{tcolorbox}

\newtcbox{\mymath}[1][]{%
    nobeforeafter, math upper, tcbox raise base,
    enhanced, colframe=blue!30!black,
    colback=blue!30, boxrule=1pt,
    #1}

\tcbset{
    highlight math style={
        enhanced,
        colframe=red!60!black,
        colback=yellow!50,
        arc=4pt,
        boxrule=1pt,
        drop fuzzy shadow
    }
    }

\usepackage{physics}
\usepackage{pgfplots}
\pgfplotsset{compat=1.17}

\linespread{1.5}

\definecolor{ao(english)}{rgb}{0.0, 0.5, 0.0}
\definecolor{byzantium}{rgb}{0.44, 0.16, 0.39}
\newunicodechar{∘}{\circ}

%%%%%%%%%%%%%%%%%%%%%%%%%%%%%%%%%%%%%%%%%%%%%%%%%%
% These are some new commands that may be useful 
% for paper writing in general. If other new commands
% are needed for your specific paper, please feel 
% free to add here. 
%
% The currently available commands are organized in: 
% 1) Systems
% 2) Quantities
% 3) Energies and units
% 4) particle species
% 5) Colors package
% 6) hyperlink
%%%%%%%%%%%%%%%%%%%%%%%%%%%%%%%%%%%%%%%%%%%%%%%%%%

\usepackage{amsmath}
\usepackage{amssymb}
\usepackage{upgreek}
\usepackage{multirow}
\usepackage{setspace}% http://ctan.org/pkg/setspace
\usepackage{fancyhdr}
\usepackage{datetime}

% 1) SYSTEMS 
\newcommand{\pp}           {pp\xspace}
\newcommand{\ppbar}        {\mbox{$\mathrm {p\overline{p}}$}\xspace}
\newcommand{\XeXe}         {\mbox{Xe--Xe}\xspace}
\newcommand{\PbPb}         {\mbox{Pb--Pb}\xspace}
\newcommand{\pA}           {\mbox{pA}\xspace}
\newcommand{\pPb}          {\mbox{p--Pb}\xspace}
\newcommand{\AuAu}         {\mbox{Au--Au}\xspace}
\newcommand{\dAu}          {\mbox{d--Au}\xspace}
\def\pA{$pA$\xspace}
\def\AA{$AA$\xspace}
\def\NN{$NN$\xspace}
\def\signn{$\sigma^{inel}_{NN}$\xspace}
\def\sigtotal{$\sigma_{\textnormal{tot}}$\xspace}
\def\mrm{\mathrm}
\def\ntrig{N_\mrm{trig}}
\newcommand{\rivet}{R\protect\scalebox{1}{IVET}\xspace}
\newcommand{\hepmc}{H\protect\scalebox{1}{EP}MC\xspace}
\newcommand{\herwig}{H\protect\scalebox{1}{ERWIG} 7\xspace}
\newcommand{\sherpa}{S\protect\scalebox{1}{HERPA}\xspace}
\newcommand{\urqmd}{U\protect\scalebox{1}{r}QMD\xspace}
\newcommand{\urqmdversion}{U\protect\scalebox{1}{r}QMD 3.4\xspace}
\newcommand{\pythia}{\protect\scalebox{1}{PYTHIA}\xspace}
\newcommand{\pythiaversion}{\protect\scalebox{1}{PYTHIA 8.2}\xspace}
\newcommand{\pythiaversionused}{\protect\scalebox{1}{PYTHIA 8.235}\xspace}
\newcommand{\pytang}{\protect\scalebox{1}{PYTHIA}/Angantyr\xspace}
\newcommand{\angantyr}{\protect\scalebox{1}{}Angantyr\xspace}
\newcommand{\pytangur}{\protect\scalebox{1}{PYTHIA}/Angantyr + U\protect\scalebox{1}{r}QMD\xspace}
\newcommand{\figref}[1]{Fig.~\ref{#1}}
\newcommand{\tabref}[1]{Tab.~\ref{#1}}
\renewcommand{\eqref}[1]{Eq.~(\ref{#1})}

% hydrodynamic simulation chain:
% TRENTo
\newcommand{\trento}{\protect\scalebox{1}{T$_{\text{R}}$ENT}o\xspace}
% KOMPOST : Linear kinetic theory propagator for initial conditions in heavy ion collisions
\newcommand{\kompost}{\protect\scalebox{1}{K$\varnothing$MP$\varnothing$ST}\xspace}
% MUSIC
\newcommand{\music}{\protect\scalebox{1}{MUSIC}\xspace}
% iSS
\newcommand{\iss}{\protect\scalebox{1}{iSS}\xspace}

% 2) QUANTITIES 
\newcommand{\s}            {\ensuremath{\sqrt{s}}\xspace}
\newcommand{\snn}          {\ensuremath{\sqrt{s_{\mathrm{NN}}}}\xspace}
\newcommand{\pt}           {\ensuremath{p_{\rm T}}\xspace}
\newcommand{\meanpt}       {$\langle p_{\mathrm{T}}\rangle$\xspace}
\newcommand{\ycms}         {\ensuremath{y_{\rm CMS}}\xspace}
\newcommand{\ylab}         {\ensuremath{y_{\rm lab}}\xspace}
\newcommand{\etarange}[1]  {\mbox{$\left | \eta \right |~<~#1$}}
\newcommand{\centbin}[2]  {\mbox{$#1-#2\%$}}
\newcommand{\ptrange}[2]  {\mbox{$#1 < p_{\mathrm{T}}\hspace{0.2cm} (\mathrm{GeV}/\mathrm{\textit{c}}) <#2$}}
\newcommand{\ptrangetrig}[2]  {\mbox{$#1 < p^{\mathrm{trigger}}_{\mathrm{T} }\hspace{0.2cm} (\mathrm{GeV}/\mathrm{\textit{c}}) <#2$}}
\newcommand{\ptrangeassoc}[2]  {\mbox{$#1 < p^{\mathrm{assoc}}_{\mathrm{T} }\hspace{0.2cm} (\mathrm{GeV}/\mathrm{\textit{c}}) <#2$}}
\newcommand{\etazerothree} {$\left|\eta \right| < 0.3$\xspace}
\newcommand{\etazerofive} {$\left|\eta \right| < 0.5$\xspace}
\newcommand{\etazeroeight} {$\left|\eta \right| < 0.8$\xspace}
\newcommand{\yrange}[1]    {\mbox{$\left | y \right |~<~#1$}}
\newcommand{\dndy}         {\ensuremath{\mathrm{d}N_\mathrm{ch}/\mathrm{d}y}\xspace}
\newcommand{\dndeta}       {\ensuremath{\mathrm{d}N_\mathrm{ch}/\mathrm{d}\eta}\xspace}
\newcommand{\dnchdydpt}   {\ensuremath{\mathrm{d}N_\mathrm{ch}/\mathrm{d}y\mathrm{d}p_{\mathrm{T}}}\xspace}
\newcommand{\dnchaadydpt}   {\ensuremath{\mathrm{d}N_\mathrm{ch}^{AA}/\mathrm{d}y\mathrm{d}p_{\mathrm{T}}}\xspace}
\newcommand{\dnchppdydpt}   {\ensuremath{\mathrm{d}N_\mathrm{ch}^{\mathrm{pp}}/\mathrm{d}y\mathrm{d}p_{\mathrm{T}}}\xspace}
\newcommand{\dnchdphi}{\ensuremath{\mathrm{d}N_\mathrm{ch}/\mathrm{d}\phi}\xspace}
\newcommand{\dnchddeltaphi}{\ensuremath{\mathrm{d}N_\mathrm{ch}/\mathrm{d}\Delta\upphi}\xspace}
\newcommand{\dndphi}{\ensuremath{\mathrm{d}N/\mathrm{d}\phi}\xspace}
\newcommand{\dnddeltaphi}{\ensuremath{\mathrm{d}N/\mathrm{d}\Delta\upphi}\xspace}
\newcommand{\avdndeta}     {\ensuremath{\langle\dndeta\rangle}\xspace}
\newcommand{\avdndetarap}  {$\langle$ dN$_{\textnormal{ch}}$/d$\eta$ $\rangle_{|\eta| < 0.5}$\xspace}
\newcommand{\dNdy}         {\ensuremath{\mathrm{d}N_\mathrm{ch}/\mathrm{d}y}\xspace}
\newcommand{\Npart}        {\ensuremath{N_\mathrm{part}}\xspace}
\newcommand{\meanNpart}    {$\langle$\ensuremath{N_\mathrm{part}}$\rangle$\xspace}
\newcommand{\ncoll}        {\ensuremath{N_\mathrm{coll}}\xspace}
\newcommand{\meanncoll}    {$\langle$\ensuremath{N_\mathrm{coll}}$\rangle$\xspace}
\newcommand{\averagencollhadronic}    {$\langle$\ensuremath{\mathrm{N}_\mathrm{coll}^{\mathrm{hadronic}}}$\rangle$\xspace}
\newcommand{\meantaa}      {$\langle$\ensuremath{T_\mathrm{AA}}$\rangle$\xspace}
\newcommand{\dEdx}         {\ensuremath{\textrm{d}E/\textrm{d}x}\xspace}
\newcommand{\RpPb}         {\ensuremath{R_{\rm pPb}}\xspace}
\newcommand{\raa}          {$R_{AA}$\xspace}
\newcommand{\vtwo}         {$v_{2}$\xspace}
\newcommand{\vtwoinitial}  {$v_{2}^{\mathrm{initial}}$\xspace}
\newcommand{\vtwofinal}    {$v_{2}^{\mathrm{final}}$\xspace}
\newcommand{\vtwofourfinal}{$v_{2}^{\mathrm{final}}\{4\}$\xspace}
\newcommand{\vtwofit}      {$v_{2}^{\mathrm{Fit}}$\xspace}
\newcommand{\vtwotwo}      {$v_{2}\{2\}$\xspace}
\newcommand{\vtwofour}     {$v_{2}\{4\}$\xspace}
\newcommand{\vtwopt}       {$v_{2}(p_{\textnormal{T}})$\xspace}
\newcommand{\vtwoptfit}    {$v_{2}^{\mathrm{Fit}}(p_{\textnormal{T}})$\xspace}
\newcommand{\nch}          {\ensuremath{N_\mathrm{ch}}\xspace}
\newcommand{\psireactionplane}          {$\Psi_{\textnormal{RP}}$\xspace}
\newcommand{\deltaphireactionplane}     {$\Delta\upphi = \phi - \Psi_{\textnormal{RP}}$\xspace}
\newcommand{\nevdnchddeltaphi}     {(1/N$_{\textnormal{ev}}$)dN$_{\textnormal{ch}}$/d$\Delta\upphi$\xspace}
\newcommand{\meannch}      {\ensuremath{\langle N_\mathrm{ch}\rangle}\xspace}
\newcommand{\etamodule}    {\ensuremath{|\eta|}\xspace}
\newcommand{\qbar}         {$\bar{\textnormal{q}}$\xspace}
\newcommand{\qqbar}        {$\textnormal{q}\bar{\textnormal{q}}$\xspace}
\newcommand{\qqbarzero}    {$\textnormal{q}_{0}\bar{\textnormal{q}}_{0}$\xspace}
\newcommand{\qqqbars}      {$\bar{\textnormal{q}}\bar{\textnormal{q}}\bar{\textnormal{q}}$\xspace}
\newcommand{\alphastrong}  {$\alpha_{\textnormal{s}}$\xspace}
\newcommand{\alphastrongdistance}  {$\alpha_{\textnormal{s}}$(R)\xspace}
\newcommand{\qtwo}         {Q$^2$\xspace}
\newcommand{\alphastrongqtwo}  {$\alpha_{\textnormal{s}}$(Q$^2$)\xspace}
\newcommand{\lambdaqcd}        {$\Lambda_{\textnormal{QCD}}$\xspace}
\newcommand{\sectionpp}        {$\sigma^{\textnormal{pp}}_{\textnormal{inel}}$\xspace}

% 3) ENERGIES, UNITS
\newcommand{\sqrts}        {$\sqrt{s}$\xspace}
\newcommand{\sqrtsnn}      {$\sqrt{s_{\mathrm{NN}}}$\xspace}
\newcommand{\nineH}        {$\sqrt{s}~=~0.9$~Te\kern-.1emV\xspace}
\newcommand{\seven}        {$\sqrt{s}~=~7$~Te\kern-.1emV\xspace}
\newcommand{\twoH}         {$\sqrt{s}~=~0.2$~Te\kern-.1emV\xspace}
\newcommand{\twosevensix}  {$\sqrt{s}~=~2.76$~Te\kern-.1emV\xspace}
\newcommand{\five}         {$\sqrt{s}~=~5.02$~Te\kern-.1emV\xspace}
\newcommand{\twohundrernn} {$\sqrt{s_{\mathrm{NN}}}=200$~Ge\kern-.1emV\xspace}
\newcommand{\twosevensixnn} {$\sqrt{s_{\mathrm{NN}}}=2.76$~Te\kern-.1emV\xspace}
\newcommand{\fivenn}       {$\sqrt{s_{\mathrm{NN}}}~=~5.02$~Te\kern-.1emV\xspace}
\newcommand{\fivefourfournn} {$\sqrt{s_{\mathrm{NN}}}=5.44$~Te\kern-.1emV\xspace}
\newcommand{\LT}           {L{\'e}vy-Tsallis\xspace}
\newcommand{\GeVc}         {Ge\kern-.1emV/$c$\xspace}
\newcommand{\MeVc}         {Me\kern-.1emV/$c$\xspace}
\newcommand{\TeV}          {Te\kern-.1emV\xspace}
\newcommand{\GeV}          {Ge\kern-.1emV\xspace}
\newcommand{\MeV}          {Me\kern-.1emV\xspace}
\newcommand{\GeVmass}      {Ge\kern-.2emV/$c^2$\xspace}
\newcommand{\MeVmass}      {Me\kern-.2emV/$c^2$\xspace}
\newcommand{\lumi}         {\ensuremath{\mathcal{L}}\xspace}
\newcommand{\fmc}         {fm\kern-.1em/$c$\xspace}

% 4) PARTICLE SPECIES 
\newcommand{\ee}           {\ensuremath{e^{+}e^{-}}} 
\newcommand{\pip}          {\ensuremath{\pi^{+}}\xspace}
\newcommand{\pim}          {\ensuremath{\pi^{-}}\xspace}
\newcommand{\kap}          {\ensuremath{\rm{K}^{+}}\xspace}
\newcommand{\kam}          {\ensuremath{\rm{K}^{-}}\xspace}
\newcommand{\pbar}         {\ensuremath{\rm\overline{p}}\xspace}
\newcommand{\kzero}        {\ensuremath{{\rm K}^{0}_{\rm{S}}}\xspace}
\newcommand{\lmb}          {\ensuremath{\Lambda}\xspace}
\newcommand{\almb}         {\ensuremath{\overline{\Lambda}}\xspace}
\newcommand{\Om}           {\ensuremath{\Omega^-}\xspace}
\newcommand{\Mo}           {\ensuremath{\overline{\Omega}^+}\xspace}
\newcommand{\X}            {\ensuremath{\Xi^-}\xspace}
\newcommand{\Ix}           {\ensuremath{\overline{\Xi}^+}\xspace}
\newcommand{\Xis}          {\ensuremath{\Xi^{\pm}}\xspace}
\newcommand{\Oms}          {\ensuremath{\Omega^{\pm}}\xspace}
\newcommand{\degree}       {\ensuremath{^{\rm o}}\xspace}
\newcommand{\comment}[1]{}

% two-particle angular correlation
\newcommand{\deltaphitriggassoc}    {$\Delta\upphi = |\phi_{\textnormal{trigger}} - \phi_{\textnormal{assoc}}|$\xspace}
\newcommand{\deltaetatriggassoc}    {$\Delta\upeta = |\eta_{\textnormal{trigger}} - \eta_{\textnormal{assoc}}|$\xspace}
\newcommand{\etatrigg}    {$\eta_{\textnormal{trigger}}$\xspace}
\newcommand{\etaassoc}    {$\eta_{\textnormal{assoc}}$\xspace}
\newcommand{\deltaphideltaeta}      {$\Delta\upphi-\Delta\upeta$\xspace}
\newcommand{\deltaphi}              {$\Delta\upphi$\xspace}
\newcommand{\moduledeltaphipitwo}   {$|\Delta\upphi| < \pi/2 $\xspace}
\newcommand{\deltaeta}              {$\Delta\upeta$\xspace}
\newcommand{\moduledeltaeta}        {$|\Delta\upeta|$\xspace}
\newcommand{\deltaphiapproxzero}    {$\Delta\upphi = 0$\xspace}
\newcommand{\deltaphiapproxpi}      {$\Delta\upphi = \pi$\xspace}
\newcommand{\deltaetaapproxzero}    {$\Delta\upeta = 0$\xspace}
\newcommand{\corrfunc}              {C($\Delta\upphi$, $\Delta\upeta$)\xspace}
\newcommand{\corrfunccorrect}              {C$_{\mathrm{correct}}(\Delta\upphi$, $\Delta\upeta$)\xspace}
\newcommand{\corrfuncmix}              {C$_{\mathrm{mix}}(\Delta\upphi$, $\Delta\upeta$)\xspace}
\newcommand{\corrfuncdeltaphi}      {C($\Delta\upphi$)\xspace}
\newcommand{\pttrigger}             {$p_{\textnormal{T}}^{\textnormal{trigger}}$\xspace}
\newcommand{\ptassoc}               {$p_{\textnormal{T}}^{\textnormal{assoc}}$\xspace}
\newcommand{\ratioyieldawaynearside}{Y$_{\textnormal{Away}}$/Y$_{\textnormal{Near}}$\xspace}

% 4) definition to references, biblatex and hyperlink
\usepackage[backend=bibtex, 
style=nature,  %style reference.
sorting=none,
firstinits=true %first name abbreviate
]{biblatex}

\usepackage{hyperref}
\hypersetup{
    colorlinks=true, %set "true" if you want colored links
    linktoc=all,     %set to "all" if you want both sections and subsections linked
    linkcolor=blue,  %choose some color if you want links to stand out
    citecolor= blue, % color of \cite{} in the text.
    urlcolor  = blue, % color of the link for the paper in references.
}

% 5) Tikz and figures
\usepackage{epsfig}
\usepackage{lmodern}
\usepackage{mathtools}
\usepackage[utf8]{luainputenc}
\usepackage{xspace}
\usepackage{tikz}
\usepackage{pgfplots}
\pgfplotsset{compat=newest}

\usetikzlibrary{positioning}
\usepackage{subcaption}

% 6) colors:
\usepackage{xcolor}
\definecolor{ao(english)}{rgb}{0.0, 0.5, 0.0} % dark green

% 7) Add lines numbers
%\usepackage{lineno}

% add pdf file to thesis:
\usepackage{pdfpages}

\hypersetup{
    colorlinks=true,% make the links colored
    linkcolor=blue
}

\usepackage{setspace}
\addbibresource{bibliography.bib}

\newcommand{\printingbibliography}{%

    \pagestyle{myheadings}
    \markright{}
    \sloppy
    \printbibliography[heading=bibintoc, % add to table of contents
                   title=Refer\^encias % Chapter name
                  ]
    \fussy%
}
\PassOptionsToPackage{table}{xcolor}

\pagestyle{fancy}
\fancyhf{}
\renewcommand{\headrulewidth}{0pt}
\fancyhead[R]{\thepage}

\geometry{a4paper,top=30mm,bottom=20mm,left=30mm,right=20mm}

\titleformat*{\section}{\bfseries\large}
\titleformat*{\subsection}{\bfseries\normalsize}

\title{Concurso Público do Instituto Federal de Sert\~ao  \\ 
EBTT \textbf{\large F\'isica}.}
\author{Andr\'e V. Silva \\ \texttt{\url{www.andrevsilva.com}}}
\date{\today}

\begin{document}

\maketitle

\tableofcontents

\noindent\rule{\linewidth}{0.4pt}\\
\newpage
\justifying

\noindent\rule{\linewidth}{0.4pt}\\
\section{Mec\^anica}
\begin{flushleft}
\subsection{Quest\~ao 41 — Força mínima para imin\^encia de movimento rampa acima}

Um bloco de massa $m$ encontra-se em repouso sobre um plano inclinado de ângulo
$\theta$ com a horizontal. Uma força $\vec{F}$ é aplicada ao bloco, formando ângulo
$\varphi$ com a direção do plano, como indicado na figura. O coeficiente de atrito
estático entre o bloco e o plano é $\mu$. Determine a intensidade mínima da força
$\vec{F}$ necessária para colocar o bloco na iminência de subir a rampa.

\begin{figure}[!h]
  \centering
  \includegraphics[scale=0.5]{figures/forca_minima.png}
\end{figure}

\subsection*{1) Equilíbrio de forças}

Projetando as forças ao longo dos eixos $\hat{x}$ (paralelo à rampa, apontando para cima) e $\hat{y}$ (normal ao plano):

\begin{equation}\label{eq:fx}
F\cos\varphi - mg\sin\theta - \mu N = 0
\end{equation}

\begin{equation}\label{eq:fy}
N - mg\cos\theta + F\sin\varphi = 0
\;\;\;\Rightarrow\;\;\;
N = mg\cos\theta - F\sin\varphi
\end{equation}

Substituindo \eqref{eq:fy} em \eqref{eq:fx}, obtemos:

\begin{equation}\label{eq:equil}
F\cos\varphi = mg\sin\theta + \mu\bigl(mg\cos\theta - F\sin\varphi\bigr).
\end{equation}

\subsection*{2) Expressão para a força aplicada}

Da equação \eqref{eq:equil}, resulta:

\begin{equation}\label{eq:fphi}
F(\varphi) = \frac{mg(\sin\theta + \mu\cos\theta)}{\cos\varphi + \mu\sin\varphi}.
\end{equation}

\subsection*{3) Maximização do denominador via Cauchy--Schwarz}

O denominador pode ser escrito como produto escalar:
\[
\cos\varphi + \mu\sin\varphi = (\cos\varphi,\;\sin\varphi)\cdot(1,\;\mu).
\]

Pela \colorbox{yellow!30}{desigualdade de Cauchy--Schwarz}:
\begin{equation}\label{eq:cauchy}
\boxed{
\cos\varphi + \mu\sin\varphi \le \sqrt{1+\mu^2}.
}
\end{equation}

A igualdade em \eqref{eq:cauchy} ocorre quando
\begin{equation}\label{eq:tanphi}
\tan\varphi^\star = \mu,
\end{equation}
isto é,
\[
\cos\varphi^\star = \frac{1}{\sqrt{1+\mu^2}},\qquad
\sin\varphi^\star = \frac{\mu}{\sqrt{1+\mu^2}}.
\]

\subsection*{4) Força mínima}

Substituindo o valor máximo do denominador \eqref{eq:cauchy} em \eqref{eq:fphi}, temos:

\begin{equation}\label{eq:fmin}
F_{\min} = \frac{mg(\sin\theta + \mu\cos\theta)}{\sqrt{1+\mu^2}}.
\end{equation}

Portanto, a força mínima aplicada que coloca o bloco na iminência de subir a rampa é dada por \eqref{eq:fmin}, atingida quando \eqref{eq:tanphi} vale.


\textbf{Alternativa correta:} \textbf{D}.
\end{flushleft}

\newpage

\noindent\rule{\linewidth}{0.6pt}\\

\begin{flushleft}
\subsection{Quest\~ao 42 — Cilindro com atrito}

Uma prancha de madeira, com comprimento $L = 1{,}0\ \text{m}$ e massa $m = 0{,}4\ \text{kg}$, 
possui um cilindro maciço e homogêneo de aço, com massa $M = 0{,}6\ \text{kg}$, 
localizado na extremidade direita da prancha. 
O sistema está em repouso sobre um plano horizontal liso. 
Uma força constante $\vec{F} = (20\ \text{N})\,\hat{x}$ é aplicada à prancha, 
fazendo com que os objetos comecem a se mover acelerados. 
O cilindro rola suavemente, sem escorregar, sobre a prancha, devido à presença de atrito entre eles. 
Desprezando o atrito entre a prancha e a superfície horizontal, bem como qualquer força de resistência do ar, 
determine o intervalo de tempo, em segundos, que o cilindro levará para cair da prancha, ou seja, 
para atingir a extremidade oposta e deixar de estar em contato com ela. 

\begin{figure}[!h]
  \centering
  \includegraphics[scale=0.5]{figures/q42.png}
\end{figure}

\begin{itemize}
\item[(A)] 0,1 s
\item[(B)] 0,2 s
\item[(C)] 0,3 s
\item[(D)] 0,4 s
\item[(E)] 0,5 s
\end{itemize}

\subsection*{1) Definição das variáveis e forças}

Seja $a_p$ a aceleração da prancha (para a direita) e $a_c$ a aceleração do centro do cilindro (para a direita),
ambas medidas no referencial inercial do solo. Seja $f$ a força de atrito horizontal exercida pela prancha sobre o cilindro
(no ponto de contato). Pela ação e reação, a prancha sofre $-f$ da parte do cilindro.

Para o \colorbox{yellow!30}{cilindro maciço homogêneo, momento de inércia em relação ao centro:}
\begin{equation}\label{eq:I}
\boxed{
I=\frac{1}{2}MR^{2}.
}
\end{equation}

Não precisamos do valor de $R$ explicitamente, apenas das relações de rotação/translação.

\subsection*{2) Equações de movimento}

Equilíbrio (segunda lei) para a prancha (força total horizontal):
\begin{equation}\label{eq:plank}
F - f = m a_p .
\end{equation}

Equação de translação para o cilindro:
\begin{equation}\label{eq:cyl_trans}
f = M a_c .
\end{equation}

Equação de rotação para o cilindro (torque causado por $f$):
\begin{equation}\label{eq:cyl_rot}
fR = I \alpha = \left(\frac{1}{2}MR^{2}\right)\alpha.
\end{equation}

\textbf{\textcolor{red}{Condição de rolamento sem escorregar entre cilindro e prancha:
a velocidade do ponto de contato do cilindro iguala a velocidade da prancha.}} Em termos das acelerações:
\begin{equation}\label{eq:constraint}
a_c - a_p = -R\alpha.
\end{equation}

(A escolha do sinal garante consistência: se a prancha acelera mais que o cilindro, o contato induz uma rotação que satisfaz \eqref{eq:constraint}.)

\subsection*{3) Eliminação das incógnitas}

Da \eqref{eq:cyl_rot} e de \eqref{eq:constraint} obtemos:
\[
fR = \tfrac{1}{2}MR^{2}\alpha \quad\Rightarrow\quad f = \tfrac{1}{2}M R \alpha.
\]
Usando \eqref{eq:constraint} $\alpha = -(a_c - a_p)/R$, resulta
\begin{equation}\label{eq:f_from_rot}
f = -\tfrac{1}{2}M (a_c - a_p).
\end{equation}

Por outro lado, pela translação do cilindro \eqref{eq:cyl_trans}:
\begin{equation}\label{eq:f_from_trans}
f = M a_c.
\end{equation}

Igualando \eqref{eq:f_from_rot} e \eqref{eq:f_from_trans}:
\[
M a_c = -\tfrac{1}{2}M (a_c - a_p).
\]
Dividindo por $M$ e rearranjando:
\[
a_c = -\tfrac{1}{2}a_c + \tfrac{1}{2}a_p
\quad\Rightarrow\quad
\tfrac{3}{2}a_c = \tfrac{1}{2}a_p
\quad\Rightarrow\quad
a_c = \tfrac{1}{3} a_p. \label{eq:ac_ap}
\]

Substituindo \eqref{eq:ac_ap} em \eqref{eq:plank} e usando \eqref{eq:cyl_trans} ($f=M a_c$):
\[
F - M a_c = m a_p.
\]
Como $a_c = a_p/3$, obtemos
\[
F - M\frac{a_p}{3} = m a_p
\quad\Rightarrow\quad
F = a_p\!\left(m + \frac{M}{3}\right).
\]
Logo a aceleração da prancha:
\begin{equation}\label{eq:ap}
a_p = \frac{F}{\,m + \dfrac{M}{3}\,} = \frac{3F}{3m+M}.
\end{equation}
E, pela \eqref{eq:ac_ap},
\begin{equation}\label{eq:ac}
a_c = \frac{a_p}{3} = \frac{F}{3m+M}.
\end{equation}

\subsection*{4) Aceleração relativa e tempo até cair}

A aceleração relativa entre prancha e cilindro (aceleração com que a prancha “afasta-se” do cilindro) é
\[
a_{\text{rel}} = a_p - a_c = a_p - \frac{a_p}{3} = \frac{2}{3}a_p.
\]
Usando \eqref{eq:ap}:
\begin{equation}\label{eq:arel}
a_{\text{rel}} = \frac{2}{3}\cdot\frac{3F}{3m+M} = \frac{2F}{3m+M}.
\end{equation}

Inicialmente a velocidade relativa é zero (sistema parte do repouso). A distância relativa a percorrer para que
o cilindro passe da extremidade direita até a esquerda da prancha é $L$. Para movimento uniformemente acelerado,
o tempo $t$ satisfaz $L = \tfrac{1}{2} a_{\text{rel}} t^{2}$, portanto
\begin{equation}\label{eq:time_general}
t = \sqrt{\frac{2L}{a_{\text{rel}}}} = \sqrt{\frac{2L(3m+M)}{2F}} = \sqrt{\frac{(3m+M)\,L}{F}}.
\end{equation}

\subsection*{5) Substituição numérica}

Dados: $m=0{,}4\ \mathrm{kg}$, $M=0{,}6\ \mathrm{kg}$, $L=1{,}0\ \mathrm{m}$, $F=20\ \mathrm{N}$.

Calcule $3m+M$:
\[
3m+M = 3(0{,}4)+0{,}6 = 1{,}2+0{,}6 = 1{,}8\ \mathrm{kg}.
\]

Substituindo em \eqref{eq:time_general}:
\[
t = \sqrt{\frac{(3m+M) L}{F}}
= \sqrt{\frac{1{,}8\times 1{,}0}{20}}
= \sqrt{\frac{1{,}8}{20}}
= \sqrt{0{,}09}
= 0{,}30\ \mathrm{s}.
\]

\bigskip
\noindent\textbf{Resposta:} $t = 0{,}3\ \mathrm{s}$. (Alternativa \textbf{C}.)


\end{flushleft}

\noindent\rule{\linewidth}{0.4pt}\\

\begin{flushleft}
\subsection{Quest\~ao 43 - Trabalho de uma força de resistência}
\noindent
Um projétil de massa $m$ é lançado verticalmente para cima a partir da posição $z=0$ com velocidade inicial 
$\vec{v} = v_0 \hat{z}$ $(v_0>0)$ no instante $t=0$. Além da força gravitacional, atua sobre ele uma força de 
resistência do ar proporcional à velocidade: $\vec{F} = - \beta m \vec{v}$, onde $\beta>0$ é o parâmetro de amortecimento. 
A aceleração da gravidade é $\vec{g} = - g \hat{z}$. Determine o trabalho realizado pela força de resistência desde o lançamento até a altura máxima.

\vspace{0.5cm}

\textcolor{red}{\textbf{Solução:}}\\

A força de resistência é:
\[
\vec{F}_r = - \beta m \vec{v} = - \beta m v \hat{z}.
\]

O trabalho realizado pela força de resistência até a altura máxima é:
\[
W_r = \int_{0}^{z_\text{max}} \vec{F}_r \cdot d\vec{z} = - \beta m \int_0^{z_\text{max}} v \, dz.
\]

A \colorbox{red!20}{equação do movimento é}:
\[
m \frac{dv}{dt} = - mg - \beta m v \quad \Rightarrow \quad \boxed{\frac{dv}{dt} + \beta v = - g.}
\]

Solução da equação diferencial:

\[
\frac{dv}{dt} + \beta v = - g \Rightarrow \frac{dv}{dt} = - g - \beta v 
\]

\[
\frac{dv}{g+ \beta v} = - dt
\]

\[
\int \frac{dv}{g+ \beta v} = - \int dt
\]

\[
\frac{ln \left( g+ \beta v \right)}{\beta} = - t + C
\]

Usando as \colorbox{green!30}{condi\c{c}\~oes de contorno do problema (quando $t=0$ e $v=v_{o}$)}:

\[C = \frac{ln \left( g+ \beta v_0 \right)}{\beta}\]

\[
\frac{ln \left( g+ \beta v \right)}{\beta} = - t + \frac{ln \left( g+ \beta v_0 \right)}{\beta}
\]

\[
\frac{ln \left( g+ \beta v \right)}{\beta} - \frac{ln \left( g+ \beta v_0 \right)}{\beta} = - t
\]

\[
ln \left( g+ \beta v \right) - ln \left( g+ \beta v_0 \right) = - \beta t
\]

\[
ln \left[\frac{\left( g+ \beta v \right)}{ \left( g+ \beta v_0 \right)}\right] = - \beta t
\]

\[
\frac{\left( g+ \beta v \right)}{ \left( g+ \beta v_0 \right)} = e^{- \beta t}
\]

\[
\left( g+ \beta v \right)= \left( g+ \beta v_0 \right) e^{- \beta t}
\]

\[
\beta v = \left( g+ \beta v_0 \right) e^{- \beta t} - g
\]

\[
\boxed{
v(t) = \left(v_0 + \frac{g}{\beta}\right) e^{-\beta t} - \frac{g}{\beta}.
}
\]

\colorbox{green!30}{Altura máxima ocorre em $t_\text{max}$ tal que $v(t_\text{max}) = 0$}:
\[
0 = \left(v_0 + \frac{g}{\beta}\right) e^{-\beta t_\text{max}} - \frac{g}{\beta} 
\quad \Rightarrow \quad 
e^{-\beta t_\text{max}} = \frac{g/\beta}{v_0 + g/\beta} 
\quad \Rightarrow \quad \boxed{t_\text{max} = \frac{1}{\beta} \ln \left( 1 + \frac{\beta v_0}{g} \right).}
\]

O trabalho da força de resistência:
\[
W_r = - \beta m \int_0^{t_\text{max}} v^2(t) \, dt
= - \beta m \int_0^{t_\text{max}} \left[ \left(v_0 + \frac{g}{\beta}\right) e^{-\beta t} - \frac{g}{\beta} \right]^2 dt.
\]

\[
W_r = - \beta m \int_0^{t_\text{max}} \left[ \left(v_0 + \frac{g}{\beta}\right) e^{-\beta t} - \frac{g}{\beta} \right]^2 dt
\]

\[
\begin{aligned}
W_r &= - \beta m \int_0^{t_\text{max}} \Bigg[ 
\left(v_0 + \frac{g}{\beta}\right)^2 e^{-2\beta t} 
- 2 \left(v_0 + \frac{g}{\beta}\right) \frac{g}{\beta} e^{-\beta t} 
+ \left(\frac{g}{\beta}\right)^2 
\Bigg] dt \\
&= - \beta m \Bigg[ 
\left(v_0 + \frac{g}{\beta}\right)^2 \textcolor{red}{\int_0^{t_\text{max}} e^{-2\beta t} dt} 
- 2 \left(v_0 + \frac{g}{\beta}\right) \frac{g}{\beta} \textcolor{orange!90}{\int_0^{t_\text{max}} e^{-\beta t} dt }
+ \left(\frac{g}{\beta}\right)^2 \textcolor{blue}{\int_0^{t_\text{max}} dt}
\Bigg]
\end{aligned}
\]

\[
\begin{aligned}
\textcolor{red}{\int_0^{t_\text{max}} e^{-2\beta t} dt} &= \frac{1 - e^{-2 \beta t_\text{max}}}{2\beta}, \\
\textcolor{orange!90}{\int_0^{t_\text{max}} e^{-\beta t} dt } &= \frac{1 - e^{-\beta t_\text{max}}}{\beta}, \\
\textcolor{blue}{\int_0^{t_\text{max}} dt} & = t_\text{max}.
\end{aligned}
\]

Integrando e substituindo \(t_\text{max}\) e \(e^{-\beta t_\text{max}}\):

\[
t_\text{max} = \frac{1}{\beta} \ln \left( 1 + \frac{\beta v_0}{g} \right)
\]

\[
1 - e^{-2\beta t_\text{max}} = 1 - e^{-2\beta \left(\frac{1}{\beta} \ln \left( 1 + \frac{\beta v_0}{g} \right)\right)}
\]

\[
1 - e^{-2\beta t_\text{max}} = 1 - e^{-\ln \left( 1 + \frac{\beta v_0}{g} \right)^2}
\]

\[
1 - e^{-2\beta t_\text{max}} = 1 - \left( 1 + \frac{\beta v_0}{g} \right)^{-2}
\]

\[
1 - e^{-2\beta t_\text{max}} = 1 - \frac{1}{\left( 1 + \frac{\beta v_0}{g} \right)^{2}}
\]

\[
1 - e^{-2\beta t_\text{max}} = \frac{\left( 1 + \frac{\beta v_0}{g} \right)^{2} -1}{\left( 1 + \frac{\beta v_0}{g} \right)^{2}}
\]

\[
1 - e^{-2\beta t_\text{max}} = \frac{1 + \frac{2\beta v_0}{g} + \frac{\beta^2 v_0^2}{g^2} - 1}{\left( 1 + \frac{\beta v_0}{g} \right)^{2}}
\]

\[
\boxed{
1 - e^{-2\beta t_\text{max}} = \frac{\frac{2\beta v_0}{g} + \frac{\beta^2 v_0^2}{g^2}}{\left( 1 + \frac{\beta v_0}{g} \right)^{2}}
}
\]

\[
\boxed{
\begin{aligned}
\textcolor{red}{\int_0^{t_\text{max}} e^{-2\beta t} dt} &= \frac{1 - e^{-2 \beta t_\text{max}}}{2\beta} =  \frac{1}{2\beta} \frac{\left[\frac{2\beta v_0}{g} + \frac{\beta^2 v_0^2}{g^2} \right]}{\left(\frac{\beta}{g}\right)^2 \left( \frac{g}{\beta} + v_0 \right)^{2}}\\
\end{aligned}
}
\]

\[
\boxed{
\begin{aligned}
\textcolor{red}{\int_0^{t_\text{max}} e^{-2\beta t} dt} &= \frac{g^2}{2 \beta^3} \frac{\left[\frac{2\beta v_0}{g} + \frac{\beta^2 v_0^2}{g^2} \right]}{\left( v_0 + \frac{g}{\beta} \right)^{2}}. \quad \checkmark
\end{aligned}
}
\]

\[
1 - e^{-\beta t_\text{max}} = 1 - e^{-\beta \left(\frac{1}{\beta} \ln \left( 1 + \frac{\beta v_0}{g} \right)\right)}
\]

\[
1 - e^{-\beta t_\text{max}} = 1 - e^{\ln \left( 1 + \frac{\beta v_0}{g} \right)^{-1}}
\]

\[
1 - e^{-\beta t_\text{max}} = 1 - \left( 1 + \frac{\beta v_0}{g} \right)^{-1}
\]

\[
1 - e^{-\beta t_\text{max}} = 1 - \frac{1}{\left( 1 + \frac{\beta v_0}{g} \right)}
\]

\[
1 - e^{-\beta t_\text{max}} = \frac{\left( 1 + \frac{\beta v_0}{g} \right) - 1}{\left( 1 + \frac{\beta v_0}{g} \right)}
\]

\[
1 - e^{-\beta t_\text{max}} = \frac{\left(\frac{\beta v_0}{g} \right)}{\left( 1 + \frac{\beta v_0}{g} \right)}
\]

\[
1 - e^{-\beta t_\text{max}} = \frac{\left(\frac{\beta v_0}{g} \right)}{\left( 1 + \frac{\beta v_0}{g} \right)} = \frac{\left(\frac{\beta v_0}{g} \right)}{\frac{\beta}{g}\left( v_0 + \frac{g}{\beta} \right)} = \frac{v_0}{\left( v_0 + \frac{g}{\beta} \right)}
\]

\[
\begin{aligned}
\textcolor{orange!90}{\int_0^{t_\text{max}} e^{-\beta t} dt } &= \frac{1 - e^{-\beta t_\text{max}}}{\beta} = \frac{\frac{v_0}{\left( v_0 + \frac{g}{\beta} \right)}}{\beta} = \frac{v_0}{\beta\left( v_0 + \frac{g}{\beta} \right)} \\
\end{aligned}
\]

\[
\begin{aligned}
\textcolor{orange!90}{\int_0^{t_\text{max}} e^{-\beta t} dt } &= \frac{v_0}{\beta\left( v_0 + \frac{g}{\beta} \right)}. \quad \checkmark \\
\end{aligned}
\]


\[
\begin{aligned}
W_r &= - \beta m \Bigg[ 
\cancel{\left(v_0 + \frac{g}{\beta}\right)^2} \left[\frac{g^2}{2 \beta^3 } \frac{\left[\frac{2\beta v_0}{g} + \frac{\beta^2 v_0^2}{g^2} \right]}{\cancel{\left( v_0 + \frac{\beta}{g} \right)^{2}}}\right]
- 2 \cancel{\left(v_0 + \frac{g}{\beta}\right)}\frac{g}{\beta} \frac{v_0}{\beta\cancel{\left( v_0 + \frac{g}{\beta} \right)}} 
+ \left(\frac{g^2}{\beta^3}\right)\ln \left( 1 + \frac{\beta v_0}{g} \right)
\Bigg]
\end{aligned}
\]

\[
\begin{aligned}
W_r &= - \beta m \Bigg[ 
\frac{g^2 \left[\frac{2\beta v_0}{g} + \frac{\beta^2 v_0^2}{g^2} \right]}{2 \beta^3}
- 2 g\frac{v_0}{\beta} 
+ \left(\frac{g^2}{\beta^3}\right)\ln \left( 1 + \frac{\beta v_0}{g} \right)
\Bigg]
\end{aligned}
\]

\[
\begin{aligned}
W_r &= - \Bigg[ 
\frac{m g^2}{2 \beta^2} \left[\frac{2\beta v_0}{g} + \frac{\beta^2 v_0^2}{g^2} \right]
- 2 mg\frac{v_0}{\beta} 
+ \left(\frac{m g^2}{\beta^2}\right)\ln \left( 1 + \frac{\beta v_0}{g} \right)
\Bigg]
\end{aligned}
\]

\[
\begin{aligned}
W_r &= - \Bigg[ 
\frac{m g v_0 }{\beta}  + \frac{m v_0^2}{2}
- 2 mg\frac{v_0}{\beta} 
+ \left(\frac{m g^2}{\beta^2}\right)\ln \left( 1 + \frac{\beta v_0}{g} \right)
\Bigg]
\end{aligned}
\]

\[
\begin{aligned}
W_r &= 
- \frac{m g v_0 }{\beta} - \frac{m v_0^2}{2}
+ \frac{2 mg v_0}{\beta} 
- \left(\frac{m g^2}{\beta^2}\right)\ln \left( 1 + \frac{\beta v_0}{g} \right)
\end{aligned}
\]

\[
\begin{aligned}
W_r &= 
\frac{m g v_0 }{\beta}
- \left(\frac{m g^2}{\beta^2}\right)\ln \left( 1 + \frac{\beta v_0}{g} \right) - \frac{m v_0^2}{2}
\end{aligned}
\]

\[
\boxed{
\begin{aligned}
W_r &=  \frac{m g v_0 }{\beta} \Bigg[1 - \left(\frac{m g^2}{\beta^2}\right)\ln \left( 1 + \frac{\beta v_0}{g} \right) \Bigg] - \frac{m v_0^2}{2}
\end{aligned}
}
\quad \blacksquare
\]


\end{flushleft}


\begin{flushleft}

\subsection{Quest\~ao 44 - P\^endulo F\'isico}
Um p\^endulo f\'isico constitu\'{\i}do por uma placa fina e homog\^enea em forma de um setor circular de raio $R$ e \^angulo 
central $\alpha$, est\'a suspenso verticalmente no centro $O$ do disco de origem. O p\^endulo \'{e} deslocado por um \^angulo 
$\theta$ em rela\c{c}\~ao \`a vertical e, em seguida, abandonado a partir do repouso para oscilar. A acelera\c{c}\~ao local 
da gravidade \'e $g$ e poss\'iveis atritos s\~ao desprez\'iveis. Assinale a alternativa que apresenta a express\~ao correta 
para a frequ\^encia angular $\omega$ de pequenas oscila\c{c}\~oes do p\^endulo f\'isico.

\begin{figure}[!h]
  \centering
  \includegraphics[scale=0.5]{figures/pendulo_fisico.png}
\end{figure}


\begin{itemize}
\item[(A)] $\displaystyle \omega=\sqrt{\frac{4g}{3R}}$
\item[(B)] $\displaystyle \omega=\sqrt{\frac{8g\cos(\alpha)}{3R\,\alpha}}$
\item[(C)] $\displaystyle \omega=\sqrt{\frac{8g\sin(\alpha/2)}{3R\,\alpha}}$
\item[(D)] $\displaystyle \omega=\sqrt{\frac{4g\sin(\alpha)}{3R\,\alpha}}$
\item[(E)] $\displaystyle \omega=\sqrt{\frac{4g\cos(\alpha/2)}{3R\,\alpha}}$
\end{itemize}

\vspace{0.5cm}

\textcolor{red}{\textbf{Solução:}}\\

Para \colorbox{yellow!30}{pequenas oscila\c{c}\~oes linearizamos $\sin\theta\approx\theta$ e usamos a equa\c{c}\~ao do p\^endulo f\'isico:}
\[
I_O\,\ddot\theta + m g h\,\theta = 0,
\]

\[
\,\ddot\theta + \frac{m g h}{I_O}\,\theta = 0,
\]

\[
\,\ddot\theta + \omega^{2}\,\theta = 0,
\]


onde $I_O$ \'e o momento de in\'ercia em rela\c{c}\~ao ao ponto de suspens\~ao $O$ (eixo perpendicular ao plano) e $h$ \'e a dist\^ancia 
do centro de massa ao ponto $O$.

\textbf{1) Massa e momento de in\'ercia:}\\
Para uma placa homog\^enea em forma de setor, a densidade superficial $\sigma$ satisfaz
\[
m=\sigma\cdot\text{área}=\sigma\left(\tfrac{1}{2}\alpha R^2\right).
\]
O momento de in\'ercia em rela\c{c}\~ao a $O$ (eixo perpendicular ao plano) é
\[
I_O=\sigma\int_{0}^{\alpha}\int_{0}^{R} r^2\;r\,dr\,d\phi
=\sigma\frac{\alpha R^4}{4}.
\]
Substituindo $\sigma = \dfrac{2m}{\alpha R^2}$ obtemos
\[
I_O=\frac{2m}{\alpha R^2}\cdot\frac{\alpha R^4}{4}=\frac{mR^2}{2}.
\]

\textbf{2) Centro de massa (distância radial $h$ a partir de $O$):}\\
O centro de massa de um setor circular encontra-se sobre a bissetriz e sua distância ao centro é
\[
\boxed{
h=r_{CM}=\frac{4R\sin(\alpha/2)}{3\alpha}.
}
\]

\textbf{3) Frequência angular:}\\
\[
\omega=\sqrt{\frac{m g h}{I_O}}
=\sqrt{\frac{m g \,\dfrac{4R\sin(\alpha/2)}{3\alpha}}{\dfrac{mR^2}{2}}}
=\sqrt{\frac{8g\sin(\alpha/2)}{3R\,\alpha}}.
\]

Portanto, a alternativa correta é \(\colorbox{green!50}{\textbf{(C)}}\).

\end{flushleft}


\begin{flushleft}
\subsection{Quest\~ao 45 - Colis\~ao Unidimensional inel\'astica}

Considere uma part\'icula de massa $m$, que se move com velocidade $v_0$, e realiza uma colis\~ao unidimensional inel\'astica com outra part\'icula de massa $M$, inicialmente em repouso. O coeficiente de restitui\c{c}\~ao do material constituinte das part\'iculas \'e denotado por $\varepsilon$. Considerando que a raz\~ao das massas das part\'iculas \'e $M/m=\lambda$, analise as assertivas abaixo:

I. A velocidade da part\'icula de massa $m$ ap\'os a colis\~ao \'e $v=v_0(1-\varepsilon\lambda)/(1+\lambda)$.\\
II. A velocidade da part\'icula de massa $M$ ap\'os a colis\~ao \'e $V=v_0(1+\varepsilon)/(1+\lambda)$.\\
III. A raz\~ao entre a energia cin\'etica adquirida pela part\'icula de massa $M$ e a energia cin\'etica inicial da part\'icula de massa $m$ \'e $\lambda(\varepsilon+1)/(\lambda+1)$.

Quais est\~ao corretas?

\begin{itemize}
\item[(A)] Apenas I.
\item[(B)] Apenas II.
\item[(C)] Apenas III.
\item[(D)] Apenas I e II.
\item[(E)] I, II e III.
\end{itemize}

\vspace{0.5cm}

\textcolor{red}{\textbf{Solução:}}\\

Pela conserva\c{c}\~ao do momento e defini\c{c}\~ao do coeficiente de restitui\c{c}\~ao:
\[
m v_0 = m v + M V,
\qquad
V - v = \varepsilon (v_0-0)=\varepsilon v_0.
\]
Da segunda equa\c{c}\~ao temos $V=v+\varepsilon v_0$. Substituindo na conserva\c{c}\~ao do momento:
\[
m v_0 = m v + M(v+\varepsilon v_0)=(m+M)v + M\varepsilon v_0.
\]
Isolando $v$:
\[
(m+M)v = v_0(m - M\varepsilon)
\quad\Rightarrow\quad
v = v_0\frac{m - M\varepsilon}{m+M}
= v_0\frac{1-\lambda\varepsilon}{1+\lambda},
\]
o que confirma a assertiva \textbf{I}.

Agora $V=v+\varepsilon v_0$:
\[
V=v_0\frac{1-\lambda\varepsilon}{1+\lambda}+\varepsilon v_0
= v_0\frac{1-\lambda\varepsilon+\varepsilon(1+\lambda)}{1+\lambda}
= v_0\frac{1+\varepsilon}{1+\lambda},
\]
confirmando a assertiva \textbf{II}.

Para a assertiva \textbf{III}, calculemos a raz\~ao das energias:
\[
\frac{K_M}{K_{m,\,\text{inicial}}}
=\frac{\tfrac{1}{2}M V^2}{\tfrac{1}{2}m v_0^2}
=\frac{M}{m}\left(\frac{V}{v_0}\right)^2
=\lambda\left(\frac{1+\varepsilon}{1+\lambda}\right)^2
=\frac{\lambda(1+\varepsilon)^2}{(1+\lambda)^2},
\]
que \textbf{não} coincide com $\dfrac{\lambda(1+\varepsilon)}{1+\lambda}$ (a dada na III). Portanto a assertiva \textbf{III} é falsa.

Assim, est\~ao corretas apenas I e II.

A resposta correta é alternativa \colorbox{green!50}{\textbf{(D)}}.

\end{flushleft}

\begin{flushleft}
\subsection{Quest\~ao 47 - Oscila\c{c}\~oes acopladas}

Dois blocos (1 e 2) de massas iguais a  $m = 0,5\,\text{kg}$ são conectados a três molas que
estão posicionadas entre duas paredes, conforme ilustrado na figura abaixo. A constante elástica das
duas molas externas é $k = 2,0\,\text{N/m}$, e a constante elástica da mola do meio $k_0 = 8,0\,\text{N/m}$. As molas
têm massa desprezível e satisfazem à lei de Hooke. Sabe-se também que quando os blocos se
encontram simultaneamente em suas respectivas posições de equilíbrio, as molas não apresentam
qualquer deformação. Considere que $x_1(t)$ e $x_2(t)$ denotam os deslocamentos dos blocos da esquerda
e da direita, respectivamente, em relação às suas posições de equilíbrio. No instante inicial $t$ = 0,
ambos os blocos 1 e 2 são soltos a partir do repouso nas posições $x_1(0)=10\,\text{cm}$ e $x_2(0)=0$,
respectivamente. Assinale a alternativa que representa a posição dos blocos como função do tempo
medido em unidades do sistema internacional.

\begin{figure}[!h]
  \centering
  \includegraphics[scale=0.5]{figures/sistema-massa-mola.png}
\end{figure}

\begin{itemize}
\item[(A)] $x_1(t)=0,05[\cos(2t)+\cos(6t)]\,\text{m}, \;\; x_2(t)=0,05[\cos(2t)-\cos(6t)]\,\text{m}$
\item[(B)] $x_1(t)=0,05[\cos(2t)+\cos(4t)]\,\text{m}, \;\; x_2(t)=0,05[\cos(4t)-\cos(2t)]\,\text{m}$
\item[(C)] $x_1(t)=0,05\cos(3t)\cos(t)\,\text{m}, \;\; x_2(t)=0,05\sin(3t)\sin(t)\,\text{m}$
\item[(D)] $x_1(t)=0,10\cos(4t)\,\text{m}, \;\; x_2(t)=0,10\sin(2t)\,\text{m}$
\item[(E)] $x_1(t)=0,10\cos(2t)\,\text{m}, \;\; x_2(t)=0,10\sin(4t)\,\text{m}$
\end{itemize}

\vspace{0.5cm}

\textcolor{red}{\textbf{Solu\c{c}\~ao:}}\\[2mm]

1) \textbf{Equa\c{c}\~oes de movimento:}  
Para o bloco 1:
\[
m\ddot{x}_1 = -k x_1 - k_0(x_1 - x_2).
\]

Para o bloco 2:
\[
m\ddot{x}_2 = -k x_2 - k_0(x_2 - x_1).
\]

\[
\Rightarrow 
\begin{cases}
\ddot{x}_1 + \dfrac{k+k_0}{m}x_1 - \dfrac{k_0}{m}x_2 = 0, \\
\ddot{x}_2 + \dfrac{k+k_0}{m}x_2 - \dfrac{k_0}{m}x_1 = 0.
\end{cases}
\]

2) \textbf{Matriz do sistema:}  
\[
\begin{bmatrix}
\ddot{x}_1 \\ \ddot{x}_2
\end{bmatrix}
= -\frac{1}{m}
\begin{bmatrix}
k+k_0 & -k_0 \\
-k_0 & k+k_0
\end{bmatrix}
\begin{bmatrix}
x_1 \\ x_2
\end{bmatrix}.
\]

Com $m=0,5$, $k=2$ e $k_0=8$:
\[
A=\frac{1}{0,5}
\begin{bmatrix}
10 & -8 \\ -8 & 10
\end{bmatrix}
=
\begin{bmatrix}
20 & -16 \\ -16 & 20
\end{bmatrix}.
\]

3) \textbf{Autovalores (modos normais):}  
\[
\det(A-\lambda I)=0 \;\;\Rightarrow\;\;
(20-\lambda)^2-(-16)^2=0,
\]
\[
(20-\lambda)^2-256=0,
\;\;\Rightarrow\;\;
20-\lambda=\pm 16.
\]
\[
\lambda_1=4, \quad \lambda_2=36.
\]

Logo, as frequ\^encias s\~ao:
\[
\omega_1=\sqrt{4}=2, \qquad \omega_2=\sqrt{36}=6.
\]

4) \textbf{Autovetores:}  
Para $\lambda_1=4$:  
\[
(20-4)x_1-16x_2=0 \;\;\Rightarrow\;\; x_1=x_2.
\]

Para $\lambda_2=36$:  
\[
(20-36)x_1-16x_2=0 \;\;\Rightarrow\;\; x_1=-x_2.
\]

\textbf{Modos normais:}
\[
\begin{cases}
\text{Modo 1 (freq. 2 rad/s): } x_1=x_2, \\
\text{Modo 2 (freq. 6 rad/s): } x_1=-x_2.
\end{cases}
\]

5) \textbf{Combina\c{c}\~ao linear:}  
Solu\c{c}\~ao geral:
\[
x_1(t)=A\cos(2t)+B\cos(6t), \quad
x_2(t)=A\cos(2t)-B\cos(6t).
\]

6) \textbf{Condi\c{c}\~oes iniciais:}  
No instante $t=0$:
\[
x_1(0)=A+B=0,10, \quad x_2(0)=A-B=0.
\]

\[
A=B=0,05.
\]

7) \textbf{Solu\c{c}\~ao final:}
\[
x_1(t)=0,05[\cos(2t)+\cos(6t)] \;\text{m}, \qquad
x_2(t)=0,05[\cos(2t)-\cos(6t)] \;\text{m}.
\]

\medskip

A resposta correta \'e a alternativa \colorbox{green!50}{\textbf{(A)}}.

\end{flushleft}


\section{Gravita\c{c}\~ao}

\begin{flushleft}
\subsection{Quest\~ao 46 - Balan\c{c}a de tor\c{c}\~ao de Cavendish}

No experimento de Henry Cavendish, de 1797, foi utilizada uma balan\c{c}a de tor\c{c}\~ao
para determinar o valor da constante gravitacional $G$ da lei da gravita\c{c}\~ao universal de Newton.
Considere uma balan\c{c}a de tor\c{c}\~ao composta por uma barra de massa desprez\'ivel e comprimento $L$,
suspensa horizontalmente pelo seu centro por um fio de tor\c{c}\~ao vertical. Duas pequenas esferas de
massa igual a $m$ est\~ao presas em cada extremidade da barra. No primeiro passo do experimento,
observa-se que, quando a barra \'e girada com um pequeno \^angulo, torcendo o fio, e depois solta, o
p\^endulo de tor\c{c}\~ao resultante sofre movimento harm\^onico simples com um per\'iodo $T$. Em seguida,
ap\'os o p\^endulo ser parado e estar em sua posi\c{c}\~ao de equil\'ibrio, um par de esferas grandes de massa
igual a $M$ s\~ao colocadas em lados opostos da barra, cada uma pr\'oxima a uma das massas $m$. Devido
\`a atra\c{c}\~ao gravitacional apenas entre cada par de massas, a barra \'e observada girando por um pequeno
\^angulo $\theta$ e depois parar nessa posi\c{c}\~ao, com cada massa $M$ a uma dist\^ancia $D$ da massa $m$
correspondente. Determine uma express\~ao para $G$ em termos das vari\'aveis dadas no problema.

\begin{figure}[!h]
  \centering
  \includegraphics[scale=0.3]{figures/henry_cavendish.png}
\end{figure}

\begin{itemize}
\item[(A)] $\,G=\dfrac{\pi^2 D^2 L^2 \theta}{M T^2}$
\item[(B)] $\,G=\dfrac{2\pi^2 D^2 L \theta}{M T^2}$
\item[(C)] $\,G=\dfrac{4\pi^2 D^2 L^2 \theta}{M T^2}$
\item[(D)] $\,G=\dfrac{\pi^2 D^2 L \theta}{m T^2}$
\item[(E)] $\,G=\dfrac{\pi^2 D^2 L \theta}{4m T^2}$
\end{itemize}

\vspace{0.5cm}

\textcolor{red}{\textbf{Solu\c{c}\~ao:}}\\[2mm]

1) \textbf{\colorbox{yellow!30}{Constante de tor\c{c}\~ao via o per\'iodo.}} Para pequenas oscila\c{c}\~oes, o p\^endulo de tor\c{c}\~ao satisfaz
\[
T=2\pi\sqrt{\frac{I}{\kappa}}
\quad\Rightarrow\quad
\boxed{\kappa=\frac{4\pi^2 I}{T^2}}.
\]
A barra \'e desprez\'ivel e h\'a duas massas $m$ a $L/2$ do eixo, logo
\[
I=2\,m\left(\frac{L}{2}\right)^2=\frac{mL^2}{2}
\;\;\Rightarrow\;\;
\kappa=\frac{4\pi^2}{T^2}\,\frac{mL^2}{2}
=\frac{2\pi^2 m L^2}{T^2}.
\]

2)  \textbf{\colorbox{yellow!30}{Equil\'ibrio com as massas $M$.}} A for\c{c}a gravitacional entre $M$ e $m$ \'e
\[
F=\frac{G m M}{D^2}.
\]
Cada for\c{c}a produz um torque de m\'odulo $F\cdot (L/2)$ em torno do centro; s\~ao duas for\c{c}as sim\'etricas, portanto o torque gravitacional total vale
\[
\tau_g=2\,F\left(\frac{L}{2}\right)=F\,L.
\]
No novo equil\'ibrio, \colorbox{yellow!30}{o torque el\'astico do fio $\tau_{\kappa}=\kappa\,\theta$ (para pequeno $\theta$)} equilibra o torque gravitacional:
\[
\kappa\,\theta=F\,L=\frac{G m M}{D^2}\,L.
\]

3) \textbf{Isolando $G$.} Substituindo $\kappa$:
\[
\frac{2\pi^2 m L^2}{T^2}\,\theta=\frac{G m M}{D^2}\,L
\;\;\Rightarrow\;\;
\boxed{G=\frac{2\pi^2 D^2 L\,\theta}{M T^2}.}
\]

\medskip

A resposta correta \'e a alternativa \colorbox{green!50}{\textbf{(B)}}.

\end{flushleft}

\begin{flushleft}
\subsection{Questão 48 - Módulo da velocidade de um satélite orbitando a Terra}
Um satélite artificial orbita a Terra em uma trajetória elíptica sob efeito apenas da
força gravitacional. O satélite passa pelo perigeu $P$ (ponto mais próximo à Terra) com velocidade $\vec{v}_p$ e
pelo apogeu $A$ (ponto mais afastado da Terra) com velocidade $\vec{v}_a$. A velocidade do satélite em um
ponto $Y$, localizado na linha que passa pela Terra e perpendicular ao eixo maior da elipse, é denotada
por $\vec{v}$. É correto afirmar que o módulo da velocidade $v$ no ponto $Y$, em termos de $v_p$ e $v_a$ , é expresso
por:

\begin{figure}[!h]
  \centering
  \includegraphics[scale=0.6]{figures/satelite.png}
\end{figure}

\begin{itemize}
\item[(A)] $v = \dfrac{v_a+v_p}{2}$
\item[(B)] $v = \dfrac{2v_av_p}{v_a+v_p}$
\item[(C)] $v = \sqrt{v_av_p}$
\item[(D)] $v = \sqrt{\dfrac{v_a^2+v_p^2}{2}}$
\item[(E)] $v = \sqrt{\dfrac{2v_a^2v_p^2}{v_a^2+v_p^2}}$
\end{itemize}

\vspace{0.5cm}

\textcolor{red}{\textbf{Solução:}}\\

Considerando a órbita elíptica com foco na Terra, usemos a equação de \emph{vis-viva} e a conservação do momento angular. Denotando por $\mu=GM$,

\[
v^2=\mu\!\left(\frac{2}{r}-\frac{1}{a}\right),
\]
onde $r$ é a distância ao foco (Terra) no ponto considerado e $a$ é o semieixo maior. Para o perigeu ($r_p$) e apogeu ($r_a$) temos
\begin{align}
v_p^2&=\mu\!\left(\frac{2}{r_p}-\frac{1}{a}\right),\label{vp}\\
v_a^2&=\mu\!\left(\frac{2}{r_a}-\frac{1}{a}\right).\label{va}
\end{align}

Subtraindo \eqref{va} de \eqref{vp} obtemos
\[
v_p^2-v_a^2=2\mu\!\left(\frac{1}{r_p}-\frac{1}{r_a}\right)
\quad\Rightarrow\quad
\mu=\frac{v_p^2-v_a^2}{2\left(\dfrac{1}{r_p}-\dfrac{1}{r_a}\right)}.
\]

O ponto $Y$ corresponde ao ângulo verdadeiro $\theta=\tfrac{\pi}{2}$, portanto
\[
r_Y=\frac{a(1-e^2)}{1+0}=a(1-e^2).
\]
Usando a relação entre os raios de perigeu/apogeu e $a$ (isto é, $r_p=a(1-e)$ e $r_a=a(1+e)$) obtemos
\[
\frac{1}{r_Y}=\frac{a}{r_pr_a}=\frac{r_p+r_a}{2r_pr_a}.
\]

Agora escrevemos a velocidade em $Y$ via vis-viva (usando a expressão em $r_p$ e eliminando $1/a$):
\[
v_Y^2=v_p^2+2\mu\!\left(\frac{1}{r_Y}-\frac{1}{r_p}\right).
\]
Substituindo $\mu$ e $\dfrac{1}{r_Y}-\dfrac{1}{r_p}=\dfrac{r_p-r_a}{2r_pr_a}$ temos
\[
v_Y^2
= v_p^2 + (v_p^2-v_a^2)\frac{r_pr_a}{r_a-r_p}\cdot\frac{r_p-r_a}{2r_pr_a}
= v_p^2 -\frac{1}{2}(v_p^2-v_a^2).
\]
Portanto
\[
v_Y^2=\frac{v_p^2+v_a^2}{2},
\]
e
\[
\boxed{\,v_Y=\sqrt{\dfrac{v_p^2+v_a^2}{2}}\,.}
\]

\textbf{Resposta:} alternativa \(\colorbox{green!50}{\textbf{D}}\).

\end{flushleft}


\section{Termodin\^amica}

\begin{flushleft}
\subsection{Quest\~ao 49 - Variação de Entropia Total}
Em um recipiente de capacidade térmica desprezível e termicamente isolado, uma quantidade de água de massa \(m_A=80\ \mathrm{g}\) 
encontra-se inicialmente à temperatura \(T_A=60^\circ\mathrm{C}\). Um cubo de gelo com massa \(m_B=20\ \mathrm{g}\) a \(T_0=0^\circ\mathrm{C}\) é 
introduzido no interior do recipiente. Sabe-se que o calor específico da água é \(c=1{.}0\ \mathrm{cal\;g^{-1}\;^\circ C^{-1}}\) e o calor 
latente de fusão do gelo a \(0^\circ\mathrm{C}\) é \(L=80\ \mathrm{cal\;g^{-1}}\). Qual é a variação de entropia total do sistema ao atingir 
o equilíbrio térmico, em unidades de \(\mathrm{cal\cdot K^{-1}}\)?

\begin{itemize}
\item[(A)] \(20\ln\!\left(\dfrac{305}{273}\right) - 80\ln\!\left(\dfrac{333}{305}\right)\)
\item[(B)] \(20\ln\!\left(\dfrac{321}{273}\right) - 80\ln\!\left(\dfrac{333}{321}\right)\)
\item[(C)] \(\dfrac{8000}{273} + 100\ln\!\left(\dfrac{323}{273}\right) - 100\ln\!\left(\dfrac{373}{323}\right)\)
\item[(D)] \(\dfrac{1600}{273} + 20\ln\!\left(\dfrac{321}{273}\right) - 80\ln\!\left(\dfrac{333}{321}\right)\)
\item[(E)] \(\dfrac{1600}{273} + 20\ln\!\left(\dfrac{305}{273}\right) - 80\ln\!\left(\dfrac{333}{305}\right)\)
\end{itemize}

\vspace{0.5cm}

\textcolor{red}{\textbf{Solução:}}\\

Calculemos a temperatura de equilíbrio \(T_f\) pelo balanço de energia:
\[
m_A c (60 - T_f) = m_B L + m_B c (T_f - 0).
\]
Substituindo \(m_A=80\), \(m_B=20\), \(c=1\), \(L=80\):
\[
80(60 - T_f) = 20\cdot 80 + 20 T_f \quad\Rightarrow\quad 4800 - 80T_f = 1600 + 20T_f
\]
\[
\Rightarrow\ 100T_f = 3200 \ \Rightarrow\ T_f = 32^\circ\mathrm{C}.
\]

Usando temperaturas absolutas \(T_K=T(^{\circ}{\rm C})+273\):
\begin{itemize}
\item Água: de \(333\ \mathrm{K}\) a \(305\ \mathrm{K}\)
\[
\Delta S_{\text{água}} = m_A c \ln\!\left(\frac{305}{333}\right) = 80\ln\!\left(\frac{305}{333}\right).
\]
\item Gelo (fusão a \(273\ \mathrm{K}\) + aquecimento de \(273\) a \(305\ \mathrm{K}\)):
\[
\Delta S_{\text{fusão}} = \frac{m_B L}{273} = \frac{20\cdot 80}{273} = \frac{1600}{273},
\]
\[
\Delta S_{\text{aquec}} = m_B c \ln\!\left(\frac{305}{273}\right)=20\ln\!\left(\frac{305}{273}\right).
\]
\end{itemize}

Somando todas as contribuições:
\[
\Delta S_{\text{total}} = \frac{1600}{273} + 20\ln\!\left(\frac{305}{273}\right) + 80\ln\!\left(\frac{305}{333}\right).
\]
Reescrevendo o último termo com sinal negativo:
\[
\Delta S_{\text{total}} = \frac{1600}{273} + 20\ln\!\left(\frac{305}{273}\right) - 80\ln\!\left(\frac{333}{305}\right).
\]

A resposta correta é alternativa \colorbox{green!50}{\textbf{(E)}}.

\end{flushleft}


\begin{flushleft}
\subsection{Quest\~ao 50 - ciclo termodinâmico: gás de fótons}
Processos termodinâmicos podem ser estendidos a partículas relativísticas, como um gás de fótons. Considere uma radiação 
eletromagnética inicialmente confinada em uma cavidade de volume \(V\), a qual está em equilíbrio térmico com as paredes 
da cavidade a uma temperatura \(T\). Essa radiação se comporta como um gás de fótons, cuja energia interna é dada por
\[
U=\frac{4\sigma}{c}\,V T^4,
\]
onde \(\sigma\) é a constante de Stefan-Boltzmann e \(c\) é a velocidade da luz no vácuo. A pressão \(P\) do gás de fótons 
é um terço da densidade volumétrica de energia \(u=U/V\), ou seja:
\[
P=\frac{4\sigma}{3c}\,T^4.
\]
Esse gás de fótons é utilizado como substância de trabalho em um ciclo termodinâmico \((A\!B\!C\!D\!A)\), composto por dois 
processos isobáricos e dois processos isocóricos, conforme ilustrado no diagrama \(P\) vs. \(V\) fornecido (retângulo com 
vértices \(A(V_0,P_0), B(V_0,2P_0), C(2V_0,2P_0), D(2V_0,P_0)\)). Calcule a eficiência \(\eta\) do ciclo para o gás de fótons.

\begin{itemize}
\item[(A)] \(\eta = 9{,}0\%\)
\item[(B)] \(\eta = 12{,}5\%\)
\item[(C)] \(\eta = 14{,}3\%\)
\item[(D)] \(\eta = 15{,}4\%\)
\item[(E)] \(\eta = 25\%\)
\end{itemize}

\vspace{0.5cm}

\textcolor{red}{\textbf{Solução:}}\\

Para o gás de fótons temos \(U=\dfrac{4\sigma}{c}VT^4\) e \(P=\dfrac{4\sigma}{3c}T^4\). Eliminando \(T^4\):
\[
T^4=\frac{3P}{a}\quad\text{com }a=\frac{4\sigma}{c},
\]
e
\[
U=aVT^4 = aV\frac{3P}{a}=3PV.
\]
Logo \(U=3PV\).

Calculemos \(U\) em cada vértice (em unidades de \(P_0V_0\)):
\[
U_A=3P_0V_0,\qquad U_B=3(2P_0)V_0=6P_0V_0,
\]
\[
U_C=3(2P_0)(2V_0)=12P_0V_0,\qquad U_D=3P_0(2V_0)=6P_0V_0.
\]

Trabalho líquido do ciclo \(W_{\text{líq}}\) é a área do retângulo:
\[
W_{\text{líq}}=(2P_0-P_0)(2V_0-V_0)=P_0V_0.
\]

Agora os calores absorvidos (processos com \(Q>0\)):

1. \(A\to B\) (isocórico): \(W_{AB}=0\), \(\Delta U_{AB}=U_B-U_A=6P_0V_0-3P_0V_0=3P_0V_0\). Assim \(Q_{AB}=3P_0V_0\).

2. \(B\to C\) (isobárico a \(P=2P_0\)): \(\Delta U_{BC}=U_C-U_B=12P_0V_0-6P_0V_0=6P_0V_0\). Trabalho \(W_{BC}=P\Delta V=2P_0(2V_0-V_0)=2P_0V_0\). Logo \(Q_{BC}=\Delta U_{BC}+W_{BC}=6P_0V_0+2P_0V_0=8P_0V_0\).

Os outros processos \(C\to D\) e \(D\to A\) liberam calor (\(Q<0\)), portanto o calor total absorvido é
\[
Q_{\text{in}}=Q_{AB}+Q_{BC}=3P_0V_0+8P_0V_0=11P_0V_0.
\]

Portanto a eficiência do ciclo é
\[
\eta=\frac{W_{\text{líq}}}{Q_{\text{in}}}=\frac{P_0V_0}{11P_0V_0}=\frac{1}{11}\approx 0{,}0909\equiv 9{,}09\%.
\]

A resposta correta é alternativa \colorbox{green!50}{\textbf{(A) \(\eta=9{,}0\%\)}}.

\end{flushleft}


\section{Eletromagnetismo}

\begin{flushleft}
\subsection{Questão 51 - Equações de Maxwell no Vácuo}
No artigo intitulado \textit{``A Dynamical Theory of the Electromagnetic Field''}, de 1865, James Clerk Maxwell formulou inicialmente 20 
equações para descrever os campos elétricos e magnéticos na natureza. Foram Oliver Heaviside e Heinrich Hertz que, duas décadas após a morte 
de Maxwell, as simplificaram em quatro, conhecidas hoje como: Lei de Gauss para eletricidade, Lei de Gauss para magnetismo, Lei de Faraday e Lei de 
Ampère-Maxwell. Essas equações relacionam os vetores campo elétrico e campo magnético e suas fontes, como cargas elétricas e correntes. Considerando 
as quatro equações de Maxwell, é possível demonstrar que campos elétricos \(\vec{E}(\vec{r},t)\) e magnéticos \(\vec{B}(\vec{r},t)\) dependentes do 
espaço e tempo, no espaço vazio, satisfazem equações de onda, cuja velocidade de propagação é dada por \(1/\sqrt{\mu_0\varepsilon_0}\), 
onde \(\varepsilon_0\) e \(\mu_0\) são a permissividade elétrica e a permeabilidade magnética do vácuo, respectivamente. 
Sobre esse conjunto de equações, assinale a alternativa \colorbox{red!20}{\textbf{INCORRETA}.}

\begin{itemize}
\item[(A)] A lei de Gauss para eletricidade estabelece que cargas elétricas estacionárias produzem um campo elétrico, e o fluxo desse campo, ao passar por qualquer superfície fechada, é proporcional à carga total contida nessa superfície.
\item[(B)] A lei de Gauss para o magnetismo estabelece que o fluxo magnético total que passa por qualquer superfície fechada é zero. Do ponto de vista experimental, esta equação descreve que as linhas de força do campo magnético não convergem nem divergem de nenhum ponto no espaço, o que implica diretamente na ausência de polos magnéticos isolados (monopolos magnéticos) na natureza.
\item[(C)] A lei da indução eletromagnética, descoberta por Michael Faraday em 1831, em uma série de experimentos, afirma que a integral de linha do campo elétrico em torno de uma curva fechada \(C\) é igual ao negativo da taxa de variação temporal do fluxo magnético através de qualquer superfície \(S\) limitada pela curva \(C\).
\item[(D)] A lei de Ampère-Maxwell afirma que a integral de linha do campo magnético em torno de qualquer curva fechada \(C\) é proporcional à soma da corrente elétrica de condução e da corrente de deslocamento através da superfície \(S\) limitada pela curva \(C\).
\item[(E)] As equações de onda do campo elétrico e magnético são invariantes por transformação de Galileu entre as coordenadas espaço-tempo de dois referenciais inerciais.
\end{itemize}

\vspace{0.5cm}

\textcolor{red}{\textbf{Solução:}}\\

A resposta correta é alternativa \colorbox{green!50}{\textbf{(E)}}.

\bigskip

\textbf{Explicação detalhada (alternativa a alternativa):}

\medskip

\textbf{(A) --- Correta.}\\
Forma integral da Lei de Gauss (eletricidade):
\[
\oint_{S} \mathbf{E}\cdot d\mathbf{A}=\frac{Q_{\text{enc}}}{\varepsilon_0},
\]
e forma diferencial:
\[
\nabla\cdot\mathbf{E}=\frac{\rho}{\varepsilon_0}.
\]
Isto significa exatamente que o fluxo do campo elétrico através de qualquer superfície fechada é proporcional à carga total contida no volume delimitado por essa superfície. A alternativa A descreve corretamente a lei.

\medskip

\textbf{(B) --- Correta.}\\
Lei de Gauss para o magnetismo (integral):
\[
\oint_{S}\mathbf{B}\cdot d\mathbf{A}=0,
\]
e diferencial:
\[
\nabla\cdot\mathbf{B}=0.
\]
Isso expressa que não há "fontes" ou "sumidouros" para \(\mathbf{B}\): as linhas de campo magnético são contínuas (laços fechados) e não existem monopolos magnéticos observados. A descrição da alternativa B está correta.

\medskip

\textbf{(C) --- Correta.}\\
Lei de Faraday (forma integral):
\[
\oint_{C}\mathbf{E}\cdot d\mathbf{l}=-\frac{d}{dt}\int_{S}\mathbf{B}\cdot d\mathbf{A},
\]
ou diferencial:
\[
\nabla\times\mathbf{E}=-\frac{\partial\mathbf{B}}{\partial t}.
\]
Isto corresponde exatamente ao enunciado: a força eletromotriz ao longo de uma curva fechada é o negativo da variação temporal do fluxo magnético pela superfície limitada. A alternativa C está correta.

\medskip

\textbf{(D) --- Correta.}\\
Lei de Ampère-Maxwell (integral):
\[
\oint_{C}\mathbf{B}\cdot d\mathbf{l}=\mu_0 I_{\text{enc}}+\mu_0\varepsilon_0\frac{d}{dt}\int_{S}\mathbf{E}\cdot d\mathbf{A},
\]
ou diferencial:
\[
\nabla\times\mathbf{B}=\mu_0\mathbf{J}+\mu_0\varepsilon_0\frac{\partial\mathbf{E}}{\partial t}.
\]
A inclusão do termo \(\mu_0\varepsilon_0\partial\mathbf{E}/\partial t\) (corrente de deslocamento) foi essencial para a consistência matemática e física das equações e para permitir equações de onda para \(\mathbf{E}\) e \(\mathbf{B}\). A alternativa D está correta.

\medskip

\textbf{(E) --- \underline{Incorreta} (explicação detalhada).}\\
As equações de Maxwell no vácuo levam às equações de onda para os campos elétrico e magnético, por exemplo (forma escalar 1D para ilustrar):
\[
\frac{\partial^2 f}{\partial x^2}-\frac{1}{c^2}\frac{\partial^2 f}{\partial t^2}=0,
\qquad c=\frac{1}{\sqrt{\mu_0\varepsilon_0}}.
\]
Essas equações de onda \emph{não} são invariantes sob transformações de Galileu. A transformação galileana entre dois referenciais que se movem com velocidade \(v\) (no eixo \(x\)) é
\[
x'=x-vt,\qquad t'=t.
\]
Aplicando a regra da cadeia:
\[
\frac{\partial}{\partial x}=\frac{\partial}{\partial x'},\qquad
\frac{\partial}{\partial t}=\frac{\partial}{\partial t'}-v\frac{\partial}{\partial x'}.
\]
Logo
\[
\frac{\partial^2}{\partial t^2}
=\frac{\partial^2}{\partial t'^2}-2v\frac{\partial^2}{\partial t'\partial x'}+v^2\frac{\partial^2}{\partial x'^2}.
\]
Substituindo na equação de onda obtemos termos mistos \(\partial^2/(\partial t'\partial x')\) e um coeficiente diferente no termo \(\partial^2/\partial x'^2\); o operador da onda não preserva sua forma original salvo nos casos triviais \(v=0\) ou \(c\to\infty\). Escrevendo:
\[
\frac{\partial^2}{\partial x'^2}-\frac{1}{c^2}\left(\frac{\partial^2}{\partial t'^2}-2v\frac{\partial^2}{\partial t'\partial x'}+v^2\frac{\partial^2}{\partial x'^2}\right)
\neq
\frac{\partial^2}{\partial x'^2}-\frac{1}{c^2}\frac{\partial^2}{\partial t'^2}.
\]
Portanto a equação de onda muda de forma sob a transformação galileana — não é invariante.

Fisicamente, isso reflete que Maxwell prediz uma velocidade de propagação \(c\) da luz que é a mesma em todos os referenciais inerciais — este fato é incompatível com a adição simples de velocidades postulada pela transformação de Galileu. A resolução histórica dessa contradição levou às transformações de Lorentz e à teoria da relatividade restrita de Einstein: as equações de Maxwell são invariantes sob transformações de Lorentz, não sob Galileu. Assim a alternativa E está errada.

\bigskip

\textbf{Conclusão:} a única alternativa \emph{incorreta} é a \(\boxed{\text{E}}\) porque as equações de onda (e, em geral, as equações de Maxwell) não são invariantes sob transformações galileanas — elas exigem invariância de Lorentz.
\end{flushleft}

\begin{flushleft}
\subsection{Quest\~ao 52 - Lei de Gauss para Eletricidade}

Uma carga elétrica pontual \(q>0\) se encontra no centro geométrico de uma superfície gaussiana cilíndrica de altura 
\(2L\) e raio \(a\). Sabendo que \(\varepsilon_0\) é a permissividade elétrica do meio em todo o espaço e que as grandezas 
são expressas no sistema internacional de unidades, analise as assertivas abaixo:

\begin{figure}[!h]
  \centering
  \includegraphics[scale=0.6]{figures/lei_de_gauss.png}
\end{figure}

I. O fluxo total do vetor campo elétrico na superfície gaussiana cilíndrica é dado por \(\dfrac{q}{2\pi\varepsilon_0RL}\).

II. Os fluxos elétricos do vetor campo elétrico através de cada uma das superfícies circulares superior (\(\phi_1\)) e inferior (\(\phi_2\)) do cilindro são dados por:
\[
\phi_1=\phi_2=\frac{q}{2\varepsilon_0}\left(1-\frac{L}{\sqrt{L^2+a^2}}\right).
\]

III. O fluxo elétrico \(\phi_3\) através da superfície lateral do cilindro é dado por:
\[
\phi_3=\frac{q}{\varepsilon_0}\,\frac{a}{\sqrt{L^2+a^2}}.
\]

Quais estão corretas?

\begin{itemize}
\item[(A)] Apenas I.
\item[(B)] Apenas II.
\item[(C)] Apenas III.
\item[(D)] Apenas II e III.
\item[(E)] I, II e III.
\end{itemize}

\vspace{0.5cm}

\textcolor{red}{\textbf{Solução:}}\\

Analisemos cada item.

\textbf{Total de fluxo (verificação de I).} Por Teorema de Gauss, o fluxo elétrico total através de uma superfície fechada que envolve a carga pontual \(q\) é
\[
\Phi_{\text{total}}=\frac{q}{\varepsilon_0}.
\]
A expressão dada em I, \(\dfrac{q}{2\pi\varepsilon_0RL}\), não tem dimensão nem forma compatível com \(q/\varepsilon_0\). Portanto \textbf{I é falsa}.

\textbf{Fluxo através de uma tampa circular (verificação de II).} Considere a tampa circular superior: a carga está no eixo, à distância \(L\) do plano da tampa, e a tampa tem raio \(a\). A semi-ângulo sólido \(\theta\) que a tampa subtende satisfa\-z \(\cos\theta=\dfrac{L}{\sqrt{L^2+a^2}}\). O ângulo sólido \(\Omega\) subtendido pela tampa (vista da carga) é
\[
\Omega = 2\pi(1-\cos\theta)=2\pi\left(1-\frac{L}{\sqrt{L^2+a^2}}\right).
\]
O fluxo através da tampa é a fração \(\Omega/(4\pi)\) do fluxo total \(q/\varepsilon_0\):
\[
\phi_1=\frac{q}{\varepsilon_0}\cdot\frac{\Omega}{4\pi}
=\frac{q}{\varepsilon_0}\cdot\frac{2\pi\left(1-\dfrac{L}{\sqrt{L^2+a^2}}\right)}{4\pi}
=\frac{q}{2\varepsilon_0}\left(1-\frac{L}{\sqrt{L^2+a^2}}\right).
\]
Como a tampa inferior é simétrica, \(\phi_2=\phi_1\). Logo a expressão em II está correta; \textbf{II é verdadeira}.

\[
\boxed{
\phi_1= \phi_2=\frac{q}{2\varepsilon_0}\left(1-\frac{L}{\sqrt{L^2+a^2}}\right).
}
\]

\textbf{Fluxo através da superfície lateral (verificação de III).} O fluxo lateral \(\phi_3\) é o fluxo total menos o fluxo pelas duas tampas:
\[
\phi_3=\Phi_{\text{total}}-2\phi_1
=\frac{q}{\varepsilon_0}-2\cdot\frac{q}{2\varepsilon_0}\left(1-\frac{L}{\sqrt{L^2+a^2}}\right)
=\frac{q}{\varepsilon_0}\cdot\frac{L}{\sqrt{L^2+a^2}}.
\]

\[
\boxed{\phi_3=\frac{q}{\varepsilon_0}\,\frac{L}{\sqrt{L^2+a^2}}}
\]
Portanto o valor correto é \(\displaystyle \phi_3=\frac{q}{\varepsilon_0}\,\frac{L}{\sqrt{L^2+a^2}}\), não o dado em III (que tem \(a\) no numerador). Assim \textbf{III é falsa}.

\vspace{0.3cm}
Conclusão: apenas a assertiva II está correta.

A resposta correta é alternativa \colorbox{green!50}{\textbf{(B) Apenas II}}.

\end{flushleft}

\begin{flushleft}
\subsection{Questão 53 - Lei de Ampère}
Uma espira retangular com lados $a$ e $b$ encontra-se situada no mesmo plano, a uma distância $D$ de um fio retilíneo muito longo, 
conforme a figura. Sabe-se que no intervalo de tempo $0 \le t \le T$ o fio conduz uma corrente elétrica que varia linearmente 
com o tempo de acordo com a função 
\[
I(t) = I_0 \left(1 - \frac{t}{T}\right),
\] 
onde $I_0$ representa a intensidade da corrente no instante de $t=0$. A constante de permeabilidade magnética do meio é $\mu_0$, 
e o fio e a espira retangular estão fixos em suas posições. Desconsiderando os efeitos de autoindutância, determine o módulo da 
força eletromotriz induzida $\mathcal{E}(t)$ na espira retangular para o intervalo de tempo $0 \le t \le T$.

\begin{figure}[!h]
  \centering
  \includegraphics[scale=0.6]{figures/lei_de_ampere.png}
\end{figure}

\begin{itemize}
\item[(A)] $|\mathcal{E}(t)| = \frac{\mu_0 a b I_0}{2 \pi D T}$
\item[(B)] $|\mathcal{E}(t)| = \frac{\mu_0 b I_0}{2 \pi T} \ln \left( 1 + \frac{a}{D} \right)$
\item[(C)] $|\mathcal{E}(t)| = \frac{\mu_0 b I_0}{2 \pi T} \left( \frac{1}{a} - \frac{1}{a+D} \right)$
\item[(D)] $|\mathcal{E}(t)| = \frac{\mu_0 b I_0}{2 \pi} \left( 1 - \frac{t}{T} \right) \ln \left( 1 + \frac{a}{D} \right)$
\item[(E)] $|\mathcal{E}(t)| = \frac{\mu_0 b I_0}{2 \pi} \left( 1 - \frac{t}{T} \right) \left( \frac{1}{a} - \frac{1}{a+D} \right)$
\end{itemize}


\subsection*{Resolução}

O fluxo magnético $\Phi_B$ através da espira devido ao fio retilíneo é dado por:
\[
\Phi_B(t) = \int_{\text{espira}} \vec{B} \cdot d\vec{S}.
\]

O campo magnético de um fio retilíneo infinito é:
\[
B(r,t) = \frac{\mu_0 I(t)}{2\pi r},
\]
onde $r$ é a distância perpendicular ao fio.

Para uma espira retangular, integramos ao longo da direção $a$ (altura da espira):
\[
\Phi_B(t) = \int_{D}^{D+a} B(r,t) \, b \, dr = b \int_{D}^{D+a} \frac{\mu_0 I(t)}{2\pi r} dr = \frac{\mu_0 b I(t)}{2\pi} \int_{D}^{D+a} \frac{dr}{r}.
\]

O resultado da integral:
\[
\Phi_B(t) = \frac{\mu_0 b I(t)}{2\pi} \ln\left(\frac{D+a}{D}\right) = \frac{\mu_0 b I(t)}{2\pi} \ln\left(1 + \frac{a}{D}\right).
\]

A força eletromotriz induzida na espira é dada pela lei de Faraday:
\[
\mathcal{E}(t) = \left| - \frac{d\Phi_B(t)}{dt} \right| = \left| - \frac{\mu_0 b}{2\pi} \ln\left(1 + \frac{a}{D}\right) \frac{d I(t)}{dt} \right|.
\]

Como $I(t) = I_0 \left(1 - \frac{t}{T}\right)$, temos:
\[
\frac{d I(t)}{dt} = - \frac{I_0}{T}.
\]

Substituindo na expressão da FEM:
\[
\mathcal{E}(t) = \frac{\mu_0 b I_0}{2\pi T} \ln\left(1 + \frac{a}{D}\right).
\]

\subsection*{Conclusão}

O módulo da força eletromotriz induzida na espira retangular é constante e dado por:
\[
\boxed{\mathcal{E}(t) = \frac{\mu_0 b I_0}{2\pi T} \ln\left(1 + \frac{a}{D}\right)}.
\]

Portanto, a alternativa correta é \colorbox{green!50}{\textbf{B}}.

\end{flushleft}

\section{\'Optica Geométrica}
\begin{flushleft}
\subsection{Quest\~ao 54 - Lente Convergente}

Uma lente delgada convergente de distância focal $f$ é colocada entre uma fonte luminosa e um anteparo, 
que estão fixos e separados por uma distância $L$, com $L \geq 4f$. Qual é a expressão para a distância entre 
as duas posições da lente que formarão imagens reais nítidas da fonte sobre o anteparo?

\begin{figure}[!h]
  \centering
  \includegraphics[scale=0.6]{figures/optica_geometrica_lente.png}
\end{figure}

\begin{itemize}
\item[(A)] $L/2$
\item[(B)] $\sqrt{L^{2}+2fL}$
\item[(C)] $\sqrt{L^{2}+4fL}$
\item[(D)] $\sqrt{L^{2}-2fL}$
\item[(E)] $\sqrt{L^{2}-4fL}$
\end{itemize}

\vspace{0.5cm}

\textcolor{red}{\textbf{Solução:}}\\

Seja $p$ a distância objeto-lente (da fonte até a lente) e $q$ a distância lente–imagem (da lente até o anteparo). Como fonte e anteparo estão separados por $L$ e a lente fica entre eles, temos
\[
p+q=L.
\]
A equação da lente delgada é
\[
\frac{1}{f}=\frac{1}{p}+\frac{1}{q}.
\]
Substituindo $q=L-p$:
\[
\frac{1}{f}=\frac{1}{p}+\frac{1}{L-p}=\frac{L}{p(L-p)}.
\]
Multiplicando ambos os lados por $p(L-p)$:
\[
p(L-p)=fL.
\]
Isso resulta na equação quadrática
\[
p^{2}-Lp+fL=0.
\]
As raízes desta equação (as duas posições possíveis da lente que formam imagem nítida no anteparo) são
\[
p_{1,2}=\frac{L\pm\sqrt{L^{2}-4fL}}{2}.
\]
A distância entre as duas posições da lente é
\[
\Delta p=|p_1-p_2|=\sqrt{L^{2}-4fL}.
\]
Portanto a alternativa correta é a \colorbox{green!50}{\textbf{(E)}}.

\end{flushleft}

\section{Interfer\^encia e Difra\c{c}\~ao}

\begin{flushleft}
\subsection{Quest\~ao 55 - Fibras \'Opticas}
\noindent

O uso das fibras óticas é uma das principais aplicações tecnológicas da óptica na área
de comunicações. Fibras cilíndricas finas de vidro ou plástico podem ser usadas para transmitir sinais
ao invés de fios metálicos. A grande vantagem está na largura de banda disponível quando o portador
são ondas eletromagnéticas. Isso permite que uma fibra carregue muito mais sinais independentes
diferentes do que um fio condutor. Existem também vantagens em termos de peso e independência
de recursos limitados. Considere uma fibra óptica composta por um núcleo de vidro fino com índice
de refração \(n_f\), cercado por uma camada de revestimento de menor densidade com índice de refração
\(n_r < n_f\). O índice de refração do ar é denotado por \(n_0\). Assinale a alternativa que expressa
corretamente o ângulo de incidência máximo (\(\theta_m\)) que qualquer raio que incida na face de entrada
seja guiado dentro da fibra por reflexões internas totais sucessivas.

\begin{figure}[!h]
  \centering
  \includegraphics[scale=0.6]{figures/reflexao_total.png}
\end{figure}

\begin{itemize}
\item[(A)] \(\displaystyle \theta_m=\arcsin\!\left(\frac{1}{n_0}\sqrt{n_f^2-n_r^2}\right)\)
\item[(B)] \(\displaystyle \theta_m=\arcsin\!\left(\frac{n_f^2}{n_0^2}-\frac{n_f^4}{n_0^2 n_r^2}\right)\)
\item[(C)] \(\displaystyle \theta_m=\arccos\!\left(\frac{1}{n_0}\sqrt{n_f^2-n_r^2}\right)\)
\item[(D)] \(\displaystyle \theta_m=\arcsin\!\left(\frac{n_f}{n_r}\right)\)
\item[(E)] \(\displaystyle \theta_m=\arccos\!\left(\frac{n_r}{n_f}\right)\)
\end{itemize}

\vspace{0.5cm}

\textcolor{red}{\textbf{Solução:}}\\

Para guiar por reflexões internas totais, o raio dentro do núcleo deve atingir a interface núcleo-revestimento com um ângulo de incidência 
igual ao mínimo permitido (ângulo crítico) \(\theta_c\) com \(\sin\theta_c=\dfrac{n_r}{n_f}\). Em termos do ângulo \(\theta'\) que o raio faz 
com o eixo da fibra, a condição limite é \(\theta'_{\max}=90^\circ-\theta_c\), portanto
\[
\sin\theta'_{\max}=\cos\theta_c=\sqrt{1-\sin^2\theta_c}=\sqrt{1-\left(\frac{n_r}{n_f}\right)^2}=\frac{\sqrt{n_f^2-n_r^2}}{n_f}.
\]
Na face de entrada, pela lei de Snell \(n_0\sin\theta_m=n_f\sin\theta'\). Substituindo \(\sin\theta'=\sin\theta'_{\max}\) obtém-se
\[
\sin\theta_m=\frac{n_f}{n_0}\cdot\frac{\sqrt{n_f^2-n_r^2}}{n_f}=\frac{1}{n_0}\sqrt{n_f^2-n_r^2},
\]
logo
\[
\theta_m=\arcsin\!\left(\frac{1}{n_0}\sqrt{n_f^2-n_r^2}\right).
\]

A resposta correta é alternativa \colorbox{green!50}{\textbf{(A)}}.

\end{flushleft}

\begin{flushleft}
\subsection{Quest\~ao 57 - Interferência da Luz} 

No trabalho intitulado “Uso do espelho de Lloyd como método de ensino de óptica no Ensino Médio” 
(Revista Brasileira de Ensino de Física, 2012), os autores propõem uma abordagem para o ensino de óptica, 
com ênfase na aplicação de conceitos básicos sobre o fenômeno da interferência da luz. Foi utilizado o experimento denominado 
“espelho de Lloyd”, que oferece uma ligação mais simples entre a óptica geométrica e a óptica física, sendo muito mais acessível 
do que a experiência das duas fendas de Young, segundo os autores. No procedimento, um padrão de franjas de interferência é observado usando-se 
luz emitida por uma fonte coerente, como um laser, e a luz refletida por uma placa de vidro (figura a). O padrão de interferência é projetado em 
um anteparo. As distâncias entre a fonte luminosa em relação ao plano da placa de vidro e o anteparo são, respectivamente, \(d = 1,0\ \mathrm{mm}\) 
e \(D = 2,0\ \mathrm{m}\) (figura b). Durante a realização do experimento, foi registrado um padrão de interferência (figura c) com 14 franjas por centímetro. 

Qual é o valor do comprimento de onda \(\lambda\) da luz que foi medido, levando em consideração as informações fornecidas?

\begin{figure}[!h]
  \centering
  \includegraphics[scale=0.6]{figures/espelho_LIoyd.png}
\end{figure}

\begin{itemize}
\item[(A)] 660 nm.
\item[(B)] 680 nm.
\item[(C)] 710 nm.
\item[(D)] 750 nm.
\item[(E)] 780 nm.
\end{itemize}

\vspace{0.5cm}

\textcolor{red}{\textbf{Solução:}}\\

No arranjo de Lloyd, a franja de interferência observada no anteparo é similar à obtida por duas fontes separadas por \(2d\) (a fonte real e a imagem especular). Para pequenas inclinações (parâmetro de aproximação de Gauss) a separação entre franjas sucessivas (franja brilhante para a próxima franja brilhante) é dada por
\[
\beta=\frac{\lambda\,D}{2d}.
\]

Da observação temos \(14\) franjas por centímetro, portanto a distância entre franjas é
\[
\beta=\frac{1\ \mathrm{cm}}{14}=\frac{0{,}01\ \mathrm{m}}{14}=7{,}142857\times 10^{-4}\ \mathrm{m}.
\]

Isolando \(\lambda\) na expressão para \(\beta\):
\[
\lambda=\frac{2d\,\beta}{D}.
\]

Substituindo os valores numéricos (\(d=1{,}0\ \mathrm{mm}=1{,}0\times10^{-3}\ \mathrm{m}\), \(D=2{,}0\ \mathrm{m}\)):
\[
\lambda=\frac{2\,(1{,}0\times10^{-3})\,(7{,}142857\times10^{-4})}{2{,}0}
=\;1{,}0\times10^{-3}\times 7{,}142857\times10^{-4}
=7{,}142857\times10^{-7}\ \mathrm{m}.
\]

Em unidades nanométricas:
\[
\lambda\approx 7{,}14\times10^{-7}\ \mathrm{m}=714\ \mathrm{nm}.
\]

Comparando com as alternativas apresentadas, o valor mais próximo é \(\mathbf{710\ \mathrm{nm}}\) (alternativa \textbf{(C)}).\\

A resposta correta é alternativa \colorbox{green!50}{\textbf{(C) 710 nm}}.
\end{flushleft}

\section{Relatividade}

\begin{flushleft}
\subsection{Quest\~ao 56 - Transformação de Lorentz}

Seja \(R\) um referencial inercial e \(R'\) um referencial inercial que se move em relação a \(R\), com velocidade constante \(\vec{v} = \beta c \hat{x}\), 
na qual \(c\) é a velocidade da luz no vácuo e \(|\beta| < 1\) é um parâmetro adimensional. Os eixos \(x, y, z\) de \(R\) são paralelos aos 
eixos \(x', y', z'\) de \(R'\), e as coordenadas espaço-tempo estão relacionadas entre si através da transformação de Lorentz. Sabe-se também que 
as origens \(O\) e \(O'\) dos referenciais \(R\) e \(R'\) são coincidentes nos instantes \(t = t' = 0\). Considere as funções de onda 
\(\Psi_\pm(x,t) = \Psi_0 \exp[ik\phi_\pm(x,t)]\), onde \(k\) é o vetor de onda e os comprimentos \(\phi_\pm\) são \(\phi_+(x,t) = x + ct\) e 
\(\phi_-(x,t) = x - ct\). É correto afirmar que no referencial \(R'\) as grandezas \(\phi'_+\) e \(\phi'_-\) estão relacionadas através da seguinte 
forma, respectivamente:

\begin{itemize}
\item[(A)] \(\phi'_+ = \phi_+(1+\beta)\) \quad e \quad \(\phi'_- = \phi_-(1-\beta)\)
\item[(B)] \(\phi'_+ = \phi_+(1-\beta)\) \quad e \quad \(\phi'_- = \phi_-(1+\beta)\)
\item[(C)] \(\phi'_+ = \phi_+\sqrt{\dfrac{1-\beta}{1+\beta}}\) \quad e \quad \(\phi'_- = \phi_-\sqrt{\dfrac{1+\beta}{1-\beta}}\)
\item[(D)] \(\phi'_+ = \phi_+\sqrt{\dfrac{1+\beta}{1-\beta}}\) \quad e \quad \(\phi'_- = \phi_-\sqrt{\dfrac{1-\beta}{1+\beta}}\)
\item[(E)] \(\phi'_+ = \dfrac{1}{\sqrt{1-\beta^2}}(\phi_+ - \beta\phi_-)\) \quad e \quad \(\phi'_- = \dfrac{1}{\sqrt{1-\beta^2}}(\phi_- + \beta\phi_+)\)
\end{itemize}

\vspace{0.5cm}

\textcolor{red}{\textbf{Solução:}}\\

Usando as transformações de Lorentz,
\[
\begin{aligned}
\phi'_+ &= x' + c t' = \gamma(x-\beta ct) + c\gamma\left(t-\beta\frac{x}{c}\right) \\
&= \gamma\big[(1-\beta)x + (1-\beta)ct\big] = \gamma(1-\beta)\,(x+ct)=\gamma(1-\beta)\,\phi_+.
\end{aligned}
\]
Analogamente,
\[
\begin{aligned}
\phi'_- &= x' - c t' = \gamma(x-\beta ct) - c\gamma\left(t-\beta\frac{x}{c}\right) \\
&= \gamma\big[(1+\beta)x - (1+\beta)ct\big] = \gamma(1+\beta)\,(x-ct)=\gamma(1+\beta)\,\phi_-.
\end{aligned}
\]
Como \(\gamma(1-\beta)=\dfrac{1-\beta}{\sqrt{1-\beta^2}}=\sqrt{\dfrac{1-\beta}{1+\beta}}\) e \(\gamma(1+\beta)=\sqrt{\dfrac{1+\beta}{1-\beta}}\), obtemos exatamente a alternativa (C).

A resposta correta é alternativa \colorbox{green!50}{\textbf{(C)}}.

\end{flushleft}

\begin{flushleft}
\subsection{Quest\~ao 58 - Energia-Momento Relativística} 

Um experimento simplificado para verificar a relação energia-momento relativística de partículas beta emitidas por uma fonte radioativa foi 
proposto recentemente por D. Jackson \textit{et al.} no \textit{American Journal of Physics}, 92, 775 (2024). No aparato experimental, as 
partículas beta, de massa de repouso \(m_0\) e carga elétrica \(q\), são emitidas por uma fonte de \({}^{204}\mathrm{Tl}\) (tálio-204) e, 
em seguida, passam por um colimador de aço que direciona suas trajetórias. Após saírem do colimador, as partículas beta seguem trajetórias 
(aproximadamente) circulares devido a um campo magnético uniforme de intensidade \(B\). Um colimador de alumínio conduz as partículas para 
um detector Geiger-Müller, que registra a chegada das partículas, permitindo a medição do raio \(R\) de suas trajetórias. Seja \(K\) a energia 
cinética de uma partícula beta incidente na região do campo magnético e, considerando efeitos relativísticos, determine o raio \(R\) dessas trajetórias
\'e dado pela expressão:

\begin{figure}[!h]
  \centering
  \includegraphics[scale=0.6]{figures/energia-momento.png}
\end{figure}


\begin{itemize}
\item[(A)] \(R=\sqrt{\dfrac{2m_0K}{(q^2B^2)}}\)
\item[(B)] \(R=\sqrt{\dfrac{m_0K}{(2q^2B^2)}}\)
\item[(C)] \(R=\sqrt{\dfrac{K(2m_0c^2-K)}{(qBc)^2}}\)
\item[(D)] \(R=\sqrt{\dfrac{K(2m_0c^2+K)}{(qBc)^2}}\)
\item[(E)] \(R=\sqrt{\dfrac{2K(m_0c^2+K)}{(qBc)^2}}\)
\end{itemize}

\vspace{0.5cm}

\textcolor{red}{\textbf{Solução:}}\\

A força magnética fornece a força centrípeta responsável pela trajetória circular:
\[
qvB=\frac{p\,v}{R}\quad\Rightarrow\quad \boxed{R=\frac{p}{qB}},
\]
onde \(p\) é o momento relativístico da partícula.

Relatividade special: a energia total é
\[
E=\gamma m_0 c^2 = K + m_0 c^2,
\]
e a relação energia-momento é
\[
E^2=(pc)^2+(m_0 c^2)^2.
\]
Isolando \(p\) obtemos
\[
(pc)^2 = E^2-(m_0 c^2)^2 = (K+m_0 c^2)^2-(m_0 c^2)^2
=K^2+2K m_0 c^2.
\]
Portanto
\[
p=\frac{\sqrt{K^2+2K m_0 c^2}}{c}=\frac{\sqrt{K\,(K+2m_0 c^2)}}{c}.
\]

Substituindo em \(R=p/(qB)\):
\[
R=\frac{1}{qB}\,\frac{\sqrt{K^2+2K m_0 c^2}}{c}
=\frac{\sqrt{K\,(K+2m_0 c^2)}}{qB c}.
\]

Escrevendo em forma equivalente (colocando o denominador \((qBc)^2\) sob o radical):
\[
\boxed{R=\sqrt{\dfrac{K\,(2m_0 c^2+K)}{(qBc)^2}}}.
\]

Assim, a alternativa que coincide com a expressão correta é a \(\mathbf{(D)}\).\\

A resposta correta é alternativa \colorbox{green!50}{\textbf{(D)}}.
\end{flushleft}

\section{Mecânica quântica em três dimensões e átomo de Hidrogênio}

\begin{flushleft}
\subsection{Quest\~ao 59 - Modelo de Bohr para positrônio} 

Recentemente, uma equipe de pesquisadores do CERN alcançou um avanço importante ao aplicar uma técnica inovadora de resfriamento a laser 
em uma amostra de positrônio, uma partícula composta por um elétron (carga elétrica \(-e\) e massa \(m\)) e um pósitron (antipartícula com 
carga positiva e massa semelhante à do elétron). Essa conquista permite uma investigação mais
precisa das propriedades do positrônio, abrindo possibilidades para estudos fundamentais, como a
produção de condensados Bose-Einstein de antimatéria e a geração de luz gama coerente, o que
poderia trazer um leque de novas aplicações.

\begin{figure}[!h]
  \centering
  \includegraphics[scale=0.6]{figures/positronio.png}
\end{figure}

Considerando o modelo de Bohr para átomos hidrogenoides, é correto afirmar que os níveis de energia de um positrônio, em função do número 
quântico principal \(n\), são dados por:

\begin{itemize}
\item[(A)] \(E_n=-3{,}4/n^2\ \mathrm{eV}\)
\item[(B)] \(E_n=-6{,}8/n^2\ \mathrm{eV}\)
\item[(C)] \(E_n=-13{,}6/n^2\ \mathrm{eV}\)
\item[(D)] \(E_n=-27{,}2/n^2\ \mathrm{eV}\)
\item[(E)] \(E_n=-25{,}0/n^2\ \mathrm{MeV}\)
\end{itemize}

\vspace{0.5cm}

\textcolor{red}{\textbf{Solução:}}\\

No modelo de Bohr para um sistema de duas partículas (átomo hidrogenoide) as energias estão dadas por
\[
E_n=-\frac{\mu e^4}{2(4\pi\varepsilon_0)^2\hbar^2}\frac{1}{n^2},
\]
onde \(\mu\) é a massa reduzida do sistema. Para o hidrogênio (núcleo com massa muito maior que \(m\)) tem-se \(\mu\approx m_e\) e portanto
\[
E_n^{\text{H}}=-\frac{m_e e^4}{2(4\pi\varepsilon_0)^2\hbar^2}\frac{1}{n^2}=-\frac{13{,}6}{n^2}\ \mathrm{eV}.
\]

No caso do positrônio, o sistema é formado por um elétron e um pósitron com a mesma massa \(m_e\), então a \colorbox{yellow}{massa reduzida} é
\[
\mu=\frac{m_e\cdot m_e}{m_e+m_e}=\frac{m_e}{2}.
\]
Substituindo na expressão para \(E_n\) obtemos
\[
E_n^{\text{Ps}}=-\frac{(m_e/2)e^4}{2(4\pi\varepsilon_0)^2\hbar^2}\frac{1}{n^2}
=-\frac{1}{2}\left(\frac{m_e e^4}{2(4\pi\varepsilon_0)^2\hbar^2}\right)\frac{1}{n^2}
=-\frac{1}{2}\cdot\frac{13{,}6}{n^2}\ \mathrm{eV}.
\]
Logo
\[
\boxed{
E_n^{\text{Ps}}=-\frac{6{,}8}{n^2}\ \mathrm{eV}}.
\]

Portanto, a alternativa correta é a \(\mathbf{(B)}\).\\

A resposta correta é alternativa \colorbox{green!50}{\textbf{(B)  \(\,E_n=-6{,}8/n^2\ \mathrm{eV}\)}}.
\end{flushleft}

\begin{flushleft}
\subsection{Quest\~ao 60 - \'Atomo de Hidrogênio} 

O estado de um elétron em um átomo de hidrogênio, na representação posição 
\(\vec{r} = x \hat{x} + y \hat{y} + z \hat{z}\), é descrito pela função de onda normalizada a seguir:

\[
\psi(\vec{r}) = \frac{1}{\sqrt{32 \pi a_0^5}} (\alpha x + \beta y + \gamma z)\exp\left(-\frac{r}{2a_0}\right),
\]
onde \(a_0\) é o raio de Bohr, \(r=\sqrt{x^2+y^2+z^2}\) e \(\alpha,\beta,\gamma\) são números reais que satisfazem \(\alpha^2+\beta^2+\gamma^2=1\). O estado \(\psi(\vec{r})\) é uma superposição das autofunções \(\psi_{nlm}(r,\theta,\phi)\) do átomo de hidrogênio, para \(n=2\).  

A tabela apresenta as autofunções normalizadas do átomo de hidrogênio em coordenadas esféricas \((r,\theta,\phi)\) para os orbitais com \(n=2\):

\[
\begin{array}{|c|c|}
\hline
\text{Estado } (n,l,m) & \text{Função de onda} \\ \hline
(2,0,0) & \psi_{200}(r,\theta,\phi) = \frac{1}{\sqrt{8\pi a_0^3}}\left(1-\frac{r}{2a_0}\right)e^{-r/2a_0} \\ \hline
(2,1,0) & \psi_{210}(r,\theta,\phi) = \frac{1}{\sqrt{8\pi a_0^3}} \frac{r}{2a_0} e^{-r/2a_0}\cos\theta \\ \hline
(2,1,\pm 1) & \psi_{21\pm1}(r,\theta,\phi) = \pm \frac{1}{\sqrt{16\pi a_0^3}}\left(\frac{r}{2a_0}\right)e^{-r/2a_0}\sin\theta e^{\pm i\phi} \\ \hline
\end{array}
\]

Seja \(\hbar=h/2\pi\). Assinale a alternativa correta que representa a probabilidade de uma medida de \(\hat{L}_z\) resultar \(+\hbar\).

\begin{itemize}
\item[(A)] \(P(L_z=+\hbar)=\tfrac{1}{2}(\alpha^2+\beta^2)\)
\item[(B)] \(P(L_z=+\hbar)=\alpha^2+\beta^2\)
\item[(C)] \(P(L_z=+\hbar)=\tfrac{1}{2}(\alpha^2-\beta^2)\)
\item[(D)] \(P(L_z=+\hbar)=\alpha^2-\beta^2\)
\item[(E)] \(P(L_z=+\hbar)=\gamma^2\)
\end{itemize}

\vspace{0.5cm}

\textcolor{red}{\textbf{Solução:}}\\

A função de onda dada pode ser reescrita em coordenadas esféricas. Usando:
\[
x = r \sin\theta \cos\phi, \quad y = r \sin\theta \sin\phi, \quad z = r \cos\theta,
\]
obtemos
\[
\psi(r,\theta,\phi) = \frac{r}{\sqrt{32 \pi a_0^5}} \exp\left(-\frac{r}{2a_0}\right)\left[\alpha \sin\theta\cos\phi + \beta \sin\theta\sin\phi + \gamma \cos\theta\right].
\]

Comparando com as autofunções da tabela, vemos que:
\[
\psi_{210}(r,\theta,\phi)\propto \frac{r}{a_0}e^{-r/2a_0}\cos\theta,
\]
\[
\psi_{21\pm1}(r,\theta,\phi)\propto \frac{r}{a_0}e^{-r/2a_0}\sin\theta e^{\pm i\phi}.
\]

Podemos escrever:
\[
\alpha \sin\theta \cos\phi + \beta \sin\theta \sin\phi = 
\frac{\alpha - i\beta}{2} \sin\theta e^{i\phi} + \frac{\alpha+i\beta}{2}\sin\theta e^{-i\phi}.
\]

Assim, o estado é uma combinação linear:
\[
\psi \sim \gamma \psi_{210} + \frac{\alpha - i\beta}{\sqrt{2}} \psi_{21,1} + \frac{\alpha+i\beta}{\sqrt{2}} \psi_{21,-1}.
\]

Portanto, o coeficiente da autofunção \(\psi_{21,1}\) (que corresponde a \(m=+1\), logo \(L_z=+\hbar\)) é
\[
c_{+1} = \frac{\alpha - i\beta}{\sqrt{2}}.
\]

A probabilidade é
\[
P(L_z=+\hbar) = |c_{+1}|^2 = \frac{1}{2}(\alpha^2+\beta^2).
\]

Logo, a alternativa correta é a \(\mathbf{(A)}\).\\

A resposta correta é alternativa \colorbox{green!50}{\textbf{(A) \(\, P(L_z=+\hbar)=\tfrac{1}{2}(\alpha^2+\beta^2)\)}}.
\end{flushleft}



%%%%%%%%%%%%%%%%%%%%%%%%%%%%%%%%%%%%%%%%%%%%%%%%%%%%%%%%%%%%%%%%%%
\newpage 

\begin{flushleft}
\textbf{Problema.} Um pêndulo de massa $m_2$ e comprimento $L$ é solto do repouso na posição $A$, que faz um ângulo $\theta$ com a vertical. A corda passa por uma roldana ideal e traciona um bloco de massa $m_1$ sobre uma mesa horizontal. Ao o pêndulo atingir o ponto mais baixo $B$, qual deve ser o menor coeficiente de atrito estático $\mu_s$ entre $m_1$ e a mesa para que $m_1$ não deslize?

\vspace{0.4cm}
\textbf{Solução.}

\textit{1) Velocidade do pêndulo em $B$.} Pela conservação de energia entre $A$ e $B$:
\[
m_2 g L\,(1-\cos\theta)=\frac12 m_2 v_B^2
\;\;\Rightarrow\;\;
v_B^2=2gL\,(1-\cos\theta).
\]

\textit{2) Tração na corda em $B$.} No ponto mais baixo, as forças radiais no pêndulo dão
\[
T_B - m_2 g = m_2\frac{v_B^2}{L}
\;\;\Rightarrow\;\;
T_B = m_2\!\left(g+\frac{v_B^2}{L}\right)
     = m_2 g\bigl[1+2(1-\cos\theta)\bigr]
     = m_2 g\,(3-2\cos\theta).
\]

Como a roldana é ideal, a tração que puxa $m_1$ na horizontal é $T_B$.

\textit{3) Condição de não deslizamento de $m_1$.} Para $m_1$ permanecer em repouso,
a força de atrito estático máxima deve ser ao menos igual à tração:
\[
f_{s,\max}=\mu_s N=\mu_s m_1 g \;\ge\; T_B.
\]
Logo, o coeficiente mínimo é
\[
\boxed{\;\mu_{s,\min}=\frac{T_B}{m_1 g}
      =\frac{m_2}{m_1}\,\bigl(3-2\cos\theta\bigr)\; }.
\]

\textbf{Observação:} O ponto $B$ é o ponto mais baixo da trajetória, onde a tração é máxima; portanto, se $m_1$ não desliza em $B$, não deslizará em nenhuma outra posição.
\end{flushleft}




%%%%%%%% Bibliography 
% Os comandos para incluir as referências bibliográficas
%\printingbibliography

\end{document}
