\section{\large \textcolor{blue}{Gravitação}}

\begin{flushleft}
\textbf{\textcolor{blue}{\Large Quest\~ao 37}}\\
\noindent
\subsection{Quest\~ao 37 - Astrônomia}
\colorbox{green!30}{Qual o astrônomo que propôs um modelo geocêntrico que
permitia descrever e prever} \colorbox{green!30}{as posições dos planetas e que, para isso, propôs que o movimento retrógrado dos planetas}
não tem sempre o mesmo aspecto e duração?

\begin{itemize}
\item[(A)] Galileu Galilei.
\item[(B)] Johannes Kepler.
\item[(C)] Cláudio Ptolomeu.
\item[(D)] Nicolau Copérnico.
\end{itemize}

\vspace{0.5cm}

\textcolor{red}{\textbf{Solução:}}\\

\section*{Resposta correta}

\[
\boxed{\text{(C) Cláudio Ptolomeu}}
\]

\section*{Explicação detalhada}

\subsection*{Quem foi Ptolomeu?}
Cláudio Ptolomeu foi um astrônomo, matemático e geógrafo grego que viveu em Alexandria, no Egito, no século II d.C. Ele escreveu a obra \textit{Almagesto}, que se tornou o principal tratado astronômico da Antiguidade e da Idade Média.

\subsection*{O que ele propôs?}
Ptolomeu refinou o antigo modelo geocêntrico (originalmente defendido por Aristóteles e Hiparco), criando um sistema geométrico e matemático capaz de:
\begin{itemize}
    \item Prever com precisão a posição dos planetas no céu em diferentes datas.
    \item Explicar por que os planetas às vezes parecem parar e andar para trás (\textit{movimento retrógrado aparente}).
\end{itemize}

\subsection*{Como ele explicou o movimento retrógrado?}
Para explicar o movimento retrógrado no \textbf{modelo geocêntrico}, Ptolomeu propôs que cada planeta não girava apenas em torno da Terra, mas fazia isso percorrendo duas trajetórias ao mesmo tempo:
\begin{itemize}
    \item Um \textbf{deferente}: círculo grande ao redor da Terra.
    \item Um \textbf{epiciclo}: círculo menor, cujo centro se move ao longo do deferente.
\end{itemize}

Esse sistema (\textit{deferente + epiciclo}) conseguia reproduzir as irregularidades do movimento dos planetas, inclusive o fato de que o movimento retrógrado não tinha sempre o mesmo tamanho nem a mesma duração para cada planeta.

\section*{Por que não as outras alternativas?}

\begin{itemize}
    \item \textbf{(A) Galileu Galilei}: Defendeu o heliocentrismo e fez observações com telescópio (\textit{séc. XVII}).
    \item \textbf{(B) Johannes Kepler}: Refinou o heliocentrismo com órbitas elípticas, rejeitando o geocentrismo (\textit{séc. XVII}).
    \item \textbf{(D) Nicolau Copérnico}: Propôs o heliocentrismo com órbitas circulares (\textit{séc. XVI}).
\end{itemize}

Somente \textbf{Ptolomeu} defendeu um modelo \textbf{geocêntrico}, consistente com as crenças da época, que já explicava as variações do movimento retrógrado.

\section*{Resumo}

\begin{center}
\small
\begin{tabular}{|c|c|c|}
\hline
\textbf{Astrônomo} & \textbf{Modelo} & \textbf{Movimento retrógrado} \\
\hline
\textbf{Ptolomeu} & Geocêntrico com epiciclos & Explicava corretamente o aspecto variável \\
\hline
Galileu & Heliocentrismo com telescópio & Observações em defesa do heliocentrismo \\
\hline
Kepler & Heliocentrismo com órbitas elípticas & Refinamento matemático \\
\hline
Copérnico & Heliocentrismo com órbitas circulares & Proposta inicial \\
\hline
\end{tabular}
\end{center}


A resposta correta é alternativa \colorbox{green!50}{\textbf{C}}.
\end{flushleft}

\noindent\rule{\linewidth}{0.6pt}\\

\section*{Gravitação Universal}

\textbf{Lei da Gravitação Universal:}
\begin{equation*}
  F = -G \frac{m_1 m_2}{r^2}
\end{equation*}

\textbf{Campo gravitacional:}
\begin{equation*}
  g = \frac{G M}{r^2}
\end{equation*}

\textbf{Energia potencial gravitacional:}
\begin{equation*}
  E_p = -\frac{G M m}{r}
\end{equation*}

\section*{Demonstração da Velocidade de Escape}

\colorbox{green!40}{A velocidade de escape é a mínima velocidade necessária para um corpo escapar da} \\
\colorbox{green!40}{gravidade de um planeta,} sem considerar resistência do ar.

\subsection*{Conservação de Energia}

Considerando um corpo de massa $m$ lançado da superfície de um planeta de massa $M$ e raio $R$:

\begin{itemize}
  \item Energia mecânica inicial:
  \[
  E_{\text{inicial}} = \frac{1}{2}mv^{2}_{e} - \frac{GMm}{R}
  \]
  \item Energia mecânica final (no infinito): 
  \[
  E_{\text{final}} = 0
  \]
\end{itemize}

Aplicando a conservação da energia:

\[
\frac{1}{2}mv^2_{e} - \frac{GMm}{R} = 0
\Rightarrow \frac{1}{2}v^2_{e} = \frac{GM}{R}
\Rightarrow v_{e} = \sqrt{\frac{2GM}{R}}
\]

\noindent
\textbf{Conclusão:} A velocidade de escape depende apenas da massa e do raio do corpo celeste, e não da massa do objeto lançado.

\noindent\rule{\linewidth}{0.6pt}\\

\begin{flushleft}
\textbf{\textcolor{blue}{\Large Questao 38}}\\
\noindent
\subsection{Quest\~ao 38 - Lei da Gravitação Universal}
Um foguete é lançado verticalmente para cima a partir da
superfície da Terra. Se a velocidade inicial do foguete for
metade da velocidade de escape da Terra, qual a altura que
o foguete atingirá, em unidades do raio da Terra (R$_{T}$)?
Despreze as influências da rotação da Terra no movimento
do foguete.

\begin{itemize}
\item[(A)] (7/3)R$_{T}$.
\item[(B)] (5/3)R$_{T}$.
\item[(C)] (2/3)R$_{T}$.
\item[(D)] (1/3)R$_{T}$.
\end{itemize}

\vspace{0.5cm}

\textcolor{red}{\textbf{Solução:}}\\

A energia mecânica total do foguete se conserva, pois desprezamos a resistência do ar.  
Na superfície da Terra (\(r = R_T\)), a energia total é a soma da energia cinética e potencial:  

\[
E_i = \frac{1}{2} m v_0^2 - \frac{G M_T m}{R_T}
\]

Na altura máxima (\(r = r_{\text{max}}\)), a velocidade do foguete é nula (\(v_f = 0\)):  

\[
E_f = 0 - \frac{G M_T m}{r_{\text{max}}}
\]

Conservação da energia mecânica: \(E_i = E_f\)  
Portanto:

\[
\frac{1}{2} m v_0^2 - \frac{G M_T m}{R_T} = - \frac{G M_T m}{r_{\text{max}}}
\]

Cancelamos \(m\) em todos os termos:

\[
\frac{1}{2} v_0^2 - \frac{G M_T}{R_T} = - \frac{G M_T}{r_{\text{max}}}
\]

Sabemos que a \textbf{velocidade de escape} é dada por:

\[
v_e = \sqrt{\frac{2 G M_T}{R_T}}
\]

Como a velocidade inicial do foguete é \(v_0 = \frac{v_e}{2}\), temos:

\[
v_0^2 = \left( \frac{v_e}{2} \right)^2 = \frac{v_e^2}{4} = \frac{1}{4} \cdot \frac{2 G M_T}{R_T} = \frac{G M_T}{2 R_T}
\]

Substituímos \(v_0^2\) na equação da energia:

\[
\frac{1}{2} \cdot \frac{G M_T}{2 R_T} - \frac{G M_T}{R_T} = - \frac{G M_T}{r_{\text{max}}}
\]

\[
\frac{G M_T}{4 R_T} - \frac{G M_T}{R_T} = - \frac{G M_T}{r_{\text{max}}}
\]

\[
-\frac{3}{4} \cdot \frac{G M_T}{R_T} = - \frac{G M_T}{r_{\text{max}}}
\]

Eliminamos o sinal e \(G M_T\):

\[
\frac{3}{4 R_T} = \frac{1}{r_{\text{max}}}
\]

Então:

\[
r_{\text{max}} = \frac{4}{3} R_T
\]

A altura máxima \(h_{\text{max}}\) acima da superfície é:

\[
h_{\text{max}} = r_{\text{max}} - R_T = \frac{4}{3} R_T - R_T = \frac{1}{3} R_T
\]

\section*{Resposta final:}

\[
\boxed{h_{\text{max}} = \frac{1}{3} R_T}
\]

O foguete atinge uma altura máxima igual a \(\frac{1}{3}\) do raio da Terra.


A resposta correta é alternativa \colorbox{green!50}{\textbf{D}}.
\end{flushleft}

\noindent\rule{\linewidth}{0.6pt}\\

\begin{flushleft}
\textbf{\textcolor{blue}{\Large Quest\~ao 39}}\\
\noindent
\subsection{Quest\~ao 39 - Lei da Gravitação Universal}
Um satélite de massa m orbita um planeta de massa M em
uma órbita circular de raio R. O tempo necessário para uma
volta completa do satélite em torno do planeta é

\begin{itemize}
\item[(A)] independente de M.
\item[(B)] proporcional a R$^{3/2}$.
\item[(C)] dependente de m.
\item[(D)] proporcional a R$^{2}$.
\end{itemize}

\vspace{0.5cm}

\textcolor{red}{\textbf{Solução:}}\\

A força gravitacional fornece a força centrípeta necessária:

\[
\frac{G M m}{R^2} = m \frac{v^2}{R}
\]

Cancelando \(m\) e resolvendo para \(v\):

\[
v = \sqrt{\frac{G M}{R}}
\]

O período \(T\) é dado por:

\[
T = \frac{2\pi R}{v}
\]

Substituindo \(v\):

\[
T =
\sqrt{\frac{4 \pi^{2} R^{3}}{G M}} =\sqrt{\frac{4 \pi^{2}}{G M}}\sqrt{R^{3}} = \sqrt{\frac{4 \pi^{2}}{G M}} R^{3/2}
\]


\section*{Resposta final:}

\[
\boxed{
T \propto R^{3/2}
}
\]


A resposta correta é alternativa \colorbox{green!50}{\textbf{B}}.
\end{flushleft}

\begin{flushleft}
\textbf{\textcolor{blue}{\Large Quest\~ao 43 - IFPA 2018 - Sistema Isolado de Estrelas Binárias}}\\
\noindent

\subsection{Quest\~ao 43 - IFPA 2018 - Sistema Isolado de Estrelas Binárias}

Considere um sistema isolado de duas estrelas binárias no espaço. Em termos da constante de gravitação $G$, da distância entre 
as estrelas $L$ e de suas massas $M_1$ e $M_2$, o período $T$ de rotação das estrelas binárias quando as massas são diferentes 
$(M_1\neq M_2)$ é dado por

\begin{itemize}
\item[(A)] $T=2\pi L\sqrt{\dfrac{G\,M_1M_2}{L}}\,$.
\item[(B)] $T=2\pi L\sqrt{\dfrac{G(M_1+M_2)}{L}}\,$.
\item[(C)] $T=2\pi\sqrt{\dfrac{G L}{(M_1+M_2)}}\,$.
\item[(D)] $T=2\pi\sqrt{\dfrac{G L}{M_1M_2}}\,$.
\item[(E)] $T=2\pi L\sqrt{\dfrac{L}{G(M_1+M_2)}}\,$.
\end{itemize}

\vspace{0.5cm}

\textcolor{red}{\textbf{Solu\c{c}\~ao:}}\\

Seja $r_1$ a distância da massa $M_1$ ao centro de massa e $r_2$ a distância da massa $M_2$ ao centro de massa. 
Pelas defini\c{c}\~oes do centro de massa e pela separa\c{c}\~ao $L$ entre os centros, temos
\[
r_1 + r_2 = L,
\qquad
M_1 r_1 = M_2 r_2.
\]
Dessas duas rela\c{c}\~oes segue-se (por exemplo isolando $r_1$):
\[
r_1=\frac{M_2}{M_1+M_2}\,L,
\qquad
r_2=\frac{M_1}{M_1+M_2}\,L,
\]


e suponha órbitas circulares. A força gravitacional que age sobre $M_1$ é
\[
F=\frac{G M_1 M_2}{L^2},
\]
que fornece a força centrípeta para $M_1$:
\[
M_1 r_1 \omega^2=\frac{G M_1 M_2}{L^2},
\]
onde $\omega$ é a velocidade angular comum. Cancelando $M_1$ e substituindo $r_1$:
\[
\omega^2=\frac{G M_2}{L^2 r_1}
=\frac{G M_2}{L^2}\cdot\frac{M_1+M_2}{M_2}
=\frac{G(M_1+M_2)}{L^3}.
\]
Logo,
\[
\omega=\sqrt{\frac{G(M_1+M_2)}{L^3}}.
\]
O período $T$ é dado por $T=2\pi/\omega$, portanto
\[
T=2\pi\sqrt{\frac{L^3}{G(M_1+M_2)}}.
\]
Reescrevendo $\sqrt{L^3}=L\sqrt{L}$ obtém-se
\[
T=2\pi L\sqrt{\frac{L}{G(M_1+M_2)}}.
\]

Portanto a alternativa correta é \colorbox{green!50}{\textbf{(E)}}.

\end{flushleft}


\begin{flushleft}
\textbf{\textcolor{blue}{\Large Quest\~ao - }}\\
\noindent

\subsection{Quest\~ao }

\begin{itemize}
\item[(A)] 
\item[(B)] 
\item[(C)]
\item[(D)] 
\item[(E)] 
\end{itemize}

\vspace{0.5cm}

\textcolor{red}{\textbf{Solução:}}\\


A resposta correta é alternativa \colorbox{green!50}{\textbf{...}}.

\end{flushleft}