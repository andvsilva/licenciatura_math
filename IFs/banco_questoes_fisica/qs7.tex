\section{\large \textcolor{blue}{Óptica geométrica}}

\begin{flushleft}
\textbf{\textcolor{blue}{\Large Quest\~ao - Entrada da Fibra Óptica — Lei de Snell}}\\
\noindent

\subsection{Quest\~ao Entrada da Fibra Óptica — Lei de Snell}

\vspace{0.5cm}

\textcolor{red}{\textbf{Solução:}}\\

\section*{Índices de Refração}

\begin{itemize}
    \item $n_1 = 1$
    \item $n_2 = 1{,}6$
    \item $n_3 = 1{,}5$
\end{itemize}

\section*{Entrada da Fibra Óptica (Raio de Luz)}

Utilizando a \textbf{Lei de Snell}, temos:

\subsection*{1. Incidência do meio $n_1$ para o meio $n_2$ (ponto 1):}
\[
n_1 \cdot \sin \theta = n_2 \cdot \sin \phi
\Rightarrow \sin \theta = 1{,}6 \cdot \sin \phi
\]

\subsection*{2. Reflexão Total Interna no ponto (2):}
\[
n_2 \cdot \sin \alpha = n_3 \cdot \sin 90^\circ
\Rightarrow 1{,}6 \cdot \sin \alpha = 1{,}5 \cdot 1 = 1{,}5
\Rightarrow \sin \alpha = \frac{1{,}5}{1{,}6}
\]

\subsection*{3. Substituindo na equação de Snell:}
\[
\sin \theta = 1{,}6 \cdot \sin \phi
\qquad
\text{e}
\qquad
\sin \alpha = \frac{1{,}5}{1{,}6} = \frac{15}{16}
\]

\subsection*{4. Cálculo de $\sin \theta$:}
\[
\sin \theta = 1{,}6 \cdot \sin \phi = \frac{15}{10} = \frac{3{,}5}{4{,}4}
\]

\section*{Identidade Trigonométrica (para reflexão total):}

Sabemos que:
\[
\phi + \alpha = 90^\circ
\Rightarrow \alpha = 90^\circ - \phi
\]

Portanto:
\[
\sin (90^\circ - \phi) = \cos \phi
\quad \Rightarrow \quad
\sin \alpha = \cos \phi
\]

Logo:
\[
\sin(90^\circ - \phi) = \sin 90^\circ \cdot \cos \phi - \cos 90^\circ \cdot \sin \phi = \cos \phi
\]

Sabemos que:

\[
\cos \phi = \frac{15}{16}
\]

Pelo fato de que:
\[
\sin^2 \phi + \cos^2 \phi = 1
\Rightarrow \sin^2 \phi = 1 - \left(\frac{15}{16}\right)^2
= \frac{256 - 225}{256} = \frac{31}{256}
\]

\[
\Rightarrow \sin \phi = \sqrt{\frac{31}{256}}
\]

Agora, usando a equação:
\[
\sin \theta = 1{,}6 \cdot \sin \phi
\Rightarrow \sin \theta = 1{,}6 \cdot \sqrt{\frac{31}{256}}
= \frac{16}{10} \cdot \sqrt{\frac{31}{256}}
= \frac{16}{10} \cdot \frac{\sqrt{31}}{16}
= \frac{\sqrt{31}}{10}
\]

Portanto, o ângulo de incidência máximo é:

\[
\boxed{
\theta = \sin^{-1} \left( \frac{\sqrt{31}}{10} \right)
}
\]

\end{flushleft}

\begin{flushleft}
\textbf{\textcolor{blue}{\Large Quest\~ao - }}\\
\noindent

\subsection{Quest\~ao }

\begin{itemize}
\item[(A)] 
\item[(B)] 
\item[(C)]
\item[(D)] 
\item[(E)] 
\end{itemize}

\vspace{0.5cm}

\textcolor{red}{\textbf{Solução:}}\\


A resposta correta é alternativa \colorbox{green!50}{\textbf{...}}.

\end{flushleft}




