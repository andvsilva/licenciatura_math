\noindent\rule{\linewidth}{0.6pt}\\

\section{\large \textcolor{blue}{As leis de conservação na Mecânica Clássica}}

\begin{flushleft}
\textbf{\textcolor{blue}{\Large Quest\~ao - Medidor de Vazão (Tubo de Venturi)}}\\
\noindent

\subsection{Quest\~ao - Medidor de Vazão (Tubo de Venturi)}

Um fluido incompressível e não viscoso escoa horizontalmente através de um tubo de Venturi. O tubo possui uma seção larga de área \( A_1 \) e uma seção estreita de área \( A_2 \), com \( A_1 > A_2 \). Dois tubos manométricos estão conectados nas duas seções, e observa-se um desnível \( h \) entre os níveis do fluido nesses tubos.

Sabendo que a diferença de altura nos tubos manométricos é devida à diferença de pressão entre as seções do tubo, determine a expressão para a velocidade do fluido \( v_1 \) na seção de maior área \( A_1 \), em função de \( g \), \( h \), \( A_1 \) e \( A_2 \).



\begin{itemize}
\item[(A)] \( v_1 = \sqrt{ \dfrac{2gh}{1 - \left( \dfrac{A_2}{A_1} \right)^2} } \)
\item[(B)] \( v_1 = \sqrt{ \dfrac{gh}{\left( \dfrac{A_1}{A_2} \right)^2 - 1} } \)
\item[(C)] \( v_1 = \sqrt{ \dfrac{2gh}{\left( \dfrac{A_1}{A_2} \right)^2 - 1} } \)
\item[(D)] \( v_1 = \dfrac{A_2}{A_1} \sqrt{ 2gh } \)
\item[(E)] \( v_1 = \sqrt{ 2gh \left( \dfrac{A_2}{A_1} \right)^2 } \)
\end{itemize}

\vspace{0.5cm}

\textcolor{red}{\textbf{Solução:}}\\

Pelo teorema de Bernoulli (sem variação de altura) e pela equação da continuidade, temos:

\[
P_1 - P_2 = \frac{\rho}{2}(v_2^2 - v_1^2) \quad \text{e} \quad v_2 = \frac{A_1}{A_2} v_1
\]

Substituindo:

\[
\rho g h = \frac{\rho}{2} \left[ \left( \frac{A_1}{A_2} \right)^2 v_1^2 - v_1^2 \right]
\Rightarrow 2gh = v_1^2 \left[ \left( \frac{A_1}{A_2} \right)^2 - 1 \right]
\]

\[
\Rightarrow v_1 = \sqrt{ \frac{2gh}{\left( \dfrac{A_1}{A_2} \right)^2 - 1} }
\]

A resposta correta é alternativa \colorbox{green!50}{\textbf{(C)}}.

\end{flushleft}





\begin{flushleft}
\textbf{\textcolor{blue}{\Large Quest\~ao - }}\\
\noindent

\subsection{Quest\~ao }

\begin{itemize}
\item[(A)] 
\item[(B)] 
\item[(C)]
\item[(D)] 
\item[(E)] 
\end{itemize}

\vspace{0.5cm}

\textcolor{red}{\textbf{Solução:}}\\


A resposta correta é alternativa \colorbox{green!50}{\textbf{...}}.

\end{flushleft}

\noindent\rule{\linewidth}{0.6pt}\\
