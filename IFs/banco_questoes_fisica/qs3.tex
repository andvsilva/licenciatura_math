\section{\large \textcolor{blue}{Oscilações e ondas}}

\begin{flushleft}
\textbf{\textcolor{blue}{\Large Quest\~ao 48 - IFS2024 - P\^endulo Simples}}\\
\noindent
\subsection{Quest\~ao 48 - IFS2024 - P\^endulo Simples}
Um pêndulo simples de comprimento \( L = 10\,m \) oscila com um ângulo máximo de oito graus \( 0{,}14\,\) rad.  
Considere a aceleração da gravidade \( g = 10\,\)m/s\(^2\). A equação diferencial que descreve o movimento do pêndulo para pequenos ângulos é dada por:
$\frac{d^2\theta}{dt^2} + \omega^2 \theta = 0$ sendo \( \omega \) a frequência angular do pêndulo e \( \theta \) o ângulo de deslocamento em função do tempo \( t \).  
Considerando as condições iniciais \( \theta(0) = \theta_0 \) e \( \frac{d\theta}{dt}(0) = 0 \), a solução geral da equação diferencial para o pêndulo é:

\begin{itemize}
\item[(A)] \( \theta(t) = 0{,}14\cos(0{,}1t) \).
\item[(B)] \( \theta(t) = 0{,}14\cos(0{,}4t) \).
\item[(C)] \( \theta(t) = 0{,}14\cos(0{,}8t) \).
\item[(D)] \( \theta(t) = 0{,}14\cos(t) \).
\end{itemize}

\vspace{0.5cm}

\textcolor{red}{\textbf{Solução:}}\\

\section*{Demonstração da equação do movimento do pêndulo simples a partir do torque}

Considere um pêndulo simples com comprimento \(L\) e massa \(m\), oscilando em torno do ponto de suspensão com um ângulo \(\theta(t)\) em relação à posição de equilíbrio vertical.

\bigskip

\textbf{1. Torque devido à força peso}

A força peso atua verticalmente para baixo com intensidade \(mg\). O torque em relação ao ponto de suspensão é:

\[
\boxed{\tau = - m g L \sin\theta,}
\]

onde o sinal negativo indica que o torque tende a restaurar o pêndulo para a posição de equilíbrio (\(\theta = 0\)).

\bigskip

\textbf{2. Momento de inércia do pêndulo simples}

Como o pêndulo é uma massa pontual no final de um fio de massa desprezível, o momento de inércia em relação ao ponto de suspensão é:

\[
\boxed{I = m L^2.}
\]

\bigskip

\textbf{3. Equação do movimento rotacional}

Aplicando a segunda lei de Newton para rotações, temos:

\[
\boxed{\tau = I \alpha,}
\]

onde \(\alpha = \frac{d^2 \theta}{dt^2}\) é a aceleração angular. Substituindo,

\[
- m g L \sin\theta = m L^2 \frac{d^2 \theta}{dt^2}.
\]

Dividindo ambos os lados por \(m L^2\):

\[
\boxed{\frac{d^2 \theta}{dt^2} + \frac{g}{L} \sin\theta = 0.}
\]

\bigskip

\textbf{4. Aproximação para pequenos ângulos}

Para pequenas oscilações, onde \(\theta \ll 1\) (rad), podemos aproximar \(\sin\theta \approx \theta\), obtendo a equação linearizada:

\[
\boxed{\frac{d^2 \theta}{dt^2} + \frac{g}{L} \theta = 0.}
\]

Definindo

\[
\omega = \sqrt{\frac{g}{L}},
\]

a equação diferencial torna-se

\[
\boxed{\frac{d^2 \theta}{dt^2} + \omega^2 \theta = 0.}
\]

\bigskip

\textbf{5. Solução da equação diferencial}

A solução geral da equação é

\[
\boxed{\theta(t) = A \cos(\omega t) + B \sin(\omega t),}
\]

onde as \colorbox{green!30}{constantes \(A\) e \(B\) são determinadas pelas condições iniciais.}

Dadas as condições:

\[
\theta(0) = \theta_0, \quad \frac{d\theta}{dt}(0) = 0,
\]

temos:

\[
\theta(0) = A = \theta_0,
\]

e

\[
\frac{d\theta}{dt} = - A \omega \sin(\omega t) + B \omega \cos(\omega t) \implies \frac{d\theta}{dt}(0) = B \omega = 0 \Rightarrow B = 0.
\]

Assim, a solução final é

\[
\theta(t) = \theta_0 \cos(\omega t) = \theta_0 \cos\left(\sqrt{\frac{g}{L}} \, t \right).
\]

\[
\boxed{
\theta(t) = \theta_0 \cos\left(\sqrt{\frac{10}{10}} \, t \right) = \theta_0 \cos\left(t \right).
}
\]

A resposta correta é alternativa \colorbox{green!50}{\textbf{D}}.
\end{flushleft}


\begin{flushleft}
\textbf{\textcolor{blue}{\Large Quest\~ao 46}}\\
\noindent
\subsection{Quest\~ao 46 - Ondas Estacionária}
Um pesquisador que está estudando a propagação de ondas em uma corda observa a seguinte situação: uma
onda estacionária se forma na corda, com nós (pontos de amplitude zero) a cada 0,5 m, amplitude de 2,0 m e
velocidade de propagação de 2,0 m/s. A equação que o pesquisador obtém para descrever a onda estacionária é

\begin{itemize}
\item[(A)] $y(x,t) = 2\sin(\pi x)\cos(4\pi t)$
\item[(B)] $y(x,t) = 2\sin(2\pi x)\cos(4\pi t)$
\item[(C)] $y(x,t) = 2\sin(2\pi x)\cos(\pi t)$
\item[(D)] $y(x,t) = 2\sin(\pi x)\cos(\pi t)$
\end{itemize}

\vspace{0.5cm}

\textcolor{red}{\textbf{Solução:}}\\

\textbf{Resolução:}

\bigskip

\textbf{Dados do problema:}
\begin{itemize}
    \item Distância entre nós consecutivos: \(0{,}5\,m\)
    \item Amplitude máxima: \(A = 2,0\,m\)
    \item Velocidade de propagação: \(v = 2,0\,m/s\)
\end{itemize}

Queremos encontrar a equação da onda estacionária no formato:
\[
y(x,t) = 2A \sin(kx) \cos(\omega t)
\]

Sabemos que o fator \(2A\) já é dado como \(2,0\), então apenas precisamos determinar \(k\) e \(\omega\).

\bigskip

\textbf{Passo 1: distância entre nós}

Em uma onda estacionária, a distância entre dois nós consecutivos é igual a \(\lambda/2\).  
Como o problema informa que essa distância é \(0{,}5\,m\), temos:
\[
\frac{\lambda}{2} = 0{,}5 \quad \Longrightarrow \quad \lambda = 1,0\,m
\]

\bigskip

\textbf{Passo 2: número de onda \(k\)}

O número de onda é dado por:
\[
k = \frac{2\pi}{\lambda} = \frac{2\pi}{1,0} = 2\pi
\]

Portanto, o fator espacial da solução é \(\sin(2\pi x)\).

\bigskip

\textbf{Passo 3: frequência angular \(\omega\)}

Usamos a relação entre velocidade, frequência e comprimento de onda:
\[
v = \lambda f \quad \Longrightarrow \quad f = \frac{v}{\lambda} = \frac{2,0}{1,0} = 2,0\,Hz
\]

E como \(\omega = 2\pi f\), temos:
\[
\omega = 2\pi \cdot 2 = 4\pi
\]

\bigskip

\textbf{Passo 4: equação final}

Substituindo os valores encontrados:
\[
y(x,t) = 2 \sin(2\pi x) \cos(4\pi t)
\]

\bigskip

\textbf{Resposta correta:}
\[
\boxed{y(x,t) = 2 \sin(2\pi x) \cos(4\pi t)}
\]

Essa equação possui duas partes principais:

\bigskip

\textbf{Parte espacial:} \(\sin(kx)\)
\begin{itemize}
    \item Determina o padrão fixo de \textbf{nós} (onde a amplitude é sempre zero) e \textbf{ventres} (onde a amplitude é máxima).
    \item Define a forma da onda ao longo do espaço.
\end{itemize}

\bigskip

\textbf{Parte temporal:} \(\cos(\omega t)\)
\begin{itemize}
    \item Descreve a oscilação harmônica no tempo.
    \item Cada ponto vibra com a frequência angular \(\omega\), mas com amplitude espacialmente determinada.
\end{itemize}


A resposta correta é alternativa \colorbox{green!50}{\textbf{B}}.
\end{flushleft}

\noindent\rule{\linewidth}{0.6pt}\\

\begin{flushleft}
\textbf{\textcolor{blue}{\Large Quest\~ao 47}}\\
\noindent
\subsection{Quest\~ao 47 - Ondas Sonoras}
Duas fontes de ondas sonoras idênticas emitem ondas com
comprimento de onda de 0,5 m em fase. As fontes estão
separadas por uma distância de 1,5 m. Haverá interferência
construtiva ao longo da linha que liga as duas fontes nas
posições:

\begin{itemize}
\item[(A)] 0,25 m, 0,75 m, 1,25 m.
\item[(B)] 0,5 m, 1,0 m, 1,25 m.
\item[(C)] 0,5 m, 1,0 m, 1,5 m.
\item[(D)] 0,25 m, 0,5 m, 1,25 m.
\end{itemize}

\vspace{0.5cm}

\textcolor{red}{\textbf{Solução:}}\\

\colorbox{yellow!30}{A diferença de caminhos entre as ondas emitidas pelas duas fontes deve ser um múltiplo} 
\colorbox{yellow!30}{inteiro de \( \lambda \) para que ocorra \textbf{interferência construtiva}:}
\[
\Delta r = m\lambda, \quad m = 0, \pm1, \pm2, \dots
\]

Colocando as fontes nos pontos \( x=0 \) e \( x=d \), ao longo do eixo \( x \), temos para um ponto \( x \):
\[
\Delta r = |x - (d-x)| = |2x - d|
\]

Para interferência construtiva:
\[
2x - d = m\lambda
\]

Resolvendo para \( x \):
\[
x = \frac{d + m\lambda}{2}
\]

Substituindo \( d = 1{,}5 \) e \( \lambda = 0{,}5 \):
\[
x = \frac{1{,}5 + 0{,}5m}{2} = 0{,}75 + 0{,}25m
\]

Para que \( x \) esteja entre \( 0 \) e \( 1{,}5 \), os valores possíveis de \( m \) são \( m = -3, -2, -1, 0, 1, 2, 3 \), o que resulta nas posições:
\[
x = 0{,}0;\ 0{,}25;\ 0{,}5;\ 0{,}75;\ 1{,}0;\ 1{,}25;\ 1{,}5 \ \mathrm{m}
\]

Entre as alternativas dadas, a correta é:
\[
\boxed{\text{(A) } 0{,}25\,m,\ 0{,}75\,m,\ 1{,}25\,m}
\]


A resposta correta é alternativa \colorbox{green!50}{\textbf{A}}.
\end{flushleft}

\noindent\rule{\linewidth}{0.6pt}\\

\section*{Equilíbrio do Corpo Rígido e da Partícula}

\textbf{Condições de equilíbrio:}
\begin{align*}
  \sum \vec{F} &= 0 \quad \text{(equilíbrio translacional)} \\
  \sum \vec{\tau} &= 0 \quad \text{(equilíbrio rotacional)}
\end{align*}

\textbf{Torque (momento de uma força):}
\begin{equation*}
  \tau = r F \sin \theta
\end{equation*}

\begin{equation*}
  \vec{\tau} = \frac{d\vec{L}}{dt}
\end{equation*}

\begin{equation*}
  \tau = I.\alpha
\end{equation*}

\begin{equation*}
  \alpha = \frac{d^{2}\theta}{dt^{2}}
\end{equation*}

\begin{equation*}
  \frac{d^{2}\theta}{dt^{2}} + \frac{g}{L}\sin\theta = 0 \quad \textrm{MHS}
\end{equation*}

\begin{equation*}
  \frac{d^{2}\theta}{dt^{2}} + \omega^{2}\sin\theta = 0 \quad \textrm{MHS}
\end{equation*}

Solução geral EDO:
\begin{equation*}
  \theta(t) = \theta_{0} \cos(\omega t + \varphi)
\end{equation*}

\subsection*{Rota\c{c}\~ao de um Corpo R\'igido}
\begin{equation*}
  \omega = \frac{d\theta}{dt}, \quad \alpha = \frac{d\omega}{dt}
\end{equation*}


\begin{flushleft}
\textbf{\textcolor{blue}{\Large Quest\~ao - }}\\
\noindent

\subsection{Quest\~ao }

\begin{itemize}
\item[(A)] 
\item[(B)] 
\item[(C)]
\item[(D)] 
\item[(E)] 
\end{itemize}

\vspace{0.5cm}

\textcolor{red}{\textbf{Solução:}}\\


A resposta correta é alternativa \colorbox{green!50}{\textbf{...}}.
\end{flushleft}

\begin{flushleft}
\textbf{\textcolor{blue}{\Large Quest\~ao - }}\\
\noindent

\subsection{Quest\~ao }

\begin{itemize}
\item[(A)] 
\item[(B)] 
\item[(C)]
\item[(D)] 
\item[(E)] 
\end{itemize}

\vspace{0.5cm}

\textcolor{red}{\textbf{Solução:}}\\


A resposta correta é alternativa \colorbox{green!50}{\textbf{...}}.
\end{flushleft}

\begin{flushleft}
\textbf{\textcolor{blue}{\Large Quest\~ao - }}\\
\noindent

\subsection{Quest\~ao }

\begin{itemize}
\item[(A)] 
\item[(B)] 
\item[(C)]
\item[(D)] 
\item[(E)] 
\end{itemize}

\vspace{0.5cm}

\textcolor{red}{\textbf{Solução:}}\\


A resposta correta é alternativa \colorbox{green!50}{\textbf{...}}.

\end{flushleft}


\begin{flushleft}
\textbf{\textcolor{blue}{\Large Quest\~ao - }}\\
\noindent

\subsection{Quest\~ao }

\begin{itemize}
\item[(A)] 
\item[(B)] 
\item[(C)]
\item[(D)] 
\item[(E)] 
\end{itemize}

\vspace{0.5cm}

\textcolor{red}{\textbf{Solução:}}\\


A resposta correta é alternativa \colorbox{green!50}{\textbf{...}}.


\end{flushleft}