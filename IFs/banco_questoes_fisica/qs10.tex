\section{\large \textcolor{blue}{Mecânica quântica em 3D e átomo de Hidrogênio}}

\begin{flushleft}
\textbf{\textcolor{blue}{\Large Quest\~ao 25 -IFSC 2023 - Mecânica Quântica}}\\
\noindent

\subsection{uest\~ao 25 -IFSC 2023 - Mecânica Quântica}

Bohr, um renomado f\'isico do s\'eculo XX, contribuiu significativamente para o desenvolvimento do modelo at\^omico ao aprimorar as ideias propostas por Rutherford. Com seu modelo, Bohr estabeleceu uma s\'erie de princ\'ipios que descreviam o comportamento dos el\'etrons em torno do n\'ucleo at\^omico, fornecendo uma explica\c{c}\~ao crucial para o fen\^omeno do espectro de emiss\~ao de gases excitados. Esses postulados foram de extrema import\^ancia para avan\c{c}ar a compreens\~ao da estrutura dos \'atomos e estabeleceram as bases fundamentais da f\'isica qu\^antica. 

Analise as assertivas abaixo em rela\c{c}\~ao aos postulados do modelo de Bohr e assinale V, se verdadeiras, ou F, se falsas.

\begin{enumerate}
    \item ( \ ) Os el\'etrons em um \'atomo est\~ao confinados em \'orbitas circulares ao redor do n\'ucleo, cujo raio da trajet\'oria $R_n = 5,3 \times 10^{-11} n^2\,\text{m}$, sendo $n$ um n\'umero inteiro correspondente \`a \'orbita do el\'etron.
    \item ( \ ) A energia de um el\'etron em uma \'orbita $n$ (com $n$ sendo um n\'umero inteiro) \'e dada por: $E_n = \frac{-13,6}{n^2} \ \text{eV}$.
    \item ( \ ) A energia dos el\'etrons \'e quantizada, ou seja, eles podem existir apenas em n\'iveis de energia espec\'ificos.
    \item ( \ ) Quando um el\'etron transita de um n\'ivel de energia mais alto para um n\'ivel de energia mais baixo, ele emite energia na forma de f\'otons.
    \item ( \ ) O modelo de Bohr explica completamente o comportamento dos el\'etrons em \'atomos.
\end{enumerate}

A ordem correta de preenchimento dos par\^enteses, de cima para baixo, \'e:

\begin{itemize}
\item[(A)] F -- F -- F -- V -- F.
\item[(B)] V -- F -- V -- F -- F.
\item[(C)] V -- F -- V -- V -- F.
\item[(D)] V -- V -- F -- V -- F.
\item[(E)] V -- F -- V -- V -- F.
\end{itemize}

\vspace{0.5cm}

\textcolor{red}{\textbf{Solu\c{c}\~ao:}}\\

A seguir analisamos cada uma das cinco assertivas, justificando porque s\~ao verdadeiras (V) ou falsas (F).

\bigskip

\textbf{Assertiva 1:} Os el\'etrons em um \'atomo est\~ao confinados em \'orbitas circulares ao redor do n\'ucleo, cujo raio da trajet\'oria 
\[
R_n = 5{,}3\times 10^{-11} n^2 \ \text{m},
\]
sendo $n$ um n\'umero inteiro.\\
No modelo de Bohr para o \`atomo de hidrog\^enio, $R_n = a_0 n^2$ com $a_0 \approx 5{,}29\times 10^{-11} \ \text{m}$, o \emph{raio de Bohr}. Afirma\c{c}\~ao correta.\\
\textbf{Conclus\~ao: V}.

\bigskip

\textbf{Assertiva 2:} A energia de um el\'etron em uma \'orbita $n$ \'e dada por 
\[
E_n = \frac{-13{,}6}{n^2} \ \text{eV}.
\]
Esta \'e a express\~ao correta para o \`atomo de hidrog\^enio, deduzida a partir da quantiza\c{c}\~ao do momento angular ($m_e v r = n\hbar$) e da condi\c{c}\~ao de equil\'ibrio centr\'ipeto.\\
\textbf{Conclus\~ao: V}.

\bigskip

\textbf{Assertiva 3:} A energia dos el\'etrons \'e quantizada, ou seja, apenas certos n\'iveis de energia s\~ao permitidos.\\
Este \'e um dos postulados centrais de Bohr.\\
\textbf{Conclus\~ao: V}.

\bigskip

\textbf{Assertiva 4:} Quando um el\'etron transita de um n\'ivel de energia mais alto para um mais baixo, ele emite energia na forma de f\'oton, com
\[
\Delta E = h\nu.
\]
Afirma\c{c}\~ao correta.\\
\textbf{Conclus\~ao: V}.

\bigskip

\textbf{Assertiva 5:} O modelo de Bohr explica completamente o comportamento dos el\'etrons em \'atomos.\\
Falsa, pois o modelo de Bohr descreve bem apenas sistemas hidrogenoides e n\~ao incorpora efeitos relativ\'isticos, spin, estrutura fina e hiperfina, nem explica \'atomos multieletr\^onicos de forma completa.\\
\textbf{Conclus\~ao: F}.

\bigskip

\textbf{Sequ\^encia final:} (1) V, (2) V, (3) V, (4) V, (5) F.

\vspace{0.5cm}

\textcolor{red}{\textbf{Resposta:}}\\
A alternativa correta deve corresponder \`a sequ\^encia \colorbox{green!50}{\textbf{V -- V -- V -- V -- F}}.

A resposta correta \'e alternativa \colorbox{green!50}{\textbf{E}}.

\end{flushleft}

\begin{flushleft}
\textbf{\textcolor{blue}{\Large Questão 27 - IFSC 2023 - Mecânica Quântica - Probabilidade}}\\
\noindent

\subsection{Questão 27 - IFSC 2023 - Mecânica Quântica - Probabilidade}

A equa\c{c}\~ao de Schr\"odinger \'e uma ferramenta fundamental na descri\c{c}\~ao do comportamento de part\'iculas qu\^anticas, 
como o el\'etron, em sistemas f\'isicos. Considere uma part\'icula confinada numa caixa unidimensional de comprimento $L=2\ \text{m}$ 
cuja fun\c{c}\~ao de onda no estado $n$ \'e dada por

\[
\psi(x)=A\sin\!\left(\frac{n\pi x}{L}\right),
\]
onde $A$ \'e a constante de normaliza\c{c}\~ao e $n$ \'e um n\'umero inteiro positivo que representa o estado de energia.

Considerando $n=3$, qual o valor aproximado da posi\c{c}\~ao $x$ (no intervalo $0<x<L$) onde a probabilidade de encontrar a part\'icula \'e m\'axima?

\begin{itemize}
\item[(A)] $0{,}25\ \text{m}.$
\item[(B)] $0{,}33\ \text{m}.$
\item[(C)] $0{,}50\ \text{m}.$
\item[(D)] $1{,}50\ \text{m}.$
\item[(E)] $2{,}00\ \text{m}.$
\end{itemize}

\vspace{0.5cm}

\textcolor{red}{\textbf{Solu\c{c}\~ao:}}\\

A função de onda para a partícula confinada na caixa unidimensional é:
\[
\psi(x) = A \sin\left(\frac{n\pi x}{L}\right),
\]
onde $n=3$ e $L = 2 \ \text{m}$.

A probabilidade de encontrar a partícula em $x$ é proporcional a:
\[
P(x) \propto |\psi(x)|^2 = A^2 \sin^2\left(\frac{3\pi x}{2}\right).
\]

O valor máximo de $P(x)$ ocorre quando:
\[
\sin^2\left(\frac{3\pi x}{2}\right) = 1.
\]
Isso significa que:
\[
\sin\left(\frac{3\pi x}{2}\right) = \pm 1.
\]

A condição para o seno atingir $\pm 1$ é:
\[
\frac{3\pi x}{2} = \frac{\pi}{2} + k\pi, \quad k \in \mathbb{Z}.
\]
Simplificando:
\[
\frac{3\pi x}{2} = \frac{\pi}{2}(1 + 2k),
\]
\[
x = \frac{1 + 2k}{3}.
\]

Para $0 < x < L = 2$ m, temos:
\[
k = 0 \ \Rightarrow \ x = \frac{1}{3} \ \text{m} \ (\approx 0,33 \ \text{m}),
\]
\[
k = 1 \ \Rightarrow \ x = 1 \ \text{m},
\]
\[
k = 2 \ \Rightarrow \ x = \frac{5}{3} \ \text{m} \ (\approx 1,67 \ \text{m}).
\]

Dentre as alternativas, o valor aproximado que aparece é:
\[
\boxed{0,33 \ \text{m}}
\]

\textcolor{red}{\textbf{Resposta correta:}} Letra B.

\end{flushleft}

\begin{flushleft}
\textbf{\textcolor{blue}{\Large Quest\~ao - }}\\
\noindent

\subsection{Quest\~ao }

\begin{itemize}
\item[(A)] 
\item[(B)] 
\item[(C)]
\item[(D)] 
\item[(E)] 
\end{itemize}

\vspace{0.5cm}

\textcolor{red}{\textbf{Solução:}}\\


A resposta correta é alternativa \colorbox{green!50}{\textbf{...}}.

\end{flushleft}