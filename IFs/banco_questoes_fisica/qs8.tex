\section{\large \textcolor{blue}{Interferência e difração}}

\begin{flushleft}
\textbf{\textcolor{blue}{\Large Quest\~ao 43}}\\
\noindent
\subsection{Quest\~ao 43 - Filmes Finos}
Luz com 650 nm de comprimento de onda incide
perpendicularmente em um filme fino de sabão, que tem
índice de refração igual a 1,30. Sabendo que esse filme está
suspenso no ar, qual a menor espessura que esse filme
deve ter para que as ondas refletidas por ele sofram
interferência construtiva?

\begin{itemize}
\item[(A)] 320 nm.
\item[(B)] 242 nm.
\item[(C)] 125 nm.
\item[(D)] 117 nm.
\end{itemize}

\vspace{0.5cm}

\textcolor{red}{\textbf{Solução:}}\\

\section*{Interferência construtiva em um filme de sabão}

\textbf{Dados:}
\begin{itemize}
    \item Comprimento de onda no ar: \( \lambda_0 = 650\,\mathrm{nm} \)
    \item Índice de refração do filme: \( n_f = 1{,}30 \)
    \item Índice de refração do ar: \( n_{ar} \approx 1 \)
\end{itemize}

O filme está suspenso no ar. Queremos a menor espessura \(e\) para que a luz refletida tenha interferência construtiva.

\subsection*{Condição de fase}

Quando a luz incide sobre a superfície do filme:
\begin{itemize}
    \item Na interface ar–sabão (\(n_\text{ar} < n_\text{sabão}\)), ocorre inversão de fase de \(\pi\) (equivalente a \(\lambda/2\)).
    \item Na interface sabão–ar (\(n_\text{sabão} > n_\text{ar}\)), não ocorre inversão.
\end{itemize}

Como há uma inversão de fase, a condição para \textbf{interferência construtiva} é:
\[
2e = \left(m + \frac{1}{2}\right) \lambda_f
\]

Para a menor espessura (\(m = 0\)):
\[
2e = \frac{\lambda_f}{2} \quad \implies \quad e = \frac{\lambda_f}{4}
\]

\subsection*{Comprimento de onda no filme}

No interior do filme, o comprimento de onda é menor:
\[
\lambda_f = \frac{\lambda_0}{n_f} = \frac{650}{1{,}30} \approx 500\,\mathrm{nm}
\]

\subsection*{Espessura mínima}

Substituindo:
\[
e_\text{mín} = \frac{\lambda_f}{4} = \frac{500}{4} = 125\,\mathrm{nm}
\]

\subsection*{Resposta final:}
\[
\boxed{e_\text{mín} = 125\,\mathrm{nm}}
\]


A resposta correta é alternativa \colorbox{green!50}{\textbf{C}}.
\end{flushleft}

\noindent\rule{\linewidth}{0.6pt}\\

\section*{Intervalo válido para o comprimento de onda de um laser}

O comprimento de onda (\( \lambda \)) de um laser depende do material ativo utilizado no laser e pode abranger diferentes regiões do espectro eletromagnético. Abaixo estão os intervalos típicos para lasers comuns:

\begin{center}
\begin{tabular}{|l|c|}
\hline
\textbf{Tipo de laser} & \textbf{Comprimento de onda (\( \lambda \))} \\
\hline
Laser ultravioleta (UV) & \(180\,\mathrm{nm} \text{ a } 400\,\mathrm{nm}\) \\
\hline
Laser visível (vermelho-violeta) & \(400\,\mathrm{nm} \text{ a } 700\,\mathrm{nm}\) \\
\hline
Laser infravermelho próximo (NIR) & \(700\,\mathrm{nm} \text{ a } 1400\,\mathrm{nm}\) \\
\hline
Laser infravermelho médio & \(1400\,\mathrm{nm} \text{ a } 3000\,\mathrm{nm}\) \\
\hline
Laser infravermelho distante & \(>3000\,\mathrm{nm}\) \\
\hline
\end{tabular}
\end{center}

\vspace{0.5cm}

\subsection*{Exemplos comuns de lasers visíveis:}
\begin{itemize}
    \item Laser vermelho (He-Ne ou diodo): \(630\,\mathrm{nm} - 680\,\mathrm{nm}\)
    \item Laser verde (Nd:YAG com dobro da frequência): \(532\,\mathrm{nm}\)
    \item Laser azul: \(405\,\mathrm{nm} - 488\,\mathrm{nm}\)
    \item Laser violeta: \( \sim 400\,\mathrm{nm} \)
\end{itemize}

\vspace{0.5cm}

Para lasers visíveis, o intervalo típico de comprimento de onda válido é aproximadamente:
\[
\boxed{400\,\mathrm{nm} \leq \lambda \leq 700\,\mathrm{nm}}
\]

\begin{flushleft}
\textbf{\textcolor{blue}{\Large Quest\~ao 44}}\\
\noindent
\subsection{Quest\~ao 44 - Difração de um feixe de luz laser}
Um feixe de luz laser incide sobre uma fenda estreita, e uma
figura de difração é observada sobre uma tela localizada a 5,0 m
da fenda. A distância vertical entre o centro do primeiro mínimo
acima do máximo central e o centro do primeiro mínimo abaixo
do máximo central é de 20 mm. Qual é a largura da fenda?

\begin{itemize}
\item[(A)] 0,30 mm.
\item[(B)] 0,45 mm.
\item[(C)] 0,55 mm.
\item[(D)] 0,65 mm.
\end{itemize}

\vspace{0.5cm}

\textcolor{red}{\textbf{Solução:}}\\

\subsection*{Passo 1: Condição para os mínimos}

Para uma fenda simples, os mínimos ocorrem em ângulos \(\theta\) tais que:
\[
a \cdot \sin\theta = m\lambda
\]
Para o primeiro mínimo (\(m=1\)):
\[
\sin\theta_1 = \frac{\lambda}{a}
\]

\vspace{0.5cm}

\subsection*{Passo 2: Relação geométrica na tela}

Na tela, a distância vertical entre o máximo central e o primeiro mínimo é aproximadamente:
\[
y_1 = L \cdot \tan\theta_1 \approx L \cdot \sin\theta_1
\]

A distância total entre o primeiro mínimo acima e o primeiro mínimo abaixo é:
\[
\Delta y = 2y_1
\]

Substituindo \(y_1\):
\[
\Delta y = 2L \cdot \sin\theta_1
\]

E como \(\sin\theta_1 = \lambda/a\):
\[
\Delta y = 2L \cdot \frac{\lambda}{a}
\]

\vspace{0.5cm}

\subsection*{Passo 3: Resolvendo para \(a\)}

Isolando \(a\):
\[
a = 2L \cdot \frac{\lambda}{\Delta y}
\]

Substituindo os valores numéricos:
\[
a = 2 \cdot 5{,}0 \cdot \frac{6{,}5 \times 10^{-7}}{0{,}020}
\]

\[
a = 10{,}0 \cdot 3{,}25 \times 10^{-5} = 3{,}25 \times 10^{-4}\,m
\]

Convertendo para milímetros:
\[
a = 0{,}325\,mm
\]

\vspace{0.5cm}

\subsection*{Resposta final:}

\[
\boxed{a \approx 0{,}325\,mm}
\]

A resposta correta é alternativa \colorbox{green!50}{\textbf{A}}.
\end{flushleft}

\noindent\rule{\linewidth}{0.6pt}\\

\begin{flushleft}
\textbf{\textcolor{blue}{\Large Quest\~ao 45 }}\\
\noindent
\subsection{Quest\~ao 42 - Rede de Difração}
Uma rede de difração possui \( 1{,}25 \times 10^{4} \) fendas uniformemente espaçadas, de forma que a largura total da rede é \( 25,0\,\mathrm{mm} \).  
Determine o ângulo \( \theta \) correspondente ao máximo de primeira ordem.

\begin{itemize}
\item[(A)] $4{,}35 \times 10^{-4} \textrm{ rad/nm}$.
\item[(B)] $5{,}26 \times 10^{-4} \textrm{ rad/nm}$.
\item[(C)] $3{,}87 \times 10^{-4} \textrm{ rad/nm}$.
\item[(D)] $2{,}19 \times 10^{-4} \textrm{ rad/nm}$.
\end{itemize}

\vspace{0.5cm}

\textcolor{red}{\textbf{Solução:}}\\

\subsection*{Dados:}
\begin{itemize}
    \item Número de fendas: \( N = 1{,}25 \times 10^4 \)
    \item Largura da rede: \( L = \SI{25,0}{mm} = 25,0 \times 10^{-3}\,\mathrm{m} \)
    \item Ordem do máximo: \( m = 1 \)
\end{itemize}

\subsection*{Passo 1: Condição para o máximo de difração}

Para um máximo de ordem \(m\), a condição de difração é:
\[
d \, \sin\theta = m\lambda
\]

Para \(m=1\) e pequenos ângulos (\( \sin\theta \approx \theta \)):
\[
\theta \approx \frac{\lambda}{d}
\]

Logo, a razão \( \theta/\lambda \) é:
\[
\frac{\theta}{\lambda} \approx \frac{1}{d}
\]

\subsection*{Passo 2: Espaçamento entre as fendas}

O espaçamento \(d\) entre fendas é dado por:
\[
d = \frac{L}{N}
\]

Substituindo os valores:
\[
d = \frac{25{,}0 \times 10^{-3}}{1{,}25 \times 10^4} = 2{,}0 \times 10^{-6}\,\mathrm{m}
\]

\subsection*{Passo 3: Calculando \( \theta/\lambda \)}

Em \(\mathrm{m}^{-1}\):
\[
\frac{\theta}{\lambda} = \frac{1}{2{,}0 \times 10^{-6}} = 5{,}0 \times 10^{5}\,\mathrm{m}^{-1}
\]

Convertendo para \(\mathrm{nm}^{-1}\), sabendo que \(1\,\mathrm{m} = 10^{9}\,\mathrm{nm}\):
\[
\frac{\theta}{\lambda} = 5{,}0 \times 10^{5} \times 10^{-9} = 5{,}0 \times 10^{-4}\,\mathrm{rad/nm}
\]

O valor mais próximo entre as alternativas é:
\[
\boxed{\frac{\theta}{\lambda} = 5{,}26 \times 10^{-4}\,\mathrm{rad/nm}}
\]


A resposta correta é alternativa \colorbox{green!50}{\textbf{B}}.
\end{flushleft}

\begin{flushleft}
\textbf{\textcolor{blue}{\Large Q46}}\\
\noindent

\subsection{Quest\~ao 46 - IFSC 2023 - Interferência/Filmes finos}

As asas das borboletas apresentam cores estruturais devido ao fenômeno de interferência causada pelas 
múltiplas reflexões internas, semelhante ao que ocorre em filmes finos. Essas cores são resultado da 
interferência construtiva e destrutiva das ondas de luz que são refletidas e transmitidas pelas camadas 
microscópicas das asas. 

A interferência construtiva ou destrutiva ocorre devido à diferença de caminho percorrido pela luz ao 
atravessar cada calha, que são as estruturas que compõem as camadas das escamas das asas da borboleta, 
e pela mudança de fase que ocorre nas múltiplas reflexões da luz. A distância entre as calhas é o principal 
fator que determina a coloração final observada.

Considere uma situação hipotética na qual a diferença de fase entre as ondas equivale a $\pi/2$. Com base 
nessas informações, no que se refere à relação entre a distância entre as calhas das asas de uma borboleta 
e a coloração exibida: haverá interferência \_\_\_\_\_\_\_\_\_\_ se a distância entre as calhas for um 
múltiplo \_\_\_\_\_\_\_\_\_\_\_\_\_\_\_\_\_\_\_.

\begin{itemize}
\item[(A)] construtiva - ímpar de meio comprimento de onda para a cor exibida
\item[(B)] construtiva - par de meio comprimento de onda para a cor exibida
\item[(C)] destrutiva - ímpar de meio comprimento de onda
\item[(D)] construtiva - par do comprimento de onda para todas as cores
\item[(E)] destrutiva - ímpar do comprimento de onda para todas as cores
\end{itemize}

\vspace{0.5cm}

\textcolor{red}{\textbf{Solução:}}\\

A interferência construtiva ocorre quando a diferença de caminho óptico resulta em uma diferença de fase de 
$2n\pi$ (múltiplos inteiros de $\lambda$). Já a interferência destrutiva ocorre para $(2n+1)\frac{\lambda}{2}$ 
(múltiplos ímpares de meio comprimento de onda).  

Neste problema, há uma diferença de fase adicional de $\pi/2$ devido à reflexão. Para que o resultado seja 
\textbf{construtivo}, a distância entre as calhas precisa compensar essa diferença extra, o que acontece quando 
o caminho óptico for um \textbf{par de meios comprimentos de onda}.  

Portanto, a resposta correta é alternativa \colorbox{green!50}{\textbf{A}}.
\end{flushleft}

\begin{flushleft}
\textbf{\textcolor{blue}{\Large Quest\~ao - }}\\
\noindent

\subsection{Quest\~ao }

\begin{itemize}
\item[(A)] 
\item[(B)] 
\item[(C)]
\item[(D)] 
\item[(E)] 
\end{itemize}

\vspace{0.5cm}

\textcolor{red}{\textbf{Solução:}}\\


A resposta correta é alternativa \colorbox{green!50}{\textbf{...}}.

\end{flushleft}