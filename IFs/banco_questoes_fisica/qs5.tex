\section{\large \textcolor{blue}{As leis da Termodinâmica}}

\begin{flushleft}
\textbf{\textcolor{blue}{\Large Quest\~ao - IFSP 2015 - Lei de Fourier da Condu\c{c}\~ao de Calor}}\\
\noindent

\subsection{Quest\~ao IFSP 2015 - Lei de Fourier da Condu\c{c}\~ao de Calor}

Em um experimento sobre condutividade térmica dos metais, uma barra metálica homogênea e de área de secção transversal uniforme, isolada termicamente do meio externo, foi colocada entre duas fontes a temperaturas diferentes ($T_A$ e $T_B$). Dois termômetros foram colocados de forma a medirem a temperatura da barra em dois pontos diferentes e estabilizaram seus valores naqueles mostrados na figura abaixo.

\vspace{0.3cm}

\includegraphics[width=0.8\textwidth]{figures/barra_termica.png}

\vspace{0.3cm}

A temperatura das fontes ($T_A$ e $T_B$) são, respectivamente:

\begin{itemize}
\item[(A)] 90\textdegree C e 20\textdegree C
\item[(B)] 125\textdegree C e 5\textdegree C
\item[(C)] 120\textdegree C e 16,6\textdegree C
\item[(D)] 95\textdegree C e 5\textdegree C
\item[(E)] 20\textdegree C e 90\textdegree C
\end{itemize}

\vspace{0.5cm}

\textcolor{red}{\textbf{Solução:}}\\

Como a barra é homogênea, de área constante e está isolada termicamente, o sistema está em equilíbrio térmico e o fluxo de calor é constante. A distribuição de temperatura é linear em cada trecho. Assim, podemos aplicar a relação:

\[
\frac{\Delta T_1}{L_1} = \frac{\Delta T_2}{L_2} = \frac{\Delta T_3}{L_3}
\]

Dividindo a barra em 3 trechos:
\begin{itemize}
\item Do ponto $T_A$ até 80\textdegree C: comprimento $x$, variação de temperatura: $T_A - 80$
\item De 80\textdegree C até 50\textdegree C: comprimento $2x$, variação de temperatura: $30$
\item De 50\textdegree C até $T_B$: comprimento $3x$, variação de temperatura: $50 - T_B$
\end{itemize}

Igualando as razões:

\[
\frac{T_A - 80}{x} = \frac{30}{2x} \Rightarrow T_A - 80 = 15 \Rightarrow T_A = 95^\circ \text{C}
\]

\[
\frac{30}{2x} = \frac{50 - T_B}{3x} \Rightarrow 15 = \frac{50 - T_B}{3} \Rightarrow 50 - T_B = 45 \Rightarrow T_B = 5^\circ \text{C}
\]

\vspace{0.3cm}

A resposta correta é a alternativa \colorbox{green!50}{\textbf{(D)}}.

\end{flushleft}


\begin{flushleft}
\textbf{\textcolor{blue}{\Large Quest\~ao 34 - IFSP 2017 - Entropia}}\\
\noindent

\subsection{Quest\~ao 34 - IFSP 2017 - Entropia}
Dois corpos de diferentes materiais e temperaturas s\~ao colocados em uma caixa termicamente isolada. 
O material 1, com 200 g e temperatura de 40\textdegree C, possui $c_1 = 300 \, \text{J/kg.K}$; e o material 2, 
com 100 g e temperatura de 100\textdegree C, possui $c_2 = 120 \, \text{J/kg.K}$. Qual a varia\c{c}\~ao de entropia do 
sistema ap\'os atingir o equil\'ibrio t\'ermico?

\begin{itemize}
\item[(A)] -0,16 J/K
\item[(B)] 0,16 J/K
\item[(C)] 5,07 J/K
\item[(D)] 72,31 J/K
\end{itemize}

\vspace{0.5cm}

\textcolor{red}{\textbf{Solução:}}\\

Como o sistema é termicamente isolado, usamos a conserva\c{c}\~ao da energia para encontrar a temperatura final de equil\'ibrio $T_f$:

\[
m_1 c_1 (T_f - T_1) + m_2 c_2 (T_f - T_2) = 0
\]

\[
0{,}2 \cdot 300 \cdot (T_f - 313{,}15) + 0{,}1 \cdot 120 \cdot (T_f - 373{,}15) = 0
\Rightarrow T_f = 323{,}15\, \text{K}
\]

A varia\c{c}\~ao de entropia total do sistema ser\'a:

\[
\Delta S = m_1 c_1 \ln \left( \frac{T_f}{T_1} \right) + m_2 c_2 \ln \left( \frac{T_f}{T_2} \right)
\]

\[
\Delta S = 0{,}2 \cdot 300 \cdot \ln\left( \frac{323{,}15}{313{,}15} \right) + 0{,}1 \cdot 120 \cdot \ln\left( \frac{323{,}15}{373{,}15} \right)
\]

\[
\Delta S \approx 60 \cdot 0{,}0314 + 12 \cdot (-0{,}1437) \approx 1{,}884 - 1{,}724 = \boxed{0{,}16 \, \text{J/K}}
\]

A resposta correta é alternativa \colorbox{green!50}{\textbf{B}}.

\end{flushleft}

\section*{Ciclos Termodinâmicos — Descrição Detalhada}

\subsection*{O que é um ciclo termodinâmico?}

Um \colorbox{yellow!40}{\textbf{ciclo termodinâmico} é uma sequência de processos termodinâmicos} realizados por um sistema (geralmente um fluido de trabalho), que retorna ao seu estado inicial ao final do ciclo.

O sistema troca calor \(Q\) com o meio externo e realiza trabalho \(W\), obedecendo à Primeira Lei da Termodinâmica:
\[
\Delta U = Q - W
\]

Como o \colorbox{yellow!40}{sistema retorna ao estado inicial (\(\Delta U = 0\))}, temos:
\[
Q_{\text{líquido}} = W_{\text{líquido}}
\]

\begin{itemize}
  \item Se \colorbox{yellow!40}{o ciclo for \textbf{motor}: transforma calor em trabalho (\(W_{\text{líquido}} > 0\)).}
  \item Se for \textbf{refrigerador/bomba de calor}: consome trabalho para transferir calor de um reservatório frio para um quente.
\end{itemize}

\subsection*{Ciclos Motores (Máquinas Térmicas)}

\subsubsection*{\colorbox{yellow!40}{Ciclo de Carnot}}

Ciclo ideal com a máxima eficiência possível entre duas temperaturas \(T_q\) (quente) e \(T_f\) (fria).

\begin{enumerate}
  \item \colorbox{green!30}{Isotérmica a \(T_q\) (expansão com entrada de calor \(Q_q\))}
  \item \colorbox{green!30}{Adiabática (expansão até \(T_f\))}
  \item \colorbox{green!30}{Isotérmica a \(T_f\) (compressão com rejeição de calor \(Q_f\))}
  \item \colorbox{green!30}{Adiabática (compressão até \(T_q\))}
\end{enumerate}

Eficiência ideal:
\[
\boxed{
\eta_C = 1 - \frac{T_f}{T_q}
}
\]

\section*{O Ciclo de Carnot é Irreversível?}

\textbf{Resposta curta:} \textbf{Não. O ciclo de Carnot é, por definição, completamente reversível.}

\subsection*{Por quê?}

O ciclo de Carnot é um modelo teórico ideal que estabelece o limite máximo de eficiência entre duas temperaturas \( T_q \) (quente) e \( T_f \) (fria). Ele é composto por quatro transformações \textbf{reversíveis}:
\begin{itemize}
  \item Duas isotérmicas reversíveis:
    \begin{itemize}
      \item Expansão isotérmica a \( T_q \) (absorve calor \( Q_q \))
      \item Compressão isotérmica a \( T_f \) (rejeita calor \( Q_f \))
    \end{itemize}
  \item Duas adiabáticas reversíveis:
    \begin{itemize}
      \item Expansão adiabática (sem troca de calor)
      \item Compressão adiabática (sem troca de calor)
    \end{itemize}
\end{itemize}

Cada processo ocorre de modo infinitamente lento, mantendo o sistema em equilíbrio e sem produção de entropia:
\[
\oint \frac{\delta Q}{T} = 0
\]

\subsection*{Na prática}

Nenhuma máquina real pode executar um ciclo de Carnot, pois:
\begin{itemize}
  \item As trocas infinitesimais de calor requerem tempo infinito.
  \item Sempre há atrito, dissipação e gradientes de temperatura.
\end{itemize}

Portanto:
\begin{center}

\textbf{Ciclo de Carnot ideal: reversível e eficiência máxima.}\\
\textbf{Máquinas reais: irreversíveis e menos eficientes.}
\end{center}

\begin{table}[h!]
\centering
\small
\caption{Comparação entre o Ciclo de Carnot e Ciclo Real}
\begin{tabular}{|c|c|c|}
\hline
\textbf{Característica} & \textbf{Ciclo de Carnot (Ideal)} & \textbf{Ciclo Real} \\ \hline
\textbf{Reversibilidade} 
& Totalmente reversível & Irreversível (perdas) \\ \hline
\textbf{Produção/entropia} 
& Zero & Maior que zero \\ \hline
\textbf{Eficiência} 
& Máxima teórica & Menor que Carnot \\ \hline
\textbf{Processos} 
& Isotérmicos/adiabáticos & Processos com dissipação\\ \hline
\textbf{Velocidade/operação} 
& Infinitamente lenta & Finita \\ \hline
\textbf{Aplicabilidade} 
& Apenas modelo teórico & Realizado em motores/máquinas \\ \hline
\end{tabular}
\end{table}


\subsubsection*{Ciclo Otto}

Modelo ideal para motores a gasolina (ignição por centelha).

\begin{enumerate}
  \item Compressão adiabática
  \item Aquecimento a volume constante (explosão da mistura combustível-ar)
  \item Expansão adiabática
  \item Resfriamento a volume constante (descarga dos gases)
\end{enumerate}

Eficiência ideal:
\[
\eta_O = 1 - \frac{1}{r^{\gamma-1}}, \quad r = \frac{V_{\text{máx}}}{V_{\text{mín}}}, \quad \gamma = \frac{c_p}{c_v}
\]

\subsubsection*{Ciclo Diesel}

Modelo para motores diesel (ignição por compressão). Difere do Otto: calor adicionado a pressão constante.

\begin{enumerate}
  \item Compressão adiabática
  \item Aquecimento a pressão constante
  \item Expansão adiabática
  \item Resfriamento a volume constante
\end{enumerate}

\subsubsection*{Ciclo de Brayton (ou Joule)}

Usado em turbinas a gás e motores a jato.

\begin{enumerate}
  \item Compressão adiabática
  \item Aquecimento a pressão constante
  \item Expansão adiabática
  \item Resfriamento a pressão constante
\end{enumerate}

\subsection*{Ciclos de Refrigeração e Bombas de Calor}

\subsubsection*{Ciclo inverso de Carnot}

Mesmo princípio do Carnot, mas “ao contrário”. Usa trabalho para transferir calor de \(T_f\) para \(T_q\).

Coeficiente de performance (COP):
\begin{itemize}
  \item Refrigerador: 
  \[
  COP_R = \frac{T_f}{T_q - T_f}
  \]
  \item Bomba de calor: 
  \[
  COP_B = \frac{T_q}{T_q - T_f}
  \]
\end{itemize}

\subsubsection*{Ciclo de Compressão de Vapor}

Usado em geladeiras e ar-condicionado.

\begin{enumerate}
  \item Compressão adiabática (fluido é comprimido e aquecido)
  \item Condensação a pressão constante (rejeita calor para o ambiente)
  \item Expansão isentrópica (queda de \(P\) e \(T\))
  \item Vaporização a pressão constante (absorve calor do ambiente interno)
\end{enumerate}

\subsection*{Resumo das Grandezas Importantes}

Eficiência térmica de uma máquina térmica:
\[
\eta = \frac{W_{\text{líquido}}}{Q_{\text{quente}}}
\]

COP para refrigeradores e bombas:
\begin{itemize}
  \item Refrigerador: \(COP_R = \frac{Q_f}{W}\)
  \item Bomba de calor: \(COP_B = \frac{Q_q}{W}\)
\end{itemize}

\subsection*{Observação Prática}

\begin{itemize}
  \item \colorbox{green!30}{Ciclos reais sempre têm perdas por atrito, irreversibilidades} e transferência de calor fora do equilíbrio — por isso a eficiência real é menor que a teórica.
  \item O \colorbox{green!30}{\textbf{Ciclo de Carnot} é um limite superior (ideal), mas impraticável na prática.}
\end{itemize}

\begin{flushleft}
\textbf{\textcolor{blue}{\Large Quest\~ao - 39 IFSC 2023 - Ciclos Termodinamicos - Diesel}}\\
\noindent

\subsection{Quest\~ao - 39 IFSC 2023 - Ciclos Termodinamicos - Diesel}

Os ciclos termodin\^amicos s\~ao fen\^omenos que envolvem a convers\~ao de energia t\'ermica em trabalho mec\^anico ou a 
realiza\c{c}\~ao de trabalho mec\^anico em um sistema. Esses ciclos abrangem uma variedade de configura\c{c}\~oes e t\^em 
aplica\c{c}\~oes em diversos campos. Um exemplo relevante \'e o ciclo Diesel, amplamente utilizado em motores de combust\~ao 
interna. Com base no exposto acima e considerando o ciclo Diesel te\'orico apresentado no gr\'afico da Figura 3 abaixo, 
relacione a Coluna 1 \`a Coluna 2.

\begin{center}
\centering
\includegraphics[scale=0.5]{figures/ciclo-termodinamico-diesel.png}
\end{center}

\bigskip

\noindent
\textbf{Coluna 1}
\begin{enumerate}
\item Curva A$\to$B
\item Curva B$\to$C
\item Curva C$\to$D
\item Curva D$\to$A
\end{enumerate}

\noindent
\textbf{Coluna 2}
\begin{itemize}
\item[(\ )] Realiza\c{c}\~ao de trabalho pelo sistema.
\item[(\ )] Transformação adiabática.
\item[(\ )] Rejei\c{c}\~ao de calor pelo sistema.
\item[(\ )] Realiza\c{c}\~ao de trabalho pelo sistema.
\end{itemize}

A ordem correta de preenchimento dos par\^enteses, de cima para baixo, \'e:

\begin{itemize}
\item[(A)] 2 -- 1 -- 4 -- 3.
\item[(B)] 3 -- 2 -- 4 -- 1.
\item[(C)] 2 -- 4 -- 1 -- 3.
\item[(D)] 3 -- 2 -- 4 -- 1.
\item[(E)] 2 -- 4 -- 2 -- 3.
\end{itemize}

\vspace{0.5cm}

\textcolor{red}{\textbf{Solução:}}\\

\textbf{Observação preliminar:} No ciclo Diesel teórico (representado no diagrama $p\times V$), as quatro transformações usuais s\~ao:
\begin{itemize}
    \item compress\~ao adiab\'atica (estado inicial $\to$ estado comprimido),
    \item aquecimento a press\~ao quase constante (adi\c{c}\~ao de calor isob\'arica),
    \item expans\~ao adiab\'atica (realiza trabalho),
    \item rejei\c{c}\~ao de calor em volume praticamente constante (isoqu\'arico/isochorico).
\end{itemize}

Agora analisamos cada curva do enunciado com base no diagrama:

\begin{enumerate}
    \item \textbf{Curva A$\to$B:} No diagrama, A est\'a em uma posi\c{c}\~ao com maior volume e menor press\~ao; ao ir para B a press\~ao aumenta e o volume diminui — trata-se de compress\~ao sem trocas de calor (adiab\'atica ideal). Portanto \textbf{A$\to$B = transforma\c{c}\~ao adiab\'atica}. (corresponde ao item \emph{Transformação adiabática}).
    \item \textbf{Curva B$\to$C:} Apresenta press\~ao aproximadamente constante enquanto o volume aumenta (seta para a direita) — caracteriza adi\c{c}\~ao de calor a press\~ao constante com expans\~ao: o sistema \textbf{realiza trabalho} sobre o meio externo. (corresponde a \emph{Realização de trabalho pelo sistema}).
    \item \textbf{Curva C$\to$D:} \'E uma expans\~ao onde a press\~ao e o volume variam com forma curva (queda de press\~ao com aumento de volume) — 
    corresponde \`a \textbf{expans\~ao adiab\'atica} (o sistema tamb\'em realiza trabalho nessa etapa). (corresponde a \emph{Realização de trabalho pelo sistema}).
    \item \textbf{Curva D$\to$A:} Apresenta varia\c{c}\~ao de press\~ao a volume praticamente constante (seta vertical) — corresponde \`a \textbf{rejei\c{c}\~ao de calor} (isoqu\'arica/isoch\'orica) que leva o sistema de volta ao estado inicial. (corresponde a \emph{Rejeição de calor pelo sistema}).
\end{enumerate}

Associando as curvas (n\'umero da Coluna 1) \`a descri\c{c}\~ao da Coluna 2 (de cima para baixo):

\begin{itemize}
    \item Realiza\c{c}\~ao de trabalho pelo sistema. \quad $\to$ Curva \textbf{2} (B$\to$C).
    \item Transformação adiabática. \quad\quad\quad\quad\ \ $\to$ Curva \textbf{1} (A$\to$B).
    \item Rejeição de calor pelo sistema. \quad\quad\ $\to$ Curva \textbf{4} (D$\to$A).
    \item Realiza\c{c}\~ao de trabalho pelo sistema. \quad $\to$ Curva \textbf{3} (C$\to$D).
\end{itemize}

Logo, a sequ\^encia \emph{de cima para baixo} \'e \(\;2\;-\;1\;-\;4\;-\;3\;\).

\vspace{0.3cm}

\textbf{Resposta:} \colorbox{green!50}{\textbf{A}}.

\end{flushleft}


\begin{flushleft}
\textbf{\textcolor{blue}{\Large Quest\~ao - }}\\
\noindent

\subsection{Quest\~ao }

\begin{itemize}
\item[(A)] 
\item[(B)] 
\item[(C)]
\item[(D)] 
\item[(E)] 
\end{itemize}

\vspace{0.5cm}

\textcolor{red}{\textbf{Solução:}}\\


A resposta correta é alternativa \colorbox{green!50}{\textbf{...}}.

\end{flushleft}

\noindent\rule{\linewidth}{0.6pt}\\
