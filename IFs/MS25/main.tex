\documentclass[a4paper,12pt]{article}
\usepackage[brazil, english]{babel}
\usepackage[utf8]{inputenc}
\usepackage[T1]{fontenc}
\usepackage{geometry}
\usepackage{setspace}
\usepackage{titlesec}
\usepackage{hyperref}
\usepackage{graphicx}
\usepackage{caption}
\usepackage{subcaption}
\usepackage{fancyhdr}
\setlength{\headheight}{15pt}
\addtolength{\topmargin}{-2.5pt}
\usepackage{xcolor}
\usepackage{amsmath, amssymb, bm}
\usepackage{mathtools}
\usepackage{cancel}
\usepackage{tikz}
\usepackage{newunicodechar}
\usepackage{ragged2e}
\usepackage{setspace}
\usepackage{tikz-3dplot} % Necessário para coordenadas 3D
\usetikzlibrary{intersections}
\usepackage{siunitx}
\usetikzlibrary{3d, arrows.meta}

\usepackage{color}
\definecolor{myblue}{rgb}{.8, .8, 1}

\usepackage{amsmath}
\usepackage{empheq}

\newlength\mytemplen
\newsavebox\mytempbox

\makeatletter
\newcommand\mybluebox{%
    \@ifnextchar[%]
       {\@mybluebox}%
       {\@mybluebox[0pt]}}

\def\@mybluebox[#1]{%
    \@ifnextchar[%]
       {\@@mybluebox[#1]}%
       {\@@mybluebox[#1][0pt]}}

\def\@@mybluebox[#1][#2]#3{
    \sbox\mytempbox{#3}%
    \mytemplen\ht\mytempbox
    \advance\mytemplen #1\relax
    \ht\mytempbox\mytemplen
    \mytemplen\dp\mytempbox
    \advance\mytemplen #2\relax
    \dp\mytempbox\mytemplen
    \colorbox{myblue}{\hspace{1em}\usebox{\mytempbox}\hspace{1em}}}
\makeatother

\usepackage[most]{tcolorbox}

\newtcbox{\mymath}[1][]{%
    nobeforeafter, math upper, tcbox raise base,
    enhanced, colframe=blue!30!black,
    colback=blue!30, boxrule=1pt,
    #1}

\tcbset{
    highlight math style={
        enhanced,
        colframe=red!60!black,
        colback=yellow!50,
        arc=4pt,
        boxrule=1pt,
        drop fuzzy shadow
    }
    }

\usepackage{physics}
\usepackage{pgfplots}
\pgfplotsset{compat=1.17}

\linespread{1.5}

\definecolor{ao(english)}{rgb}{0.0, 0.5, 0.0}
\definecolor{byzantium}{rgb}{0.44, 0.16, 0.39}
\newunicodechar{∘}{\circ}

%%%%%%%%%%%%%%%%%%%%%%%%%%%%%%%%%%%%%%%%%%%%%%%%%%
% These are some new commands that may be useful 
% for paper writing in general. If other new commands
% are needed for your specific paper, please feel 
% free to add here. 
%
% The currently available commands are organized in: 
% 1) Systems
% 2) Quantities
% 3) Energies and units
% 4) particle species
% 5) Colors package
% 6) hyperlink
%%%%%%%%%%%%%%%%%%%%%%%%%%%%%%%%%%%%%%%%%%%%%%%%%%

\usepackage{amsmath}
\usepackage{amssymb}
\usepackage{upgreek}
\usepackage{multirow}
\usepackage{setspace}% http://ctan.org/pkg/setspace
\usepackage{fancyhdr}
\usepackage{datetime}

% 1) SYSTEMS
\newcommand{\btc}               {\textbf{BTC}}
\newcommand{\btcspace}          {\textbf{BTC} }
\newcommand{\pow}               {\textbf{PoW}}

% 4) definition to references, biblatex and hyperlink
\usepackage[backend=bibtex, 
style=nature,  %style reference.
sorting=none,
firstinits=true %first name abbreviate
]{biblatex}

\usepackage{hyperref}
\hypersetup{
    colorlinks=true, %set "true" if you want colored links
    linktoc=all,     %set to "all" if you want both sections and subsections linked
    linkcolor=blue,  %choose some color if you want links to stand out
    citecolor= blue, % color of \cite{} in the text.
    urlcolor  = blue, % color of the link for the paper in references.
}

% 5) Tikz and figures
\usepackage{epsfig}
\usepackage{lmodern}
\usepackage{mathtools}
\usepackage[utf8]{luainputenc}
\usepackage{xspace}
\usepackage{tikz}
\usepackage{pgfplots}
\pgfplotsset{compat=newest}

\usetikzlibrary{positioning}
\usepackage{subcaption}

% 6) colors:
\usepackage{xcolor}
\definecolor{ao(english)}{rgb}{0.0, 0.5, 0.0} % dark green

% 7) Add lines numbers
%\usepackage{lineno}

% add pdf file to thesis:
\usepackage{pdfpages}

\hypersetup{
    colorlinks=true,% make the links colored
    linkcolor=blue
}

\usepackage{setspace}
\addbibresource{bibliography.bib}

\newcommand{\printingbibliography}{%

    \pagestyle{myheadings}
    \markright{}
    \sloppy
    \printbibliography[heading=bibintoc, % add to table of contents
                   title=Refer\^encias % Chapter name
                  ]
    \fussy%
}
\PassOptionsToPackage{table}{xcolor}

\pagestyle{fancy}
\fancyhf{}
\renewcommand{\headrulewidth}{0pt}
\fancyhead[R]{\thepage}

\geometry{a4paper,top=30mm,bottom=20mm,left=30mm,right=20mm}

\titleformat*{\section}{\bfseries\large}
\titleformat*{\subsection}{\bfseries\normalsize}

\title{ \textbf{\large IFMS 2025 - Concurso N$^{\circ}$ 20/2025 - EBTT - F\'isica}}
\author{Andr\'e V. Silva \\ \texttt{\url{www.andrevsilva.com}}}
\date{\today}

\begin{document}

\maketitle
\noindent\rule{\linewidth}{0.4pt}\\

\justifying

\begin{flushleft}
\textbf{\textcolor{blue}{\Large Q11}}\\
\noindent
Uma usina termelétrica opera um ciclo de Carnot entre dois reservatórios térmicos: um a 
\SI{800}{\kelvin} e outro a \SI{300}{\kelvin}. A usina recebe \SI{500}{\mega\joule} de calor da 
fonte quente por ciclo e realiza trabalho sobre um gerador elétrico. No entanto, devido a perdas 
operacionais e imperfeições no sistema, a eficiência real da usina é 60\% da eficiência teórica do ciclo 
de Carnot. Com base nessas informações, qual é o trabalho efetivo realizado pela usina em cada ciclo?

\begin{itemize}
\item[(A)] 90 MJ.
\item[(B)] 25 MJ.
\item[(C)] 300 MJ.
\item[(D)] 312,5 MJ.
\item[(E)] 187,5 MJ.
\end{itemize}

\vspace{0.5cm}

\textcolor{red}{\textbf{Solução:}}\\

\begin{itemize}
    \item Temperatura da fonte quente: \( T_q = \SI{800}{\kelvin} \)
    \item Temperatura da fonte fria: \( T_f = \SI{300}{\kelvin} \)
    \item Calor recebido por ciclo: \( Q_q = \SI{500}{\mega\joule} \)
    \item Eficiência real: \( \eta_{\text{real}} = 0{,}60 \cdot \eta_{\text{Carnot}} \)
\end{itemize}

\vspace{1em}
A eficiência teórica do ciclo de Carnot é dada por:

\[
\eta_{\text{Carnot}} = 1 - \frac{T_f}{T_q} = 1 - \frac{300}{800} = 1 - 0{,}375 = 0{,}625
\]

Eficiência real da usina:

\[
\eta_{\text{real}} = 0{,}60 \cdot 0{,}625 = 0{,}375
\]

O trabalho efetivo realizado por ciclo é:

\[
W = \eta_{\text{real}} \cdot Q_q = 0{,}375 \cdot \SI{500}{\mega\joule} = \SI{187.5}{\mega\joule}
\]

\[
\boxed{W = \SI{187.5}{\mega\joule}}
\]

A resposta correta é alternativa \colorbox{green!50}{\textbf{E}}.

\end{flushleft}
\noindent\rule{\linewidth}{0.4pt}\\
\begin{flushleft}
\textbf{\textcolor{blue}{\Large Q12}}\\
Teoria da Relatividade Restrita de Einstein
trouxe mudanças profundas na compreensão do
espaço e do tempo. Um dos conceitos
fundamentais é a dilatação temporal, que implica
que o tempo não é absoluto e depende do
referencial do observador.
Tendo isso em vista, considere que dois
observadores, A e B, estejam analisando o
movimento de uma partícula. O observador A está
em repouso em um laboratório na Terra, enquanto
o observador B viaja em uma nave a uma
velocidade relativística v em relação a A. Com
base nas previsões da Relatividade Restrita, é
correto afirmar 

\textcolor{red}{\textbf{Solução:}}\\

\textbf{(A)} o tempo medido pelo observador B será sempre15
menor do que o tempo medido pelo observador
A, independentemente da velocidade da nave.\\
\colorbox{green!50}{\textbf{(B)}} a dilatação do tempo significa que um relógio em
movimento em relação a um referencial inercial
sempre parecerá atrasado em relação a um
relógio em repouso nesse referencial.\\
\textbf{(C)} se a nave de B viajar a uma velocidade maior do
que a velocidade da luz no vácuo, o fator de
Lorentz se tornaria negativo, implicando a
possibilidade de viajar para o passado.\\
\textbf{(D)} o efeito da dilatação do tempo desaparece
completamente quando a velocidade relativa
entre A e B é menor do que a metade da
velocidade da luz no vácuo.\\
\textbf{(E)} a dilatação temporal ocorre apenas quando a
velocidade relativa entre dois referenciais é
superior a 80\% da velocidade da luz no vácuo.

\end{flushleft}

%%%%%%%% Bibliography 
% Os comandos para incluir as referências bibliográficas
%\printingbibliography

\end{document}
