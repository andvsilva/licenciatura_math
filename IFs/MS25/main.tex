\documentclass[a4paper,12pt]{article}
\usepackage[brazil, english]{babel}
\usepackage[utf8]{inputenc}
\usepackage[T1]{fontenc}
\usepackage{geometry}
\usepackage{setspace}
\usepackage{titlesec}
\usepackage{hyperref}
\usepackage{graphicx}
\usepackage{caption}
\usepackage{subcaption}
\usepackage{fancyhdr}
\setlength{\headheight}{15pt}
\addtolength{\topmargin}{-2.5pt}
\usepackage{xcolor}
\usepackage{amsmath, amssymb, bm}
\usepackage{mathtools}
\usepackage{cancel}
\usepackage{tikz}
\usepackage{newunicodechar}
\usepackage{ragged2e}
\usepackage{setspace}
\usepackage{tikz-3dplot} % Necessário para coordenadas 3D
\usetikzlibrary{intersections}
\usepackage{siunitx}
\usetikzlibrary{3d, arrows.meta}

\usepackage{color}
\definecolor{myblue}{rgb}{.8, .8, 1}

\usepackage{amsmath}
\usepackage{empheq}

\newlength\mytemplen
\newsavebox\mytempbox

\makeatletter
\newcommand\mybluebox{%
    \@ifnextchar[%]
       {\@mybluebox}%
       {\@mybluebox[0pt]}}

\def\@mybluebox[#1]{%
    \@ifnextchar[%]
       {\@@mybluebox[#1]}%
       {\@@mybluebox[#1][0pt]}}

\def\@@mybluebox[#1][#2]#3{
    \sbox\mytempbox{#3}%
    \mytemplen\ht\mytempbox
    \advance\mytemplen #1\relax
    \ht\mytempbox\mytemplen
    \mytemplen\dp\mytempbox
    \advance\mytemplen #2\relax
    \dp\mytempbox\mytemplen
    \colorbox{myblue}{\hspace{1em}\usebox{\mytempbox}\hspace{1em}}}
\makeatother

\usepackage[most]{tcolorbox}

\newtcbox{\mymath}[1][]{%
    nobeforeafter, math upper, tcbox raise base,
    enhanced, colframe=blue!30!black,
    colback=blue!30, boxrule=1pt,
    #1}

\tcbset{
    highlight math style={
        enhanced,
        colframe=red!60!black,
        colback=yellow!50,
        arc=4pt,
        boxrule=1pt,
        drop fuzzy shadow
    }
    }

\usepackage{physics}
\usepackage{pgfplots}
\pgfplotsset{compat=1.17}

\linespread{1.5}

\definecolor{ao(english)}{rgb}{0.0, 0.5, 0.0}
\definecolor{byzantium}{rgb}{0.44, 0.16, 0.39}
\newunicodechar{∘}{\circ}

%%%%%%%%%%%%%%%%%%%%%%%%%%%%%%%%%%%%%%%%%%%%%%%%%%
% These are some new commands that may be useful 
% for paper writing in general. If other new commands
% are needed for your specific paper, please feel 
% free to add here. 
%
% The currently available commands are organized in: 
% 1) Systems
% 2) Quantities
% 3) Energies and units
% 4) particle species
% 5) Colors package
% 6) hyperlink
%%%%%%%%%%%%%%%%%%%%%%%%%%%%%%%%%%%%%%%%%%%%%%%%%%

\usepackage{amsmath}
\usepackage{amssymb}
\usepackage{upgreek}
\usepackage{multirow}
\usepackage{setspace}% http://ctan.org/pkg/setspace
\usepackage{fancyhdr}
\usepackage{datetime}

% 1) SYSTEMS
\newcommand{\btc}               {\textbf{BTC}}
\newcommand{\btcspace}          {\textbf{BTC} }
\newcommand{\pow}               {\textbf{PoW}}

% 4) definition to references, biblatex and hyperlink
\usepackage[backend=bibtex, 
style=nature,  %style reference.
sorting=none,
firstinits=true %first name abbreviate
]{biblatex}

\usepackage{hyperref}
\hypersetup{
    colorlinks=true, %set "true" if you want colored links
    linktoc=all,     %set to "all" if you want both sections and subsections linked
    linkcolor=blue,  %choose some color if you want links to stand out
    citecolor= blue, % color of \cite{} in the text.
    urlcolor  = blue, % color of the link for the paper in references.
}

% 5) Tikz and figures
\usepackage{epsfig}
\usepackage{lmodern}
\usepackage{mathtools}
\usepackage[utf8]{luainputenc}
\usepackage{xspace}
\usepackage{tikz}
\usepackage{pgfplots}
\pgfplotsset{compat=newest}

\usetikzlibrary{positioning}
\usepackage{subcaption}

% 6) colors:
\usepackage{xcolor}
\definecolor{ao(english)}{rgb}{0.0, 0.5, 0.0} % dark green

% 7) Add lines numbers
%\usepackage{lineno}

% add pdf file to thesis:
\usepackage{pdfpages}

\hypersetup{
    colorlinks=true,% make the links colored
    linkcolor=blue
}

\usepackage{setspace}
\addbibresource{bibliography.bib}

\newcommand{\printingbibliography}{%

    \pagestyle{myheadings}
    \markright{}
    \sloppy
    \printbibliography[heading=bibintoc, % add to table of contents
                   title=Refer\^encias % Chapter name
                  ]
    \fussy%
}
\PassOptionsToPackage{table}{xcolor}

\pagestyle{fancy}
\fancyhf{}
\renewcommand{\headrulewidth}{0pt}
\fancyhead[R]{\thepage}

\geometry{a4paper,top=30mm,bottom=20mm,left=30mm,right=20mm}

\titleformat*{\section}{\bfseries\large}
\titleformat*{\subsection}{\bfseries\normalsize}

\title{ \textbf{\large IFMS 2025 - Concurso N$^{\circ}$ 20/2025 - EBTT - F\'isica}}
\author{Andr\'e V. Silva \\ \texttt{\url{www.andrevsilva.com}}}
\date{\today}

\begin{document}

\maketitle
\noindent\rule{\linewidth}{0.4pt}\\

\justifying

\begin{flushleft}
\textbf{\textcolor{blue}{\Large Q11}}\\
\noindent
Uma usina termelétrica opera um ciclo de Carnot entre dois reservatórios térmicos: um a 
\SI{800}{\kelvin} e outro a \SI{300}{\kelvin}. A usina recebe \SI{500}{\mega\joule} de calor da 
fonte quente por ciclo e realiza trabalho sobre um gerador elétrico. No entanto, devido a perdas 
operacionais e imperfeições no sistema, a eficiência real da usina é 60\% da eficiência teórica do ciclo 
de Carnot. Com base nessas informações, qual é o trabalho efetivo realizado pela usina em cada ciclo?

\begin{itemize}
\item[(A)] 90 MJ.
\item[(B)] 25 MJ.
\item[(C)] 300 MJ.
\item[(D)] 312,5 MJ.
\item[(E)] 187,5 MJ.
\end{itemize}

\vspace{0.5cm}

\textcolor{red}{\textbf{Solução:}}\\

\begin{itemize}
    \item Temperatura da fonte quente: \( T_q = \SI{800}{\kelvin} \)
    \item Temperatura da fonte fria: \( T_f = \SI{300}{\kelvin} \)
    \item Calor recebido por ciclo: \( Q_q = \SI{500}{\mega\joule} \)
    \item Eficiência real: \( \eta_{\text{real}} = 0{,}60 \cdot \eta_{\text{Carnot}} \)
\end{itemize}

\vspace{1em}
A eficiência teórica do ciclo de Carnot é dada por:

\[
\eta_{\text{Carnot}} = 1 - \frac{T_f}{T_q} = 1 - \frac{300}{800} = 1 - 0{,}375 = 0{,}625
\]

Eficiência real da usina:

\[
\eta_{\text{real}} = 0{,}60 \cdot 0{,}625 = 0{,}375
\]

O trabalho efetivo realizado por ciclo é:

\[
W = \eta_{\text{real}} \cdot Q_q = 0{,}375 \cdot \SI{500}{\mega\joule} = \SI{187.5}{\mega\joule}
\]

\[
\boxed{W = \SI{187.5}{\mega\joule}}
\]

A resposta correta é alternativa \colorbox{green!50}{\textbf{E}}.

\end{flushleft}
\noindent\rule{\linewidth}{0.4pt}\\
\begin{flushleft}
\textbf{\textcolor{blue}{\Large Q12}}\\
Teoria da Relatividade Restrita de Einstein
trouxe mudanças profundas na compreensão do
espaço e do tempo. Um dos conceitos
fundamentais é a dilatação temporal, que implica
que o tempo não é absoluto e depende do
referencial do observador.
Tendo isso em vista, considere que dois
observadores, A e B, estejam analisando o
movimento de uma partícula. O observador A está
em repouso em um laboratório na Terra, enquanto
o observador B viaja em uma nave a uma
velocidade relativística v em relação a A. Com
base nas previsões da Relatividade Restrita, é
correto afirmar 

\textcolor{red}{\textbf{Solução:}}\\

\textbf{(A)} o tempo medido pelo observador B será sempre15
menor do que o tempo medido pelo observador
A, independentemente da velocidade da nave.\\
\colorbox{green!50}{\textbf{(B)}} a dilatação do tempo significa que um relógio em
movimento em relação a um referencial inercial
sempre parecerá atrasado em relação a um
relógio em repouso nesse referencial.\\
\textbf{(C)} se a nave de B viajar a uma velocidade maior do
que a velocidade da luz no vácuo, o fator de
Lorentz se tornaria negativo, implicando a
possibilidade de viajar para o passado.\\
\textbf{(D)} o efeito da dilatação do tempo desaparece
completamente quando a velocidade relativa
entre A e B é menor do que a metade da
velocidade da luz no vácuo.\\
\textbf{(E)} a dilatação temporal ocorre apenas quando a
velocidade relativa entre dois referenciais é
superior a 80\% da velocidade da luz no vácuo.

\end{flushleft}


\begin{flushleft}
\textbf{\textcolor{blue}{\Large Q13}}\\
Uma boia no oceano oscila verticalmente devido à passagem de ondas periódicas de comprimento de onda 
igual a \(20\,\text{m}\) e frequência de \(0{,}5\,\text{Hz}\). Um barco se aproxima da boia em linha 
reta com velocidade constante de \(10\,\text{m/s}\), movendo-se na direção oposta à propagação das ondas.

Com base no exposto, determine a frequência das ondas que atingem o barco e assinale a alternativa correta.

\begin{itemize}
\item[(A)] 0,65 Hz 
\item[(B)] 0,75 Hz.
\item[(C)] 0,85 Hz.
\item[(D)] 0,90 Hz.
\item[(E)] 1,00 Hz.
\end{itemize}

\vspace{0.5cm}

\textcolor{red}{\textbf{Solução:}}\\

\noindent
Sabemos que a frequência observada por um receptor em movimento, no caso de \colorbox{yellow!20}{ondas} \colorbox{yellow!20}{mecânicas (como ondas do mar), 
é dada pela fórmula do \textbf{efeito Doppler}:}

\begin{equation}
f' = f_0 \cdot \left( \frac{v + v_o}{v} \right)
\end{equation}

\noindent
onde:
\begin{itemize}
  \item \( f' \) é a frequência observada pelo barco,
  \item \( f_0 = 0{,}5\,\text{Hz} \) é a frequência da onda percebida pela boia (fonte estacionária),
  \item \( v \) é a velocidade de propagação da onda,
  \item \( v_o = 10\,\text{m/s} \) é a velocidade do barco (\textbf{positiva}, pois o barco se aproxima da fonte).
\end{itemize}

\noindent
Como o comprimento de onda é \( \lambda = 20\,\text{m} \) e a frequência \( f_0 = 0{,}5\,\text{Hz} \), podemos calcular a velocidade da onda:

\begin{equation}
v = \lambda \cdot f_0 = 20 \cdot 0{,}5 = 10\,\text{m/s}
\end{equation}

\noindent
Substituindo os valores na equação do efeito Doppler:

\begin{equation}
f' = 0{,}5 \cdot \left( \frac{10 + 10}{10} \right) = 0{,}5 \cdot \left( \frac{20}{10} \right) = 0{,}5 \cdot 2 = 1{,}0\,\text{Hz}
\end{equation}

\noindent
\textbf{Resposta:} A frequência das ondas percebida pelo barco é \colorbox{green!50}{\textbf{E}}: \( \boxed{1{,}0\,\text{Hz}} \).

\end{flushleft}

\begin{flushleft}
\textbf{\textcolor{blue}{\Large Q14}}\\

Uma indústria química deseja preparar uma solução misturando dois líquidos miscíveis: um solvente A 
com densidade \( \rho_A = 0{,}80\,\text{g/cm}^3 \) e um solvente B com densidade \( \rho_B = 1{,}20\,\text{g/cm}^3 \).
No preparo, os técnicos misturam \(1{,}2\,\text{L}\) do solvente A com \(0{,}8\,\text{L}\) do solvente B. 
Entretanto, devido às interações moleculares, ocorre uma contração volumétrica de 5\% no volume total da mistura.
Com base nessas informações, determine o valor aproximado da densidade final da mistura e assinale a alternativa correta.


\begin{itemize}
\item[(A)] 0,96 g/cm³.
\item[(B)] 1,01 g/cm³.
\item[(C)] 1,04 g/cm³.
\item[(D)] 1,08 g/cm³.
\item[(E)] 1,12 g/cm³.
\end{itemize}

\vspace{0.5cm}

\textcolor{red}{\textbf{Solução:}}\\

\noindent
Vamos calcular a densidade final da mistura considerando:

\begin{itemize}
  \item Solvente A: densidade \( \rho_A = 0{,}80\,\text{g/cm}^3 \), volume \( V_A = 1{,}2\,\text{L} = 1200\,\text{cm}^3 \)
  \item Solvente B: densidade \( \rho_B = 1{,}20\,\text{g/cm}^3 \), volume \( V_B = 0{,}8\,\text{L} = 800\,\text{cm}^3 \)
\end{itemize}

\noindent
Calculamos as massas dos dois solventes:

\begin{align*}
m_A &= \rho_A \cdot V_A = 0{,}80 \cdot 1200 = 960\,\text{g} \\
m_B &= \rho_B \cdot V_B = 1{,}20 \cdot 800 = 960\,\text{g}
\end{align*}

\noindent
A massa total da mistura é:

\[
m_{\text{total}} = m_A + m_B = 960 + 960 = 1920\,\text{g}
\]

\noindent
O volume inicial da mistura seria:

\[
V_{\text{inicial}} = V_A + V_B = 1200 + 800 = 2000\,\text{cm}^3
\]

\noindent
Como ocorre uma contração volumétrica de 5\%, o volume final da mistura é:

\begin{align*}
V_{\text{final}} &= V_{\text{inicial}} \cdot (1 - 0{,}05) \\
&= 2000 \cdot 0{,}95 = 1900\,\text{cm}^3
\end{align*}

\noindent
Agora, calculamos a densidade final da mistura:

\[
\rho_{\text{mistura}} = \frac{m_{\text{total}}}{V_{\text{final}}} = \frac{1920}{1900} \approx 1{,}01\,\text{g/cm}^3
\]

\noindent
A densidade final da mistura é aproximadamente \( \boxed{1{,}01\,\text{g/cm}^3} \), alternativa \colorbox{green!50}{\textbf{B}}.

\end{flushleft}

\begin{flushleft}
\textbf{\textcolor{blue}{\Large Q15}}\\
Duas cargas puntiformes \( q_1 = +4\,\mu\text{C} \) e \( q_2 = -2\,\mu\text{C} \) estão fixas no vácuo a uma distância de \( 0{,}6\,\text{m} \) uma da outra. Um ponto \( P \) está localizado no ponto médio entre as duas cargas.

Sabendo que a constante eletrostática no vácuo é \( k = 9{,}0 \times 10^9\,\text{N} \cdot \text{m}^2/\text{C}^2 \), determine:

\begin{itemize}
    \item \colorbox{green!20}{o campo elétrico resultante no ponto \( P \);}
    \item \colorbox{green!20}{o potencial elétrico no ponto \( P \).}
\end{itemize}

Assinale a alternativa correta.
\begin{itemize}
\item[(A)] 2,0.10$^{5}$ N/C, 6,0.10$^{3}$V.
\item[(B)] 6,0.10$^{5}$ N/C, 1,8.10$^{5}$V.
\item[(C)] 3,0.10$^{5}$ N/C, -6,0.10$^{3}$V.
\item[(D)] 6,0.10$^{5}$ N/C, 6,0.10$^{4}$V.
\item[(E)] 6,0.10$^{5}$ N/C, -6,0.10$^{4}$V.
\end{itemize}

\vspace{0.5cm}

\textcolor{red}{\textbf{Solução:}}\\

As cargas são \( q_1 = +4\,\mu\text{C} = 4 \times 10^{-6}\,\text{C} \) e \( q_2 = -2\,\mu\text{C} = -2 \times 10^{-6}\,\text{C} \), 
separadas por uma distância de \( 0{,}6\,\text{m} \). O ponto \( P \) está no ponto médio entre elas, ou seja, a \( d = 0{,}3\,\text{m} \) de cada carga.

\vspace{0.5cm}
\textbf{1) Campo Elétrico no ponto \( P \):}

A direção do campo elétrico gerado por uma carga positiva é para fora da carga, e por uma carga negativa, é para dentro da carga. Assim:

- O campo elétrico devido a \( q_1 \) no ponto \( P \) aponta para a direita.
- O campo elétrico devido a \( q_2 \) no ponto \( P \) também aponta para a direita (pois é negativo e o campo aponta na direção oposta à carga).

Ambos os campos têm mesma direção e sentido, então somamos os módulos:

\[
E_1 = k \frac{|q_1|}{d^2} = 9{,}0 \times 10^9 \cdot \frac{4 \times 10^{-6}}{(0{,}3)^2}
= 9{,}0 \times 10^9 \cdot \frac{4 \times 10^{-6}}{0{,}09}
= 4{,}0 \times 10^5\,\text{N/C}
\]

\[
E_2 = k \frac{|q_2|}{d^2} = 9{,}0 \times 10^9 \cdot \frac{2 \times 10^{-6}}{(0{,}3)^2}
= 9{,}0 \times 10^9 \cdot \frac{2 \times 10^{-6}}{0{,}09}
= 2{,}0 \times 10^5\,\text{N/C}
\]

\[
E_{\text{total}} = E_1 + E_2 = 4{,}0 \times 10^5 + 2{,}0 \times 10^5 = 6{,}0 \times 10^5\,\text{N/C} \quad (\text{para a direita})
\]

\vspace{0.5cm}
\textbf{2) Potencial Elétrico no ponto \( P \):}

O potencial elétrico é uma grandeza escalar, então somamos algebricamente:

\[
V = V_1 + V_2
= k \frac{q_1}{d} + k \frac{q_2}{d}
= k \cdot \left( \frac{q_1 + q_2}{d} \right)
\]

\[
V = 9{,}0 \times 10^9 \cdot \frac{(4 - 2) \times 10^{-6}}{0{,}3}
= 9{,}0 \times 10^9 \cdot \frac{2 \times 10^{-6}}{0{,}3}
= 6{,}0 \times 10^4\,\text{V}
\]

\vspace{0.5cm}
\textbf{Resposta final:}

\begin{itemize}
    \item Campo elétrico: \( \boxed{6{,}0 \times 10^5\,\text{N/C}} \)
    \item Potencial elétrico: \( \boxed{6{,}0 \times 10^4\,\text{V}} \)
\end{itemize}

Alternativa correta: \colorbox{green!50}{\textbf{D)}}

\end{flushleft}

\begin{flushleft}
\textbf{\textcolor{blue}{\Large Q16}}\\
Em um experimento de eletromagnetismo, um estudante conecta um solenoide longo 
a uma fonte de corrente contínua (CC) e observa a geração de um campo magnético 
em seu interior. O solenoide possui \( 500 \) espiras, um comprimento de \( 25\,\text{cm} \) 
e é percorrido por uma corrente elétrica de \( 2{,}0\,\text{A} \). Sabendo que a 
permeabilidade magnética do vácuo é \( \mu_0 = 12 \times 10^{-7}\,\text{T} \cdot \text{m/A} \), 
determine a intensidade do campo magnético no interior do solenoide e assinale a alternativa correta.

\begin{itemize}
\item[(A)] 4,8 $\times$ 10$^{-3}$ T
\item[(B)] 5,0 $\times$ 10$^{-3}$ T
\item[(C)] 6,3 $\times$ 10$^{-3}$ T
\item[(D)] 8,0 $\times$ 10$^{-3}$ T
\item[(E)] 9,5 $\times$ 10$^{-3}$ T
\end{itemize}

\vspace{0.5cm}

\textcolor{red}{\textbf{Solução:}}\\

O campo magnético \( B \) no interior de um solenoide ideal (longo) é dado por:

\[
B = \mu_0 \cdot n \cdot I
\]

onde:
\begin{itemize}
    \item \( \mu_0 = 12 \times 10^{-7}\,\text{T} \cdot \text{m/A} \) (permeabilidade magnética do vácuo),
    \item \( n = \dfrac{N}{L} \) é a densidade linear de espiras (número de espiras por metro),
    \item \( N = 500 \) é o número total de espiras,
    \item \( L = 25\,\text{cm} = 0{,}25\,\text{m} \) é o comprimento do solenoide,
    \item \( I = 2{,}0\,\text{A} \) é a corrente que percorre o solenoide.
\end{itemize}

Calculando \colorbox{green!20}{a densidade linear de espiras:}

\[
n = \frac{N}{L} = \frac{500}{0{,}25} = 2000\,\text{espiras/m}
\]

Substituindo os valores na fórmula do campo magnético:

\[
B = \mu_0 \cdot n \cdot I
= (12 \times 10^{-7}) \cdot (2000) \cdot (2{,}0)
\]

\[
B = 12 \cdot 2000 \cdot 2 \times 10^{-7}
= 48000 \times 10^{-7}
= 4{,}8 \times 10^{-3}\,\text{T}
\]

A intensidade do campo magnético no interior do solenoide é:

\[
\boxed{4{,}8 \times 10^{-3}\,\text{T}}
\]

Alternativa correta: \colorbox{green!50}{\textbf{A)}}

\end{flushleft}

\begin{flushleft}
\textbf{\textcolor{blue}{\Large Q17}}\\
A temperatura \( T \) de um reservatório de água, em graus Celsius, varia com o tempo \( t \), em horas, de acordo com a função quadrática:

\[
T(t) = -2t^2 + 12t + 20
\]

Diante disso, assinale a alternativa que apresenta o instante \( t \) em que a temperatura atinge seu valor máximo.
\begin{itemize}
\item[(A)] 2 horas.
\item[(B)] 3 horas.
\item[(C)] 4 horas.
\item[(D)] 5 horas.
\item[(E)] 6 horas.
\end{itemize}

\vspace{0.5cm}

\textcolor{red}{\textbf{Solução:}}\\

A função que descreve a temperatura em função do tempo é dada por:

\[
T(t) = -2t^2 + 12t + 20
\]

Essa é uma função quadrática da forma geral:

\[
T(t) = at^2 + bt + c
\]

com os coeficientes:
\[
a = -2, \quad b = 12, \quad c = 20
\]

Como o coeficiente \( a \) é negativo, a parábola é voltada para baixo, o que significa que o valor máximo da função ocorre no vértice da parábola.

O tempo \( t \) em que a temperatura atinge seu valor máximo é dado pela fórmula do vértice:

\[
t = -\frac{b}{2a}
\]

Substituindo os valores:

\[
t = -\frac{12}{2 \cdot (-2)} = -\frac{12}{-4} = 3
\]

\textbf{Portanto, a temperatura atinge seu valor máximo no instante \( t = 3 \) horas.}

Alternativa correta: \colorbox{green!50}{\textbf{B)}}

\end{flushleft}


\begin{flushleft}
\textbf{\textcolor{blue}{\Large Q18}}\\
A potência fornecida por uma fonte de calor depende do tempo conforme a função
P(t) = 100 + 20t, em que t está em minutos e P em Watts. Essa fonte é usada para aquecer uma
amostra de água, aumentando sua temperatura em 75$^{\circ}$C ao longo de 5 minutos. Considere que
toda a energia fornecida pela fonte tenha sido transferida integralmente para a amostra. Tendo
isso em vista, determine a massa da amostra em gramas e assinale a alternativa correta.
Dados:
Calor específico da água: 1 cal/g$^{\circ}$C.
1 cal = 4 J.

\begin{itemize}
\item[(A)] 50g.
\item[(B)] 150g.
\item[(C)] 300g.
\item[(D)] 450g.
\item[(E)] 600g.
\end{itemize}

\vspace{0.5cm}

\textcolor{red}{\textbf{Solução:}}\\

A potência fornecida por uma fonte de calor varia com o tempo segundo a função:

\[
P(t) = 100 + 20t
\]

onde $t$ está em \textbf{minutos} e $P(t)$ em \textbf{watts} (1 W = 1 J/s).

Como a unidade de tempo padrão no SI é o segundo, devemos reescrever a função usando $t$ em segundos.

Sabemos que $$1\,\text{min} = 60\,\text{s} \Rightarrow t_{\text{min}} = \frac{t_{\text{s}}}{60}$$

\[
P(t_{\text{s}}) = 100 + 20 \cdot \left(\frac{t_{\text{s}}}{60}\right)
= 100 + \frac{t_{\text{s}}}{3}
\]

Agora calculamos a energia fornecida pela fonte ao longo de 5 minutos ($300\,\text{s}$):

\[
E = \int_0^{300} \left(100 + \frac{t}{3} \right) dt 
\]
\[
E = \left[100t + \frac{t^2}{6} \right]_0^{300}
\]

\[
E = 100 \cdot 300 + \frac{300^2}{6} = 30000 + \frac{90000}{6}
\]
\[
E = 30000 + 15000 = 45000\,\text{J}
\]

Sabemos que essa energia foi integralmente utilizada para aquecer a água.

\textbf{Convertendo para calorias:}

\[
Q = \frac{45000}{4} = 11250\,\text{cal}
\]

\textbf{Usando a equação do calor:}

\[
Q = m \cdot c \cdot \Delta \theta
\]

onde:

\begin{itemize}
    \item $Q = 11250\,\text{cal}$
    \item $c = 1\,\text{cal/g}^\circ\text{C}$
    \item $\Delta \theta = 75^\circ\text{C}$
\end{itemize}

\[
11250 = m \cdot 1 \cdot 75 \Rightarrow m = \frac{11250}{75} = \boxed{150\,\text{g}}
\]

\vspace{0.3cm}
\textbf{Resposta final:} \fbox{150 g}, alternativa \colorbox{green!50}{\textbf{B}}.

\end{flushleft}


\begin{flushleft}
\textbf{\textcolor{blue}{\Large Q19}}\\


\begin{itemize}
\item[(A)] 
\item[(B)] 
\item[(C)] 
\item[(D)] 
\item[(E)] 
\end{itemize}

\vspace{0.5cm}

\textcolor{red}{\textbf{Solução:}}\\

\end{flushleft}

\begin{flushleft}
\textbf{\textcolor{blue}{\Large Q20}}\\

\begin{itemize}
\item[(A)] 
\item[(B)] 
\item[(C)] 
\item[(D)] 
\item[(E)] 
\end{itemize}

\vspace{0.5cm}

\textcolor{red}{\textbf{Solução:}}\\

\end{flushleft}


\begin{flushleft}
\textbf{\textcolor{blue}{\Large Q21}}\\

\begin{itemize}
\item[(A)] 
\item[(B)] 
\item[(C)] 
\item[(D)] 
\item[(E)] 
\end{itemize}

\vspace{0.5cm}

\textcolor{red}{\textbf{Solução:}}\\

\end{flushleft}

%%%%%%%% Bibliography 
% Os comandos para incluir as referências bibliográficas
%\printingbibliography

\end{document}
