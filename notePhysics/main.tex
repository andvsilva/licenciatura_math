\documentclass[a4paper,12pt]{article}
\usepackage[brazil, english]{babel}
\usepackage[utf8]{inputenc}
\usepackage[T1]{fontenc}
\usepackage{geometry}
\usepackage{setspace}
\usepackage{titlesec}
\usepackage{hyperref}
\usepackage{graphicx}
\usepackage{caption}
\usepackage{subcaption}
\usepackage{fancyhdr}
\setlength{\headheight}{15pt}
\addtolength{\topmargin}{-2.5pt}
\usepackage{xcolor}
\usepackage{amsmath, amssymb, bm}
\usepackage{mathtools}
\usepackage{cancel}
\usepackage{tikz}
\usepackage{newunicodechar}
\usepackage{ragged2e}
\usepackage{setspace}
\usepackage{tikz-3dplot} % Necessário para coordenadas 3D
\usetikzlibrary{intersections}
\usepackage{siunitx}
\usetikzlibrary{3d, arrows.meta}
\usepackage{multicol}
\setlength{\columnsep}{1cm}

\usepackage{color}
\definecolor{myblue}{rgb}{.8, .8, 1}

\usepackage{amsmath}
\usepackage{empheq}

\newlength\mytemplen
\newsavebox\mytempbox

\makeatletter
\newcommand\mybluebox{%
    \@ifnextchar[%]
       {\@mybluebox}%
       {\@mybluebox[0pt]}}

\def\@mybluebox[#1]{%
    \@ifnextchar[%]
       {\@@mybluebox[#1]}%
       {\@@mybluebox[#1][0pt]}}

\def\@@mybluebox[#1][#2]#3{
    \sbox\mytempbox{#3}%
    \mytemplen\ht\mytempbox
    \advance\mytemplen #1\relax
    \ht\mytempbox\mytemplen
    \mytemplen\dp\mytempbox
    \advance\mytemplen #2\relax
    \dp\mytempbox\mytemplen
    \colorbox{myblue}{\hspace{1em}\usebox{\mytempbox}\hspace{1em}}}
\makeatother

\usepackage[most]{tcolorbox}

\newtcbox{\mymath}[1][]{%
    nobeforeafter, math upper, tcbox raise base,
    enhanced, colframe=blue!30!black,
    colback=blue!30, boxrule=1pt,
    #1}

\tcbset{
    highlight math style={
        enhanced,
        colframe=red!60!black,
        colback=yellow!50,
        arc=4pt,
        boxrule=1pt,
        drop fuzzy shadow
    }
    }

\usepackage{physics}
\usepackage{pgfplots}
\pgfplotsset{compat=1.17}

\linespread{1}

\definecolor{ao(english)}{rgb}{0.0, 0.5, 0.0}
\definecolor{byzantium}{rgb}{0.44, 0.16, 0.39}
\newunicodechar{∘}{\circ}

%%%%%%%%%%%%%%%%%%%%%%%%%%%%%%%%%%%%%%%%%%%%%%%%%%
% These are some new commands that may be useful 
% for paper writing in general. If other new commands
% are needed for your specific paper, please feel 
% free to add here. 
%
% The currently available commands are organized in: 
% 1) Systems
% 2) Quantities
% 3) Energies and units
% 4) particle species
% 5) Colors package
% 6) hyperlink
%%%%%%%%%%%%%%%%%%%%%%%%%%%%%%%%%%%%%%%%%%%%%%%%%%

\usepackage{amsmath}
\usepackage{amssymb}
\usepackage{upgreek}
\usepackage{multirow}
\usepackage{setspace}% http://ctan.org/pkg/setspace
\usepackage{fancyhdr}
\usepackage{datetime}

% 1) SYSTEMS
\newcommand{\btc}               {\textbf{BTC}}
\newcommand{\btcspace}          {\textbf{BTC} }
\newcommand{\pow}               {\textbf{PoW}}

% 4) definition to references, biblatex and hyperlink
\usepackage[backend=bibtex, 
style=nature,  %style reference.
sorting=none,
firstinits=true %first name abbreviate
]{biblatex}

\usepackage{hyperref}
\hypersetup{
    colorlinks=true, %set "true" if you want colored links
    linktoc=all,     %set to "all" if you want both sections and subsections linked
    linkcolor=blue,  %choose some color if you want links to stand out
    citecolor= blue, % color of \cite{} in the text.
    urlcolor  = blue, % color of the link for the paper in references.
}

% 5) Tikz and figures
\usepackage{epsfig}
\usepackage{lmodern}
\usepackage{mathtools}
\usepackage[utf8]{luainputenc}
\usepackage{xspace}
\usepackage{tikz}
\usepackage{pgfplots}
\pgfplotsset{compat=newest}

\usetikzlibrary{positioning}
\usepackage{subcaption}

% 6) colors:
\usepackage{xcolor}
\definecolor{ao(english)}{rgb}{0.0, 0.5, 0.0} % dark green

% 7) Add lines numbers
%\usepackage{lineno}

% add pdf file to thesis:
\usepackage{pdfpages}

\hypersetup{
    colorlinks=true,% make the links colored
    linkcolor=blue
}

\usepackage{setspace}
\addbibresource{bibliography.bib}

\newcommand{\printingbibliography}{%

    \pagestyle{myheadings}
    \markright{}
    \sloppy
    \printbibliography[heading=bibintoc, % add to table of contents
                   title=Refer\^encias % Chapter name
                  ]
    \fussy%
}
\PassOptionsToPackage{table}{xcolor}

\pagestyle{fancy}
\fancyhf{}
\renewcommand{\headrulewidth}{0pt}
\fancyhead[R]{\thepage}

\geometry{a4paper,top=30mm,bottom=20mm,left=30mm,right=20mm}

\titleformat*{\section}{\bfseries\large}
\titleformat*{\subsection}{\bfseries\normalsize}

\title{ \textbf{\large Note - Physics}}
\author{Andr\'e V. Silva}
\date{\today}

\begin{document}

\begin{multicols}{2}

\justifying

%\noindent\rule{\linewidth}{0.4pt}\\
\section{Cinemática Escalar/Vetorial}

\section{Conceitos Fundamentais}
\begin{itemize}
  \item \textbf{Movimento}: variação da posição no tempo em relação a um referencial.
  \item \textbf{Repouso}: posição constante em relação ao referencial.
  \item \textbf{Referencial}: sistema usado como base para descrever o movimento.
\end{itemize}

\section{Cinemática Escalar (1D)}
\begin{itemize}
  \item \textbf{Posição}: $s$
  \item \textbf{Deslocamento}: $\Delta s = s_f - s_0$
  \item \textbf{Velocidade média}: $v_m = \frac{\Delta s}{\Delta t}$
  \item \textbf{Aceleração média}: $a_m = \frac{\Delta v}{\Delta t}$
\end{itemize}

\section{Movimento Retilíneo Uniforme (MRU)}
\begin{itemize}
  \item Velocidade constante: $a = 0$
  \item Equação horária: $s = s_0 + v t$
\end{itemize}

\section{Movimento Retilíneo Uniformemente Variado (MRUV)}
\begin{itemize}
  \item Aceleração constante.
  \item $s = s_0 + v_0 t + \frac{1}{2} a t^2$
  \item $v = v_0 + a t$
  \item $v^2 = v_0^2 + 2a(s - s_0)$
  \item $\Delta s = \frac{(v + v_0)}{2} \cdot t$
\end{itemize}

\section{Cinemática Vetorial (2D e 3D)}
\begin{itemize}
  \item posição: $\vec{r}(t) = x(t)\hat{i} + y(t)\hat{j} + z(t)\hat{k}$
  \item Deslocamento vetorial: $\Delta \vec{r} = \vec{r}_f - \vec{r}_0$
  \item Velocidade vetorial: $\vec{v}(t) = \frac{d\vec{r}}{dt}$
  \item Aceleração vetorial: $\vec{a}(t) = \frac{d\vec{v}}{dt}$
\end{itemize}

\section{Lançamento Oblíquo}
\textbf{Separação dos movimentos:}
\begin{itemize}
  \item Horizontal (MRU): $x(t) = x_0 + v_{0x} t$
  \item Vertical (MRUV): $y(t) = y_0 + v_{0y} t - \frac{1}{2} g t^2$
\end{itemize}

\textbf{Outras fórmulas:}
\begin{itemize}
  \item Velocidade inicial: $\vec{v}_0 = v_0 \cos\theta\,\hat{i} + v_0 \sin\theta\,\hat{j}$
  \item Alcance: $A = \frac{v_0^2 \sin(2\theta)}{g}$
  \item Altura máxima: $H = \frac{v_0^2 \sin^2\theta}{2g}$
  \item Tempo de subida: $t_s = \frac{v_0 \sin\theta}{g}$
  \item Tempo total: $t = \frac{2 v_0 \sin\theta}{g}$
\end{itemize}

\section{Gráficos}
\begin{itemize}
  \item $s \times t$: inclinação = velocidade.
  \item $v \times t$: área = deslocamento; inclinação = aceleração.
  \item $a \times t$: área = variação da velocidade.
\end{itemize}

\section{Tipos de Movimento}
\begin{itemize}
  \item MRU:
    \begin{itemize}
      \item[$\rightarrow$] $v > 0$: progressivo
      \item[$\rightarrow$] $v < 0$: retrógrado
    \end{itemize}
  \item MRUV:
    \begin{itemize}
      \item[$\rightarrow$] $v \cdot a > 0$: acelerado
      \item[$\rightarrow$] $v \cdot a < 0$: retardado
    \end{itemize}
\end{itemize}

%%%%%%%%%%%%%%%%%%%%%%%%%%%%%%%%%%%%%%%%%%%%%%%%%%%%%%%
\noindent\rule{\linewidth}{1pt}\\
\section{Estática e Dinâmica}

\section{Conceitos Fundamentais}

\begin{itemize}
  \item \textbf{Grandezas escalares}: possuem apenas módulo (ex: massa, tempo).
  \item \textbf{Grandezas vetoriais}: possuem módulo, direção e sentido (ex: força, aceleração).
  \item \textbf{Força resultante}: vetor que representa o efeito combinado de todas as forças aplicadas.
  \item \textbf{Diagrama de corpo livre}: representação de todas as forças atuantes sobre um corpo.
\end{itemize}

\section{Equilíbrio do Corpo Rígido e da Partícula}

\textbf{Condições de equilíbrio:}
\begin{align*}
  \sum \vec{F} &= 0 \quad \text{(equilíbrio translacional)} \\
  \sum \vec{\tau} &= 0 \quad \text{(equilíbrio rotacional)}
\end{align*}

\textbf{Torque (momento de uma força):}
\begin{equation*}
  \tau = r F \sin \theta
\end{equation*}

\begin{equation*}
  \vec{\tau} = \frac{d\vec{L}}{dt}
\end{equation*}

\begin{equation*}
  \tau = I.\alpha
\end{equation*}

\begin{equation*}
  \alpha = \frac{d^{2}\theta}{dt^{2}}
\end{equation*}

\begin{equation*}
  \frac{d^{2}\theta}{dt^{2}} + \frac{g}{L}\sin\theta = 0 \quad \textrm{MHS}
\end{equation*}

\begin{equation*}
  \frac{d^{2}\theta}{dt^{2}} + \omega^{2}\sin\theta = 0 \quad \textrm{MHS}
\end{equation*}

Solução geral EDO:
\begin{equation*}
  \theta(t) = \theta_{0} \cos(\omega t + \varphi)
\end{equation*}

\subsection{Rota\c{c}\~ao de um Corpo R\'igido}
\begin{equation*}
  \omega = \frac{d\theta}{dt}, \quad \alpha = \frac{d\omega}{dt}
\end{equation*}

\subsection{Relação entre coroas e catracas}

Como a corrente impõe a mesma velocidade linear na periferia das duas engrenagens:

\begin{equation*}
  v_{\text{coroa}} = v_{\text{catraca}} = R_c \cdot \omega_c = R_k \cdot \omega_k
\end{equation*}

\section{Leis Fundamentais da Dinâmica (Leis de Newton)}

\begin{itemize}
  \item \textbf{1ª Lei (Inércia)}: um corpo em repouso ou em MRU permanece assim se a força resultante for nula.
  \item \textbf{2ª Lei}: Princ\'ipio Fundamental da Din\^amica:
  \begin{equation*}
    \vec{F}_{\text{resultante}} = m \vec{a}
  \end{equation*}
  \item \textbf{3ª Lei (Ação e Reação)}: forças trocadas entre dois corpos são iguais em módulo, mesma direção e sentidos opostos.
\end{itemize}

\section{Gravitação Universal}

\textbf{Lei da Gravitação Universal:}
\begin{equation*}
  F = G \frac{m_1 m_2}{r^2}
\end{equation*}

\textbf{Campo gravitacional:}
\begin{equation*}
  g = \frac{G M}{r^2}
\end{equation*}

\textbf{Energia potencial gravitacional:}
\begin{equation*}
  E_p = -\frac{G M m}{r}
\end{equation*}

\section{Demonstração da Velocidade de Escape}

A velocidade de escape é a mínima velocidade necessária para um corpo escapar da gravidade de um planeta, sem considerar resistência do ar.

\subsection{Conservação de Energia}

Considerando um corpo de massa $m$ lançado da superfície de um planeta de massa $M$ e raio $R$:

\begin{itemize}
  \item Energia mecânica inicial:
  \[
  E_{\text{inicial}} = \frac{1}{2}mv^2 - \frac{GMm}{R}
  \]
  \item Energia mecânica final (no infinito): 
  \[
  E_{\text{final}} = 0
  \]
\end{itemize}

Aplicando a conservação da energia:

\[
\frac{1}{2}mv^2 - \frac{GMm}{R} = 0
\Rightarrow \frac{1}{2}v^2 = \frac{GM}{R}
\Rightarrow v = \sqrt{\frac{2GM}{R}}
\]

\noindent
\textbf{Conclusão:} A velocidade de escape depende apenas da massa e do raio do corpo celeste, e não da massa do objeto lançado.

\section{Forças no Movimento Circular}

\begin{align*}
  F_c &= \frac{m v^2}{r} \quad \text{(força centrípeta)} \\
  v &= \omega r \quad \text{(velocidade tangencial)} \\
  a_c &= \frac{v^2}{r} \quad \text{(aceleração centrípeta)}
\end{align*}

\section{Impulso e Quantidade de Movimento}

\textbf{Quantidade de movimento:}
\begin{equation*}
  \vec{p} = m \vec{v}
\end{equation*}

\textbf{Impulso:}
\begin{equation*}
  \vec{I} = \vec{F} \Delta t
\end{equation*}

\textbf{Teorema do impulso:}
\begin{equation*}
  \vec{I} = \Delta \vec{p}
\end{equation*}

\section{Coeficiente de Restituição}

O \textbf{coeficiente de restituição} (\( e \)) é uma medida que descreve a \textbf{elasticidade} de uma colisão entre dois corpos. Ele relaciona a velocidade relativa dos corpos antes e depois da colisão.

\subsection{Definição}

O coeficiente de restituição é dado por:

\[
e = \frac{v_{\text{afastamento}}}{v_{\text{aproximação}}} = \frac{v'_{2} - v'_{1}}{v_{1} - v_{2}}
\]

Onde:
\begin{itemize}
  \item \( e \): coeficiente de restituição
  \item \( v_{\text{final}} \): velocidade relativa dos corpos após a colisão
  \item \( v_{\text{inicial}} \): velocidade relativa dos corpos antes da colisão
\end{itemize}

\subsection{Valores do Coeficiente de Restituição}

\begin{itemize}
  \item \( e = 1 \): colisão \textbf{perfeitamente elástica} (não há perda de energia cinética).
  \item \( e = 0 \): colisão \textbf{perfeitamente inelástica} (os corpos permanecem juntos após a colisão).
  \item \( 0 < e < 1 \): colisão \textbf{parcialmente elástica}.
\end{itemize}

\section{Momento de Inércia}

O \textbf{momento de inércia} ($I$) representa a resistência de um corpo à variação de seu movimento rotacional em torno de um eixo. Ele depende da distribuição de massa em relação ao eixo de rotação.

\subsection*{1. Momento de inércia para ponto material}
\[
I = m \cdot r^2
\]
\begin{itemize}
    \item $m$: massa do ponto (kg)
    \item $r$: distância ao eixo de rotação (m)
\end{itemize}

\subsection{2. Momento de inércia para corpos contínuos}
\[
I = \int r^2 \, \mathrm{d}m
\]

\subsection{Momentos de inércia usuais}

\begin{itemize}
    \item Disco ou cilindro sólido (eixo pelo centro, perpendicular à base):
    \[
    I = \frac{1}{2} M R^2
    \]
    \item Aro ou anel fino (eixo pelo centro, perpendicular ao plano):
    \[
    I = M R^2
    \]
    \item Esfera sólida (eixo pelo centro):
    \[
    I = \frac{2}{5} M R^2
    \]
    \item Esfera oca (casca esférica fina):
    \[
    I = \frac{2}{3} M R^2
    \]
    \item Haste fina (eixo pelo centro, perpendicular à haste):
    \[
    I = \frac{1}{12} M L^2
    \]
    \item Haste fina (eixo pela extremidade):
    \[
    I = \frac{1}{3} M L^2
    \]
\end{itemize}

\subsection{Teorema dos Eixos Paralelos (Steiner)}
Seja $I_{cm}$ o momento de inércia em relação ao centro de massa, então para um eixo paralelo a uma distância $d$:

\[
I = I_{cm} + M d^2
\]

\section{Referenciais Inerciais e Não Inerciais}

\subsection*{Referencial Inercial}

\textbf{Definição:} Um referencial é dito \textbf{inercial}, quando em repouso relativo às estrelas fixas distantes. se nele a \textbf{Primeira Lei de Newton} é válida:

\[
\text{Se } \vec{F}_{\text{resultante}} = 0 \Rightarrow \vec{v} = \text{constante}
\]

\textbf{Características:}
\begin{itemize}
  \item Não está acelerado.
  \item Permite aplicação direta das Leis de Newton.
  \item Exemplo: Espaço profundo, ou um carro em movimento retilíneo uniforme.
\end{itemize}

\subsection{Referencial Não Inercial}

\textbf{Definição:} Um referencial é \textbf{não inercial} quando está acelerado em relação a um inercial. Nele, a Primeira Lei de Newton \textbf{não se aplica diretamente} sem a introdução de forças fictícias.

\textbf{Características:}
\begin{itemize}
  \item Está acelerado (linear ou rotacionalmente).
  \item Surgem \textbf{forças fictícias}, como:
  \begin{itemize}
    \item Força centrífuga(pseudoforça)
    \item Força de Coriolis: é uma força fictícia que aparece em referenciais rotacionais, como a Terra, e age 
    sobre corpos em movimento dentro desse referencial.
    O efeito Coriolis pode ser visto como uma correção dinâmica que aparece quando a conservação do momento angular 
    é "quebrada" por se analisar o movimento num referencial rotativo.
    \[
        \vec{F}_C = -2m \vec{\omega} \times \vec{v}
    \]
    \begin{itemize}
        \item \( \vec{F}_C \): força de Coriolis
        \item \( m \): massa do corpo
        \item \( \vec{\omega} \): vetor velocidade angular do referencial
        \item \( \vec{v} \): velocidade do corpo no referencial
    \end{itemize}
    \item Força inercial (associada à aceleração do referencial)
  \end{itemize}
  \item Exemplo: Interior of a carro that freia/brusca mudança de direço, Terra (com rotaço), elevador em aceleraço.
  \item A terra não é uma referencial inercial: $$a_{cp} = \frac{g}{290} = 0.03369 m/s^2$$, com boa aproximação podemos assumir a terra um referencial inercial.
\end{itemize}

\subsection{Observação:}
Essa força só aparece em referenciais acelerados (não inerciais) e é essencial para descrever corretamente movimentos na superfície da Terra.


\subsection{Comparação Resumida}

\begin{tabular}{|c|c|c|}
\hline
& \textbf{Referencial}& \textbf{Referencial}\\
\textbf{Aspecto} & \textbf{Inercial} & \textbf{Não Inercial} \\
\hline
Aceleração & Nula & $\neq$ de zero \\
L. Newton & Válidas & $\exists$ $\vec{F}$ fictícias \\
Exemplo & deep space & Carro em curva \\
\hline
\end{tabular}



\section{Trabalho e Energia Cinética}

\textbf{Trabalho de uma força constante:}
\begin{equation*}
  W = F d \cos \theta
\end{equation*}

\textbf{Energia cinética:}
\begin{equation*}
  E_c = \frac{1}{2} m v^2
\end{equation*}

\textbf{Teorema da energia cinética:}
\begin{equation*}
  W_{\text{resultante}} = \Delta E_c
\end{equation*}

\section{Força de Atrito}

\textbf{Atrito estático:}
\begin{equation*}
  f_e \leq \mu_e N
\end{equation*}

\textbf{Atrito cinético:}
\begin{equation*}
  f_c = \mu_c N
\end{equation*}

\section{9. Energia Potencial}

\textbf{Potencial gravitacional:}
\begin{equation*}
  E_p = m g h
\end{equation*}

\textbf{Potencial elástica:}
\begin{equation*}
  E_{p,\text{el}} = \frac{1}{2} k x^2
\end{equation*}

\section{Conservação da Energia Mecânica}

\textbf{Em sistemas conservativos:}
\begin{equation*}
  E_m = E_c + E_p = \text{constante}
\end{equation*}

\section{Lei de Hooke}

\textbf{Força elástica:}
\begin{equation*}
  F = -k x
\end{equation*}

\subsection{Constante da mola ao ser cortada ao meio}

Se uma mola ideal com constante elástica \( k \) e comprimento \( L \) for cortada ao meio, cada metade terá comprimento \( \frac{L}{2} \).

Como a constante elástica é inversamente proporcional ao comprimento:

\[
k \propto \frac{1}{L}
\]

Ao cortar a mola:

\[
k' = \frac{kL}{L/2} = 2k
\]

Logo, cada metade da mola terá constante elástica \( 2k \), ou seja, será duas vezes mais rígida que a mola original.

\textbf{Energia potencial armazenada:}
\begin{equation*}
  E_{p,\text{el}} = \frac{1}{2} k x^2
\end{equation*}

\subsection{Leis de Kepler}

\begin{itemize}
  \item \textbf{1ª Lei:} Órbitas elípticas, com o Sol em um dos focos.
  \item \textbf{2ª Lei:} Áreas iguais em tempos iguais. \textit{Consequência direta da 
  conservação do momento angular:}
        \[
        \vec{L} = \vec{r} \times \vec{p} = \text{constante}
        \]

        \[ \frac{dA}{dt} = \frac{1}{2} |\vec{r} \times \vec{v}| = \text{constante} \]

  \item \textbf{3ª Lei:} \( \frac{T^2}{R^3} = \frac{4 \pi^2}{GM}  \)
\end{itemize}

\section{Conser. Momento Angular}

O \textbf{momento angular} \( \vec{L} \) de um corpo é dado por:

\[
\vec{L} = I \cdot \vec{\omega}
\]

Onde:
\begin{itemize}
  \item \( \vec{L} \): momento angular
  \item \( I \): momento de inércia
  \item \( \vec{\omega} \): velocidade angular
\end{itemize}

\subsection{Princípio da Conservação}

Se o \textbf{torque resultante externo} sobre um sistema é nulo:

\[
\vec{L}_{\text{inicial}} = \vec{L}_{\text{final}}
\quad \Rightarrow \quad I_1 \omega_1 = I_2 \omega_2
\]

\subsection*{Aplicações}

\begin{itemize}
  \item Patinadores puxando os braços e girando mais rápido
  \item Estrelas colapsando em pulsares
  \item Satélites e giroscópios
\end{itemize}

%%%%%%%%%%%%%%%%%%%%%%%%%%%%%%%%%%%%%%%%%%
\noindent\rule{\linewidth}{1pt}\\
\section{Hidrostática}

O \textbf{Princípio de Arquimedes} afirma que:

\begin{quote}
    \textit{Todo corpo total ou parcialmente imerso em um fluido em repouso sofre a ação de uma 
    força vertical para cima, denominada empuxo, de intensidade igual ao peso do fluido deslocado.}
\end{quote}

Essa força de empuxo é dada por:

\[
\boxed{
E = \rho_f \cdot g \cdot V_d
}
\]

\section{Fluido em Equilíbrio}
Fluido em repouso está sujeito apenas a forças normais e pressões. A pressão se transmite igualmente em todas as direções no interior do fluido.

\section{Conceito de Pressão}
\[
P = \frac{F}{A}
\]
Unidade: Pascal (Pa), onde \(1\, \text{Pa} = 1\, \text{N/m}^2\).

\section{Densidade}
\[
\rho = \frac{m}{V}
\]
Unidade: \(\text{kg/m}^3\). Densidade da água: \(\rho_{\text{água}} = 10^3\, \text{kg/m}^3\).

\section{Pressão de uma Coluna de Líquido}
\[
P = \rho g h
\]
Onde: \(\rho\) é a densidade, \(g\) a gravidade, \(h\) a profundidade.

\section{Conservação da Massa (Eq. da Continuidade)}

Para um fluido incompressível:

\[
A_1 v_1 = A_2 v_2
\]

Onde:
\begin{itemize}
  \item \( A \): área da seção transversal
  \item \( v \): velocidade do fluido
\end{itemize}

\section{Equação de Bernoulli}

Expressa a conservação da energia para fluidos ideais:

\[
P + \frac{1}{2} \rho v^2 + \rho g h = \text{constante}
\]

Onde:
\begin{itemize}
  \item \( P \): pressão
  \item \( \rho \): densidade
  \item \( v \): velocidade
  \item \( h \): altura
\end{itemize}

\section{Princípio de Pascal}
Uma variação de pressão aplicada a um fluido incompressível em equilíbrio transmite-se integralmente a todos os pontos do fluido e às paredes do recipiente.

\section{Pressão Atmosférica}
Pressão exercida pelo ar ao nível do mar:
\[
P_{\text{atm}} \approx 1{,}0 \times 10^5\, \text{Pa} = 1\, \text{atm}
\]

\section{Experiência de Torricelli}
\[
P_{\text{atm}} = \rho g h, \text{com } h = 0{,}76\, \text{m (coluna de mercúrio)}
\]

\section{Lei de Stevin}
\[
\Delta P = \rho g \Delta h
\]
Válida para qualquer ponto de um mesmo fluido em equilíbrio.

\section{Vasos Comunicantes}
Se o fluido for o mesmo, os níveis de líquido se igualam:
\[
h_1 = h_2
\quad \text{(para } \rho_1 = \rho_2)
\]

\section{Conservação do Momento Angular}

O \textbf{momento angular} \( \vec{L} \) de um corpo é dado por:

\[
\vec{L} = I \cdot \vec{\omega}
\]

Onde:
\begin{itemize}
  \item \( \vec{L} \): momento angular
  \item \( I \): momento de inércia
  \item \( \vec{\omega} \): velocidade angular
\end{itemize}

\subsection{Princípio da Conservação}

Se o \textbf{torque resultante externo} sobre um sistema é nulo:

\[
\vec{L}_{\text{inicial}} = \vec{L}_{\text{final}}
\quad \Rightarrow \quad I_1 \omega_1 = I_2 \omega_2
\]

\subsection{Aplicações}

\begin{itemize}
  \item Patinadores puxando os braços e girando mais rápido
  \item Estrelas colapsando em pulsares
  \item Satélites e giroscópios
\end{itemize}
\section{Prensa Hidráulica}
Aplicação do Princípio de Pascal:
\[
\frac{F_1}{A_1} = \frac{F_2}{A_2}
\]
Permite multiplicar força aplicando pressão a um fluido entre dois êmbolos de áreas diferentes.

\section{Equilíbrio térmico e temperatura}
\begin{itemize}
    \item Temperatura é uma medida da energia cinética média das partículas.
    \item Dois corpos estão em \textbf{equilíbrio térmico} quando não trocam mais calor entre si.
    \item \textbf{Lei Zero da Termodinâmica}: Se $A$ está em equilíbrio com $B$, e $B$ com $C$, então $A$ está em equilíbrio com $C$.
\end{itemize}

%%%%%%%%%%%%%%%%%%%%%%%%%%%%%%%%%%%%%%%%%%
\noindent\rule{\linewidth}{1pt}\\

\section{Escalas termométricas}

\begin{itemize}
    \item Principais escalas: Celsius $(^\circ C)$, Fahrenheit $(^\circ F)$, Kelvin $(K)$.
    \item Conversões:
    \begin{align*}
        T(K) &= T(^\circ C) + 273{,}15 \\
        T(^\circ F) &= \frac{9}{5}T(^\circ C) + 32
    \end{align*}
\end{itemize}

\section{Dilatação dos sólidos e líquidos}
\begin{itemize}
    \item Dilatação linear: $\Delta L = L_0 \alpha \Delta T$
    \item Dilatação superficial: $\Delta A = A_0 \cdot 2\alpha \cdot \Delta T$
    \item Dilatação volumétrica: $\Delta V = V_0 \beta \Delta T$, com $\beta = 3\alpha$
    \item Dilatação aparente dos líquidos: $\Delta V_{ap}$  = $V_0 (\gamma_{\text{líq}} -\beta_{rec}) \Delta T$
\end{itemize}

\section{Estudo térmico dos gases}
\begin{itemize}
    \item Gases ideais obedecem à equação de estado e ignoram interações intermoleculares.
    \item Variáveis de estado: $P$, $V$, $T$, $n$.
    \item Hipóteses: moléculas puntiformes, colisões elásticas, movimento aleatório.
\end{itemize}

\section{Lei geral dos gases perfeitos}
\begin{equation*}
    PV = nRT
\end{equation*}
Onde:
\begin{itemize}
    \item $R = 8{,}31 \, \text{J/mol·K}$ (constante dos gases)
\end{itemize}

\section{Equação de Clapeyron}
\begin{equation*}
    \frac{P_1V_1}{T_1} = \frac{P_2V_2}{T_2}
\end{equation*}

\section{Transformações gasosas}
\begin{itemize}
    \item \textbf{Isotérmica} ($\Delta T = 0$): $PV = \text{constante}$ \hfill (Lei de Boyle)
    \item \textbf{Isobárica} ($\Delta P = 0$): $\frac{V}{T} = \text{constante}$ \hfill (Lei de Charles)
    \item \textbf{Isocórica} ($\Delta V = 0$): $\frac{P}{T} = \text{constante}$ \hfill (Lei de Gay-Lussac)
    \item \textbf{adiabático} ($Q = 0$) \hfill
    \begin{itemize}
        \item[$\rightarrow$] $PV^{\gamma} = \text{constante}$
        \item[$\rightarrow$] $TV^{\gamma -1} = \text{constante}$
        \item[$\rightarrow$] $P^{1-\gamma}T^{\gamma} = \text{constante}$
        \item[$\rightarrow$] $\gamma = \frac{C_{p}}{C_{v}}$
        \item[$\rightarrow$] g\'as monoat\^omico: $\gamma = \frac{5}{3}$
        \item[$\rightarrow$] g\'as diat\^omico: $\gamma = \frac{7}{5}$
        \item[$\rightarrow$] $C_{p} = C_{v} + R$
        \item[$\rightarrow$] U = $\frac{3}{2}nRT$ Energia Interna, gases monoat\^omico
        \item[$\rightarrow$] U = $\frac{f}{2}nRT$, f graus de liberdade.
        \item[$\rightarrow$] W = $\int \frac{nRT}{V}dV$
    \end{itemize}
\end{itemize}

\section{Princípio da conservação da energia}
\begin{itemize}
    \item Energia interna se conserva em sistemas isolados.
    \item Base do \textbf{Primeiro Princípio da Termodinâmica}.
\end{itemize}

\section{Mudanças de estado físico}
\begin{itemize}
    \item Fusão, vaporização, solidificação, condensação, sublimação.
    \item Ocorrem à temperatura constante.
    \item Energia envolvida depende da massa e do calor latente.
\end{itemize}

\section{Quantidade de calor}
\begin{itemize}
    \item Calor sensível: $Q = mc\Delta T$
    \item Calor latente: $Q = mL$
    \item Unidade no SI: Joule (J)
\end{itemize}

\section{Propagação do calor}
\subsection{Condução térmica -  Lei de Fourier da Condução de Calor}
\begin{equation*}
\boxed{Q = -kA\frac{dT}{dx}}
\end{equation*}

\subsection{Convecção}
\begin{itemize}
    \item Transferência por movimentação de massa em fluidos.
\end{itemize}

\subsection{Radiação}
\begin{equation*}
    P = \sigma A T^4
\end{equation*}
Onde $\sigma = 5{,}67 \times 10^{-8} \, \text{W/m}^2\text{K}^4$

\section{Princ\'ipios da Termodinâmica}

\subsection{Primeiro Princípio}
\begin{equation*}
    \Delta U = Q - W   \hspace{0.5cm} \longrightarrow \hspace{0.5cm} Q = W + \Delta U
\end{equation*}
\subsection{Segundo Princípio}
\begin{itemize}
    \item O calor não flui espontaneamente de um corpo frio para um corpo quente.
    \item Entropia tende a aumentar.
\end{itemize}

\section*{O que é entropia?}
A entropia (\(S\)) é uma função de estado que mede o grau de desordem de um sistema, a quantidade de microestados possíveis, e a irreversibilidade de processos.

\section*{Definição termodinâmica}
Para processos reversíveis:

\[
\Delta S = \int \frac{dQ_{\text{rev}}}{T}
\]

Para temperatura constante (isotérmico):

\[
\Delta S = \frac{Q_{\text{rev}}}{T}
\]

\section*{Segunda Lei da Termodinâmica}

\[
\Delta S_{\text{total}} \geq 0
\]

\begin{itemize}
  \item \( \Delta S_{\text{total}} = 0 \): processo reversível
  \item \( \Delta S_{\text{total}} > 0 \): processo irreversível
\end{itemize}

\section*{Entropia estatística (Boltzmann)}

\[
S = k_B \ln \Omega
\]

\begin{itemize}
  \item \( k_B = 1{,}38 \times 10^{-23} \, \text{J/K} \)
  \item \( \Omega \): número de microestados possíveis
\end{itemize}

\section*{Unidade}
Joules por kelvin (J/K)

\section*{Exemplos onde a entropia aumenta}
\begin{itemize}
  \item Derretimento de gelo
  \item Expansão de gás
  \item Mistura de substâncias
\end{itemize}

\subsection{Terceiro Princípio}
\begin{itemize}
    \item A entropia de um cristal perfeito é zero no zero absoluto $(0\,K)$.
\end{itemize}

\section{Definição}

Uma máquina térmica converte calor em trabalho, operando entre duas fontes térmicas.

\section{Rendimento}

\[
\eta = \frac{W}{Q_f} = \frac{Q_f - Q_r}{Q_f} = 1 - \frac{Q_r}{Q_f}
\]

\begin{itemize}
  \item \( \eta \): rendimento
  \item \( W \): trabalho útil
  \item \( Q_q \): calor absorvido da fonte quente
  \item \( Q_f \): calor rejeitado à fonte fria
\end{itemize}

\section{Rendimento da Máquina de Carnot}

\[
\eta_{\text{Carnot}} = 1 - \frac{T_f}{T_q}
\]

\section{Refrigerador/Termodinâmica}

O \textbf{refrigerador} transfere calor do meio frio para o quente, usando trabalho externo.

\begin{itemize}
  \item Retira calor \( Q_f \) do interior (frio).
  \item Rejeita calor \( Q_q \) para o exterior (quente).
  \item Gasta trabalho \( W \): \quad \( Q_q = Q_f + W \)
\end{itemize}

\textbf{Coeficiente de performance (COP):}

\[
K = \frac{Q_f}{W}
\]

\textbf{Quanto maior \( K \)}, mais eficiente é o refrigerador.

\begin{itemize}
  \item \( T_q \): temperatura da fonte quente (em K)
  \item \( T_f \): temperatura da fonte fria (em K)
\end{itemize}

\section{Equivalente mecânico do calor}
\begin{itemize}
    \item Experiência de Joule:
    \begin{equation*}
        1 \, \text{cal} = 4{,}186 \, \text{J}
    \end{equation*}
\end{itemize}

%%%%%%%%%%%%%%%%%%%%%%%%%%%%%%%%%%%%%%%%%%
\noindent\rule{\linewidth}{1pt}\\

\section{Movimento Vibratório e Ondulatório}

\subsection{Movimento Periódico}
\begin{itemize}
    \item \textbf{Amplitude} $(A)$: valor máximo da oscilação.
    \item \textbf{Período} $(T)$: tempo para uma oscilação completa.
    \item \textbf{Frequência} $(f)$: número de oscilações por segundo, $f = \dfrac{1}{T}$.
    \item Unidade de frequência: hertz $(\text{Hz})$.
\end{itemize}

\section{Movimento Harmônico Simples (MHS)}

\begin{itemize}
    \item Posição em função do tempo:
    \[
        x(t) = A \cos(\omega t + \varphi)
    \]
    \item $\omega$: frequência angular, $\omega = 2\pi f$
    \item $\varphi$: fase inicial
    \item Velocidade: $v(t) = -A\omega \sin(\omega t + \varphi)$
    \item Aceleração: $a(t) = -A\omega^2 \cos(\omega t + \varphi) = -\omega^2 x(t)$
\end{itemize}

\section{Oscilador Harmônico}
\begin{itemize}
    \item Sistema massa-mola:
    \[
        F = -kx \quad \Rightarrow \quad m\ddot{x} = -kx
    \]
    \item Solução: MHS com:
    \[
        \omega = \sqrt{\frac{k}{m}}, \quad T = 2\pi\sqrt{\frac{m}{k}}, \quad f = \frac{1}{2\pi}\sqrt{\frac{k}{m}}
    \]
\end{itemize}

\section{Pêndulo Simples}
\begin{itemize}
    \item Para pequenos ângulos $(\theta < 10^\circ)$, o movimento é aproximadamente harmônico:
    \[
        T = 2\pi \sqrt{\frac{L}{g}}
    \]
    \item Onde $L$ é o comprimento e $g$ a aceleração da gravidade.
\end{itemize}

\section{Classificação das Ondas}
\begin{itemize}
    \item \textbf{Quanto à natureza:}
    \begin{itemize}
        \item Mecânicas (necessitam meio): som, ondas em corda.
        \item Eletromagnéticas (propagam no vácuo): luz, micro-ondas.
    \end{itemize}
    \item \textbf{Quanto à direção da vibração:}
    \begin{itemize}
        \item Transversais: vibração $\bot$ propagação (ex: luz).
        \item Longitudinais: vibração $\|$ propagação (ex: som).
    \end{itemize}
\end{itemize}

\section{Velocidade de propagação de uma onda unidimensional}
\[
    v = \lambda f
\]
Onde:
\begin{itemize}
    \item $\lambda$ é o comprimento de onda.
    \item $f$ é a frequência.
\end{itemize}

\section{Ondas Periódicas}
\[
    y(x, t) = A \cos(kx - \omega t + \varphi)
\]
\begin{itemize}
    \item $k = \dfrac{2\pi}{\lambda}$: número de onda
    \item $\omega = 2\pi f $: frequência angular
\end{itemize}

\section{Reflexão e refração de um pulso numa corda}
\begin{itemize}
    \item Reflexão em extremidade fixa: inversão de fase.
    \item Reflexão em extremidade livre: sem inversão.
    \item Refração: mudança de meio altera velocidade e comprimento de onda.
\end{itemize}

\section{Frente de onda}
\begin{itemize}
    \item Superfície formada por pontos que vibram em fase.
    \item Representa a forma da propagação (plana, esférica, etc).
\end{itemize}

\section{Fenômenos Ondulatórios}
\subsection{Reflexão}
\begin{itemize}
    \item Onda retorna ao encontrar um obstáculo.
    \item Lei da reflexão: ângulo de incidência = ângulo de reflexão.
\end{itemize}

\subsection{Refração}
\begin{itemize}
    \item Mudança de direção ao passar de um meio para outro com velocidade diferente.
    \item A frequência permanece constante.
\end{itemize}

\subsection{Difração}
\begin{itemize}
    \item Capacidade de contornar obstáculos e atravessar fendas.
    \item Mais evidente quando $\lambda \sim$ dimensão da fenda.
\end{itemize}

\subsubsection{Dispersão em uma Rede de Difração}

\subsubsection{Equação da Rede de Difração}
A condição para os máximos de interferência é dada por:
\[
d \sin\theta = m \lambda
\]
Onde:
\begin{itemize}
  \item \( d \): espaçamento entre as fendas (passo da rede)
  \item \( \theta \): ângulo de difração
  \item \( m \): ordem do máximo (inteiro)
  \item \( \lambda \): comprimento de onda da luz
\end{itemize}

\subsubsection{Dispersão Angular}
A dispersão angular \( D \) é definida como a variação do ângulo de difração com o comprimento de onda:
\[
D = \frac{d\theta}{d\lambda} = \frac{m}{d \cos\theta}
\]

\begin{itemize}
  \item A dispersão aumenta com a ordem \( m \)
  \item A dispersão é maior para redes com menor espaçamento \( d \) (mais linhas por mm)
  \item Aumenta com o ângulo \( \theta \), pois \( \cos\theta \) diminui
\end{itemize}

\subsection{Polarização}
\begin{itemize}
    \item Ocorre apenas com ondas transversais.
    \item Restrição da direção de oscilação.
\end{itemize}

\subsection{Superposição}
\begin{itemize}
    \item Ondas que se encontram somam-se ponto a ponto.
    \item Pode ser construtiva (reforço) ou destrutiva (cancelamento).
\end{itemize}

\subsection{Ondas estacionárias}
\begin{itemize}
    \item Resultam da superposição de duas ondas idênticas que se propagam em sentidos opostos.
    \item Formam nós (amplitude nula) e ventres (amplitude máxima).
\end{itemize}

\subsection{Interferência de ondas bidimensionais}
\begin{itemize}
    \item Padrões de interferência gerados por duas fontes coerentes.
    \item Franja de interferência depende da diferença de caminho óptico:
    \[
        \Delta s = n\lambda \quad (\text{interferência construtiva})
    \]
    \[
        \Delta s = \left(n + \frac{1}{2}\right)\lambda \quad (\text{interferência destrutiva})
    \]
\end{itemize}

\section{Condições para Interferência em Filmes Finos (Incidência Normal)}

Quando a luz incide perpendicularmente a um filme fino de espessura \( d \) e índice de refração \( n \), a diferença 
de caminho óptico entre os dois feixes refletidos é:

\[
\Delta = 2nd
\]

A condição de interferência depende da ocorrência (ou não) de inversão de fase ao refletir nas interfaces. 

\subsection*{Interferência Construtiva}

\begin{itemize}
  \item \textbf{Com inversão de fase em uma das interfaces:}
  \[
  2nd = \left( m + \frac{1}{2} \right) \lambda
  \]
  
  \item \textbf{Sem inversão de fase (ou inversão em ambas):}
  \[
  2nd = m \lambda
  \]
\end{itemize}

\subsection*{Interferência Destrutiva}

\begin{itemize}
  \item \textbf{Com inversão de fase em uma das interfaces:}
  \[
  2nd = m \lambda
  \]
  
  \item \textbf{Sem inversão de fase (ou inversão em ambas):}
  \[
  2nd = \left( m + \frac{1}{2} \right) \lambda
  \]
\end{itemize}

\noindent Onde:
\begin{itemize}
  \item \( n \) é o índice de refração do filme;
  \item \( d \) é a espessura do filme;
  \item \( \lambda \) é o comprimento de onda da luz no ar;
  \item \( m \in \mathbb{Z} \) é a ordem da interferência.
\end{itemize}

%%%%%%%%%%%%%%%%%%%%%%%%%%%%%%%%%%%%%%%%%%
\noindent\rule{\linewidth}{1pt}\\

\section{Acústica: Natureza e características do som}

\section{Natureza do Som}
\begin{itemize}
    \item O som é uma \textbf{onda mecânica longitudinal} que se propaga em meios materiais (sólidos, líquidos e gases).
    \item É gerado por um corpo em vibração e necessita de um meio material para se propagar (não se propaga no vácuo).
    \item A propagação ocorre devido à compressão e rarefação das partículas do meio.
    \item A velocidade do som depende do meio e da sua temperatura. No ar, a $\approx 340\,\text{m/s}$ (a $20^\circ$C).
\end{itemize}

\section{Características Físicas do Som}
\begin{itemize}
    \item \textbf{Frequência} $(f)$: número de vibrações por segundo. Está relacionada à altura do som (grave ou agudo).
    \begin{itemize}
        \item Sons audíveis: $20\,\text{Hz} \leq f \leq 20\,000\,\text{Hz}$
        \item Infrassons: $f < 20\,\text{Hz}$ \quad Ultrassons: $f > 20\,000\,\text{Hz}$
    \end{itemize}
    
    \item \textbf{Intensidade} $(I)$: quantidade de energia transportada pela onda sonora por unidade de área. Relaciona-se com o volume (forte ou fraco).
    \[
        I = \frac{P}{A}
    \]
    Onde $P$ é a potência da fonte sonora e $A$ é a área.
    
    \item \textbf{Nível Sonoro} $(\beta)$: medido em decibéis (dB).
    \[
        \beta = 10 \log \left( \frac{I}{I_0} \right)
    \]
    Onde $I_0 = 10^{-12}\,\text{W/m}^2$ é a intensidade de referência.
    
    \item \textbf{Timbre}: característica que permite distinguir sons de mesma frequência e intensidade produzidos por fontes diferentes. Está relacionado com a forma da onda sonora e os harmônicos presentes.
    
    \item \textbf{Velocidade do som} $(v)$: depende da densidade e da rigidez do meio. É maior em sólidos, intermediária em líquidos e menor em gases.
    \[
        v = \sqrt{\frac{E}{\rho}} \quad \text{(em sólidos)} \qquad
    \]

    \[
        v = \sqrt{\frac{\gamma R T}{M}} \quad \text{(em gases ideais)}
    \]
    Onde $E$ é o módulo de elasticidade, $\rho$ é a densidade, $\gamma$ é o coeficiente adiabático, $R$ é a constante universal dos gases, $T$ é a temperatura e $M$ a massa molar.
\end{itemize}

\section{Fenômenos Acústicos}
\begin{itemize}
    \item \textbf{Reflexão do som}: retorno do som ao encontrar obstáculos (eco).
    \item \textbf{Refração}: mudança de direção ao passar de um meio para outro com velocidade distinta.
    \item \textbf{Difração}: contorno de obstáculos e passagem por frestas.
    \item \textbf{Interferência}: superposição de ondas sonoras, gerando reforço ou cancelamento.
    \item \textbf{Ressonância}: amplificação das vibrações quando a frequência natural de um sistema coincide com a frequência da fonte sonora.
    \item \textbf{Efeito Doppler}: variação aparente da frequência sonora devido ao movimento relativo entre fonte e observador.
\end{itemize}

\section{Aplicações e Limites da Audição Humana}
\begin{itemize}
    \item A audição humana é sensível a frequências entre aproximadamente $20\,\text{Hz}$ e $20\,\text{kHz}$.
    \item Sons com intensidade acima de $120\,\text{dB}$ podem causar dor (limiar da dor).
    \item Utilizações práticas: ultrassonografia, sonar, acústica de ambientes, isolamento acústico.
\end{itemize}

%%%%%%%%%%%%%%%%%%%%%%%%%%%%%%%%%%%%%%%%%%
\noindent\rule{\linewidth}{1pt}\\

\section{Óptica e Ondulatória}

\section{Óptica Geométrica}

\subsection{Propagação da Luz}
\begin{itemize}
    \item A luz propaga-se em linha reta em meios homogêneos e transparentes.
    \item Três princípios fundamentais: propagação retilínea, reversibilidade e independência dos raios de luz.
\end{itemize}

\subsection{Espelhos Planos}
\begin{itemize}
    \item A imagem formada é virtual, direita e do mesmo tamanho do objeto.
    \item Propriedades: simetria em relação ao plano do espelho, conservação do ângulo de incidência.
\end{itemize}

\subsection{Refração da Luz e Índice de Refração}
\begin{itemize}
    \item Refração: mudança de direção da luz ao passar de um meio para outro.
    \item Lei de Snell-Descartes:
    \[
        n_1 \sin \theta_1 = n_2 \sin \theta_2
    \]
    \item Índice de refração:
    \[
        n = \frac{c}{v}
    \]
    Onde $c$ é a velocidade da luz no vácuo e $v$ no meio.
\end{itemize}

\subsection{Reflexão Total}
\begin{itemize}
    \item Ocorre quando a luz passa de um meio mais refringente para um menos refringente com ângulo maior que o ângulo crítico.
    \item Aplicação: fibras ópticas.
    \item $n_{1}.\sin(\theta_{1}) = n_{2}.\sin(90^{\circ})$
\end{itemize}

\subsection{Lâminas e Prismas}
\begin{itemize}
    \item Lâminas planas provocam apenas deslocamento lateral do feixe de luz.
    \item Prismas desviam e dispersam a luz branca em seus componentes (dispersão).
\end{itemize}

\subsection{Dispersão da Luz}
\begin{itemize}
    \item A velocidade da luz depende do comprimento de onda no meio material.
    \item Cada cor sofre um desvio diferente ao atravessar prismas, formando o espectro visível.
\end{itemize}

\subsection{Lentes Esféricas}
\begin{itemize}
    \item Podem ser convergentes(f > 0) ou divergentes.
    \item Equação de Gauss para lentes delgadas:
    \[
        \frac{1}{f} = \frac{1}{p} + \frac{1}{p'}
    \]
    Onde $f$ é a distância focal, $p$ a distância do objeto e $p'$ da imagem.
    \item $f = \frac{R}{2}$
    \item Aumento Linear: A$ = \frac{i}{O} = -\frac{p}{p'}$
    \item c\^oncavo: convergente (imagem: Real ou virtual)
    \item convexo: divergente (somente virtual)
    \begin{itemize}
        \item Espelhos convexos ampliam o campo de visão e diminuem a imagem.
    \end{itemize}
\end{itemize}

\section{Equação dos Fabricantes de Lentes}

A \textbf{equação dos fabricantes de lentes} relaciona o índice de refração da lente, os raios de curvatura das faces e a vergência (potência) da lente.

\subsection{Fórmula:}

\[
\frac{1}{f} = (n - 1) \left( \frac{1}{R_1} - \frac{1}{R_2} \right)
\]

Onde:
\begin{itemize}
  \item \( f \): distância focal da lente (em metros)
  \item \( n \): índice de refração do material da lente
  \item \( R_1 \): raio de curvatura da face anterior
  \item \( R_2 \): raio de curvatura da face posterior
\end{itemize}

\textbf{Observações sobre os sinais:}
\begin{itemize}
  \item \( R > 0 \): centro de curvatura à direita da superfície
  \item \( R < 0 \): centro de curvatura à esquerda da superfície
  \item Para lentes convergentes (como bi-convexas), \( f > 0 \)
  \item Para lentes divergentes (como bi-côncavas), \( f < 0 \)
\end{itemize}

\subsection{Vergência (potência da lente):}

\[
\mathcal{P} = \frac{1}{f} \quad \text{(em dioptrias, D)}
\]

\subsection{Associação de Lentes Delgadas}
\begin{itemize}
    \item Potência de associação:
    \[
        P_{\text{eq}} = P_1 + P_2 + \cdots + P_n
    \]
    Onde $P = \frac{1}{f}$ (com $f$ em metros e $P$ em dioptrias).
\end{itemize}

\subsection{Formação de Imagens}
\begin{itemize}
    \item Utiliza-se construção geométrica com raios notáveis.
    \item A natureza da imagem (real ou virtual, direita ou invertida, aumentada ou reduzida) depende da posição do objeto em relação ao foco e ao centro óptico.
\end{itemize}

\subsection{Instrumentos Ópticos}
\begin{itemize}
    \item \textbf{Lupa}: lente convergente que aumenta o tamanho angular do objeto observado.
    \item \textbf{Microscópio simples}: uma única lente convergente usada como lupa.
    \item \textbf{Luneta astronômica}: utiliza duas lentes — objetiva (imagem real e invertida) e ocular (amplia a imagem).
\end{itemize}

%%%%%%%%%%%%%%%%%%%%%%%%%%%%%%%%%%%%%%%%%%
\noindent\rule{\linewidth}{1pt}\\

\section{Óptica Física e Ondulatória}

\subsection{Natureza da Luz}
\begin{itemize}
    \item A luz possui natureza dual: comporta-se como onda (fenômenos de interferência, difração e polarização) e como partícula (efeito fotoelétrico).
    \item Como onda, é uma onda eletromagnética transversal.
\end{itemize}

\subsection{Fenômenos de Interferência}
\begin{itemize}
    \item Superposição de ondas que resulta em reforço (interferência construtiva) ou cancelamento (destrutiva).
\end{itemize}

\subsection{Experiência de Young}
\begin{itemize}
    \item Demonstra a natureza ondulatória da luz.
    \item Fenda dupla produz padrões de interferência em um anteparo.
    \item Distância entre franjas:
    \[
        \Delta y = \frac{\lambda L}{d}
    \]
    Onde $\lambda$ é o comprimento de onda da luz, $L$ a distância até o anteparo e $d$ a distância entre fendas.
\end{itemize}

\subsection{Polarização da Luz}
\begin{itemize}
    \item Luz natural é não polarizada (os vetores do campo elétrico vibram em todos os planos perpendiculares à direção de propagação).
    \item Polarização restringe a vibração da luz a um plano.
    \item Evidência da natureza transversal da luz.
\end{itemize}

\section{Síntese}
\begin{itemize}
    \item A óptica estuda tanto a propagação da luz (geométrica) quanto seus aspectos ondulatórios (física).
    \item Instrumentos ópticos e fenômenos ondulatórios da luz são essenciais para tecnologias modernas (óptica oftálmica, telescópios, interferômetros, fibras ópticas).
\end{itemize}

\section*{Velocidade da Luz/Constantes Fundamentais}

A velocidade da luz no vácuo é dada por:

\[
c = \frac{1}{\sqrt{\varepsilon_0 \mu_0}}
\]

Onde:
\begin{itemize}
  \item \( c \): velocidade da luz no vácuo
  \item \( \varepsilon_0 \): permissividade elétrica do vácuo
  \item \( \mu_0 \): permeabilidade magnética do vácuo
\end{itemize}

\subsection{Valores das constantes:}

\[
\varepsilon_0 \approx 8{,}854 \times 10^{-12} \, \text{F/m}
\quad \text{(farads por metro)}
\]

\[
\mu_0 = 4\pi \times 10^{-7} \, \text{N/A}^2
\quad \text{(newtons por ampère ao quadrado)}
\]

\subsection*{Resultado:}

\[
c \approx 3{,}0 \times 10^8 \, \text{m/s}
\]


%%%%%%%%%%%%%%%%%%%%%%%%%%%%%%%%%%%%%%%%%%
\noindent\rule{\linewidth}{1pt}\\

\section{Eletricidade e Magnetismo}

\section{Eletrostática}

\subsection{Eletrização}
\begin{itemize}
    \item Métodos: atrito, contato e indução.
    \item Cargas elétricas: positivas e negativas, quantizadas e conservadas.
\end{itemize}

\subsection{Lei de Coulomb}
\begin{equation*}
    F = k \frac{|q_1 q_2|}{r^2}
\end{equation*}
\begin{itemize}
    \item Força de interação entre duas cargas puntiformes no vácuo.
\end{itemize}

\subsection{Potencial Elétrico}
O potencial elétrico gerado por uma distribuição contínua de carga é dado por:

\begin{equation*}
V(\vec{r}) = \frac{1}{4\pi \varepsilon_0} \int \frac{dq}{|\vec{r} - \vec{r'}|}
\end{equation*}

Onde:
\begin{itemize}
  \item \( \vec{r} \): ponto onde se calcula o potencial,
  \item \( \vec{r'} \): ponto onde está o elemento de carga \( dq \),
  \item \( \varepsilon_0 \): permissividade do vácuo.
\end{itemize}

\section{Tipos de Distribuição}

\subsection{Distribuição Linear de Carga (fio)}

Densidade linear: \( \lambda = \frac{dq}{dl} \)

\[
V(\vec{r}) = \frac{1}{4\pi \varepsilon_0} \int \frac{\lambda \, dl'}{|\vec{r} - \vec{r'}|}
\]

\subsection{Distribuição Superficial de Carga (superfície)}

Densidade superficial: \( \sigma = \frac{dq}{dA} \)

\[
V(\vec{r}) = \frac{1}{4\pi \varepsilon_0} \int \frac{\sigma \, dA'}{|\vec{r} - \vec{r'}|}
\]

\subsection{Distribuição Volumétrica de Carga (volume)}

Densidade volumétrica: \( \rho = \frac{dq}{dV} \)

\[
V(\vec{r}) = \frac{1}{4\pi \varepsilon_0} \int \frac{\rho \, dV'}{|\vec{r} - \vec{r'}|}
\]

\section{Observações}

\begin{itemize}
  \item O potencial elétrico é uma grandeza escalar.
  \item A simetria do sistema pode facilitar os cálculos.
  \item Para pontos distantes, pode-se usar aproximações (ex: dipolo).
\end{itemize}

\subsection{Campo de Forças Coulombianas e Campo Elétrico}
\begin{equation*}
    \vec{E} = \frac{\vec{F}}{q} = k \frac{Q}{r^2} \hat{r}
\end{equation*}
\begin{itemize}
    \item Campo elétrico gerado por uma carga pontual.
    \item $\vec{E} = - \nabla V$
\end{itemize}

\subsection{Linhas de Força}
\begin{itemize}
    \item Representação gráfica da direção e sentido do campo elétrico.
    \item Saem de cargas positivas e entram em cargas negativas.
\end{itemize}

\subsection{Trabalho e Potencial Eletrostático}
\begin{itemize}
    \item Potencial elétrico: 
    \[
        V = k \frac{Q}{r}
    \]
    \item Energia potencial elétrica: 
    \[
        U = qV
    \]
    \item Trabalho da força elétrica:
    \[
        W = -\Delta U
    \]
\end{itemize}

\section{Corrente Contínua e Resistência}

\subsection{Corrente Elétrica}
\begin{itemize}
    \item Corrente: 
    \[
        I = \frac{dQ}{dt}
    \]
    \item Sentido convencional: do positivo para o negativo.
\end{itemize}

\subsection{Resistência Elétrica e Lei de Ohm}
\[
    U = R I
\]

\[ 
P = Ri^{2} = R.\left(\frac{U}{R}\right)^{2} = \frac{U^{2}}{R} = U.i
\]

\section*{Densidade de Carga Elétrica}

A densidade de carga descreve como a carga elétrica ($Q$) está distribuída no espaço. Existem três tipos principais:

\subsection*{1. Densidade Linear de Carga}

Utilizada quando a carga está distribuída ao longo de uma linha (como em fios):

\[
\lambda = \frac{dQ}{dl}
\]

\begin{itemize}
    \item $\lambda$: densidade linear de carga (C/m)
    \item $dQ$: quantidade infinitesimal de carga
    \item $dl$: elemento infinitesimal de comprimento
\end{itemize}

\subsection*{2. Densidade Superficial de Carga}

Usada quando a carga está distribuída sobre uma superfície (como em chapas condutoras):

\[
\sigma = \frac{dQ}{dA}
\]

\begin{itemize}
    \item $\sigma$: densidade superficial de carga (C/m²)
    \item $dA$: elemento de área
\end{itemize}

\subsection*{3. Densidade Volumétrica de Carga}

Usada quando a carga está distribuída em todo um volume:

\[
\rho = \frac{dQ}{dV}
\]

\begin{itemize}
    \item $\rho$: densidade volumétrica de carga (C/m³)
    \item $dV$: elemento de volume
\end{itemize}

\subsection{Campo Elétrico: Placa Infinita}

\begin{equation*}
    \vec{E} = \frac{\sigma}{\epsilon_{0}}\hat{x}
\end{equation*}

\subsection{Associação de Resistores}
\begin{itemize}
    \item Série: $R_{eq} = R_1 + R_2 + \dots$
    \item Paralelo: $\frac{1}{R_{eq}} = \frac{1}{R_1} + \frac{1}{R_2} + \dots$
\end{itemize}

\subsection{Resistividade e Temperatura}
\[
    R = \rho \frac{L}{A}, \quad \rho(T) = \rho_0[1 + \alpha(T - T_0)]
\]

\subsection{Efeito Joule}
\[
    Q = R I^2 t
\]

\subsection{Geradores e Receptores}
\begin{itemize}
    \item Geradores fornecem energia elétrica (ex: baterias).
    \item Receptores consomem energia elétrica (ex: motores).
    \item Equação geral do gerador:
    \[
        U = \mathcal{E} - r I
    \]
    \item Para receptores:
    \[
        U = \mathcal{E} + r I
    \]
\end{itemize}

\subsection{Pilhas em Série e Paralelo}
\begin{itemize}
    \item Série: $ \mathcal{E}_{eq} = \mathcal{E}_1 + \mathcal{E}_2 + \dots$
    \item Paralelo: mesma $ \mathcal{E}$, menor resistência interna.
\end{itemize}

\subsection{Leis de Kirchhoff}
\begin{itemize}
    \item Lei das malhas (tensões): soma das ddps em um circuito fechado é zero.
    \item Lei dos nós (correntes): soma das correntes que entram num nó é igual à soma das que saem.
\end{itemize}

\subsection{Instrumentos de Medida}
\begin{itemize}
    \item \textbf{Amperímetro}: mede corrente — ligado em série.
    \item \textbf{Voltímetro}: mede tensão — ligado em paralelo.
    \item \textbf{Multímetro}: mede corrente, tensão e resistência.
    \item \textbf{Ponte de Wheatstone}: circuito para medir resistências desconhecidas com alta precisão.
\end{itemize}

%%%%%%%%%%%%%%%%%%%%%%%%%%%%%%%%%%%%%%%%%%
\noindent\rule{\linewidth}{1pt}\\

\section{Magnetismo e Indução}

\subsection{Campo Magnético Gerado por Corrente Elétrica}
\begin{itemize}
    \item Fio retilíneo:
    \[
        \oint \vec{B}.d\vec{l} = \mu_{0}I_{eng}, \rightarrow B = \frac{\mu_0 I}{2\pi r}
    \]
    \item Espira circular:
    \[
        B = \frac{\mu_0 I}{2R}
    \]
    \item Solenóide (interior):
    \[
        B = \mu_0 n I
    \]
    Onde $n$ é o número de espiras por unidade de comprimento.
\end{itemize}

\section{Campo Magnético de uma Onda Eletromagnética}

Uma onda eletromagnética é composta por um campo elétrico $\vec{E}$ e um campo magnético $\vec{B}$ que oscilam perpendicularmente entre si e à direção de propagação.

\subsection{Representação vetorial}

Em uma onda plana no vácuo propagando-se na direção $x$, temos:

\begin{align*}
\vec{E}(x, t) &= E_0 \cos(kx - \omega t) \, \hat{y} \\
\vec{B}(x, t) &= B_0 \cos(kx - \omega t) \, \hat{z}
\end{align*}

\begin{itemize}
  \item $\vec{E}$: campo elétrico (direção $\hat{y}$)
  \item $\vec{B}$: campo magnético (direção $\hat{z}$)
  \item Propagação: direção $\hat{x}$
  \item $E_0$ e $B_0$: amplitudes dos campos
  \item $k = \dfrac{2\pi}{\lambda}$: número de onda
  \item $\omega = 2\pi f$: frequência angular
\end{itemize}

\subsection*{Relação entre os campos}

Como a onda se propaga no vácuo, temos:

\[
\frac{E_0}{B_0} = c
\quad \Rightarrow \quad
B_0 = \frac{E_0}{c}
\]

\begin{itemize}
  \item $c$: velocidade da luz no vácuo ($c \approx 3 \times 10^8~\text{m/s}$)
\end{itemize}

\subsection*{Direções ortogonais}

A orientação dos vetores segue a regra da mão direita:

\[
\vec{E} \times \vec{B} = \text{direção da propagação da onda}
\]

\subsection*{Energia transportada pela onda}

A densidade de energia média associada aos campos é:

\[
u = \frac{1}{2} \varepsilon_0 E^2 + \frac{1}{2} \frac{B^2}{\mu_0}
\]

E o vetor de Poynting (densidade de fluxo de energia) (Energia/Área.tempo):

\[
\vec{S} = \frac{1}{\mu_0} \vec{E} \times \vec{B}
\]

\subsection{Força Magn\'etica e Força El\'etrica}

\begin{itemize}
    \item For\c{c}a de Lorentz:
    \[
        \vec{F} = q\vec{E} + q \vec{v} \times \vec{B}
    \]

    \item Regra/M\~ao Direita: $ F = |q| v B \sin\theta$
\end{itemize}

Se a partícula tem massa $m$ e entra perpendicularmente no campo magnético:

\begin{itemize}
    \item Raio da trajetória:
    \begin{equation*}
        R = \frac{m \cdot v}{|q| \cdot B}
    \end{equation*}

    \item Período do movimento:
    \begin{equation*}
        T = \frac{2\pi m}{|q| \cdot B}
    \end{equation*}
\end{itemize}

\subsection{Trabalho realizado pela força magnética}
\begin{equation*}
    W = \vec{F} \cdot \vec{d} = 0
\end{equation*}

\section{Força Magnética sobre um Fio com Corrente}

\begin{equation*}
    \vec{F} = I \cdot \vec{L} \times \vec{B}
\end{equation*}

\begin{equation*}
    F = I \cdot L \cdot B \cdot \sin\theta.
\end{equation*}
Regra da M\~ao Esquerda

\subsection{Eletroímã}
\begin{itemize}
    \item Solenóide com núcleo ferromagnético, que se magnetiza quando a corrente circula.
\end{itemize}

\subsection{Indução Eletromagnética}
\begin{itemize}
    \item Lei de Faraday:
\[         \mathcal{E} = -\frac{d\Phi_B}{dt}     \] Onde $\Phi_B = B \cdot A \cdot \cos\theta$ is a magnético flux.
\end{itemize}

In an enrolamento with N espiras, temos that a força eletromotriz is $\mathcal{E}$ (cada espira) times N. 

\subsection{Lei de Lenz}
\begin{itemize}
    \item O sentido da corrente induzida é tal que seu campo magnético se opõe à variação do fluxo que a gerou.
\end{itemize}

\subsection{Campo Elétrico Induzido}
\begin{itemize}
    \item Um campo elétrico pode ser gerado por variação de campo magnético:
    \[
        \nabla \times \vec{E} = -\frac{\partial \vec{B}}{\partial t}
    \]
\end{itemize}

\section{Equações de Maxwell com Corrente de Deslocamento}

As equações de Maxwell descrevem os campos elétrico ($\vec{E}$) e magnético ($\vec{B}$). A quarta equação, conhecida como Lei de Ampère-Maxwell, inclui a \textbf{corrente de deslocamento}.

\subsection*{Forma Diferencial (Leis)}

\begin{align*}
\text{Gauss (elétrica)} \quad & \nabla \cdot \vec{E} = \frac{\rho}{\varepsilon_0} \\
\text{Gauss (magnética)} \quad & \nabla \cdot \vec{B} = 0 \\
\text{Faraday da indução} \quad & \nabla \times \vec{E} = -\frac{\partial \vec{B}}{\partial t} \\
\text{Ampère-Maxwell} \quad & \nabla \times \vec{B} = \mu_0 \vec{J} + \mu_0 \varepsilon_0 \frac{\partial \vec{E}}{\partial t}
\end{align*}

\subsubsection{Corrente de deslocamento (forma diferencial)}
\[
\vec{J}_d = \varepsilon_0 \frac{\partial \vec{E}}{\partial t}
\]

\subsection{Forma Integral}

\begin{align*}
\text{Gauss} \quad & \oint_{\partial V} \vec{E} \cdot d\vec{A} = \frac{1}{\varepsilon_0} \int_V \rho \, dV \\
\text{Magnetismo} \quad & \oint_{\partial V} \vec{B} \cdot d\vec{A} = 0 \\
\text{Faraday} \quad & \oint_{\partial S} \vec{E} \cdot d\vec{l} = -\frac{d}{dt} \int_S \vec{B} \cdot d\vec{A} \\
\text{Ampère-Maxwell} \quad & \oint_{\partial S} \vec{B} \cdot d\vec{l} = \mu_0 \int_S \vec{J} \cdot d\vec{A} + \\ \mu_0 \varepsilon_0 \frac{d}{dt} \int_S \vec{E} \cdot d\vec{A}
\end{align*}

\subsection{Corrente de deslocamento (forma integral)}
\[
I_d = \varepsilon_0 \frac{d}{dt} \int_S \vec{E} \cdot d\vec{A}
\]

\subsection*{Constantes envolvidas}

\begin{itemize}
  \item $\varepsilon_0$ — permissividade elétrica do vácuo
  \item $\mu_0$ — permeabilidade magnética do vácuo
  \item $\rho$ — densidade de carga elétrica
  \item $\vec{J}$ — densidade de corrente elétrica
  \item $\vec{J}_d$ — densidade de corrente de deslocamento
\end{itemize}

%%%%%%%%%%%%%%%%%%%%%%%%%%%%%%%%%%%%%%%%%%
\noindent\rule{\linewidth}{1pt}\\

\section{Física Moderna}

\section{Radiação do Corpo Negro e Constante de Planck}
\begin{itemize}
    \item Um corpo negro ideal absorve toda radiação incidente.
    \item A distribuição espectral da energia emitida depende da temperatura.
    \item Planck introduziu a quantização da energia:
    \[
        E = h\nu
    \]
    onde $h \approx 6{,}626 \times 10^{-34} \, \text{J} \cdot \text{s}$ é a constante de Planck.
\end{itemize}

\textbf{Lei de Stefan-Boltzmann:}
\[
P = \sigma A T^4,
\]
onde \( P \) é a potência total (W), \( A \) a área (m²), \( T \) a temperatura (K) e
\[
\sigma = 5{,}670 \times 10^{-8} \; \si{\watt\per\meter\squared\per\kelvin\tothe{4}}.
\]

\vspace{0.5cm}

\textbf{Lei de Wien:}
\[
\lambda_{\max} T = b,
\]
onde \(\lambda_{\max}\) é o comprimento de onda de máxima emissão (m), \( T \) a temperatura (K) e
\[
b = 2{,}898 \times 10^{-3} \; \si{\meter \kelvin}.
\]

\section{Efeito Fotoelétrico}
\begin{itemize}
    \item A luz incide sobre um metal e ejeta elétrons.
    \item Einstein explicou usando fótons com energia $E = h\nu$.
    \item Equação do efeito fotoelétrico:
    \[
        K_{\text{máx}} = h\nu - \phi
    \]
    onde $\phi$ é a função trabalho do material.
\end{itemize}

\section{Efeito Compton}
\begin{itemize}
    \item Espalhamento de fótons por elétrons livres.
    \item Mostra o comportamento corpuscular da radiação:
    \[
        \Delta \lambda = \lambda' - \lambda = \frac{h}{m_e c} (1 - \cos \theta)
    \]
\end{itemize}

\section{Dualidade Onda-Partícula}
\begin{itemize}
    \item Toda partícula possui propriedades ondulatórias:
    \[
        \lambda = \frac{h}{p}
    \]
    (relação de De Broglie).
    \item Confirmada por experimentos de difração de elétrons.
\end{itemize}

\section{Trabalho sobre uma carga elétrica}

Quando uma carga elétrica \( q \) se desloca em um campo elétrico, o \textbf{trabalho} realizado pelo campo é dado por:

\[
W = q \cdot \Delta V
\]

Onde:
\begin{itemize}
  \item \( W \): trabalho (em joules)
  \item \( q \): carga elétrica (em coulombs)
  \item \( \Delta V \): diferença de potencial elétrico (em volts)
\end{itemize}

\textbf{Observações:}
\begin{itemize}
  \item Se \( \Delta V > 0 \), o campo realiza trabalho negativo (carga contra o campo).
  \item Se \( \Delta V < 0 \), o campo realiza trabalho positivo (carga a favor do campo).
  \item O trabalho depende do potencial inicial e final, não do caminho percorrido (campo conservativo).
\end{itemize}

\section{Teoria da Relatividade Restrita}
\subsection{Postulados}
\begin{enumerate}
    \item As leis da Física são as mesmas em todos os referenciais inerciais.
    \item A velocidade da luz no vácuo é a mesma para todos os observadores inerciais.
\end{enumerate}

\subsection{Consequências}
\begin{itemize}
    \item Dilatação do tempo:
    \[
        \Delta t = \frac{\Delta t_0}{\sqrt{1 - \frac{v^2}{c^2}}}
    \]
    \item Contração do comprimento:
    \[
        L = L_0 \sqrt{1 - \frac{v^2}{c^2}}
    \]
\end{itemize}

\subsection{Energia Relativística}
\begin{itemize}
    \item Energia total:
    \[
        E = \gamma m c^2
    \quad \text{com } \gamma = \frac{1}{\sqrt{1 - \frac{v^2}{c^2}}}
    \]
    \item Energia em repouso:
    \[
        E_0 = m c^2
    \]
    \item Relação energia-momento:
    \[
        E^2 = (pc)^2 + (m c^2)^2
    \]
\end{itemize}

\section{Modelos Atômicos}
\subsection{Rutherford}
\begin{itemize}
    \item Descoberta do núcleo atômico.
    \item Átomo com núcleo positivo e elétrons ao redor.
    \item Instável segundo a eletrodinâmica clássica.
\end{itemize}

\subsection{Bohr}
\begin{itemize}
    \item Níveis de energia quantizados:
    \[
        E_n = -\frac{13{,}6}{n^2} \, \text{eV}
    \]
    \item Transições entre níveis explicam linhas espectrais do hidrogênio.
    \item As órbitas estáveis(estacionárias) ao redor do núcleo sem emitir radiação.
    \item Mudar de órbita $\rightarrow$ \textbf{Absorção} ou \textbf{Emissão}.
\end{itemize}

\subsection{Modelo Atômico de Bohr}

No modelo de Bohr, a luz é emitida quando um elétron salta de um nível mais externo ($n_i$) para um mais interno ($n_f$), liberando um fóton com energia específica:

\[
E = h \nu = \frac{hc}{\lambda}
\]

\noindent A fórmula de Rydberg permite calcular o comprimento de onda emitido:

\[
\frac{1}{\lambda} = R \left( \frac{1}{n_f^2} - \frac{1}{n_i^2} \right)
\]

\noindent Onde:
\begin{itemize}
  \item $\lambda$ é o comprimento de onda (m),
  \item $R$ é a constante de Rydberg ($R \approx 1{,}097 \times 10^7 \, \text{m}^{-1}$),
  \item $n_i$ é o nível inicial (mais energético),
  \item $n_f$ é o nível final (menos energético).
\end{itemize}

\subsubsection{Séries Espectrais do Hidrogênio}

\begin{itemize}
  \item \textbf{Lyman}: região ultravioleta (UV), com transições para $n_f = 1$ e $n_i > 1$.
  \item \textbf{Balmer}: região visível, com transições para $n_f = 2$ e $n_i > 2$.
  \item \textbf{Paschen}: região do infravermelho, com transições para $n_f = 3$ e $n_i > 3$.
  \item \textbf{Brackett}: região do infravermelho, com transições para $n_f = 4$ e $n_i > 4$.
  \item \textbf{Pfund}: região do infravermelho, com transições para $n_f = 5$ e $n_i > 5$.
\end{itemize}

\subsubsection{Observações}
\begin{itemize}
    \item A série de Balmer é a única cujas linhas estão no espectro visível.
    \item As outras séries estão em faixas não visíveis (UV e infravermelho).
    \item Cada transição corresponde a uma linha espectral com comprimento de onda específico.
\end{itemize}

\section{Princípio da Incerteza de Heisenberg}
\begin{itemize}
    \item É impossível conhecer simultaneamente posição e momento com precisão arbitrária:
    \[
        \Delta x \cdot \Delta p \geq \frac{\hbar}{2}
    \]
    onde $\hbar = \frac{h}{2\pi}$.
\end{itemize}

\section{Radioatividade}
\begin{itemize}
    \item Decaimento espontâneo de núcleos instáveis.
    \item Três tipos principais:
    \begin{itemize}
        \item \textbf{Alfa} ($\alpha$): emissão de núcleo de hélio.
        \item \textbf{Beta} ($\beta^-$): emissão de elétron (ou pósitron em $\beta^+$).
        \item \textbf{Gama} ($\gamma$): radiação eletromagnética de alta energia.
    \end{itemize}
    \item Lei do decaimento:
    \[
        N(t) = N_0 e^{-\lambda t}
    \]
    \item Meia-vida:
    \[
        t_{1/2} = \frac{\ln 2}{\lambda}
    \]
\end{itemize}

\section{Energia Nuclear}
\begin{itemize}
    \item Baseada na equivalência massa-energia de Einstein.
    \item Fissão nuclear: divisão de núcleos pesados (ex: $^{235}$U).
    \item Fusão nuclear: união de núcleos leves (ex: deuterônio + trítio).
    \item Liberação de energia:
    \[
        \Delta E = \Delta m \cdot c^2
    \]
    \item Aplicações: reatores nucleares, armas, medicina.
\end{itemize}

\end{multicols}

%%%%%%%% Bibliography 
% Os comandos para incluir as referências bibliográficas
%\printingbibliography

\end{document}
