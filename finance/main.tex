\documentclass[a4paper,12pt]{article}
\usepackage[brazil, english]{babel}
\usepackage[utf8]{inputenc}
\usepackage[T1]{fontenc}
\usepackage{geometry}
\usepackage{setspace}
\usepackage{titlesec}
\usepackage{hyperref}
\usepackage{graphicx}
\usepackage{caption}
\usepackage{subcaption}
\usepackage{fancyhdr}

%%%%%%%%%%%%%%%%%%%%%%%%%%%%%%%%%%%%%%%%%%%%%%%%%%
% These are some new commands that may be useful 
% for paper writing in general. If other new commands
% are needed for your specific paper, please feel 
% free to add here. 
%
% The currently available commands are organized in: 
% 1) Systems
% 2) Quantities
% 3) Energies and units
% 4) particle species
% 5) Colors package
% 6) hyperlink
%%%%%%%%%%%%%%%%%%%%%%%%%%%%%%%%%%%%%%%%%%%%%%%%%%

\usepackage{amsmath}
\usepackage{amssymb}
\usepackage{upgreek}
\usepackage{multirow}
\usepackage{setspace}% http://ctan.org/pkg/setspace
\usepackage{fancyhdr}
\usepackage{datetime}

% 1) SYSTEMS
\newcommand{\btc}               {\textbf{BTC}}
\newcommand{\btcspace}          {\textbf{BTC} }
\newcommand{\pow}               {\textbf{PoW}}

% 4) definition to references, biblatex and hyperlink
\usepackage[backend=bibtex, 
style=nature,  %style reference.
sorting=none,
firstinits=true %first name abbreviate
]{biblatex}

\usepackage{hyperref}
\hypersetup{
    colorlinks=true, %set "true" if you want colored links
    linktoc=all,     %set to "all" if you want both sections and subsections linked
    linkcolor=blue,  %choose some color if you want links to stand out
    citecolor= blue, % color of \cite{} in the text.
    urlcolor  = blue, % color of the link for the paper in references.
}

% 5) Tikz and figures
\usepackage{epsfig}
\usepackage{lmodern}
\usepackage{mathtools}
\usepackage[utf8]{luainputenc}
\usepackage{xspace}
\usepackage{tikz}
\usepackage{pgfplots}
\pgfplotsset{compat=newest}

\usetikzlibrary{positioning}
\usepackage{subcaption}

% 6) colors:
\usepackage{xcolor}
\definecolor{ao(english)}{rgb}{0.0, 0.5, 0.0} % dark green

% 7) Add lines numbers
%\usepackage{lineno}

% add pdf file to thesis:
\usepackage{pdfpages}

\hypersetup{
    colorlinks=true,% make the links colored
    linkcolor=blue
}

\usepackage{setspace}
\addbibresource{bibliography.bib}

\newcommand{\printingbibliography}{%

    \pagestyle{myheadings}
    \markright{}
    \sloppy
    \printbibliography[heading=bibintoc, % add to table of contents
                   title=Refer\^encias % Chapter name
                  ]
    \fussy%
}
\PassOptionsToPackage{table}{xcolor}

\pagestyle{fancy}
\fancyhf{}
\renewcommand{\headrulewidth}{0pt}
\fancyhead[R]{\thepage}

\geometry{a4paper,top=30mm,bottom=20mm,left=30mm,right=20mm}

\titleformat*{\section}{\bfseries\large}
\titleformat*{\subsection}{\bfseries\normalsize}

\title{ \textbf{\large Análise do Fluxo de Caixa e sua Importância na Gestão Financeira Empresarial}}
\author{ANDRÉ VIEIRA DA SILVA}
\date{17 de Outubro de 2023}

\begin{document}

\maketitle

\selectlanguage{brazil}

\begin{abstract}
\textbf{PALAVRAS CHAVE: }.
\end{abstract}

\selectlanguage{english}
\begin{abstract}

\textbf{KEYWORDS: }.
\end{abstract}

\selectlanguage{brazil}

\newpage
%\hypersetup{linkcolor=blue}
\tableofcontents
\newpage

\setstretch{1.3} % Altere o valor 1.2 para o valor desejado


\section{Introdução}

\subsection{Apresentação do problema}
\subsection{Objetivos do estudo}
\subsection{Justificativa}

\section{Revisão de Literatura}

\subsection{Conceitos de matemática financeira}
\subsection{Importância do fluxo de caixa na gestão financeira}
\subsection{Métodos de análise de fluxo de caixa}
\subsection{Relação entre contabilidade financeira e fluxo de caixa}

\section{Metodologia}

\subsection{Descrição da metodologia de pesquisa}
\subsection{Coleta de dados}
\subsection{Análise de dados}

\section{Fluxo de Caixa: Teoria e Prática}

\subsection{Elaboração de um fluxo de caixa}
\subsection{Demonstração de como registrar transações financeiras}
\subsection{Exemplos práticos de cálculos financeiros}

\section{Análise do Fluxo de Caixa Empresarial}

\subsection{Interpretação das demonstrações de fluxo de caixa}
\subsection{Identificação de tendências financeiras}
\subsection{Tomada de decisões com base nas informações do fluxo de caixa}

\section{Integração com Contabilidade Financeira}

\subsection{Relação entre o balanço patrimonial e o fluxo de caixa}
\subsection{Demonstração do resultado do exercício (DRE) e o fluxo de caixa}
\subsection{Como a contabilidade financeira influencia o fluxo de caixa}

\section{Estudo de Caso}

\subsection{Aplicação dos conceitos em um caso real de uma empresa}
\subsection{Análise do fluxo de caixa e sua relação com o desempenho financeiro}

\section{Considerações Finais}

\subsection{Resumo das principais conclusões}
\subsection{Sugestões para futuras pesquisas}

%%%%%%%% Bibliography 
% Os comandos para incluir as referências bibliográficas
\printingbibliography

\end{document}
