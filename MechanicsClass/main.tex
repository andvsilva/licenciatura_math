\documentclass[a4paper,12pt]{article}
\usepackage[brazil, english]{babel}
\usepackage[utf8]{inputenc}
\usepackage[T1]{fontenc}
\usepackage{geometry}
\usepackage{setspace}
\usepackage{titlesec}
\usepackage{hyperref}
\usepackage{graphicx}
\usepackage{caption}
\usepackage{subcaption}
\usepackage{fancyhdr}
\setlength{\headheight}{15pt}
\addtolength{\topmargin}{-2.5pt}
\usepackage{xcolor}
\usepackage{amsmath, amssymb, bm}
\usepackage{mathtools}
\usepackage{cancel}
\usepackage{tikz}
\usepackage{newunicodechar}
\usepackage{ragged2e}
\usepackage{setspace}
\usepackage{tikz-3dplot} % Necessário para coordenadas 3D
\usetikzlibrary{intersections}
\usepackage{siunitx}
\usetikzlibrary{3d, arrows.meta}

\usepackage{color}
\definecolor{myblue}{rgb}{.8, .8, 1}

\usepackage{amsmath}
\usepackage{empheq}

\newlength\mytemplen
\newsavebox\mytempbox

\makeatletter
\newcommand\mybluebox{%
    \@ifnextchar[%]
       {\@mybluebox}%
       {\@mybluebox[0pt]}}

\def\@mybluebox[#1]{%
    \@ifnextchar[%]
       {\@@mybluebox[#1]}%
       {\@@mybluebox[#1][0pt]}}

\def\@@mybluebox[#1][#2]#3{
    \sbox\mytempbox{#3}%
    \mytemplen\ht\mytempbox
    \advance\mytemplen #1\relax
    \ht\mytempbox\mytemplen
    \mytemplen\dp\mytempbox
    \advance\mytemplen #2\relax
    \dp\mytempbox\mytemplen
    \colorbox{myblue}{\hspace{1em}\usebox{\mytempbox}\hspace{1em}}}
\makeatother

\usepackage[most]{tcolorbox}

\newtcbox{\mymath}[1][]{%
    nobeforeafter, math upper, tcbox raise base,
    enhanced, colframe=blue!30!black,
    colback=blue!30, boxrule=1pt,
    #1}

\tcbset{
    highlight math style={
        enhanced,
        colframe=red!60!black,
        colback=yellow!50,
        arc=4pt,
        boxrule=1pt,
        drop fuzzy shadow
    }
    }

\usepackage{physics}
\usepackage{pgfplots}
\pgfplotsset{compat=1.17}

\linespread{1.5}

\definecolor{ao(english)}{rgb}{0.0, 0.5, 0.0}
\definecolor{byzantium}{rgb}{0.44, 0.16, 0.39}
\newunicodechar{∘}{\circ}

%%%%%%%%%%%%%%%%%%%%%%%%%%%%%%%%%%%%%%%%%%%%%%%%%%
% These are some new commands that may be useful 
% for paper writing in general. If other new commands
% are needed for your specific paper, please feel 
% free to add here. 
%
% The currently available commands are organized in: 
% 1) Systems
% 2) Quantities
% 3) Energies and units
% 4) particle species
% 5) Colors package
% 6) hyperlink
%%%%%%%%%%%%%%%%%%%%%%%%%%%%%%%%%%%%%%%%%%%%%%%%%%

\usepackage{amsmath}
\usepackage{amssymb}
\usepackage{upgreek}
\usepackage{multirow}
\usepackage{setspace}% http://ctan.org/pkg/setspace
\usepackage{fancyhdr}
\usepackage{datetime}

% 1) SYSTEMS 
\newcommand{\pp}           {pp\xspace}
\newcommand{\ppbar}        {\mbox{$\mathrm {p\overline{p}}$}\xspace}
\newcommand{\XeXe}         {\mbox{Xe--Xe}\xspace}
\newcommand{\PbPb}         {\mbox{Pb--Pb}\xspace}
\newcommand{\pA}           {\mbox{pA}\xspace}
\newcommand{\pPb}          {\mbox{p--Pb}\xspace}
\newcommand{\AuAu}         {\mbox{Au--Au}\xspace}
\newcommand{\dAu}          {\mbox{d--Au}\xspace}
\def\pA{$pA$\xspace}
\def\AA{$AA$\xspace}
\def\NN{$NN$\xspace}
\def\signn{$\sigma^{inel}_{NN}$\xspace}
\def\sigtotal{$\sigma_{\textnormal{tot}}$\xspace}
\def\mrm{\mathrm}
\def\ntrig{N_\mrm{trig}}
\newcommand{\rivet}{R\protect\scalebox{1}{IVET}\xspace}
\newcommand{\hepmc}{H\protect\scalebox{1}{EP}MC\xspace}
\newcommand{\herwig}{H\protect\scalebox{1}{ERWIG} 7\xspace}
\newcommand{\sherpa}{S\protect\scalebox{1}{HERPA}\xspace}
\newcommand{\urqmd}{U\protect\scalebox{1}{r}QMD\xspace}
\newcommand{\urqmdversion}{U\protect\scalebox{1}{r}QMD 3.4\xspace}
\newcommand{\pythia}{\protect\scalebox{1}{PYTHIA}\xspace}
\newcommand{\pythiaversion}{\protect\scalebox{1}{PYTHIA 8.2}\xspace}
\newcommand{\pythiaversionused}{\protect\scalebox{1}{PYTHIA 8.235}\xspace}
\newcommand{\pytang}{\protect\scalebox{1}{PYTHIA}/Angantyr\xspace}
\newcommand{\angantyr}{\protect\scalebox{1}{}Angantyr\xspace}
\newcommand{\pytangur}{\protect\scalebox{1}{PYTHIA}/Angantyr + U\protect\scalebox{1}{r}QMD\xspace}
\newcommand{\figref}[1]{Fig.~\ref{#1}}
\newcommand{\tabref}[1]{Tab.~\ref{#1}}
\renewcommand{\eqref}[1]{Eq.~(\ref{#1})}

% hydrodynamic simulation chain:
% TRENTo
\newcommand{\trento}{\protect\scalebox{1}{T$_{\text{R}}$ENT}o\xspace}
% KOMPOST : Linear kinetic theory propagator for initial conditions in heavy ion collisions
\newcommand{\kompost}{\protect\scalebox{1}{K$\varnothing$MP$\varnothing$ST}\xspace}
% MUSIC
\newcommand{\music}{\protect\scalebox{1}{MUSIC}\xspace}
% iSS
\newcommand{\iss}{\protect\scalebox{1}{iSS}\xspace}

% 2) QUANTITIES 
\newcommand{\s}            {\ensuremath{\sqrt{s}}\xspace}
\newcommand{\snn}          {\ensuremath{\sqrt{s_{\mathrm{NN}}}}\xspace}
\newcommand{\pt}           {\ensuremath{p_{\rm T}}\xspace}
\newcommand{\meanpt}       {$\langle p_{\mathrm{T}}\rangle$\xspace}
\newcommand{\ycms}         {\ensuremath{y_{\rm CMS}}\xspace}
\newcommand{\ylab}         {\ensuremath{y_{\rm lab}}\xspace}
\newcommand{\etarange}[1]  {\mbox{$\left | \eta \right |~<~#1$}}
\newcommand{\centbin}[2]  {\mbox{$#1-#2\%$}}
\newcommand{\ptrange}[2]  {\mbox{$#1 < p_{\mathrm{T}}\hspace{0.2cm} (\mathrm{GeV}/\mathrm{\textit{c}}) <#2$}}
\newcommand{\ptrangetrig}[2]  {\mbox{$#1 < p^{\mathrm{trigger}}_{\mathrm{T} }\hspace{0.2cm} (\mathrm{GeV}/\mathrm{\textit{c}}) <#2$}}
\newcommand{\ptrangeassoc}[2]  {\mbox{$#1 < p^{\mathrm{assoc}}_{\mathrm{T} }\hspace{0.2cm} (\mathrm{GeV}/\mathrm{\textit{c}}) <#2$}}
\newcommand{\etazerothree} {$\left|\eta \right| < 0.3$\xspace}
\newcommand{\etazerofive} {$\left|\eta \right| < 0.5$\xspace}
\newcommand{\etazeroeight} {$\left|\eta \right| < 0.8$\xspace}
\newcommand{\yrange}[1]    {\mbox{$\left | y \right |~<~#1$}}
\newcommand{\dndy}         {\ensuremath{\mathrm{d}N_\mathrm{ch}/\mathrm{d}y}\xspace}
\newcommand{\dndeta}       {\ensuremath{\mathrm{d}N_\mathrm{ch}/\mathrm{d}\eta}\xspace}
\newcommand{\dnchdydpt}   {\ensuremath{\mathrm{d}N_\mathrm{ch}/\mathrm{d}y\mathrm{d}p_{\mathrm{T}}}\xspace}
\newcommand{\dnchaadydpt}   {\ensuremath{\mathrm{d}N_\mathrm{ch}^{AA}/\mathrm{d}y\mathrm{d}p_{\mathrm{T}}}\xspace}
\newcommand{\dnchppdydpt}   {\ensuremath{\mathrm{d}N_\mathrm{ch}^{\mathrm{pp}}/\mathrm{d}y\mathrm{d}p_{\mathrm{T}}}\xspace}
\newcommand{\dnchdphi}{\ensuremath{\mathrm{d}N_\mathrm{ch}/\mathrm{d}\phi}\xspace}
\newcommand{\dnchddeltaphi}{\ensuremath{\mathrm{d}N_\mathrm{ch}/\mathrm{d}\Delta\upphi}\xspace}
\newcommand{\dndphi}{\ensuremath{\mathrm{d}N/\mathrm{d}\phi}\xspace}
\newcommand{\dnddeltaphi}{\ensuremath{\mathrm{d}N/\mathrm{d}\Delta\upphi}\xspace}
\newcommand{\avdndeta}     {\ensuremath{\langle\dndeta\rangle}\xspace}
\newcommand{\avdndetarap}  {$\langle$ dN$_{\textnormal{ch}}$/d$\eta$ $\rangle_{|\eta| < 0.5}$\xspace}
\newcommand{\dNdy}         {\ensuremath{\mathrm{d}N_\mathrm{ch}/\mathrm{d}y}\xspace}
\newcommand{\Npart}        {\ensuremath{N_\mathrm{part}}\xspace}
\newcommand{\meanNpart}    {$\langle$\ensuremath{N_\mathrm{part}}$\rangle$\xspace}
\newcommand{\ncoll}        {\ensuremath{N_\mathrm{coll}}\xspace}
\newcommand{\meanncoll}    {$\langle$\ensuremath{N_\mathrm{coll}}$\rangle$\xspace}
\newcommand{\averagencollhadronic}    {$\langle$\ensuremath{\mathrm{N}_\mathrm{coll}^{\mathrm{hadronic}}}$\rangle$\xspace}
\newcommand{\meantaa}      {$\langle$\ensuremath{T_\mathrm{AA}}$\rangle$\xspace}
\newcommand{\dEdx}         {\ensuremath{\textrm{d}E/\textrm{d}x}\xspace}
\newcommand{\RpPb}         {\ensuremath{R_{\rm pPb}}\xspace}
\newcommand{\raa}          {$R_{AA}$\xspace}
\newcommand{\vtwo}         {$v_{2}$\xspace}
\newcommand{\vtwoinitial}  {$v_{2}^{\mathrm{initial}}$\xspace}
\newcommand{\vtwofinal}    {$v_{2}^{\mathrm{final}}$\xspace}
\newcommand{\vtwofourfinal}{$v_{2}^{\mathrm{final}}\{4\}$\xspace}
\newcommand{\vtwofit}      {$v_{2}^{\mathrm{Fit}}$\xspace}
\newcommand{\vtwotwo}      {$v_{2}\{2\}$\xspace}
\newcommand{\vtwofour}     {$v_{2}\{4\}$\xspace}
\newcommand{\vtwopt}       {$v_{2}(p_{\textnormal{T}})$\xspace}
\newcommand{\vtwoptfit}    {$v_{2}^{\mathrm{Fit}}(p_{\textnormal{T}})$\xspace}
\newcommand{\nch}          {\ensuremath{N_\mathrm{ch}}\xspace}
\newcommand{\psireactionplane}          {$\Psi_{\textnormal{RP}}$\xspace}
\newcommand{\deltaphireactionplane}     {$\Delta\upphi = \phi - \Psi_{\textnormal{RP}}$\xspace}
\newcommand{\nevdnchddeltaphi}     {(1/N$_{\textnormal{ev}}$)dN$_{\textnormal{ch}}$/d$\Delta\upphi$\xspace}
\newcommand{\meannch}      {\ensuremath{\langle N_\mathrm{ch}\rangle}\xspace}
\newcommand{\etamodule}    {\ensuremath{|\eta|}\xspace}
\newcommand{\qbar}         {$\bar{\textnormal{q}}$\xspace}
\newcommand{\qqbar}        {$\textnormal{q}\bar{\textnormal{q}}$\xspace}
\newcommand{\qqbarzero}    {$\textnormal{q}_{0}\bar{\textnormal{q}}_{0}$\xspace}
\newcommand{\qqqbars}      {$\bar{\textnormal{q}}\bar{\textnormal{q}}\bar{\textnormal{q}}$\xspace}
\newcommand{\alphastrong}  {$\alpha_{\textnormal{s}}$\xspace}
\newcommand{\alphastrongdistance}  {$\alpha_{\textnormal{s}}$(R)\xspace}
\newcommand{\qtwo}         {Q$^2$\xspace}
\newcommand{\alphastrongqtwo}  {$\alpha_{\textnormal{s}}$(Q$^2$)\xspace}
\newcommand{\lambdaqcd}        {$\Lambda_{\textnormal{QCD}}$\xspace}
\newcommand{\sectionpp}        {$\sigma^{\textnormal{pp}}_{\textnormal{inel}}$\xspace}

% 3) ENERGIES, UNITS
\newcommand{\sqrts}        {$\sqrt{s}$\xspace}
\newcommand{\sqrtsnn}      {$\sqrt{s_{\mathrm{NN}}}$\xspace}
\newcommand{\nineH}        {$\sqrt{s}~=~0.9$~Te\kern-.1emV\xspace}
\newcommand{\seven}        {$\sqrt{s}~=~7$~Te\kern-.1emV\xspace}
\newcommand{\twoH}         {$\sqrt{s}~=~0.2$~Te\kern-.1emV\xspace}
\newcommand{\twosevensix}  {$\sqrt{s}~=~2.76$~Te\kern-.1emV\xspace}
\newcommand{\five}         {$\sqrt{s}~=~5.02$~Te\kern-.1emV\xspace}
\newcommand{\twohundrernn} {$\sqrt{s_{\mathrm{NN}}}=200$~Ge\kern-.1emV\xspace}
\newcommand{\twosevensixnn} {$\sqrt{s_{\mathrm{NN}}}=2.76$~Te\kern-.1emV\xspace}
\newcommand{\fivenn}       {$\sqrt{s_{\mathrm{NN}}}~=~5.02$~Te\kern-.1emV\xspace}
\newcommand{\fivefourfournn} {$\sqrt{s_{\mathrm{NN}}}=5.44$~Te\kern-.1emV\xspace}
\newcommand{\LT}           {L{\'e}vy-Tsallis\xspace}
\newcommand{\GeVc}         {Ge\kern-.1emV/$c$\xspace}
\newcommand{\MeVc}         {Me\kern-.1emV/$c$\xspace}
\newcommand{\TeV}          {Te\kern-.1emV\xspace}
\newcommand{\GeV}          {Ge\kern-.1emV\xspace}
\newcommand{\MeV}          {Me\kern-.1emV\xspace}
\newcommand{\GeVmass}      {Ge\kern-.2emV/$c^2$\xspace}
\newcommand{\MeVmass}      {Me\kern-.2emV/$c^2$\xspace}
\newcommand{\lumi}         {\ensuremath{\mathcal{L}}\xspace}
\newcommand{\fmc}         {fm\kern-.1em/$c$\xspace}

% 4) PARTICLE SPECIES 
\newcommand{\ee}           {\ensuremath{e^{+}e^{-}}} 
\newcommand{\pip}          {\ensuremath{\pi^{+}}\xspace}
\newcommand{\pim}          {\ensuremath{\pi^{-}}\xspace}
\newcommand{\kap}          {\ensuremath{\rm{K}^{+}}\xspace}
\newcommand{\kam}          {\ensuremath{\rm{K}^{-}}\xspace}
\newcommand{\pbar}         {\ensuremath{\rm\overline{p}}\xspace}
\newcommand{\kzero}        {\ensuremath{{\rm K}^{0}_{\rm{S}}}\xspace}
\newcommand{\lmb}          {\ensuremath{\Lambda}\xspace}
\newcommand{\almb}         {\ensuremath{\overline{\Lambda}}\xspace}
\newcommand{\Om}           {\ensuremath{\Omega^-}\xspace}
\newcommand{\Mo}           {\ensuremath{\overline{\Omega}^+}\xspace}
\newcommand{\X}            {\ensuremath{\Xi^-}\xspace}
\newcommand{\Ix}           {\ensuremath{\overline{\Xi}^+}\xspace}
\newcommand{\Xis}          {\ensuremath{\Xi^{\pm}}\xspace}
\newcommand{\Oms}          {\ensuremath{\Omega^{\pm}}\xspace}
\newcommand{\degree}       {\ensuremath{^{\rm o}}\xspace}
\newcommand{\comment}[1]{}

% two-particle angular correlation
\newcommand{\deltaphitriggassoc}    {$\Delta\upphi = |\phi_{\textnormal{trigger}} - \phi_{\textnormal{assoc}}|$\xspace}
\newcommand{\deltaetatriggassoc}    {$\Delta\upeta = |\eta_{\textnormal{trigger}} - \eta_{\textnormal{assoc}}|$\xspace}
\newcommand{\etatrigg}    {$\eta_{\textnormal{trigger}}$\xspace}
\newcommand{\etaassoc}    {$\eta_{\textnormal{assoc}}$\xspace}
\newcommand{\deltaphideltaeta}      {$\Delta\upphi-\Delta\upeta$\xspace}
\newcommand{\deltaphi}              {$\Delta\upphi$\xspace}
\newcommand{\moduledeltaphipitwo}   {$|\Delta\upphi| < \pi/2 $\xspace}
\newcommand{\deltaeta}              {$\Delta\upeta$\xspace}
\newcommand{\moduledeltaeta}        {$|\Delta\upeta|$\xspace}
\newcommand{\deltaphiapproxzero}    {$\Delta\upphi = 0$\xspace}
\newcommand{\deltaphiapproxpi}      {$\Delta\upphi = \pi$\xspace}
\newcommand{\deltaetaapproxzero}    {$\Delta\upeta = 0$\xspace}
\newcommand{\corrfunc}              {C($\Delta\upphi$, $\Delta\upeta$)\xspace}
\newcommand{\corrfunccorrect}              {C$_{\mathrm{correct}}(\Delta\upphi$, $\Delta\upeta$)\xspace}
\newcommand{\corrfuncmix}              {C$_{\mathrm{mix}}(\Delta\upphi$, $\Delta\upeta$)\xspace}
\newcommand{\corrfuncdeltaphi}      {C($\Delta\upphi$)\xspace}
\newcommand{\pttrigger}             {$p_{\textnormal{T}}^{\textnormal{trigger}}$\xspace}
\newcommand{\ptassoc}               {$p_{\textnormal{T}}^{\textnormal{assoc}}$\xspace}
\newcommand{\ratioyieldawaynearside}{Y$_{\textnormal{Away}}$/Y$_{\textnormal{Near}}$\xspace}

% 4) definition to references, biblatex and hyperlink
\usepackage[backend=bibtex, 
style=nature,  %style reference.
sorting=none,
firstinits=true %first name abbreviate
]{biblatex}

\usepackage{hyperref}
\hypersetup{
    colorlinks=true, %set "true" if you want colored links
    linktoc=all,     %set to "all" if you want both sections and subsections linked
    linkcolor=blue,  %choose some color if you want links to stand out
    citecolor= blue, % color of \cite{} in the text.
    urlcolor  = blue, % color of the link for the paper in references.
}

% 5) Tikz and figures
\usepackage{epsfig}
\usepackage{lmodern}
\usepackage{mathtools}
\usepackage[utf8]{luainputenc}
\usepackage{xspace}
\usepackage{tikz}
\usepackage{pgfplots}
\pgfplotsset{compat=newest}

\usetikzlibrary{positioning}
\usepackage{subcaption}

% 6) colors:
\usepackage{xcolor}
\definecolor{ao(english)}{rgb}{0.0, 0.5, 0.0} % dark green

% 7) Add lines numbers
%\usepackage{lineno}

% add pdf file to thesis:
\usepackage{pdfpages}

\hypersetup{
    colorlinks=true,% make the links colored
    linkcolor=blue
}

\usepackage{setspace}
\addbibresource{bibliography.bib}

\newcommand{\printingbibliography}{%

    \pagestyle{myheadings}
    \markright{}
    \sloppy
    \printbibliography[heading=bibintoc, % add to table of contents
                   title=Refer\^encias % Chapter name
                  ]
    \fussy%
}
\PassOptionsToPackage{table}{xcolor}

\pagestyle{fancy}
\fancyhf{}
\renewcommand{\headrulewidth}{0pt}
\fancyhead[R]{\thepage}

\geometry{a4paper,top=30mm,bottom=20mm,left=30mm,right=20mm}

\titleformat*{\section}{\bfseries\large}
\titleformat*{\subsection}{\bfseries\normalsize}

\title{ \textbf{\large } }
\author{}
\date{ter 03 jun 2025 19:09:14}

\begin{document}

\maketitle
\noindent\rule{\linewidth}{0.4pt}\\

\justifying

\begin{flushleft}
\textbf{\textcolor{blue}{\Large \textbf{1.}}}\\
\noindent
Um estudante chuta um bloco sem atrito com uma velocidade inicial \(v_0\), de modo que ele desliza reto para cima ao longo de um plano que está inclinado com um ângulo \(\theta\) acima da horizontal. 

\begin{itemize}
    \item[(a)] Escreva a segunda lei de Newton para o bloco e resolva-a para obter a posição do bloco como uma função do tempo.
    \item[(b)] Quanto tempo o bloco levará até retornar à sua posição original?
\end{itemize}

\vspace{0.5cm}
\section*{Solução}

\subsection*{(a) \colorbox{yellow!20}{Segunda Lei de Newton e Posição em função do tempo}}

Como o plano está \textbf{sem atrito}, a única força atuando sobre o bloco ao longo do plano inclinado é a componente do peso. Projetando o peso na direção do plano, temos:

\[
F = -mg\sin\theta
\]

Aplicando a segunda lei de Newton:

\[
F = ma \Rightarrow -mg\sin\theta = ma \Rightarrow a = -g\sin\theta
\]

A aceleração é constante, então usamos a equação horária do movimento uniformemente acelerado (MRUA):

\[
x(t) = x_0 + v_0 t + \frac{1}{2} a t^2
\]

Supondo que a posição inicial seja \(x_0 = 0\), temos:

\[
x(t) = v_0 t - \frac{1}{2} g \sin\theta \cdot t^2
\]

\textbf{Portanto, a posição do bloco em função do tempo é:}

\[
\boxed{x(t) = v_0 t - \frac{1}{2} g \sin\theta \cdot t^2}
\]

\vspace{0.5cm}
\subsection*{(b) \colorbox{yellow!20}{Tempo para retornar à posição original}}

O bloco retorna à posição original quando \(x(t) = 0\). Usamos a equação obtida:

\[
v_0 t - \frac{1}{2} g \sin\theta \cdot t^2 = 0
\]

Colocando \(t\) em evidência:

\[
t \left(v_0 - \frac{1}{2} g \sin\theta \cdot t\right) = 0
\]

As soluções são:
\[
t = 0 \quad \text{(instante inicial)}, \quad t = \frac{2v_0}{g \sin\theta}
\]

\textbf{Portanto, o tempo total até o bloco retornar à posição inicial é:}

\[
\boxed{t = \frac{2v_0}{g \sin\theta}}
\]

\end{flushleft}

\noindent\rule{\linewidth}{0.4pt}\\

\begin{flushleft}
\textbf{\textcolor{blue}{\Large \textbf{2.}}}\\
\noindent
Considere uma esfera (diâmetro \( D \), densidade \( \rho_{\text{esf}} \)) caindo no ar (densidade \( \rho_{\text{ar}} \)) e assuma que a força de arrasto é puramente quadrática.

\begin{enumerate}
    \item[(a)] Use a equação (1) (com \( \kappa = 1/4 \) para a esfera) para mostrar que a velocidade limite será dada pela equação (2).
    \[
    f_{\text{quad}} = \kappa \rho_{\text{ar}} A v^2 \tag{1}
    \]
    \[
    v_{\text{lim}} = \sqrt{ \frac{8}{3} D g \cdot \frac{\rho_{\text{esf}}}{\rho_{\text{ar}}} } \tag{2}
    \]

    \item[(b)] Use esse resultado para mostrar que em duas esferas do mesmo tamanho, a mais densa irá eventualmente cair mais rápido.

    \item[(c)] Para duas esferas do mesmo material, mostre que a maior irá eventualmente cair mais rapidamente.
\end{enumerate}

\vspace{0.5cm}
\section*{Solução}

\subsection*{(a) \colorbox{yellow!20}{Derivação da velocidade limite}}

No regime de queda com velocidade constante (velocidade limite), a força peso é equilibrada pela força de arrasto quadrática:

\[
\boxed{
P = f_{\text{quad}}
}
\]

A força peso de uma esfera de densidade \( \rho_{\text{esf}} \), volume \( V \), e aceleração gravitacional \( g \) é:

\[
P = m g = \rho_{\text{esf}} V g = \rho_{\text{esf}} \cdot \frac{\pi D^3}{6} \cdot g
\]

A área de seção transversal da esfera é:

\[
A = \frac{\pi D^2}{4}
\]

A força de arrasto é:

\[
f_{\text{quad}} = \kappa \rho_{\text{ar}} A v_{\text{lim}}^2 = \kappa \rho_{\text{ar}} \cdot \frac{\pi D^2}{4} \cdot v_{\text{lim}}^2
\]

Igualando as forças:

\[
\rho_{\text{esf}} \cdot \frac{\pi D^3}{6} \cdot g = \kappa \rho_{\text{ar}} \cdot \frac{\pi D^2}{4} \cdot v_{\text{lim}}^2
\]

Cancelando \( \pi \) dos dois lados e isolando \( v_{\text{lim}}^2 \):

\[
\rho_{\text{esf}} \cdot \frac{D^3}{6} \cdot g = \kappa \rho_{\text{ar}} \cdot \frac{D^2}{4} \cdot v_{\text{lim}}^2
\]

\[
v_{\text{lim}}^2 = \frac{ \rho_{\text{esf}} \cdot D^3 \cdot g }{6 \kappa \rho_{\text{ar}} \cdot D^2 / 4 } = \frac{4 \rho_{\text{esf}} D g}{6 \kappa \rho_{\text{ar}}}
\]

Substituindo \( \kappa = \frac{1}{4} \):

\[
v_{\text{lim}}^2 = \frac{4 \rho_{\text{esf}} D g}{6 \cdot \frac{1}{4} \rho_{\text{ar}} } = \frac{16}{6} \cdot \frac{\rho_{\text{esf}} D g}{\rho_{\text{ar}} } = \frac{8}{3} \cdot \frac{\rho_{\text{esf}} D g}{\rho_{\text{ar}} }
\]

Logo,

\[
\boxed{
v_{\text{lim}} = \sqrt{ \frac{8}{3} D g \cdot \frac{\rho_{\text{esf}}}{\rho_{\text{ar}}} }} \quad \checkmark
\]

\subsection*{(b) Comparando densidades com mesmo diâmetro}

\colorbox{yellow!20}{Se duas esferas têm o mesmo diâmetro \( D \)}, então:

\[
v_{\text{lim}} \propto \sqrt{ \rho_{\text{esf}} }
\]

Ou seja, quanto maior for \( \rho_{\text{esf}} \), maior será \( v_{\text{lim}} \). \colorbox{red!15}{Logo, a esfera mais densa cai mais} 
\colorbox{red!15}{rapidamente.}

\subsection*{(c) Comparando tamanhos com mesmo material}

Se duas esferas são do mesmo material, então \( \rho_{\text{esf}} \) é constante. Logo:

\[
v_{\text{lim}} \propto \sqrt{D}
\]

Ou seja, \colorbox{red!15}{quanto maior o diâmetro \( D \), maior a velocidade limite.} Assim, a esfera maior cairá mais rapidamente.


\end{flushleft}

\noindent\rule{\linewidth}{0.4pt}\\

\begin{flushleft}
\textbf{\textcolor{blue}{\Large \textbf{3.}}}\\
Uma partícula de massa \( m \) está se movendo sobre uma mesa sem atrito e está presa a uma mola de massa 
desprezível, cujo lado oposto passa através de um orifício sobre a mesa, onde ela é fixada. 
Inicialmente, a partícula está se movendo em um círculo de raio \( r_0 \) com velocidade angular \( \omega_0 \), 
mas agora puxamos a mola na direção do orifício até um comprimento \( r \) ser formado entre o orifício e a partícula. 
Qual é a velocidade angular da partícula nesse momento?


\vspace{0.5cm}
\section*{Solução}

\colorbox{yellow!20}{Como não há torques externos, o \textbf{momento angular} da partícula em relação ao orifício} 
\colorbox{yellow!20}{é conservado.}

\subsection*{Momento Angular Inicial}

\[
L_0 = I_0 \omega_0 = m r_0^2 \omega_0
\]

\subsection*{Momento Angular Final}

\[
L = I \omega = m r^2 \omega
\]

\subsection*{Conservação do Momento Angular}

\[
L_0 = L \Rightarrow m r_0^2 \omega_0 = m r^2 \omega
\]

Cancelando a massa \( m \) dos dois lados:

\[
r_0^2 \omega_0 = r^2 \omega
\]

Resolvendo para \( \omega \):

\[
\boxed{\omega = \frac{r_0^2}{r^2} \omega_0}
\]

\section*{\colorbox{green!15}{Interpretação Física}}

Como \( r < r_0 \), segue que \( \omega > \omega_0 \). Isso significa que a velocidade angular da partícula aumenta 
à medida que ela se aproxima do orifício. Esse efeito é análogo ao de uma patinadora que recolhe os braços para girar 
mais rápido: \colorbox{yellow!20}{ao reduzir o momento de} \colorbox{yellow!20}{inércia, a velocidade angular aumenta 
para conservar o momento angular total.}

\end{flushleft}


\noindent\rule{\linewidth}{0.4pt}\\

\begin{flushleft}
\textbf{\textcolor{blue}{\Large \textbf{4.}}}\\

Considere um foguete sujeito a uma força de resistência linear, \( f = -bv \), e mais nenhuma outra força externa. 
Use a equação (3) para mostrar que, se o foguete parte do repouso e expele a massa a uma constante \( k = -\dot{m} \), então, 
a sua velocidade será dada pela equação (4).

\begin{equation}
    m \dot{v} = -\dot{m} v_{\text{ex}} + F^{\text{ext}} \tag{3}
\end{equation}

\begin{equation}
    v = \frac{k}{b} v_{\text{ex}} \left[ 1 - \left( \frac{m}{m_0} \right)^{b/k} \right] \tag{4}
\end{equation}

\vspace{0.5cm}
\section*{Solução}

A equação (3) pode ser escrita como:

\[
m \dot{v} = -\dot{m} v_{\text{ex}} - bv
\]

Como \( \dot{m} = -k \), temos \( -\dot{m} = k \), logo:

\[
m \dot{v} = k v_{\text{ex}} - bv
\]

Dividindo ambos os lados por \( m \):

\[
\dot{v} = \frac{k}{m} v_{\text{ex}} - \frac{b}{m} v \tag{5}
\]

Agora, usamos a relação entre massa e tempo. Como \( \dot{m} = -k \), temos:

\[
m(t) = m_0 - kt
\]

Queremos eliminar \( t \) e resolver em função de \( m \). Escrevemos \( \dot{v} \) como derivada em relação à massa \( m \) usando a regra da cadeia:

\[
\frac{dv}{dt} = \frac{dv}{dm} \cdot \frac{dm}{dt} = -k \frac{dv}{dm}
\]

Substituindo isso na equação (5):

\[
-k \frac{dv}{dm} = \frac{k}{m} v_{\text{ex}} - \frac{b}{m} v
\]

Multiplicando ambos os lados por \( m \):

\[
-k m \frac{dv}{dm} = k v_{\text{ex}} - b v
\]

Multiplicando ambos os lados por \( \frac{1}{k} \):

\[
- m \frac{dv}{dm} = v_{\text{ex}} - \frac{b}{k} v
\]

Reorganizando:

\[
\frac{dv}{dm} + \frac{b}{k m} v = \frac{v_{\text{ex}}}{m}
\]

Essa é uma equação linear de primeira ordem para \( v(m) \). O fator integrante é:

\[
\mu(m) = \exp\left( \int \frac{b}{k m} \, dm \right) = m^{b/k}
\]

Multiplicando ambos os lados da equação por \( \mu(m) \):

\[
m^{b/k} \frac{dv}{dm} + \frac{b}{k} m^{b/k - 1} v = v_{\text{ex}} m^{b/k - 1}
\]

O lado esquerdo é a derivada do produto:

\[
\frac{d}{dm} \left( m^{b/k} v \right) = v_{\text{ex}} m^{b/k - 1}
\]

Integrando ambos os lados:

\[
\int \frac{d}{dm} \left( m^{b/k} v \right) \, dm = \int v_{\text{ex}} m^{b/k - 1} \, dm
\]

\[
m^{b/k} v = v_{\text{ex}} \cdot \frac{m^{b/k}}{b/k} + C
\]

\[
m^{b/k} v = \frac{k}{b} v_{\text{ex}} m^{b/k} + C
\]

Dividindo ambos os lados por \( m^{b/k} \):

\[
v = \frac{k}{b} v_{\text{ex}} + \frac{C}{m^{b/k}}
\]

Para encontrar a constante \( C \), \colorbox{yellow!20}{usamos a condição inicial: no instante em que \( m = m_0 \),} 
\colorbox{yellow!20}{\( v = 0 \):}

\[
0 = \frac{k}{b} v_{\text{ex}} + \frac{C}{m_0^{b/k}} \Rightarrow \boxed{C = - \frac{k}{b} v_{\text{ex}} m_0^{b/k}}
\]

Substituindo \( C \) na equação da velocidade:

\[
v = \frac{k}{b} v_{\text{ex}} - \frac{k}{b} v_{\text{ex}} \left( \frac{m_0}{m} \right)^{b/k}
\]

\[
v = \frac{k}{b} v_{\text{ex}} \left[ 1 - \left( \frac{m_0}{m} \right)^{b/k} \right]
\]

Finalmente, reescrevendo como na equação (4):

\[
\boxed{v = \frac{k}{b} v_{\text{ex}} \left[ 1 - \left( \frac{m}{m_0} \right)^{b/k} \right]} \quad \checkmark
\]

\end{flushleft}

%%%%%%%% Bibliography 
% Os comandos para incluir as referências bibliográficas
%\printingbibliography

\end{document}
