\documentclass[a4paper,12pt]{article}
\usepackage[brazil]{babel}
\usepackage[utf8]{inputenc}
\usepackage[T1]{fontenc}
\usepackage{geometry}
\usepackage{setspace}
\usepackage{titlesec}
\usepackage{hyperref}
\usepackage{graphicx}
\usepackage{caption}
\usepackage{subcaption}
\usepackage{fancyhdr}
\usepackage{xcolor}

%%%%%%%%%%%%%%%%%%%%%%%%%%%%%%%%%%%%%%%%%%%%%%%%%%
% These are some new commands that may be useful 
% for paper writing in general. If other new commands
% are needed for your specific paper, please feel 
% free to add here. 
%
% The currently available commands are organized in: 
% 1) Systems
% 2) Quantities
% 3) Energies and units
% 4) particle species
% 5) Colors package
% 6) hyperlink
%%%%%%%%%%%%%%%%%%%%%%%%%%%%%%%%%%%%%%%%%%%%%%%%%%

\usepackage{amsmath}
\usepackage{amssymb}
\usepackage{upgreek}
\usepackage{multirow}
\usepackage{setspace}% http://ctan.org/pkg/setspace
\usepackage{fancyhdr}
\usepackage{datetime}

% 1) SYSTEMS 
\newcommand{\pp}           {pp\xspace}
\newcommand{\ppbar}        {\mbox{$\mathrm {p\overline{p}}$}\xspace}
\newcommand{\XeXe}         {\mbox{Xe--Xe}\xspace}
\newcommand{\PbPb}         {\mbox{Pb--Pb}\xspace}
\newcommand{\pA}           {\mbox{pA}\xspace}
\newcommand{\pPb}          {\mbox{p--Pb}\xspace}
\newcommand{\AuAu}         {\mbox{Au--Au}\xspace}
\newcommand{\dAu}          {\mbox{d--Au}\xspace}
\def\pA{$pA$\xspace}
\def\AA{$AA$\xspace}
\def\NN{$NN$\xspace}
\def\signn{$\sigma^{inel}_{NN}$\xspace}
\def\sigtotal{$\sigma_{\textnormal{tot}}$\xspace}
\def\mrm{\mathrm}
\def\ntrig{N_\mrm{trig}}
\newcommand{\rivet}{R\protect\scalebox{1}{IVET}\xspace}
\newcommand{\hepmc}{H\protect\scalebox{1}{EP}MC\xspace}
\newcommand{\herwig}{H\protect\scalebox{1}{ERWIG} 7\xspace}
\newcommand{\sherpa}{S\protect\scalebox{1}{HERPA}\xspace}
\newcommand{\urqmd}{U\protect\scalebox{1}{r}QMD\xspace}
\newcommand{\urqmdversion}{U\protect\scalebox{1}{r}QMD 3.4\xspace}
\newcommand{\pythia}{\protect\scalebox{1}{PYTHIA}\xspace}
\newcommand{\pythiaversion}{\protect\scalebox{1}{PYTHIA 8.2}\xspace}
\newcommand{\pythiaversionused}{\protect\scalebox{1}{PYTHIA 8.235}\xspace}
\newcommand{\pytang}{\protect\scalebox{1}{PYTHIA}/Angantyr\xspace}
\newcommand{\angantyr}{\protect\scalebox{1}{}Angantyr\xspace}
\newcommand{\pytangur}{\protect\scalebox{1}{PYTHIA}/Angantyr + U\protect\scalebox{1}{r}QMD\xspace}
\newcommand{\figref}[1]{Fig.~\ref{#1}}
\newcommand{\tabref}[1]{Tab.~\ref{#1}}
\renewcommand{\eqref}[1]{Eq.~(\ref{#1})}

% hydrodynamic simulation chain:
% TRENTo
\newcommand{\trento}{\protect\scalebox{1}{T$_{\text{R}}$ENT}o\xspace}
% KOMPOST : Linear kinetic theory propagator for initial conditions in heavy ion collisions
\newcommand{\kompost}{\protect\scalebox{1}{K$\varnothing$MP$\varnothing$ST}\xspace}
% MUSIC
\newcommand{\music}{\protect\scalebox{1}{MUSIC}\xspace}
% iSS
\newcommand{\iss}{\protect\scalebox{1}{iSS}\xspace}

% 2) QUANTITIES 
\newcommand{\s}            {\ensuremath{\sqrt{s}}\xspace}
\newcommand{\snn}          {\ensuremath{\sqrt{s_{\mathrm{NN}}}}\xspace}
\newcommand{\pt}           {\ensuremath{p_{\rm T}}\xspace}
\newcommand{\meanpt}       {$\langle p_{\mathrm{T}}\rangle$\xspace}
\newcommand{\ycms}         {\ensuremath{y_{\rm CMS}}\xspace}
\newcommand{\ylab}         {\ensuremath{y_{\rm lab}}\xspace}
\newcommand{\etarange}[1]  {\mbox{$\left | \eta \right |~<~#1$}}
\newcommand{\centbin}[2]  {\mbox{$#1-#2\%$}}
\newcommand{\ptrange}[2]  {\mbox{$#1 < p_{\mathrm{T}}\hspace{0.2cm} (\mathrm{GeV}/\mathrm{\textit{c}}) <#2$}}
\newcommand{\ptrangetrig}[2]  {\mbox{$#1 < p^{\mathrm{trigger}}_{\mathrm{T} }\hspace{0.2cm} (\mathrm{GeV}/\mathrm{\textit{c}}) <#2$}}
\newcommand{\ptrangeassoc}[2]  {\mbox{$#1 < p^{\mathrm{assoc}}_{\mathrm{T} }\hspace{0.2cm} (\mathrm{GeV}/\mathrm{\textit{c}}) <#2$}}
\newcommand{\etazerothree} {$\left|\eta \right| < 0.3$\xspace}
\newcommand{\etazerofive} {$\left|\eta \right| < 0.5$\xspace}
\newcommand{\etazeroeight} {$\left|\eta \right| < 0.8$\xspace}
\newcommand{\yrange}[1]    {\mbox{$\left | y \right |~<~#1$}}
\newcommand{\dndy}         {\ensuremath{\mathrm{d}N_\mathrm{ch}/\mathrm{d}y}\xspace}
\newcommand{\dndeta}       {\ensuremath{\mathrm{d}N_\mathrm{ch}/\mathrm{d}\eta}\xspace}
\newcommand{\dnchdydpt}   {\ensuremath{\mathrm{d}N_\mathrm{ch}/\mathrm{d}y\mathrm{d}p_{\mathrm{T}}}\xspace}
\newcommand{\dnchaadydpt}   {\ensuremath{\mathrm{d}N_\mathrm{ch}^{AA}/\mathrm{d}y\mathrm{d}p_{\mathrm{T}}}\xspace}
\newcommand{\dnchppdydpt}   {\ensuremath{\mathrm{d}N_\mathrm{ch}^{\mathrm{pp}}/\mathrm{d}y\mathrm{d}p_{\mathrm{T}}}\xspace}
\newcommand{\dnchdphi}{\ensuremath{\mathrm{d}N_\mathrm{ch}/\mathrm{d}\phi}\xspace}
\newcommand{\dnchddeltaphi}{\ensuremath{\mathrm{d}N_\mathrm{ch}/\mathrm{d}\Delta\upphi}\xspace}
\newcommand{\dndphi}{\ensuremath{\mathrm{d}N/\mathrm{d}\phi}\xspace}
\newcommand{\dnddeltaphi}{\ensuremath{\mathrm{d}N/\mathrm{d}\Delta\upphi}\xspace}
\newcommand{\avdndeta}     {\ensuremath{\langle\dndeta\rangle}\xspace}
\newcommand{\avdndetarap}  {$\langle$ dN$_{\textnormal{ch}}$/d$\eta$ $\rangle_{|\eta| < 0.5}$\xspace}
\newcommand{\dNdy}         {\ensuremath{\mathrm{d}N_\mathrm{ch}/\mathrm{d}y}\xspace}
\newcommand{\Npart}        {\ensuremath{N_\mathrm{part}}\xspace}
\newcommand{\meanNpart}    {$\langle$\ensuremath{N_\mathrm{part}}$\rangle$\xspace}
\newcommand{\ncoll}        {\ensuremath{N_\mathrm{coll}}\xspace}
\newcommand{\meanncoll}    {$\langle$\ensuremath{N_\mathrm{coll}}$\rangle$\xspace}
\newcommand{\averagencollhadronic}    {$\langle$\ensuremath{\mathrm{N}_\mathrm{coll}^{\mathrm{hadronic}}}$\rangle$\xspace}
\newcommand{\meantaa}      {$\langle$\ensuremath{T_\mathrm{AA}}$\rangle$\xspace}
\newcommand{\dEdx}         {\ensuremath{\textrm{d}E/\textrm{d}x}\xspace}
\newcommand{\RpPb}         {\ensuremath{R_{\rm pPb}}\xspace}
\newcommand{\raa}          {$R_{AA}$\xspace}
\newcommand{\vtwo}         {$v_{2}$\xspace}
\newcommand{\vtwoinitial}  {$v_{2}^{\mathrm{initial}}$\xspace}
\newcommand{\vtwofinal}    {$v_{2}^{\mathrm{final}}$\xspace}
\newcommand{\vtwofourfinal}{$v_{2}^{\mathrm{final}}\{4\}$\xspace}
\newcommand{\vtwofit}      {$v_{2}^{\mathrm{Fit}}$\xspace}
\newcommand{\vtwotwo}      {$v_{2}\{2\}$\xspace}
\newcommand{\vtwofour}     {$v_{2}\{4\}$\xspace}
\newcommand{\vtwopt}       {$v_{2}(p_{\textnormal{T}})$\xspace}
\newcommand{\vtwoptfit}    {$v_{2}^{\mathrm{Fit}}(p_{\textnormal{T}})$\xspace}
\newcommand{\nch}          {\ensuremath{N_\mathrm{ch}}\xspace}
\newcommand{\psireactionplane}          {$\Psi_{\textnormal{RP}}$\xspace}
\newcommand{\deltaphireactionplane}     {$\Delta\upphi = \phi - \Psi_{\textnormal{RP}}$\xspace}
\newcommand{\nevdnchddeltaphi}     {(1/N$_{\textnormal{ev}}$)dN$_{\textnormal{ch}}$/d$\Delta\upphi$\xspace}
\newcommand{\meannch}      {\ensuremath{\langle N_\mathrm{ch}\rangle}\xspace}
\newcommand{\etamodule}    {\ensuremath{|\eta|}\xspace}
\newcommand{\qbar}         {$\bar{\textnormal{q}}$\xspace}
\newcommand{\qqbar}        {$\textnormal{q}\bar{\textnormal{q}}$\xspace}
\newcommand{\qqbarzero}    {$\textnormal{q}_{0}\bar{\textnormal{q}}_{0}$\xspace}
\newcommand{\qqqbars}      {$\bar{\textnormal{q}}\bar{\textnormal{q}}\bar{\textnormal{q}}$\xspace}
\newcommand{\alphastrong}  {$\alpha_{\textnormal{s}}$\xspace}
\newcommand{\alphastrongdistance}  {$\alpha_{\textnormal{s}}$(R)\xspace}
\newcommand{\qtwo}         {Q$^2$\xspace}
\newcommand{\alphastrongqtwo}  {$\alpha_{\textnormal{s}}$(Q$^2$)\xspace}
\newcommand{\lambdaqcd}        {$\Lambda_{\textnormal{QCD}}$\xspace}
\newcommand{\sectionpp}        {$\sigma^{\textnormal{pp}}_{\textnormal{inel}}$\xspace}

% 3) ENERGIES, UNITS
\newcommand{\sqrts}        {$\sqrt{s}$\xspace}
\newcommand{\sqrtsnn}      {$\sqrt{s_{\mathrm{NN}}}$\xspace}
\newcommand{\nineH}        {$\sqrt{s}~=~0.9$~Te\kern-.1emV\xspace}
\newcommand{\seven}        {$\sqrt{s}~=~7$~Te\kern-.1emV\xspace}
\newcommand{\twoH}         {$\sqrt{s}~=~0.2$~Te\kern-.1emV\xspace}
\newcommand{\twosevensix}  {$\sqrt{s}~=~2.76$~Te\kern-.1emV\xspace}
\newcommand{\five}         {$\sqrt{s}~=~5.02$~Te\kern-.1emV\xspace}
\newcommand{\twohundrernn} {$\sqrt{s_{\mathrm{NN}}}=200$~Ge\kern-.1emV\xspace}
\newcommand{\twosevensixnn} {$\sqrt{s_{\mathrm{NN}}}=2.76$~Te\kern-.1emV\xspace}
\newcommand{\fivenn}       {$\sqrt{s_{\mathrm{NN}}}~=~5.02$~Te\kern-.1emV\xspace}
\newcommand{\fivefourfournn} {$\sqrt{s_{\mathrm{NN}}}=5.44$~Te\kern-.1emV\xspace}
\newcommand{\LT}           {L{\'e}vy-Tsallis\xspace}
\newcommand{\GeVc}         {Ge\kern-.1emV/$c$\xspace}
\newcommand{\MeVc}         {Me\kern-.1emV/$c$\xspace}
\newcommand{\TeV}          {Te\kern-.1emV\xspace}
\newcommand{\GeV}          {Ge\kern-.1emV\xspace}
\newcommand{\MeV}          {Me\kern-.1emV\xspace}
\newcommand{\GeVmass}      {Ge\kern-.2emV/$c^2$\xspace}
\newcommand{\MeVmass}      {Me\kern-.2emV/$c^2$\xspace}
\newcommand{\lumi}         {\ensuremath{\mathcal{L}}\xspace}
\newcommand{\fmc}         {fm\kern-.1em/$c$\xspace}

% 4) PARTICLE SPECIES 
\newcommand{\ee}           {\ensuremath{e^{+}e^{-}}} 
\newcommand{\pip}          {\ensuremath{\pi^{+}}\xspace}
\newcommand{\pim}          {\ensuremath{\pi^{-}}\xspace}
\newcommand{\kap}          {\ensuremath{\rm{K}^{+}}\xspace}
\newcommand{\kam}          {\ensuremath{\rm{K}^{-}}\xspace}
\newcommand{\pbar}         {\ensuremath{\rm\overline{p}}\xspace}
\newcommand{\kzero}        {\ensuremath{{\rm K}^{0}_{\rm{S}}}\xspace}
\newcommand{\lmb}          {\ensuremath{\Lambda}\xspace}
\newcommand{\almb}         {\ensuremath{\overline{\Lambda}}\xspace}
\newcommand{\Om}           {\ensuremath{\Omega^-}\xspace}
\newcommand{\Mo}           {\ensuremath{\overline{\Omega}^+}\xspace}
\newcommand{\X}            {\ensuremath{\Xi^-}\xspace}
\newcommand{\Ix}           {\ensuremath{\overline{\Xi}^+}\xspace}
\newcommand{\Xis}          {\ensuremath{\Xi^{\pm}}\xspace}
\newcommand{\Oms}          {\ensuremath{\Omega^{\pm}}\xspace}
\newcommand{\degree}       {\ensuremath{^{\rm o}}\xspace}
\newcommand{\comment}[1]{}

% two-particle angular correlation
\newcommand{\deltaphitriggassoc}    {$\Delta\upphi = |\phi_{\textnormal{trigger}} - \phi_{\textnormal{assoc}}|$\xspace}
\newcommand{\deltaetatriggassoc}    {$\Delta\upeta = |\eta_{\textnormal{trigger}} - \eta_{\textnormal{assoc}}|$\xspace}
\newcommand{\etatrigg}    {$\eta_{\textnormal{trigger}}$\xspace}
\newcommand{\etaassoc}    {$\eta_{\textnormal{assoc}}$\xspace}
\newcommand{\deltaphideltaeta}      {$\Delta\upphi-\Delta\upeta$\xspace}
\newcommand{\deltaphi}              {$\Delta\upphi$\xspace}
\newcommand{\moduledeltaphipitwo}   {$|\Delta\upphi| < \pi/2 $\xspace}
\newcommand{\deltaeta}              {$\Delta\upeta$\xspace}
\newcommand{\moduledeltaeta}        {$|\Delta\upeta|$\xspace}
\newcommand{\deltaphiapproxzero}    {$\Delta\upphi = 0$\xspace}
\newcommand{\deltaphiapproxpi}      {$\Delta\upphi = \pi$\xspace}
\newcommand{\deltaetaapproxzero}    {$\Delta\upeta = 0$\xspace}
\newcommand{\corrfunc}              {C($\Delta\upphi$, $\Delta\upeta$)\xspace}
\newcommand{\corrfunccorrect}              {C$_{\mathrm{correct}}(\Delta\upphi$, $\Delta\upeta$)\xspace}
\newcommand{\corrfuncmix}              {C$_{\mathrm{mix}}(\Delta\upphi$, $\Delta\upeta$)\xspace}
\newcommand{\corrfuncdeltaphi}      {C($\Delta\upphi$)\xspace}
\newcommand{\pttrigger}             {$p_{\textnormal{T}}^{\textnormal{trigger}}$\xspace}
\newcommand{\ptassoc}               {$p_{\textnormal{T}}^{\textnormal{assoc}}$\xspace}
\newcommand{\ratioyieldawaynearside}{Y$_{\textnormal{Away}}$/Y$_{\textnormal{Near}}$\xspace}

% 4) definition to references, biblatex and hyperlink
\usepackage[backend=bibtex, 
style=nature,  %style reference.
sorting=none,
firstinits=true %first name abbreviate
]{biblatex}

\usepackage{hyperref}
\hypersetup{
    colorlinks=true, %set "true" if you want colored links
    linktoc=all,     %set to "all" if you want both sections and subsections linked
    linkcolor=blue,  %choose some color if you want links to stand out
    citecolor= blue, % color of \cite{} in the text.
    urlcolor  = blue, % color of the link for the paper in references.
}

% 5) Tikz and figures
\usepackage{epsfig}
\usepackage{lmodern}
\usepackage{mathtools}
\usepackage[utf8]{luainputenc}
\usepackage{xspace}
\usepackage{tikz}
\usepackage{pgfplots}
\pgfplotsset{compat=newest}

\usetikzlibrary{positioning}
\usepackage{subcaption}

% 6) colors:
\usepackage{xcolor}
\definecolor{ao(english)}{rgb}{0.0, 0.5, 0.0} % dark green

% 7) Add lines numbers
%\usepackage{lineno}

% add pdf file to thesis:
\usepackage{pdfpages}

\hypersetup{
    colorlinks=true,% make the links colored
    linkcolor=blue
}

\usepackage{setspace}
\addbibresource{bibliography.bib}

\newcommand{\printingbibliography}{%

    \pagestyle{myheadings}
    \markright{}
    \sloppy
    \printbibliography[heading=bibintoc, % add to table of contents
                   title=Refer\^encias % Chapter name
                  ]
    \fussy%
}
\PassOptionsToPackage{table}{xcolor}
\renewcommand{\baselinestretch}{1.5}

\pagestyle{fancy}
\fancyhf{}
\renewcommand{\headrulewidth}{0pt}
\fancyhead[R]{\thepage}

\geometry{a4paper,top=30mm,bottom=20mm,left=30mm,right=20mm}

\titleformat*{\section}{\bfseries\large}
\titleformat*{\subsection}{\bfseries\normalsize}

\title{ \textbf{\large Notas: Mec\^anica Qu\^antica - MQ} }
\author{Andr\'e V. Silva}
\date{\today}

\begin{document}

\maketitle

MQ $\rightarrow$ Teoria fundamental para compreens\~ao dos fen\^omenos f\'isicos.

MQ descreve a intera\c{c}\~ao entre mat\'eria e radia\c{c}\~ao

S\'eculo XIX eram descritas por teorias distintas:

\begin{itemize}
    \item Corpos Materiais: Leis da Mec\^anica Newtoniana
    \item Radia\~c{c}\~ao: Teoria Eletromagn\'etica
    \item intera\c{c}\~ao radia\c{c}\~ao-materia: For\c{c}a de Lorentz.
\end{itemize}

Os fen\^omenos que deram origem a teoria da MQ foram:

\begin{itemize}
    \item Radia\c{c}\~ao de Corpo Negro
    \item Efeito Fotoel\'etrico
    \item Efeito Comptom
\end{itemize}

\newpage
\section*{2. Bound state of a particle in a ``delta function potential''}

Consider a particle whose Hamiltonian \( H \) [operator defined by formula (D-10) of chapter I] is:
\[
H = -\frac{\hbar^2}{2m} \frac{d^2}{dx^2} - \alpha \delta(x),
\]
where \( \alpha \) is a positive constant whose dimensions are to be found.

\begin{enumerate}
    \item[(a)] Integrate the eigenvalue equation of \( H \) between \( -\epsilon \) and \( +\epsilon \). Letting \( \epsilon \to 0 \), show that the derivative of the eigenfunction \( \varphi(x) \) presents a discontinuity at \( x = 0 \) and determine it in terms of \( \alpha \), \( m \), and \( \varphi(0) \).

    \textbf{Resolu\c{c}\~ao}:\\
    A equação de Schrödinger independente do tempo é:

    \[
    H \varphi(x) = E \varphi(x),
    \]
\[
-\frac{\hbar^2}{2m} \frac{d^2 \varphi(x)}{dx^2} - \alpha \delta(x) \varphi(x) = E \varphi(x),
\]
onde \( E \) é a energia da partícula. Vamos integrar essa equação em torno de \( x = 0 \).

\paragraph{Integração da equação:}
Integrando ambos os lados da equação de Schrödinger em \( x \in [-\epsilon, +\epsilon] \), temos:
\[
\int_{-\epsilon}^{+\epsilon} \left[ -\frac{\hbar^2}{2m} \frac{d^2 \varphi(x)}{dx^2} - \alpha \delta(x) \varphi(x) \right] dx = \int_{-\epsilon}^{+\epsilon} E \varphi(x) dx.
\]

1. O termo com \( \delta(x) \):
\[
\int_{-\epsilon}^{+\epsilon} \delta(x) \varphi(x) dx = \varphi(0),
\]
pois \( \delta(x) \) só contribui em \( x = 0 \).

2. O termo com \( \frac{d^2 \varphi}{dx^2} \):
\[
\int_{-\epsilon}^{+\epsilon} \frac{d^2 \varphi}{dx^2} dx = \frac{d\varphi}{dx} \bigg|_{x=0^+} - \frac{d\varphi}{dx} \bigg|_{x=0^-}.
\]

3. Como \( \epsilon \to 0 \), o termo \( \int_{-\epsilon}^{+\epsilon} E \varphi(x) dx \) tende a zero.

Assim, temos:
\[
\frac{\hbar^2}{2m} \left[ \frac{d\varphi}{dx} \bigg|_{x=0^+} - \frac{d\varphi}{dx} \bigg|_{x=0^-} \right] = -\alpha \varphi(0).
\]

\paragraph{Resultado:}
A descontinuidade da derivada é:
\[
\frac{d\varphi}{dx} \bigg|_{x=0^+} - \frac{d\varphi}{dx} \bigg|_{x=0^-} = -\frac{2m\alpha}{\hbar^2} \varphi(0).
\]
    
    \item[(b)] Assume that the energy \( E \) of the particle is negative (bound state). The eigenfunction \( \varphi(x) \) can then be written:
    \[
    \varphi(x) =
    \begin{cases}
        A_1 e^{\rho x} + A_1' e^{-\rho x}, & x < 0, \\
        A_2 e^{\rho x} + A_2' e^{-\rho x}, & x > 0.
    \end{cases}
    \]
    Express the constant \( \rho \) in terms of \( E \) and \( m \). Using the results of the preceding question, calculate the matrix \( M \) defined by:
    \[
    \begin{pmatrix}
    A_2 \\ A_2'
    \end{pmatrix}
    = M
    \begin{pmatrix}
    A_1 \\ A_1'
    \end{pmatrix}.
    \]
    Then, using the condition that \( \varphi(x) \) must be square-integrable, find the possible values of the energy. Calculate the corresponding normalized wave functions.

    \textbf{Resolu\c{c}\~ao}:\\

    A função de onda \( \varphi(x) \) para \( E < 0 \) (estado ligado) é escrita como:
\begin{equation}
\varphi(x) =
\begin{cases}
A_1 e^{\rho x} + A_1' e^{-\rho x}, & x < 0, \\
A_2 e^{\rho x} + A_2' e^{-\rho x}, & x > 0,
\end{cases}
\end{equation}

onde \( \rho > 0 \) é uma constante relacionada à energia \( E \).

\subsection*{Determinando \( \rho \)}
Para \( x \neq 0 \), o potencial \( -\alpha \delta(x) \) não contribui, e a equação de Schrödinger é:
\begin{equation}
-\frac{\hbar^2}{2m} \frac{d^2 \varphi(x)}{dx^2} = E \varphi(x).
\end{equation}
Rearranjando:
\begin{equation}
\frac{d^2 \varphi(x)}{dx^2} = -\frac{2m|E|}{\hbar^2} \varphi(x),
\end{equation}

onde \( E = -|E| \), pois o estado é ligado (\( E < 0 \)).

A solução geral dessa equação diferencial é exponencial:
\begin{equation}
\varphi(x) \propto e^{\pm \rho x}, \quad \text{com } \rho = \sqrt{\frac{2m|E|}{\hbar^2}}.
\end{equation}

\subsection*{Condição de continuidade em \( x = 0 \)}
A função de onda \( \varphi(x) \) deve ser contínua em \( x = 0 \). Assim:
\begin{equation}
\varphi(0^-) = \varphi(0^+).
\end{equation}
Substituindo:
\begin{equation}
A_1 + A_1' = A_2 + A_2'.
\end{equation}

\subsection*{Condição de descontinuidade da derivada}
A descontinuidade da derivada é dada por:
\begin{equation}
\frac{d\varphi}{dx} \Big|_{x=0^+} - \frac{d\varphi}{dx} \Big|_{x=0^-} = -\frac{2m\alpha}{\hbar^2} \varphi(0).
\end{equation}

Calculando as derivadas:
\begin{itemize}
    \item Para \( x > 0 \):
    \begin{equation}
    \frac{d\varphi}{dx} \Big|_{x=0^+} = \rho A_2 - \rho A_2'.
    \end{equation}
    \item Para \( x < 0 \):
    \begin{equation}
    \frac{d\varphi}{dx} \Big|_{x=0^-} = \rho A_1' - \rho A_1.
    \end{equation}
\end{itemize}

Substituindo na equação da descontinuidade:
\begin{equation}
\left( \rho A_2 - \rho A_2' \right) - \left( \rho A_1' - \rho A_1 \right) = -\frac{2m\alpha}{\hbar^2} \varphi(0).
\end{equation}

Reorganizando:
\begin{equation}
\rho (A_2 - A_2' - A_1' + A_1) = -\frac{2m\alpha}{\hbar^2} \varphi(0).
\end{equation}

Como \( \varphi(0) = A_1 + A_1' = A_2 + A_2' \), temos:
\begin{equation}
\rho (A_2 - A_2' - A_1' + A_1) = -\frac{2m\alpha}{\hbar^2} (A_1 + A_1').
\end{equation}

\subsection*{Construção da matriz \( M \)}
A matriz \( M \) é definida como:
\begin{equation}
\begin{pmatrix}
A_2 \\
A_2'
\end{pmatrix}
= M
\begin{pmatrix}
A_1 \\
A_1'
\end{pmatrix},
\end{equation}

\begin{alignat}{4}
    A_2 = M_{11} A_1 + M_{12} A_1' \\
    A_2' = M_{21} A_1 + M_{22} A_1'
 \end{alignat}

com \( M \) escrita explicitamente a partir das condições de continuidade e descontinuidade. Da continuidade:
\begin{equation}
A_1 + A_1' = A_2 + A_2', \quad \text{ou seja, } A_2 = A_1 + A_1' - A_2'.
\end{equation}

\begin{equation}
    A_1 + A_1' = A_2 + A_2', \quad \text{ou seja, } A_2' = A_1 - A_2 - A_1'.
\end{equation}

Da descontinuidade da derivada:
\begin{equation}
\rho (A_2 - A_2' - A_1' + A_1) = -\frac{2m\alpha}{\hbar^2} (A_1 + A_1').
\end{equation}

Substituindo \( A_2 \) em função de \( A_1, A_1' \) e \( A_2' \), temos:
\begin{equation}
\rho \big[(A_1 + A_1' - A_2') - A_2' - A_1' + A_1 \big] = -\frac{2m\alpha}{\hbar^2} (A_1 + A_1').
\end{equation}
Simplificando:

\begin{equation}
2\rho A_1 - 2\rho A_2' = -\frac{2m\alpha}{\hbar^2} (A_1 + A_1').
\end{equation}

\begin{equation}
- 2\rho A_2' = -\frac{2m\alpha}{\hbar^2} (A_1 + A_1') - 2\rho A_1
\end{equation}

\begin{equation}
A_2' = \frac{m\alpha}{\hbar^2 \rho} (A_1 + A_1') + A_1
\end{equation}

\begin{equation}
\boxed{A_2' =  \left(\frac{m\alpha}{\hbar^2 \rho} + 1\right)A_1 + \frac{m\alpha}{\hbar^2 \rho}A_1'}
\end{equation}

\begin{equation}
\boxed{A_2' = M_{21} A_1 + M_{22} A_1'}
\end{equation}

Da descontinuidade da derivada:
\begin{equation}
\rho (A_2 - A_2' - A_1' + A_1) = -\frac{2m\alpha}{\hbar^2} (A_1 + A_1').
\end{equation}

Substituindo \( A_2' \) em função de \( A_1, A_1' \) e \( A_2 \), temos:
\begin{equation}
\rho \big[(A_2 - A_2' - A_1' + A_1 \big] = -\frac{2m\alpha}{\hbar^2} (A_1 + A_1').
\end{equation}

\begin{equation}
\rho \big[(A_2 - A_1 + A_2 - A_1' - A_1' + A_1 \big] = -\frac{2m\alpha}{\hbar^2} (A_1 + A_1').
\end{equation}

\begin{equation}
    2\rho \big[(A_2 - A_1'\big] = -\frac{2m\alpha}{\hbar^2} (A_1 + A_1').
\end{equation}

\begin{equation}
\boxed{A_2 =  -\frac{m\alpha}{\hbar^2\rho}A_1 + \left(1 - \frac{m\alpha}{\hbar^2\rho}\right)A_1'}
\end{equation}

\begin{equation}
\boxed{A_2 = M_{11} A_1 + M_{12} A_1'}
\end{equation}

Simplificando:

Reorganizando os termos, obtemos \( A_2 \) e \( A_2' \) em função de \( A_1 \) e \( A_1' \), definindo assim os elementos de \( M \):
\begin{equation}
M = 
\begin{pmatrix}
    -\frac{m\alpha}{\hbar^2\rho} & \left(1 - \frac{m\alpha}{\hbar^2\rho}\right) \\
    \left(\frac{m\alpha}{\hbar^2 \rho} + 1\right) &  \frac{m\alpha}{\hbar^2 \rho}
\end{pmatrix}.
\end{equation}
    
    \item[(c)] Trace these wave functions graphically. Give an order of magnitude for their width \( \Delta x \).

    \textbf{Resolu\c{c}\~ao}:\\

    Trace as funções de onda \( \varphi(x) \) para os valores permitidos de \( E \). A largura \( \Delta x \) pode ser estimada como:
\[
\Delta x \sim \frac{1}{\rho} = \frac{\hbar}{\sqrt{2m|E|}}.
\]
    
    \item[(d)] What is the probability \( \mathcal{P}(p) \) that a measurement of the momentum of the particle in one of the normalized stationary states calculated above will give a result included between \( p \) and \( p + dp \)? For what value of \( p \) is this probability maximum? In what domain, of dimension \( \Delta p \), does it take on non-negligible values? Give an order of magnitude for the product \( \Delta x \cdot \Delta p \).

    \textbf{Resolu\c{c}\~ao}:\\

    A probabilidade \( \mathcal{P}(p) \) de encontrar o momento entre \( p \) e \( p + dp \) é dada pela transformada de Fourier da função de onda:
\[
\mathcal{P}(p) \propto \left| \int_{-\infty}^{+\infty} \varphi(x) e^{-ipx/\hbar} dx \right|^2.
\]

\paragraph{Passos:}
1. Calcule \( \mathcal{P}(p) \) explicitamente para \( \varphi(x) \).
2. Encontre o valor de \( p \) para o qual \( \mathcal{P}(p) \) é máxima.
3. Determine a largura \( \Delta p \) onde \( \mathcal{P}(p) \) é significativa.

\paragraph{Incerteza:}
A relação de incerteza deve ser verificada:
\[
\Delta x \cdot \Delta p \sim \hbar.
\]
    
\end{enumerate}







%%%%%%%% Bibliography 
% Os comandos para incluir as referências bibliográficas
%\printingbibliography

\end{document}
