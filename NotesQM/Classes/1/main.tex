\documentclass[a4paper,12pt]{article}
\usepackage[brazil]{babel}
\usepackage[utf8]{inputenc}
\usepackage[T1]{fontenc}
\usepackage{geometry}
\usepackage{setspace}
\usepackage{titlesec}
\usepackage{hyperref}
\usepackage{graphicx}
\usepackage{caption}
\usepackage{subcaption}
\usepackage{fancyhdr}
\usepackage{xcolor}

%%%%%%%%%%%%%%%%%%%%%%%%%%%%%%%%%%%%%%%%%%%%%%%%%%
% These are some new commands that may be useful 
% for paper writing in general. If other new commands
% are needed for your specific paper, please feel 
% free to add here. 
%
% The currently available commands are organized in: 
% 1) Systems
% 2) Quantities
% 3) Energies and units
% 4) particle species
% 5) Colors package
% 6) hyperlink
%%%%%%%%%%%%%%%%%%%%%%%%%%%%%%%%%%%%%%%%%%%%%%%%%%

\usepackage{amsmath}
\usepackage{amssymb}
\usepackage{upgreek}
\usepackage{multirow}
\usepackage{setspace}% http://ctan.org/pkg/setspace
\usepackage{fancyhdr}
\usepackage{datetime}

% 1) SYSTEMS
\newcommand{\btc}               {\textbf{BTC}}
\newcommand{\btcspace}          {\textbf{BTC} }
\newcommand{\pow}               {\textbf{PoW}}

% 4) definition to references, biblatex and hyperlink
\usepackage[backend=bibtex, 
style=nature,  %style reference.
sorting=none,
firstinits=true %first name abbreviate
]{biblatex}

\usepackage{hyperref}
\hypersetup{
    colorlinks=true, %set "true" if you want colored links
    linktoc=all,     %set to "all" if you want both sections and subsections linked
    linkcolor=blue,  %choose some color if you want links to stand out
    citecolor= blue, % color of \cite{} in the text.
    urlcolor  = blue, % color of the link for the paper in references.
}

% 5) Tikz and figures
\usepackage{epsfig}
\usepackage{lmodern}
\usepackage{mathtools}
\usepackage[utf8]{luainputenc}
\usepackage{xspace}
\usepackage{tikz}
\usepackage{pgfplots}
\pgfplotsset{compat=newest}

\usetikzlibrary{positioning}
\usepackage{subcaption}

% 6) colors:
\usepackage{xcolor}
\definecolor{ao(english)}{rgb}{0.0, 0.5, 0.0} % dark green

% 7) Add lines numbers
%\usepackage{lineno}

% add pdf file to thesis:
\usepackage{pdfpages}

\hypersetup{
    colorlinks=true,% make the links colored
    linkcolor=blue
}

\usepackage{setspace}
\addbibresource{bibliography.bib}

\newcommand{\printingbibliography}{%

    \pagestyle{myheadings}
    \markright{}
    \sloppy
    \printbibliography[heading=bibintoc, % add to table of contents
                   title=Refer\^encias % Chapter name
                  ]
    \fussy%
}
\PassOptionsToPackage{table}{xcolor}
\renewcommand{\baselinestretch}{1.5}

\pagestyle{fancy}
\fancyhf{}
\renewcommand{\headrulewidth}{0pt}
\fancyhead[R]{\thepage}

\geometry{a4paper,top=30mm,bottom=20mm,left=30mm,right=20mm}

\titleformat*{\section}{\bfseries\large}
\titleformat*{\subsection}{\bfseries\normalsize}

\title{ \textbf{\large Notas: Mec\^anica Qu\^antica - MQ} }
\author{Andr\'e V. Silva}
\date{\today}

\begin{document}

\maketitle

MQ $\rightarrow$ Teoria fundamental para compreens\~ao dos fen\^omenos f\'isicos.

MQ descreve a intera\c{c}\~ao entre mat\'eria e radia\c{c}\~ao

S\'eculo XIX eram descritas por teorias distintas:

\begin{itemize}
    \item Corpos Materiais: Leis da Mec\^anica Newtoniana
    \item Radia\~c{c}\~ao: Teoria Eletromagn\'etica
    \item intera\c{c}\~ao radia\c{c}\~ao-materia: For\c{c}a de Lorentz.
\end{itemize}

Os fen\^omenos que deram origem a teoria da MQ foram:

\begin{itemize}
    \item Radia\c{c}\~ao de Corpo Negro
    \item Efeito Fotoel\'etrico
    \item Efeito Comptom
\end{itemize}

\newpage
\section*{2. Bound state of a particle in a ``delta function potential''}

Consider a particle whose Hamiltonian \( H \) [operator defined by formula (D-10) of chapter I] is:
\[
H = -\frac{\hbar^2}{2m} \frac{d^2}{dx^2} - \alpha \delta(x),
\]
where \( \alpha \) is a positive constant whose dimensions are to be found.

\begin{enumerate}
    \item[(a)] Integrate the eigenvalue equation of \( H \) between \( -\epsilon \) and \( +\epsilon \). 
    Letting \( \epsilon \to 0 \), show that the derivative of the eigenfunction \( \varphi(x) \) 
    presents a discontinuity at \( x = 0 \) and determine it in terms of \( \alpha \), \( m \), and \( \varphi(0) \).

    \textbf{Resolu\c{c}\~ao}:\\
    A equação de Schrödinger independente do tempo é:

    \[
    H \varphi(x) = E \varphi(x),
    \]
\[
-\frac{\hbar^2}{2m} \frac{d^2 \varphi(x)}{dx^2} - \alpha \delta(x) \varphi(x) = E \varphi(x),
\]
onde \( E \) é a energia da partícula. Vamos integrar essa equação em torno de \( x = 0 \).

\paragraph{Integração da equação:}
Integrando ambos os lados da equação de Schrödinger em \( x \in [-\epsilon, +\epsilon] \), temos:
\[
\int_{-\epsilon}^{+\epsilon} \left[ -\frac{\hbar^2}{2m} \frac{d^2 \varphi(x)}{dx^2} - \alpha \delta(x) \varphi(x) \right] dx = \int_{-\epsilon}^{+\epsilon} E \varphi(x) dx.
\]

1. O termo com \( \delta(x) \):
\[
\int_{-\epsilon}^{+\epsilon} \delta(x) \varphi(x) dx = \varphi(0),
\]
pois \( \delta(x) \) só contribui em \( x = 0 \).

2. O termo com \( \frac{d^2 \varphi}{dx^2} \):
\[
\int_{-\epsilon}^{+\epsilon} \frac{d^2 \varphi}{dx^2} dx = \frac{d\varphi}{dx} \bigg|_{x=0^+} - \frac{d\varphi}{dx} \bigg|_{x=0^-}.
\]

3. Como \( \epsilon \to 0 \), o termo \( \int_{-\epsilon}^{+\epsilon} E \varphi(x) dx \) tende a zero.

Assim, temos:
\[
\frac{\hbar^2}{2m} \left[ \frac{d\varphi}{dx} \bigg|_{x=0^+} - \frac{d\varphi}{dx} \bigg|_{x=0^-} \right] = -\alpha \varphi(0).
\]

\paragraph{Resultado:}
A descontinuidade da derivada é:
\[
\frac{d\varphi}{dx} \bigg|_{x=0^+} - \frac{d\varphi}{dx} \bigg|_{x=0^-} = -\frac{2m\alpha}{\hbar^2} \varphi(0).
\]
    
    \item[(b)] Assume that the energy \( E \) of the particle is negative (bound state). The eigenfunction \( \varphi(x) \) can then be written:
    \[
    \varphi(x) =
    \begin{cases}
        A_1 e^{\rho x} + A_1' e^{-\rho x}, & x < 0, \\
        A_2 e^{\rho x} + A_2' e^{-\rho x}, & x > 0.
    \end{cases}
    \]
    Express the constant \( \rho \) in terms of \( E \) and \( m \). Using the results of the preceding question, calculate the matrix \( M \) defined by:
    \[
    \begin{pmatrix}
    A_2 \\ A_2'
    \end{pmatrix}
    = M
    \begin{pmatrix}
    A_1 \\ A_1'
    \end{pmatrix}.
    \]
    Then, using the condition that \( \varphi(x) \) must be square-integrable, find the possible values of the energy. Calculate the corresponding normalized wave functions.

    \textbf{Resolu\c{c}\~ao}:\\

    A função de onda \( \varphi(x) \) para \( E < 0 \) (estado ligado) é escrita como:
\begin{equation}
\varphi(x) =
\begin{cases}
A_1 e^{\rho x} + A_1' e^{-\rho x}, & x < 0, \\
A_2 e^{\rho x} + A_2' e^{-\rho x}, & x > 0,
\end{cases}
\end{equation}

onde \( \rho > 0 \) é uma constante relacionada à energia \( E \).

\subsection*{Determinando \( \rho \)}
Para \( x \neq 0 \), o potencial \( -\alpha \delta(x) \) não contribui, e a equação de Schrödinger é:
\begin{equation}
-\frac{\hbar^2}{2m} \frac{d^2 \varphi(x)}{dx^2} = E \varphi(x).
\end{equation}
Rearranjando:
\begin{equation}
\frac{d^2 \varphi(x)}{dx^2} = -\frac{2m|E|}{\hbar^2} \varphi(x),
\end{equation}

onde \( E = -|E| \), pois o estado é ligado (\( E < 0 \)).

A solução geral dessa equação diferencial é exponencial:
\begin{equation}
\varphi(x) \propto e^{\pm \rho x}, \quad \text{com } \rho = \sqrt{\frac{2m|E|}{\hbar^2}}.
\end{equation}

\subsection*{Condição de continuidade em \( x = 0 \)}
A função de onda \( \varphi(x) \) deve ser contínua em \( x = 0 \). Assim:
\begin{equation}
\varphi(0^-) = \varphi(0^+).
\end{equation}
Substituindo:
\begin{equation}
A_1 + A_1' = A_2 + A_2'.
\end{equation}

\subsection*{Condição de descontinuidade da derivada}
A descontinuidade da derivada é dada por:
\begin{equation}
\frac{d\varphi}{dx} \Big|_{x=0^+} - \frac{d\varphi}{dx} \Big|_{x=0^-} = -\frac{2m\alpha}{\hbar^2} \varphi(0).
\end{equation}

Calculando as derivadas:
\begin{itemize}
    \item Para \( x > 0 \):
    \begin{equation}
    \frac{d\varphi}{dx} \Big|_{x=0^+} = \rho A_2 - \rho A_2'.
    \end{equation}
    \item Para \( x < 0 \):
    \begin{equation}
    \frac{d\varphi}{dx} \Big|_{x=0^-} = \rho A_1' - \rho A_1.
    \end{equation}
\end{itemize}

Substituindo na equação da descontinuidade:
\begin{equation}
\left( \rho A_2 - \rho A_2' \right) - \left( \rho A_1' - \rho A_1 \right) = -\frac{2m\alpha}{\hbar^2} \varphi(0).
\end{equation}

Reorganizando:
\begin{equation}
\rho (A_2 - A_2' - A_1' + A_1) = -\frac{2m\alpha}{\hbar^2} \varphi(0).
\end{equation}

Como \( \varphi(0) = A_1 + A_1' = A_2 + A_2' \), temos:
\begin{equation}
\rho (A_2 - A_2' - A_1' + A_1) = -\frac{2m\alpha}{\hbar^2} (A_1 + A_1').
\end{equation}

\subsection*{Construção da matriz \( M \)}
A matriz \( M \) é definida como:
\begin{equation}
\begin{pmatrix}
A_2 \\
A_2'
\end{pmatrix}
= M
\begin{pmatrix}
A_1 \\
A_1'
\end{pmatrix},
\end{equation}

\begin{alignat}{4}
    A_2 = M_{11} A_1 + M_{12} A_1' \\
    A_2' = M_{21} A_1 + M_{22} A_1'
 \end{alignat}

com \( M \) escrita explicitamente a partir das condições de continuidade e descontinuidade. Da continuidade:
\begin{equation}
A_1 + A_1' = A_2 + A_2', \quad \text{ou seja, } A_2 = A_1 + A_1' - A_2'.
\end{equation}

\begin{equation}
    A_1 + A_1' = A_2 + A_2', \quad \text{ou seja, } A_2' = A_1 - A_2 - A_1'.
\end{equation}

Da descontinuidade da derivada:
\begin{equation}
\rho (A_2 - A_2' - A_1' + A_1) = -\frac{2m\alpha}{\hbar^2} (A_1 + A_1').
\end{equation}

Substituindo \( A_2 \) em função de \( A_1, A_1' \) e \( A_2' \), temos:
\begin{equation}
\rho \big[(A_1 + A_1' - A_2') - A_2' - A_1' + A_1 \big] = -\frac{2m\alpha}{\hbar^2} (A_1 + A_1').
\end{equation}
Simplificando:

\begin{equation}
2\rho A_1 - 2\rho A_2' = -\frac{2m\alpha}{\hbar^2} (A_1 + A_1').
\end{equation}

\begin{equation}
- 2\rho A_2' = -\frac{2m\alpha}{\hbar^2} (A_1 + A_1') - 2\rho A_1
\end{equation}

\begin{equation}
A_2' = \frac{m\alpha}{\hbar^2 \rho} (A_1 + A_1') + A_1
\end{equation}

\begin{equation}
\boxed{A_2' =  \left(\frac{m\alpha}{\hbar^2 \rho} + 1\right)A_1 + \frac{m\alpha}{\hbar^2 \rho}A_1'}
\end{equation}

\begin{equation}
\boxed{A_2' = M_{21} A_1 + M_{22} A_1'}
\end{equation}

Da descontinuidade da derivada:
\begin{equation}
\rho (A_2 - A_2' - A_1' + A_1) = -\frac{2m\alpha}{\hbar^2} (A_1 + A_1').
\end{equation}

Substituindo \( A_2' \) em função de \( A_1, A_1' \) e \( A_2 \), temos:
\begin{equation}
\rho \big[(A_2 - A_2' - A_1' + A_1 \big] = -\frac{2m\alpha}{\hbar^2} (A_1 + A_1').
\end{equation}

\begin{equation}
\rho \big[(A_2 - A_1 + A_2 - A_1' - A_1' + A_1 \big] = -\frac{2m\alpha}{\hbar^2} (A_1 + A_1').
\end{equation}

\begin{equation}
    2\rho \big[(A_2 - A_1'\big] = -\frac{2m\alpha}{\hbar^2} (A_1 + A_1').
\end{equation}

\begin{equation}
\boxed{A_2 =  -\frac{m\alpha}{\hbar^2\rho}A_1 + \left(1 - \frac{m\alpha}{\hbar^2\rho}\right)A_1'}
\end{equation}

\begin{equation}
\boxed{A_2 = M_{11} A_1 + M_{12} A_1'}
\end{equation}

Simplificando:

Reorganizando os termos, obtemos \( A_2 \) e \( A_2' \) em função de \( A_1 \) e \( A_1' \), definindo assim os elementos de \( M \):
\begin{equation}
M = 
\begin{pmatrix}
    -\frac{m\alpha}{\hbar^2\rho} & \left(1 - \frac{m\alpha}{\hbar^2\rho}\right) \\
    \left(\frac{m\alpha}{\hbar^2 \rho} + 1\right) &  \frac{m\alpha}{\hbar^2 \rho}
\end{pmatrix}.
\end{equation}
    
    \item[(c)] Trace these wave functions graphically. Give an order of magnitude for their width \( \Delta x \).

    \textbf{Resolu\c{c}\~ao}:\\

    Trace as funções de onda \( \varphi(x) \) para os valores permitidos de \( E \). A largura \( \Delta x \) pode ser estimada como:
\[
\Delta x \sim \frac{1}{\rho} = \frac{\hbar}{\sqrt{2m|E|}}.
\]
    
    \item[(d)] What is the probability \( \mathcal{P}(p) \) that a measurement of the momentum of the particle in one of the normalized stationary states calculated above will give a result included between \( p \) and \( p + dp \)? For what value of \( p \) is this probability maximum? In what domain, of dimension \( \Delta p \), does it take on non-negligible values? Give an order of magnitude for the product \( \Delta x \cdot \Delta p \).

    \textbf{Resolu\c{c}\~ao}:\\

    A probabilidade \( \mathcal{P}(p) \) de encontrar o momento entre \( p \) e \( p + dp \) é dada pela transformada de Fourier da função de onda:
\[
\mathcal{P}(p) \propto \left| \int_{-\infty}^{+\infty} \varphi(x) e^{-ipx/\hbar} dx \right|^2.
\]

\paragraph{Passos:}
1. Calcule \( \mathcal{P}(p) \) explicitamente para \( \varphi(x) \).
2. Encontre o valor de \( p \) para o qual \( \mathcal{P}(p) \) é máxima.
3. Determine a largura \( \Delta p \) onde \( \mathcal{P}(p) \) é significativa.

\paragraph{Incerteza:}
A relação de incerteza deve ser verificada:
\[
\Delta x \cdot \Delta p \sim \hbar.
\]
    
\end{enumerate}







%%%%%%%% Bibliography 
% Os comandos para incluir as referências bibliográficas
%\printingbibliography

\end{document}
