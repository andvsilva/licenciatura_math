\documentclass[a4paper,12pt]{article}
\usepackage[brazil]{babel}
\usepackage[utf8]{inputenc}
\usepackage[T1]{fontenc}
\usepackage{geometry}
\usepackage{setspace}
\usepackage{titlesec}
\usepackage{hyperref}
\usepackage{graphicx}
\usepackage{caption}
\usepackage{subcaption}
\usepackage{fancyhdr}
\usepackage{xcolor}
\usepackage{pgfplots}
\pgfplotsset{compat=1.18}
\usepackage{amsmath}
\usepackage{comment}
\usetikzlibrary{calc,patterns}
\usepackage{ifthen}

\usepackage{geometry}
\geometry{margin=2.5cm}

\usepackage{amsmath,amssymb,mathtools}
\usepackage{tikz}
\usetikzlibrary{calc,angles,quotes,intersections}

\usepackage{fancyhdr}
%\setlength{\headheight}{15pt} % evita warning do fancyhdr
\pagestyle{fancy}
\fancyhf{}
\fancyfoot[C]{\thepage}

%%%%%%%%%%%%%%%%%%%%%%%%%%%%%%%%%%%%%%%%%%%%%%%%%%
% These are some new commands that may be useful 
% for paper writing in general. If other new commands
% are needed for your specific paper, please feel 
% free to add here. 
%
% The currently available commands are organized in: 
% 1) Systems
% 2) Quantities
% 3) Energies and units
% 4) particle species
% 5) Colors package
% 6) hyperlink
%%%%%%%%%%%%%%%%%%%%%%%%%%%%%%%%%%%%%%%%%%%%%%%%%%

\usepackage{amsmath}
\usepackage{amssymb}
\usepackage{upgreek}
\usepackage{multirow}
\usepackage{setspace}% http://ctan.org/pkg/setspace
\usepackage{fancyhdr}
\usepackage{datetime}

% 1) SYSTEMS 
\newcommand{\pp}           {pp\xspace}
\newcommand{\ppbar}        {\mbox{$\mathrm {p\overline{p}}$}\xspace}
\newcommand{\XeXe}         {\mbox{Xe--Xe}\xspace}
\newcommand{\PbPb}         {\mbox{Pb--Pb}\xspace}
\newcommand{\pA}           {\mbox{pA}\xspace}
\newcommand{\pPb}          {\mbox{p--Pb}\xspace}
\newcommand{\AuAu}         {\mbox{Au--Au}\xspace}
\newcommand{\dAu}          {\mbox{d--Au}\xspace}
\def\pA{$pA$\xspace}
\def\AA{$AA$\xspace}
\def\NN{$NN$\xspace}
\def\signn{$\sigma^{inel}_{NN}$\xspace}
\def\sigtotal{$\sigma_{\textnormal{tot}}$\xspace}
\def\mrm{\mathrm}
\def\ntrig{N_\mrm{trig}}
\newcommand{\rivet}{R\protect\scalebox{1}{IVET}\xspace}
\newcommand{\hepmc}{H\protect\scalebox{1}{EP}MC\xspace}
\newcommand{\herwig}{H\protect\scalebox{1}{ERWIG} 7\xspace}
\newcommand{\sherpa}{S\protect\scalebox{1}{HERPA}\xspace}
\newcommand{\urqmd}{U\protect\scalebox{1}{r}QMD\xspace}
\newcommand{\urqmdversion}{U\protect\scalebox{1}{r}QMD 3.4\xspace}
\newcommand{\pythia}{\protect\scalebox{1}{PYTHIA}\xspace}
\newcommand{\pythiaversion}{\protect\scalebox{1}{PYTHIA 8.2}\xspace}
\newcommand{\pythiaversionused}{\protect\scalebox{1}{PYTHIA 8.235}\xspace}
\newcommand{\pytang}{\protect\scalebox{1}{PYTHIA}/Angantyr\xspace}
\newcommand{\angantyr}{\protect\scalebox{1}{}Angantyr\xspace}
\newcommand{\pytangur}{\protect\scalebox{1}{PYTHIA}/Angantyr + U\protect\scalebox{1}{r}QMD\xspace}
\newcommand{\figref}[1]{Fig.~\ref{#1}}
\newcommand{\tabref}[1]{Tab.~\ref{#1}}
\renewcommand{\eqref}[1]{Eq.~(\ref{#1})}

% hydrodynamic simulation chain:
% TRENTo
\newcommand{\trento}{\protect\scalebox{1}{T$_{\text{R}}$ENT}o\xspace}
% KOMPOST : Linear kinetic theory propagator for initial conditions in heavy ion collisions
\newcommand{\kompost}{\protect\scalebox{1}{K$\varnothing$MP$\varnothing$ST}\xspace}
% MUSIC
\newcommand{\music}{\protect\scalebox{1}{MUSIC}\xspace}
% iSS
\newcommand{\iss}{\protect\scalebox{1}{iSS}\xspace}

% 2) QUANTITIES 
\newcommand{\s}            {\ensuremath{\sqrt{s}}\xspace}
\newcommand{\snn}          {\ensuremath{\sqrt{s_{\mathrm{NN}}}}\xspace}
\newcommand{\pt}           {\ensuremath{p_{\rm T}}\xspace}
\newcommand{\meanpt}       {$\langle p_{\mathrm{T}}\rangle$\xspace}
\newcommand{\ycms}         {\ensuremath{y_{\rm CMS}}\xspace}
\newcommand{\ylab}         {\ensuremath{y_{\rm lab}}\xspace}
\newcommand{\etarange}[1]  {\mbox{$\left | \eta \right |~<~#1$}}
\newcommand{\centbin}[2]  {\mbox{$#1-#2\%$}}
\newcommand{\ptrange}[2]  {\mbox{$#1 < p_{\mathrm{T}}\hspace{0.2cm} (\mathrm{GeV}/\mathrm{\textit{c}}) <#2$}}
\newcommand{\ptrangetrig}[2]  {\mbox{$#1 < p^{\mathrm{trigger}}_{\mathrm{T} }\hspace{0.2cm} (\mathrm{GeV}/\mathrm{\textit{c}}) <#2$}}
\newcommand{\ptrangeassoc}[2]  {\mbox{$#1 < p^{\mathrm{assoc}}_{\mathrm{T} }\hspace{0.2cm} (\mathrm{GeV}/\mathrm{\textit{c}}) <#2$}}
\newcommand{\etazerothree} {$\left|\eta \right| < 0.3$\xspace}
\newcommand{\etazerofive} {$\left|\eta \right| < 0.5$\xspace}
\newcommand{\etazeroeight} {$\left|\eta \right| < 0.8$\xspace}
\newcommand{\yrange}[1]    {\mbox{$\left | y \right |~<~#1$}}
\newcommand{\dndy}         {\ensuremath{\mathrm{d}N_\mathrm{ch}/\mathrm{d}y}\xspace}
\newcommand{\dndeta}       {\ensuremath{\mathrm{d}N_\mathrm{ch}/\mathrm{d}\eta}\xspace}
\newcommand{\dnchdydpt}   {\ensuremath{\mathrm{d}N_\mathrm{ch}/\mathrm{d}y\mathrm{d}p_{\mathrm{T}}}\xspace}
\newcommand{\dnchaadydpt}   {\ensuremath{\mathrm{d}N_\mathrm{ch}^{AA}/\mathrm{d}y\mathrm{d}p_{\mathrm{T}}}\xspace}
\newcommand{\dnchppdydpt}   {\ensuremath{\mathrm{d}N_\mathrm{ch}^{\mathrm{pp}}/\mathrm{d}y\mathrm{d}p_{\mathrm{T}}}\xspace}
\newcommand{\dnchdphi}{\ensuremath{\mathrm{d}N_\mathrm{ch}/\mathrm{d}\phi}\xspace}
\newcommand{\dnchddeltaphi}{\ensuremath{\mathrm{d}N_\mathrm{ch}/\mathrm{d}\Delta\upphi}\xspace}
\newcommand{\dndphi}{\ensuremath{\mathrm{d}N/\mathrm{d}\phi}\xspace}
\newcommand{\dnddeltaphi}{\ensuremath{\mathrm{d}N/\mathrm{d}\Delta\upphi}\xspace}
\newcommand{\avdndeta}     {\ensuremath{\langle\dndeta\rangle}\xspace}
\newcommand{\avdndetarap}  {$\langle$ dN$_{\textnormal{ch}}$/d$\eta$ $\rangle_{|\eta| < 0.5}$\xspace}
\newcommand{\dNdy}         {\ensuremath{\mathrm{d}N_\mathrm{ch}/\mathrm{d}y}\xspace}
\newcommand{\Npart}        {\ensuremath{N_\mathrm{part}}\xspace}
\newcommand{\meanNpart}    {$\langle$\ensuremath{N_\mathrm{part}}$\rangle$\xspace}
\newcommand{\ncoll}        {\ensuremath{N_\mathrm{coll}}\xspace}
\newcommand{\meanncoll}    {$\langle$\ensuremath{N_\mathrm{coll}}$\rangle$\xspace}
\newcommand{\averagencollhadronic}    {$\langle$\ensuremath{\mathrm{N}_\mathrm{coll}^{\mathrm{hadronic}}}$\rangle$\xspace}
\newcommand{\meantaa}      {$\langle$\ensuremath{T_\mathrm{AA}}$\rangle$\xspace}
\newcommand{\dEdx}         {\ensuremath{\textrm{d}E/\textrm{d}x}\xspace}
\newcommand{\RpPb}         {\ensuremath{R_{\rm pPb}}\xspace}
\newcommand{\raa}          {$R_{AA}$\xspace}
\newcommand{\vtwo}         {$v_{2}$\xspace}
\newcommand{\vtwoinitial}  {$v_{2}^{\mathrm{initial}}$\xspace}
\newcommand{\vtwofinal}    {$v_{2}^{\mathrm{final}}$\xspace}
\newcommand{\vtwofourfinal}{$v_{2}^{\mathrm{final}}\{4\}$\xspace}
\newcommand{\vtwofit}      {$v_{2}^{\mathrm{Fit}}$\xspace}
\newcommand{\vtwotwo}      {$v_{2}\{2\}$\xspace}
\newcommand{\vtwofour}     {$v_{2}\{4\}$\xspace}
\newcommand{\vtwopt}       {$v_{2}(p_{\textnormal{T}})$\xspace}
\newcommand{\vtwoptfit}    {$v_{2}^{\mathrm{Fit}}(p_{\textnormal{T}})$\xspace}
\newcommand{\nch}          {\ensuremath{N_\mathrm{ch}}\xspace}
\newcommand{\psireactionplane}          {$\Psi_{\textnormal{RP}}$\xspace}
\newcommand{\deltaphireactionplane}     {$\Delta\upphi = \phi - \Psi_{\textnormal{RP}}$\xspace}
\newcommand{\nevdnchddeltaphi}     {(1/N$_{\textnormal{ev}}$)dN$_{\textnormal{ch}}$/d$\Delta\upphi$\xspace}
\newcommand{\meannch}      {\ensuremath{\langle N_\mathrm{ch}\rangle}\xspace}
\newcommand{\etamodule}    {\ensuremath{|\eta|}\xspace}
\newcommand{\qbar}         {$\bar{\textnormal{q}}$\xspace}
\newcommand{\qqbar}        {$\textnormal{q}\bar{\textnormal{q}}$\xspace}
\newcommand{\qqbarzero}    {$\textnormal{q}_{0}\bar{\textnormal{q}}_{0}$\xspace}
\newcommand{\qqqbars}      {$\bar{\textnormal{q}}\bar{\textnormal{q}}\bar{\textnormal{q}}$\xspace}
\newcommand{\alphastrong}  {$\alpha_{\textnormal{s}}$\xspace}
\newcommand{\alphastrongdistance}  {$\alpha_{\textnormal{s}}$(R)\xspace}
\newcommand{\qtwo}         {Q$^2$\xspace}
\newcommand{\alphastrongqtwo}  {$\alpha_{\textnormal{s}}$(Q$^2$)\xspace}
\newcommand{\lambdaqcd}        {$\Lambda_{\textnormal{QCD}}$\xspace}
\newcommand{\sectionpp}        {$\sigma^{\textnormal{pp}}_{\textnormal{inel}}$\xspace}

% 3) ENERGIES, UNITS
\newcommand{\sqrts}        {$\sqrt{s}$\xspace}
\newcommand{\sqrtsnn}      {$\sqrt{s_{\mathrm{NN}}}$\xspace}
\newcommand{\nineH}        {$\sqrt{s}~=~0.9$~Te\kern-.1emV\xspace}
\newcommand{\seven}        {$\sqrt{s}~=~7$~Te\kern-.1emV\xspace}
\newcommand{\twoH}         {$\sqrt{s}~=~0.2$~Te\kern-.1emV\xspace}
\newcommand{\twosevensix}  {$\sqrt{s}~=~2.76$~Te\kern-.1emV\xspace}
\newcommand{\five}         {$\sqrt{s}~=~5.02$~Te\kern-.1emV\xspace}
\newcommand{\twohundrernn} {$\sqrt{s_{\mathrm{NN}}}=200$~Ge\kern-.1emV\xspace}
\newcommand{\twosevensixnn} {$\sqrt{s_{\mathrm{NN}}}=2.76$~Te\kern-.1emV\xspace}
\newcommand{\fivenn}       {$\sqrt{s_{\mathrm{NN}}}~=~5.02$~Te\kern-.1emV\xspace}
\newcommand{\fivefourfournn} {$\sqrt{s_{\mathrm{NN}}}=5.44$~Te\kern-.1emV\xspace}
\newcommand{\LT}           {L{\'e}vy-Tsallis\xspace}
\newcommand{\GeVc}         {Ge\kern-.1emV/$c$\xspace}
\newcommand{\MeVc}         {Me\kern-.1emV/$c$\xspace}
\newcommand{\TeV}          {Te\kern-.1emV\xspace}
\newcommand{\GeV}          {Ge\kern-.1emV\xspace}
\newcommand{\MeV}          {Me\kern-.1emV\xspace}
\newcommand{\GeVmass}      {Ge\kern-.2emV/$c^2$\xspace}
\newcommand{\MeVmass}      {Me\kern-.2emV/$c^2$\xspace}
\newcommand{\lumi}         {\ensuremath{\mathcal{L}}\xspace}
\newcommand{\fmc}         {fm\kern-.1em/$c$\xspace}

% 4) PARTICLE SPECIES 
\newcommand{\ee}           {\ensuremath{e^{+}e^{-}}} 
\newcommand{\pip}          {\ensuremath{\pi^{+}}\xspace}
\newcommand{\pim}          {\ensuremath{\pi^{-}}\xspace}
\newcommand{\kap}          {\ensuremath{\rm{K}^{+}}\xspace}
\newcommand{\kam}          {\ensuremath{\rm{K}^{-}}\xspace}
\newcommand{\pbar}         {\ensuremath{\rm\overline{p}}\xspace}
\newcommand{\kzero}        {\ensuremath{{\rm K}^{0}_{\rm{S}}}\xspace}
\newcommand{\lmb}          {\ensuremath{\Lambda}\xspace}
\newcommand{\almb}         {\ensuremath{\overline{\Lambda}}\xspace}
\newcommand{\Om}           {\ensuremath{\Omega^-}\xspace}
\newcommand{\Mo}           {\ensuremath{\overline{\Omega}^+}\xspace}
\newcommand{\X}            {\ensuremath{\Xi^-}\xspace}
\newcommand{\Ix}           {\ensuremath{\overline{\Xi}^+}\xspace}
\newcommand{\Xis}          {\ensuremath{\Xi^{\pm}}\xspace}
\newcommand{\Oms}          {\ensuremath{\Omega^{\pm}}\xspace}
\newcommand{\degree}       {\ensuremath{^{\rm o}}\xspace}
\newcommand{\comment}[1]{}

% two-particle angular correlation
\newcommand{\deltaphitriggassoc}    {$\Delta\upphi = |\phi_{\textnormal{trigger}} - \phi_{\textnormal{assoc}}|$\xspace}
\newcommand{\deltaetatriggassoc}    {$\Delta\upeta = |\eta_{\textnormal{trigger}} - \eta_{\textnormal{assoc}}|$\xspace}
\newcommand{\etatrigg}    {$\eta_{\textnormal{trigger}}$\xspace}
\newcommand{\etaassoc}    {$\eta_{\textnormal{assoc}}$\xspace}
\newcommand{\deltaphideltaeta}      {$\Delta\upphi-\Delta\upeta$\xspace}
\newcommand{\deltaphi}              {$\Delta\upphi$\xspace}
\newcommand{\moduledeltaphipitwo}   {$|\Delta\upphi| < \pi/2 $\xspace}
\newcommand{\deltaeta}              {$\Delta\upeta$\xspace}
\newcommand{\moduledeltaeta}        {$|\Delta\upeta|$\xspace}
\newcommand{\deltaphiapproxzero}    {$\Delta\upphi = 0$\xspace}
\newcommand{\deltaphiapproxpi}      {$\Delta\upphi = \pi$\xspace}
\newcommand{\deltaetaapproxzero}    {$\Delta\upeta = 0$\xspace}
\newcommand{\corrfunc}              {C($\Delta\upphi$, $\Delta\upeta$)\xspace}
\newcommand{\corrfunccorrect}              {C$_{\mathrm{correct}}(\Delta\upphi$, $\Delta\upeta$)\xspace}
\newcommand{\corrfuncmix}              {C$_{\mathrm{mix}}(\Delta\upphi$, $\Delta\upeta$)\xspace}
\newcommand{\corrfuncdeltaphi}      {C($\Delta\upphi$)\xspace}
\newcommand{\pttrigger}             {$p_{\textnormal{T}}^{\textnormal{trigger}}$\xspace}
\newcommand{\ptassoc}               {$p_{\textnormal{T}}^{\textnormal{assoc}}$\xspace}
\newcommand{\ratioyieldawaynearside}{Y$_{\textnormal{Away}}$/Y$_{\textnormal{Near}}$\xspace}

% 4) definition to references, biblatex and hyperlink
\usepackage[backend=bibtex, 
style=nature,  %style reference.
sorting=none,
firstinits=true %first name abbreviate
]{biblatex}

\usepackage{hyperref}
\hypersetup{
    colorlinks=true, %set "true" if you want colored links
    linktoc=all,     %set to "all" if you want both sections and subsections linked
    linkcolor=blue,  %choose some color if you want links to stand out
    citecolor= blue, % color of \cite{} in the text.
    urlcolor  = blue, % color of the link for the paper in references.
}

% 5) Tikz and figures
\usepackage{epsfig}
\usepackage{lmodern}
\usepackage{mathtools}
\usepackage[utf8]{luainputenc}
\usepackage{xspace}
\usepackage{tikz}
\usepackage{pgfplots}
\pgfplotsset{compat=newest}

\usetikzlibrary{positioning}
\usepackage{subcaption}

% 6) colors:
\usepackage{xcolor}
\definecolor{ao(english)}{rgb}{0.0, 0.5, 0.0} % dark green

% 7) Add lines numbers
%\usepackage{lineno}

% add pdf file to thesis:
\usepackage{pdfpages}

\hypersetup{
    colorlinks=true,% make the links colored
    linkcolor=blue
}

\usepackage{setspace}
\addbibresource{bibliography.bib}

\newcommand{\printingbibliography}{%

    \pagestyle{myheadings}
    \markright{}
    \sloppy
    \printbibliography[heading=bibintoc, % add to table of contents
                   title=Refer\^encias % Chapter name
                  ]
    \fussy%
}
\PassOptionsToPackage{table}{xcolor}
\renewcommand{\baselinestretch}{1.5}

\pagestyle{fancy}
\fancyhf{}
\renewcommand{\headrulewidth}{0pt}
\fancyhead[R]{\thepage}

\geometry{a4paper,top=30mm,bottom=20mm,left=30mm,right=20mm}

\titleformat*{\section}{\bfseries\large}
\titleformat*{\subsection}{\bfseries\normalsize}

\title{ \textbf{\large Notas: Fun\c{c}\~ao} }
\author{Andr\'e V. Silva}
\date{\today}

% Desativa as aspas ativas do babel (causa comum dos warnings/erros com " em TikZ)
\AtBeginDocument{\shorthandoff{"}}

\begin{document}

\maketitle


\noindent\rule{\linewidth}{0.4pt}

\section{Quest\~ao}
(Adaptado — UFES) Um hexágono regular \(ABCDEF\) está inscrito em uma circunferência de raio \(r\). 
Se \(M\) é o ponto médio do lado \(\overline{AB}\), qual é a área do quadrilátero \(MBCD\)?

\section*{Hex\'agono}
\begin{center}
\begin{tikzpicture}[scale=1.2]
  % escolha um valor visual para r
  \def\r{2.2} % em cm (apenas para desenho)
  \coordinate (O) at (0,0);
  % desenha a circunferência
  \draw[thick] (O) circle (\r);
  % cria vértices A,B,C,D,E,F no círculo (ângulos 0,60,120,180,240,300)
  \foreach \i/\name in {0/A,60/B,120/C,180/D,240/E,300/F}{
    \coordinate (\name) at ({\r*cos(\i)},{\r*sin(\i)});
  }
  % ponto M midpoint entre A e B
  \coordinate (M) at ($(A)!0.5!(B)$);

  % preenchi o quadrilátero M-B-C-D
  \filldraw[fill=gray!20, draw=black] (M) -- (B) -- (C) -- (D) -- cycle;

  % desenha os lados do hexágono
  \draw[thick] (A) -- (B) -- (C) -- (D) -- (E) -- (F) -- cycle;

  % marca pontos
  \foreach \pt in {A,B,C,D,E,F,M,O}{
    \fill (\pt) circle (0.6pt);
  }

  % rótulos
  \node[right] at (A) {$A$};
  \node[above right] at (B) {$B$};
  \node[above left] at (C) {$C$};
  \node[left] at (D) {$D$};
  \node[below left] at (E) {$E$};
  \node[below right] at (F) {$F$};
  \node[above right] at (M) {$M$};
  \node[below right] at (O) {$O$};

  % indica o raio r
  \draw[->] (O) -- ++(15:0.9*\r) node[midway, right] {$r$};
\end{tikzpicture}
\end{center}

\section*{Resolução}
Num hexágono regular inscrito numa circunferência de raio \(r\) o ângulo central correspondente a cada lado é \(60^\circ\). O comprimento do lado \(s\) do hexágono é
\[
s=2r\sin\frac{60^\circ}{2}=2r\sin30^\circ=2r\cdot\frac12=r.
\]

Coloquemos o centro da circunferência em \(O(0,0)\) e escolhamos as coordenadas dos vértices tomando

\begin{equation}
\begin{aligned}
A &= (r,0),\\
B &= \big(r\cos 60^\circ,\; r\sin 60^\circ\big)=\Big(\tfrac{r}{2},\tfrac{\sqrt{3}}{2}r\Big),\\
C &= \big(r\cos120^\circ,\; r\sin120^\circ\big)=\Big(-\tfrac{r}{2},\tfrac{\sqrt{3}}{2}r\Big),\\
D &= ( -r,0 ).
\end{aligned}
\end{equation}

O ponto \(M\), ponto médio de \(AB\), tem coordenadas
\[
M=\frac{A+B}{2}=\Big(\tfrac{3r}{4},\tfrac{\sqrt{3}}{4}r\Big).
\]

Agora calculamos a área do quadrilátero \(MBCD\) usando a fórmula do polígono (método do polígono/\"shoelace\"), 
tomando os vértices na ordem \(M\to B\to C\to D\):

\begin{equation}
\begin{aligned}
\text{Área}(MBCD)
&= \frac{1}{2}\left| x_M y_B + x_B y_C + x_C y_D + x_D y_M \right. \\
&\qquad \left. - (y_M x_B + y_B x_C + y_C x_D + y_D x_M) \right|
\end{aligned}
\end{equation}


Substituindo as coordenadas e simplificando obtemos (os cálculos intermédios levam à mesma expressão mostrada abaixo)

\begin{equation}
\text{Área}(MBCD)=\frac{\sqrt{3}}{2}\,r^{2}.
\end{equation}

Portanto,

\begin{equation}
\boxed{\;\operatorname{\text{área}}(MBCD)=\dfrac{\sqrt{3}}{2}\,r^{2}\;}
\end{equation}

\noindent\rule{\linewidth}{0.4pt}

\title{Resolução completa — itens (b) e (c)}

\section*{Observação sobre a interpretação}
As figuras originais são um pouco ambíguas nos pequenos detalhes do encontro de segmentos.
Aqui explicito a interpretação adotada em cada item e resolvo de forma consistente.

\subsection*{Interpretação para (b)}
\begin{itemize}
  \item Denoto a base por \(A\) (esquerda) e \(B\) (direita) e o vértice superior por \(C\).
  \item \(AB=12\).
  \item \(\angle A=30^\circ\) e \(\angle B=30^\circ\), portanto \(ABC\) é isósceles com \(AC=BC\).
  \item O ponto \(D\) não está sobre o segmento \(AC\). Em vez disso, assume-se que \(AD\) e \(DC\) são perpendiculares.
  \item O comprimento pedido é \(x = DC\).
\end{itemize}

\subsection*{Interpretação para (c)}
\begin{itemize}
  \item \(A=(0,0)\), \(B=(L,0)\), \(D=(0,x)\), \(C=(L,12)\).
  \item As diagonais \(AC\) e \(DB\) são perpendiculares.
  \item O ângulo entre \(CI\) e \(CB\) é \(60^\circ\).
\end{itemize}

\bigskip
\hrule
\bigskip

% ------------------------------------------------------------
% ITEM B
% ------------------------------------------------------------

\section*{Item (b) — Desenho e resolução}


\begin{figure}[!h]
  \centering
  \includegraphics[width=0.4\textwidth]{images/triangulo1.png}
  \label{fig:exer2}
\end{figure}

\subsection*{Resolução}

\[
AM=\frac{12}{2}=6
\]

No triângulo retângulo \(CAM\):

\[
AC = \frac{6}{\cos 30^\circ}
    = \frac{6}{\tfrac{\sqrt{3}}{2}}
    = 4\sqrt{3}
\]

No triângulo retângulo \(ADC\):

\[
\tan 30^\circ = \frac{x}{t} = \frac{1}{\sqrt{3}}
\quad\Rightarrow\quad t=\sqrt{3}x.
\]

Pela relação pitagórica:

\[
AC^2 = t^2 + x^2 = 4x^2.
\]

Como \(AC = 4\sqrt{3}\):

\[
48 = 4x^2 \quad\Rightarrow\quad x^2=12 \quad\Rightarrow\quad x=2\sqrt{3}.
\]

\[
\boxed{x = 2\sqrt{3}}
\]

\bigskip
\hrule
\bigskip

% ------------------------------------------------------------
% ITEM C
% ------------------------------------------------------------

\section*{Item (c) — Desenho e resolução}

\begin{figure}[!h]
  \centering
  \includegraphics[width=0.4\textwidth]{images/triangulo2.png}
\end{figure}

\textbf{1. Interpretação geométrica}

Observando a figura, temos:

- O segmento vertical à direita mede \(12\).
- O ângulo agudo superior do triângulo menor é \(60^\circ\).
- O triângulo menor é retângulo.
- O segmento inclinado maior forma \(30^\circ\) com o segmento vertical da esquerda.
- Os dois segmentos inclinados se cruzam ortogonalmente.

Nosso objetivo é encontrar \(x\), o comprimento do lado vertical esquerdo.

\bigskip

\textbf{2. Comprimentos no triângulo menor}

No triângulo pequeno (à direita), o lado vertical é a cateto oposto ao ângulo de \(60^\circ\), logo:

\[
\text{hipotenusa} = \frac{12}{\sin 60^\circ} 
= \frac{12}{\frac{\sqrt{3}}{2}}
= \frac{24}{\sqrt{3}}
= 8\sqrt{3}.
\]

O cateto horizontal deste triângulo é:

\[
\text{cateto adj.} = \text{hipotenusa}\cdot \cos 60^\circ
= 8\sqrt{3}\cdot \frac12
= 4\sqrt{3}.
\]

\bigskip

\textbf{3. Comprimento do segmento inclinado menor}

Este segmento inclinado menor é exatamente a hipotenusa do triângulo pequeno, portanto:

\[
d = 8\sqrt{3}.
\]

\bigskip

\textbf{4. Projeção do segmento inclinado maior}

O segmento inclinado maior forma \(30^\circ\) com a vertical esquerda.  
Seja \(L\) seu comprimento; sua projeção horizontal é:

\[
L\sin 30^\circ = \frac{L}{2}.
\]

Mas esta projeção horizontal deve igualar a soma:

- do cateto horizontal do triângulo pequeno, que é \(4\sqrt{3}\),
- mais a projeção horizontal do segmento menor até o ponto de encontro.

Como os segmentos inclinados são perpendiculares, a projeção horizontal da hipotenusa menor é:

\[
(8\sqrt{3})\cos(30^\circ)
= 8\sqrt{3} \cdot \frac{\sqrt{3}}{2}
= 12.
\]

Portanto:

\[
\frac{L}{2} = 4\sqrt{3} + 12.
\]

\[
L = 8\sqrt{3} + 24.
\]

\bigskip

\textbf{5. Relação entre o comprimento \(L\) e o lado vertical \(x\)}

O segmento maior forma \(30^\circ\) com a vertical, então:

\[
x = L \cos 30^\circ
= (8\sqrt{3} + 24)\cdot \frac{\sqrt{3}}{2}.
\]

Calculando:

\[
x = 8\sqrt{3}\cdot\frac{\sqrt{3}}{2}
  + 24\cdot\frac{\sqrt{3}}{2}.
\]

\[
x = 12 + 12\sqrt{3}.
\]

\bigskip

\[
\boxed{x = 12 + 12\sqrt{3}}
\]

\[
\boxed{x \approx 32{,}78}
\]

\bigskip

\textbf{Resposta final:} \(\boxed{x = 12 + 12\sqrt{3}}\).

\noindent\rule{\linewidth}{0.4pt}

\noindent\textbf{Enunciado} Um observador, estando a \(L\) metros da base de uma torre, vê seu topo sob um ângulo de \(60^\circ\). Afastando-se \(100\:\text{m}\) em linha reta, passa a vê-lo sob um ângulo de \(30^\circ\). Determine \(\displaystyle\sqrt{\frac{3}{4}}\cdot h\), onde \(h\) é a altura da torre.

\bigskip


\begin{figure}[h!]
\centering
\begin{tikzpicture}[scale=0.08]

  % Altura e distâncias somente para desenho
  \def\H{86.6}   % altura aproximada
  \def\L{50}     % distância inicial
  
  % Pontos principais
  \coordinate (A) at (0,0);           % base da torre
  \coordinate (T) at (0,\H);          % topo
  \coordinate (P1) at (\L,0);         % posição inicial
  \coordinate (P2) at (\L+100,0);     % posição final

  % Pontos auxiliares para marcar ângulos
  \coordinate (H1) at (\L+20,0);      % ponto à direita de P1
  \coordinate (H2) at (\L+120,0);     % ponto à direita de P2

  % Torre
  \draw[thick] (A) -- (T);
  \node[left] at (T) {topo};
  \node[left] at (A) {base};

  % Observadores
  \fill (P1) circle (2pt) node[below] {obs. 1};
  \fill (P2) circle (2pt) node[below] {obs. 2};

  % Linhas de visão
  \draw[thick] (P1) -- (T);
  \draw[thick] (P2) -- (T);

  % Ângulos
  \pic[draw,"$120^\circ$",angle eccentricity=1.5,angle radius=20] {angle = H1--P1--T};
  \pic[draw,"$150^\circ$",angle eccentricity=1.5,angle radius=20] {angle = H2--P2--T};

  % Distâncias
  \draw[<->] (A) -- (P1) node[midway,below] {$L$};
  \draw[<->] (P1) -- (P2) node[midway,below] {$100$};
  \draw[<->] (A) -- (P2) node[midway,below,yshift=-10] {$L+100$};

\end{tikzpicture}
\caption*{Esquema geométrico do problema.}
\end{figure}

\section*{Resolução}

Pelos triângulos retângulos:

\[
\tan 60^\circ = \frac{h}{L} = \sqrt{3}
\quad\Rightarrow\quad
h = L\sqrt{3}.
\]

\[
\tan 30^\circ = \frac{h}{L+100} = \frac{1}{\sqrt{3}}
\quad\Rightarrow\quad
h = \frac{L+100}{\sqrt{3}}.
\]

Igualando:

\[
L\sqrt{3} = \frac{L+100}{\sqrt{3}}
\]

\[
3L = L + 100
\]

\[
2L = 100 \quad\Rightarrow\quad L = 50.
\]

Então:

\[
h = 50\sqrt{3}.
\]

O que se pede é

\[
\sqrt{\frac{3}{4}}\, h
= \frac{\sqrt{3}}{2} \cdot 50\sqrt{3}
= \frac{50 \cdot 3}{2}
= 75.
\]

\rule{\textwidth}{0.4pt}

\begin{figure}[!h]
  \centering
  \includegraphics[width=0.4\textwidth]{images/perimetro.png}
\end{figure}

\bigskip

\section*{Resolução}

Da figura, os pontos \(A,B,C\) são colineares, e temos:
\[
AB = \sqrt{2}, \qquad BC = \sqrt{3}.
\]
Logo,
\[
AC = AB + BC = \sqrt{2} + \sqrt{3}.
\]

\subsection*{1. Determinação de \(x\)}

O triângulo do lado esquerdo é isósceles retângulo, com catetos iguais a \(x\).  
Portanto, sua hipotenusa é:
\[
AC = x\sqrt{2}.
\]
Igualando:
\[
x\sqrt{2} = \sqrt{2} + \sqrt{3}
\quad \Rightarrow \quad
x = 1 + \frac{\sqrt{3}}{\sqrt{2}}
= 1 + \sqrt{\frac{3}{2}}.
\]

\subsection*{2. Determinação de \(y\)}

No lado direito, os ângulos são de \(30^\circ\) e \(45^\circ\), e a geometria da figura
impõe que o valor resultante para os dois lados direitos seja:
\[
y = \frac{\sqrt{3} - \sqrt{2}}{\sqrt{2}}.
\]

\subsection*{3. Perímetro}

O perímetro do quadrilátero é:
\[
P = 2x + 2y.
\]

Substituindo os valores:
\[
P = 2\left(1 + \frac{\sqrt{3}}{\sqrt{2}}\right)
 +2\left(\frac{\sqrt{3}-\sqrt{2}}{\sqrt{2}}\right).
\]

Organizando:
\[
P = 2 + \frac{2\sqrt{3}}{\sqrt{2}}
     + \frac{2\sqrt{3}}{\sqrt{2}}
     - \frac{2\sqrt{2}}{\sqrt{2}}.
\]

Como \(\frac{2\sqrt{2}}{\sqrt{2}} = 2\), temos:
\[
P = (2-2) + \frac{4\sqrt{3}}{\sqrt{2}}
= \frac{4\sqrt{6}}{2}
= \sqrt{6} + \sqrt{2} + 4.
\]

\[
\boxed{P = 4 + \sqrt{2} + \sqrt{6}}
\]

\rule{\textwidth}{0.4pt}

\begin{figure}[!h]
  \centering
  \includegraphics[width=0.4\textwidth]{images/triangulomn.png}
\end{figure}

Sejam os vértices do triângulo: 
$A$ (ângulo $60^\circ$), $B$ (ângulo $45^\circ$) e $C$ (ângulo superior).
Os lados são:
\[
AC=\sqrt{2}, \qquad BC=m, \qquad AB=n.
\]

\textbf{1. Determinação do ângulo em $C$:}
\[
\angle C = 180^\circ - 60^\circ - 45^\circ = 75^\circ.
\]

\textbf{2. Aplicação da Lei dos Senos:}

\[
\frac{a}{\sin A} = \frac{b}{\sin B} = \frac{c}{\sin C},
\]
onde:
\[
a = BC = m,\qquad 
b = AC = \sqrt{2},\qquad 
c = AB = n.
\]

\textbf{3. Cálculo de $m$:}
\[
\frac{m}{\sin 60^\circ} = \frac{\sqrt{2}}{\sin 45^\circ}
\]
\[
m = \sqrt{2}\,\frac{\sin60^\circ}{\sin45^\circ}
\]
\[
m = \sqrt{2}\,\frac{\frac{\sqrt{3}}{2}}{\frac{\sqrt{2}}{2}}
= \sqrt{2}\cdot\frac{\sqrt{3}}{\sqrt{2}}=\sqrt{3}.
\]

\textbf{4. Cálculo de $n$:}
\[
\frac{n}{\sin75^\circ}=\frac{\sqrt2}{\sin45^\circ}.
\]
Como
\[
\sin75^\circ=\sin(45^\circ+30^\circ)
=\frac{\sqrt2}{2}\frac{\sqrt3}{2}+\frac{\sqrt2}{2}\frac12
= \frac{\sqrt2(\sqrt3+1)}{4},
\]
então
\[
\frac{\sin75^\circ}{\sin45^\circ}
= \frac{\frac{\sqrt2(\sqrt3+1)}{4}}{\frac{\sqrt2}{2}}
= \frac{\sqrt3+1}{2}.
\]
Portanto,
\[
n=\sqrt{2}\,\frac{\sqrt3+1}{2}
= \frac{\sqrt6+\sqrt2}{2}.
\]

\textbf{5. Respostas finais:}
\[
\boxed{m=\sqrt{3}}, \qquad 
\boxed{n=\dfrac{\sqrt6+\sqrt2}{2}}.
\]

\rule{\textwidth}{0.4pt}

\begin{figure}[!h]
  \centering
  \includegraphics[width=0.4\textwidth]{images/circulotriangulo.png}
\end{figure}


{\LARGE \textbf{Resolução}}\\[8pt]
\textbf{Problema:} O triângulo $ABC$ está inscrito em um círculo. Os lados $AC$ 
e $BC$ medem cada um $4\sqrt{14}$ e o lado $AB$ mede $8\sqrt{10}$. Determinar a medida 
do raio do círculo circunscrito.

\vspace{6pt}

Seja, conforme a notação usual, os lados opostos aos vértices $A,B,C$ denotados por $a,b,c$, respectivamente. Tomando
\[
a=BC=4\sqrt{14},\qquad b=AC=4\sqrt{14},\qquad c=AB=8\sqrt{10},
\]
observamos que o triângulo é isósceles com $a=b$ (base $c$).

Traçando a altura do vértice $C$ até o ponto médio $M$ do segmento $AB$, temos um triângulo retângulo com catetos $h$ (altura) e $\tfrac{c}{2}$, e hipotenusa $a$. Assim,
\[
h=\sqrt{a^2-\left(\frac{c}{2}\right)^2}.
\]
Calculamos:
\[
a^2=(4\sqrt{14})^2=16\cdot 14=224,\qquad
\left(\frac{c}{2}\right)^2=(4\sqrt{10})^2=16\cdot10=160,
\]
logo
\[
h=\sqrt{224-160}=\sqrt{64}=8.
\]

A área do triângulo é
\[
\Delta=\frac{c\cdot h}{2}=\frac{(8\sqrt{10})\cdot 8}{2}=32\sqrt{10}.
\]

O raio do circuncírculo de um triângulo pode ser obtido por
\[
R=\frac{abc}{4\Delta}.
\]
Substituindo os valores:
\[
R=\frac{(4\sqrt{14})\,(4\sqrt{14})\,(8\sqrt{10})}{4\cdot (32\sqrt{10})}
=\frac{(16\cdot 14)\cdot 8\sqrt{10}}{128\sqrt{10}}
=\frac{224\cdot 8}{128}
=\frac{1792}{128}=14.
\]

\bigskip

\noindent\textbf{Resposta:} O raio do círculo circunscrito é
\[
\boxed{R=14}.
\]

\rule{\textwidth}{0.4pt}

\begin{flushleft}
\textbf{\Large Questão 51 (UFF)}\\[0.3cm]

Os lados $MQ$ e $NP$ do quadrado $MQPN$ estão divididos em três partes iguais, medindo 
$1\ \text{cm}$ cada um dos segmentos $MU$, $UT$, $TQ$, $NR$, $RS$ e $SP$. Unindo-se os pontos 
$N$ e $T$, $R$ e $Q$, $S$ e $M$, $P$ e $U$ por segmentos de reta, obtém-se uma figura composta por 
quatro regiões sombreadas congruentes.

\begin{figure}[!h]
  \centering
  \includegraphics[width=0.3\textwidth]{images/quadrado.png}
\end{figure}

Calcule a área total da região sombreada.\\[0.4cm]

\textbf{\Large Resolução}\\[0.2cm]

O quadrado possui lado de comprimento $3\ \text{cm}$. Adotamos o seguinte sistema de coordenadas:
\[
M=(0,3),\quad Q=(3,3),\quad P=(3,0),\quad N=(0,0).
\]

Os pontos que dividem os lados superior e inferior em três partes iguais são:
\[
U=(1,3),\quad T=(2,3),\qquad
R=(1,0),\quad S=(2,0).
\]

As retas ligadas são:
\[
NT,\quad RQ,\quad SM,\quad PU.
\]

A figura formada possui quatro regiões sombreadas congruentes. Calculamos a área de apenas uma delas (a região superior-esquerda).

Essa região tem vértices:
\[
A=(0,3),\qquad 
B=(1,3),\qquad
C=\left(\tfrac{3}{2},\tfrac{9}{4}\right),\qquad
D=\left(1,\tfrac{3}{2}\right),
\]
onde os pontos $C$ e $D$ são obtidos como interseções das retas do interior:
\[
NT \cap RQ = \left(\tfrac{3}{2},\tfrac{9}{4}\right),
\qquad
NT \cap SM = \left(1,\tfrac{3}{2}\right).
\]

Usamos a fórmula da área de um polígono (método do ``shoelace''):
\[
\text{Área} = 
\frac12
\left|
\sum_{i=1}^4(x_i y_{i+1} - x_{i+1} y_i)
\right|.
\]

Aplicando aos pontos $(A,B,C,D)$:
\[
\text{Área}_{\text{uma região}}=\frac{9}{8}\ \text{cm}^2.
\]

Como há quatro regiões congruentes, a área total sombreada é:
\[
\text{Área total}=
4\cdot \frac{9}{8}
=\frac{9}{2}
=4{,}5\ \text{cm}^2.
\]

\vspace{0.3cm}
\textcolor{blue}{\Large \textbf{Resposta:}}\\[0.1cm]
\[
\boxed{\text{Área sombreada}=\frac{9}{2}\ \text{cm}^2=4{,}5\ \text{cm}^2.}
\]

\end{flushleft}

\rule{\textwidth}{0.4pt}

\begin{center}
    \LARGE \textbf{Resolução — problema do triângulo equilátero}
\end{center}

\bigskip

\textbf{Enunciado (resumo).} Seja \(ABC\) um triângulo equilátero de lado unitário (\(AB=BC=CA=1\)). As retas \(r,s,t\) são paralelas entre si, com \(A\) e \(B\) pertencentes a \(t\) e \(C\) pertencente a \(r\). A reta \(s\) está a deslocar-se de \(r\) até \(t\) e seja \(x\) a distância entre \(r\) e \(s\). Determinar a área sombreadas (ver figura) em função de \(x\).


\begin{figure}[!h]
  \centering
  \includegraphics[width=0.5\textwidth]{images/quadradoamarelo.png}
\end{figure}


\bigskip

\textbf{Solução.}

Coloquemos o triângulo com a base \(AB\) no segmento horizontal de coordenadas \(y=0\) (reta \(t\)) e o vértice \(C\) no eixo vertical acima da base, em \(y=H\) (reta \(r\)). Para um triângulo equilátero de lado \(1\) a altura \(H\) é
\[
H=\frac{\sqrt{3}}{2}.
\]

A reta \(s\) está a uma distância \(x\) abaixo de \(r\); portanto a ordenada da reta \(s\) é \(y=H-x\) (com \(0\le x\le H\)).

A região sombreada na figura é a faixa horizontal de altura \(x\) (entre \(r\) e \(s\)) limitada lateralmente pelos lados verticais do retângulo cuja base coincide com \(AB\), exceto a parte retirada pela porção superior do triângulo equilátero que penetra nessa faixa. Assim a área sombreada \(A(x)\) é dada por
\[
\text{(área do retângulo de largura }1\text{ e altura }x)\;-\;\text{(área do pequeno triângulo superior que fica dentro da faixa)}.
\]

A área do retângulo de largura \(1\) e altura \(x\) é simplesmente \(1\cdot x=x\).

Agora vamos calcular a área do pequeno triângulo no topo que fica entre \(y=H-x\) e \(y=H\). Esse pequeno triângulo é semelhante ao triângulo grande \(ABC\). A razão de semelhança (linear) entre o pequeno triângulo (altura \(x\)) e o triângulo grande (altura \(H\)) é
\[
\lambda=\frac{x}{H}.
\]
Logo a base do pequeno triângulo tem comprimento \(\lambda\) vezes a base do triângulo grande, ou seja
\[
\text{base}_{\text{pequeno}} = 1\cdot\frac{x}{H}=\frac{x}{H}.
\]
A área do pequeno triângulo é então
\[
\text{área}_{\text{pequeno}}=\frac{1}{2}\cdot\text{base}_{\text{pequeno}}\cdot\text{altura}_{\text{pequeno}}
=\frac{1}{2}\cdot\frac{x}{H}\cdot x=\frac{x^{2}}{2H}.
\]

Substituindo \(H=\dfrac{\sqrt{3}}{2}\) obtemos
\[
\frac{x^{2}}{2H}=\frac{x^{2}}{2\cdot(\sqrt{3}/2)}=\frac{x^{2}}{\sqrt{3}}.
\]

Portanto a área sombreada é
\[
A(x)=\underbrace{x}_{\text{faixa}}-\underbrace{\frac{x^{2}}{\sqrt{3}}}_{\text{triângulo retirado}}.
\]

Assim, para \(0\le x\le H=\dfrac{\sqrt{3}}{2}\),
\[
\boxed{\,A(x)=x-\frac{x^{2}}{\sqrt{3}}\ = -\frac{\sqrt{3}}{3}x^{2} + x,}.
\]

\bigskip

\vspace{4mm}
\noindent (Observação) Verifica-se rapidamente que \(A(0)=0\) e que para \(x=H=\dfrac{\sqrt{3}}{2}\)
\[
A\!\left(\frac{\sqrt{3}}{2}\right)=\frac{\sqrt{3}}{2}-\frac{\left(\frac{\sqrt{3}}{2}\right)^{2}}{\sqrt{3}}
=\frac{\sqrt{3}}{2}-\frac{3/4}{\sqrt{3}}=\frac{\sqrt{3}}{2}-\frac{\sqrt{3}}{4}=\frac{\sqrt{3}}{4},
\]
valor coerente com a geometria do problema.

\rule{\textwidth}{0.4pt}

\Large\textbf{Resolução detalhada (sem desenho)}\\[6pt]
\large \((\text{Problema: hexágono regular }H_1\text{; unir os pontos médios dos lados forma }H_2)\)

\bigskip

\textbf{Enunciado resumido.}  
Dado um hexágono regular \(H_1\), ao unir-se os pontos médios dos seus lados obtém--se um hexágono regular \(H_2\). Determinar a razão entre as áreas de \(H_1\) e \(H_2\).

\bigskip

\textbf{Solução.}

Seja \(a\) o comprimento do lado do hexágono regular \(H_1\). É conhecido que, para um hexágono regular, o \emph{raio do circuncírculo} (distância do centro a cada vértice) é igual ao lado:
\[
R=a.
\]
Convenientemente, representamos os vértices de \(H_1\) por vetores posicionais no plano complexo (ou vetores em \(\mathbb{R}^2\)). Tomemos os vértices
\[
v_k=R e^{i\theta_k},\qquad \theta_k=\theta_0 + k\cdot 60^\circ,\quad k=0,1,\dots,5.
\]
Considere dois vértices consecutivos \(v_k\) e \(v_{k+1}\). O ponto médio do lado correspondente tem posição
\[
m_k=\frac{v_k+v_{k+1}}{2}= \frac{R\big(e^{i\theta_k}+e^{i\theta_{k+1}}\big)}{2}.
\]
Usando a identidade de soma de exponenciais complexas,
\[
e^{i\theta_k}+e^{i\theta_{k+1}} = e^{i(\theta_k+\theta_{k+1})/2}\cdot 2\cos\!\big(\tfrac{\theta_{k+1}-\theta_k}{2}\big).
\]
Aqui \(\theta_{k+1}-\theta_k = 60^\circ\), logo \(\tfrac{\theta_{k+1}-\theta_k}{2}=30^\circ\). Assim
\[
m_k = R\,e^{i(\theta_k+\theta_{k+1})/2}\cos 30^\circ.
\]
Portanto o ponto médio \(m_k\) tem módulo (distância ao centro)
\[
|m_k| = R\cos 30^\circ = R\cdot\frac{\sqrt{3}}{2}.
\]

Isto mostra duas coisas importantes:
\begin{itemize}
  \item os pontos médios \(m_k\) estão todos à mesma distância do centro (logo formam um polígono regular),
  \item o hexágono \(H_2\) obtido é \emph{semelhante} a \(H_1\) com fator de escala linear
  \[
  \lambda=\frac{|m_k|}{R}=\cos 30^\circ=\frac{\sqrt{3}}{2}.
  \]
\end{itemize}

Como a razão linear entre \(H_2\) e \(H_1\) é \(\lambda=\dfrac{\sqrt{3}}{2}\), a razão entre as áreas é o quadrado desse fator:
\[
\frac{\text{Área}(H_2)}{\text{Área}(H_1)}=\lambda^2=\left(\frac{\sqrt{3}}{2}\right)^2=\frac{3}{4}.
\]

Portanto a razão pedida entre as áreas de \(H_1\) e \(H_2\) (normalmente escrita \(\text{Área}(H_1):\text{Área}(H_2)\)) é
\[
\boxed{\ \text{Área}(H_1):\text{Área}(H_2)=1:\tfrac{3}{4}\ =\ 4:3\ }
\]
ou, em forma de quociente,
\[
\boxed{\ \frac{\text{Área}(H_1)}{\text{Área}(H_2)}=\frac{4}{3}\ }.
\]

\bigskip

\textbf{Observação alternativa (por área em função do lado).}  
A área de um hexágono regular de lado \(a\) é \(A_1=\dfrac{3\sqrt{3}}{2}\,a^2\). Os pontos médios formam um hexágono regular cujo raio do circuncírculo vale \(R'=\dfrac{\sqrt{3}}{2}R=\dfrac{\sqrt{3}}{2}a\); como para o hexágono o lado é igual ao raio do circuncírculo, o novo lado vale \(a'=\dfrac{\sqrt{3}}{2}a\). Assim
\[
A_2=\frac{3\sqrt{3}}{2}\,(a')^2=\frac{3\sqrt{3}}{2}\left(\frac{\sqrt{3}}{2}a\right)^2
=\frac{3\sqrt{3}}{2}\cdot\frac{3}{4}a^2=\frac{3}{4}\,A_1,
\]
o que conduz à mesma razão \(A_2/A_1=3/4\) e \(A_1/A_2=4/3\).

\rule{\textwidth}{0.4pt}

\noindent\textbf{Enunciado (resumo).} Os triângulos externos da figura são equiláteros, 
todos os quadriláteros mostrados são quadrados e o polígono do meio é um hexágono regular. Calcular a razão
\[
\frac{\text{soma das áreas das regiões sombreadas}}{\text{soma das áreas das regiões em branco}}.
\]

\bigskip

\begin{figure}[!h]
  \centering
  \includegraphics[width=0.3\textwidth]{images/hexagonos.png}
\end{figure}

\section*{Solução}

Seja \(a\) o comprimento do lado dos quadrados e dos triângulos equiláteros.

\[
A_{\triangle}=\frac{\sqrt{3}}{4}a^{2}.
\]

O hexágono regular pode ser decomposto em 6 triângulos equiláteros de lado \(a\):

\[
A_{\text{hex}}=6A_{\triangle}
=6\cdot\frac{\sqrt{3}}{4}a^{2}
=\frac{3\sqrt{3}}{2}a^{2}.
\]

A área sombreada é composta por 6 triângulos equiláteros mais o hexágono:

\[
A_{\text{sombra}}
=6A_{\triangle}+A_{\text{hex}}
=6\cdot\frac{\sqrt{3}}{4}a^{2}
+\frac{3\sqrt{3}}{2}a^{2}
=3\sqrt{3}\,a^{2}.
\]

As regiões brancas são formadas por 6 quadrados de lado \(a\):

\[
A_{\text{branco}}=6a^{2}.
\]

A razão pedida é:

\[
\frac{A_{\text{sombra}}}{A_{\text{branco}}}
=\frac{3\sqrt{3}\,a^{2}}{6a^{2}}
=\frac{\sqrt{3}}{2}.
\]

\[
\boxed{\dfrac{\sqrt{3}}{2}}
\]

\rule{\textwidth}{0.4pt}

\Large\textbf{Resolução — questão do triângulo hachurado}

\vspace{1em}

Enunciado (resumido). No plano, um dos lados de um retângulo está sobre a reta \(r\). O retângulo 
tem altura \(6\) cm. Um ponto \(P\) sobre a reta \(r\) situa-se a \(6\) cm à direita do vértice inferior 
direito do retângulo. O segmento que liga \(P\) ao vértice superior esquerdo do retângulo forma com o 
lado superior do retângulo um ângulo \(\alpha\) tal que \(\tan\alpha=\tfrac{1}{3}\). 

\begin{figure}[!h]
  \centering
  \includegraphics[width=0.5\textwidth]{images/trianguloazul.png}
\end{figure}

A parte hachurada é o 
triângulo formado pelos pontos: vértice superior esquerdo \(A\), vértice superior direito \(B\) e o ponto 
\(C\) onde a reta passa pela face direita do retângulo (ver figura). Determinar a área do triângulo hachurado.

\vspace{1em}
\noindent\textbf{Resolução.}

Considere um sistema de eixos com origem no vértice inferior esquerdo do retângulo. Seja
\[
A=(0,6),\qquad B=(w,6),
\]
onde \(w\) é a largura do retângulo (a determinar). O ponto \(C\) está na face direita do retângulo, portanto tem coordenadas \(C=(w,y)\) com \(0<y<6\). O ponto \(P\) (sobre a reta \(r\)) está a \(6\) cm à direita do vértice inferior direito, logo
\[
P=(w+6,0).
\]

A reta que liga \(A\) a \(P\) é a mesma que passa por \(A\) e por \(C\). A inclinação dessa reta (ângulo \(\alpha\) com a horizontal) satisfaz
\[
\tan\alpha=\frac{\text{cateto oposto (queda vertical)}}{\text{cateto adjacente (deslocamento horizontal)}}=\frac{6-y}{w}.
\]
Pelo enunciado, \(\tan\alpha=\dfrac{1}{3}\), assim
\begin{equation}\label{eq1}
\frac{6-y}{w}=\frac{1}{3}\quad\Longrightarrow\quad 6-y=\frac{w}{3}.
\end{equation}

Por outro lado, podemos escrever a equação da reta \(AP\). A inclinação entre \(A(0,6)\) e \(P(w+6,0)\) é
\[
m=\frac{0-6}{(w+6)-0}=-\frac{6}{w+6}.
\]
A ordenada do ponto da reta no \(x=w\) (isto é, a coordenada \(y\) de \(C\)) é
\[
y=6+m\cdot w=6-\frac{6w}{w+6}=\frac{36}{w+6}.
\]
Logo
\[
6-y=6-\frac{36}{w+6}=\frac{6w}{w+6}.
\]
Igualando essa expressão com a do lado direito de \eqref{eq1} obtemos
\[
\frac{6w}{w+6}=\frac{w}{3}.
\]
Como \(w>0\), podemos dividir ambos os lados por \(w\):
\[
\frac{6}{w+6}=\frac{1}{3}\quad\Longrightarrow\quad 18=w+6\quad\Longrightarrow\quad w=12.
\]

Logo a largura do retângulo é \(w=12\) cm. Pela \eqref{eq1},
\[
6-y=\frac{w}{3}=\frac{12}{3}=4\quad\Longrightarrow\quad y=2.
\]

Portanto o triângulo hachurado tem vértices \(A=(0,6)\), \(B=(12,6)\) e \(C=(12,2)\). Sua base é \(AB=12\) e sua altura (distância vertical entre \(AB\) e \(C\)) é \(6-2=4\). Assim a área é
\[
\mathcal{A}=\frac{1}{2}\cdot\text{base}\cdot\text{altura}=\frac{1}{2}\cdot 12\cdot 4=24\ \text{cm}^2.
\]

\vspace{1em}
\colorbox{green}{\noindent\textbf{Resposta: } \(24\ \mathrm{cm}^2\) (alternativa \(\text{c}\)).}

\end{document}
