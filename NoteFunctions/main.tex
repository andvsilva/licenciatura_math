\documentclass[a4paper,12pt]{article}
\usepackage[brazil]{babel}
\usepackage[utf8]{inputenc}
\usepackage[T1]{fontenc}
\usepackage{geometry}
\usepackage{setspace}
\usepackage{titlesec}
\usepackage{hyperref}
\usepackage{graphicx}
\usepackage{caption}
\usepackage{subcaption}
\usepackage{fancyhdr}
\usepackage{xcolor}

%%%%%%%%%%%%%%%%%%%%%%%%%%%%%%%%%%%%%%%%%%%%%%%%%%
% These are some new commands that may be useful 
% for paper writing in general. If other new commands
% are needed for your specific paper, please feel 
% free to add here. 
%
% The currently available commands are organized in: 
% 1) Systems
% 2) Quantities
% 3) Energies and units
% 4) particle species
% 5) Colors package
% 6) hyperlink
%%%%%%%%%%%%%%%%%%%%%%%%%%%%%%%%%%%%%%%%%%%%%%%%%%

\usepackage{amsmath}
\usepackage{amssymb}
\usepackage{upgreek}
\usepackage{multirow}
\usepackage{setspace}% http://ctan.org/pkg/setspace
\usepackage{fancyhdr}
\usepackage{datetime}

% 1) SYSTEMS 
\newcommand{\pp}           {pp\xspace}
\newcommand{\ppbar}        {\mbox{$\mathrm {p\overline{p}}$}\xspace}
\newcommand{\XeXe}         {\mbox{Xe--Xe}\xspace}
\newcommand{\PbPb}         {\mbox{Pb--Pb}\xspace}
\newcommand{\pA}           {\mbox{pA}\xspace}
\newcommand{\pPb}          {\mbox{p--Pb}\xspace}
\newcommand{\AuAu}         {\mbox{Au--Au}\xspace}
\newcommand{\dAu}          {\mbox{d--Au}\xspace}
\def\pA{$pA$\xspace}
\def\AA{$AA$\xspace}
\def\NN{$NN$\xspace}
\def\signn{$\sigma^{inel}_{NN}$\xspace}
\def\sigtotal{$\sigma_{\textnormal{tot}}$\xspace}
\def\mrm{\mathrm}
\def\ntrig{N_\mrm{trig}}
\newcommand{\rivet}{R\protect\scalebox{1}{IVET}\xspace}
\newcommand{\hepmc}{H\protect\scalebox{1}{EP}MC\xspace}
\newcommand{\herwig}{H\protect\scalebox{1}{ERWIG} 7\xspace}
\newcommand{\sherpa}{S\protect\scalebox{1}{HERPA}\xspace}
\newcommand{\urqmd}{U\protect\scalebox{1}{r}QMD\xspace}
\newcommand{\urqmdversion}{U\protect\scalebox{1}{r}QMD 3.4\xspace}
\newcommand{\pythia}{\protect\scalebox{1}{PYTHIA}\xspace}
\newcommand{\pythiaversion}{\protect\scalebox{1}{PYTHIA 8.2}\xspace}
\newcommand{\pythiaversionused}{\protect\scalebox{1}{PYTHIA 8.235}\xspace}
\newcommand{\pytang}{\protect\scalebox{1}{PYTHIA}/Angantyr\xspace}
\newcommand{\angantyr}{\protect\scalebox{1}{}Angantyr\xspace}
\newcommand{\pytangur}{\protect\scalebox{1}{PYTHIA}/Angantyr + U\protect\scalebox{1}{r}QMD\xspace}
\newcommand{\figref}[1]{Fig.~\ref{#1}}
\newcommand{\tabref}[1]{Tab.~\ref{#1}}
\renewcommand{\eqref}[1]{Eq.~(\ref{#1})}

% hydrodynamic simulation chain:
% TRENTo
\newcommand{\trento}{\protect\scalebox{1}{T$_{\text{R}}$ENT}o\xspace}
% KOMPOST : Linear kinetic theory propagator for initial conditions in heavy ion collisions
\newcommand{\kompost}{\protect\scalebox{1}{K$\varnothing$MP$\varnothing$ST}\xspace}
% MUSIC
\newcommand{\music}{\protect\scalebox{1}{MUSIC}\xspace}
% iSS
\newcommand{\iss}{\protect\scalebox{1}{iSS}\xspace}

% 2) QUANTITIES 
\newcommand{\s}            {\ensuremath{\sqrt{s}}\xspace}
\newcommand{\snn}          {\ensuremath{\sqrt{s_{\mathrm{NN}}}}\xspace}
\newcommand{\pt}           {\ensuremath{p_{\rm T}}\xspace}
\newcommand{\meanpt}       {$\langle p_{\mathrm{T}}\rangle$\xspace}
\newcommand{\ycms}         {\ensuremath{y_{\rm CMS}}\xspace}
\newcommand{\ylab}         {\ensuremath{y_{\rm lab}}\xspace}
\newcommand{\etarange}[1]  {\mbox{$\left | \eta \right |~<~#1$}}
\newcommand{\centbin}[2]  {\mbox{$#1-#2\%$}}
\newcommand{\ptrange}[2]  {\mbox{$#1 < p_{\mathrm{T}}\hspace{0.2cm} (\mathrm{GeV}/\mathrm{\textit{c}}) <#2$}}
\newcommand{\ptrangetrig}[2]  {\mbox{$#1 < p^{\mathrm{trigger}}_{\mathrm{T} }\hspace{0.2cm} (\mathrm{GeV}/\mathrm{\textit{c}}) <#2$}}
\newcommand{\ptrangeassoc}[2]  {\mbox{$#1 < p^{\mathrm{assoc}}_{\mathrm{T} }\hspace{0.2cm} (\mathrm{GeV}/\mathrm{\textit{c}}) <#2$}}
\newcommand{\etazerothree} {$\left|\eta \right| < 0.3$\xspace}
\newcommand{\etazerofive} {$\left|\eta \right| < 0.5$\xspace}
\newcommand{\etazeroeight} {$\left|\eta \right| < 0.8$\xspace}
\newcommand{\yrange}[1]    {\mbox{$\left | y \right |~<~#1$}}
\newcommand{\dndy}         {\ensuremath{\mathrm{d}N_\mathrm{ch}/\mathrm{d}y}\xspace}
\newcommand{\dndeta}       {\ensuremath{\mathrm{d}N_\mathrm{ch}/\mathrm{d}\eta}\xspace}
\newcommand{\dnchdydpt}   {\ensuremath{\mathrm{d}N_\mathrm{ch}/\mathrm{d}y\mathrm{d}p_{\mathrm{T}}}\xspace}
\newcommand{\dnchaadydpt}   {\ensuremath{\mathrm{d}N_\mathrm{ch}^{AA}/\mathrm{d}y\mathrm{d}p_{\mathrm{T}}}\xspace}
\newcommand{\dnchppdydpt}   {\ensuremath{\mathrm{d}N_\mathrm{ch}^{\mathrm{pp}}/\mathrm{d}y\mathrm{d}p_{\mathrm{T}}}\xspace}
\newcommand{\dnchdphi}{\ensuremath{\mathrm{d}N_\mathrm{ch}/\mathrm{d}\phi}\xspace}
\newcommand{\dnchddeltaphi}{\ensuremath{\mathrm{d}N_\mathrm{ch}/\mathrm{d}\Delta\upphi}\xspace}
\newcommand{\dndphi}{\ensuremath{\mathrm{d}N/\mathrm{d}\phi}\xspace}
\newcommand{\dnddeltaphi}{\ensuremath{\mathrm{d}N/\mathrm{d}\Delta\upphi}\xspace}
\newcommand{\avdndeta}     {\ensuremath{\langle\dndeta\rangle}\xspace}
\newcommand{\avdndetarap}  {$\langle$ dN$_{\textnormal{ch}}$/d$\eta$ $\rangle_{|\eta| < 0.5}$\xspace}
\newcommand{\dNdy}         {\ensuremath{\mathrm{d}N_\mathrm{ch}/\mathrm{d}y}\xspace}
\newcommand{\Npart}        {\ensuremath{N_\mathrm{part}}\xspace}
\newcommand{\meanNpart}    {$\langle$\ensuremath{N_\mathrm{part}}$\rangle$\xspace}
\newcommand{\ncoll}        {\ensuremath{N_\mathrm{coll}}\xspace}
\newcommand{\meanncoll}    {$\langle$\ensuremath{N_\mathrm{coll}}$\rangle$\xspace}
\newcommand{\averagencollhadronic}    {$\langle$\ensuremath{\mathrm{N}_\mathrm{coll}^{\mathrm{hadronic}}}$\rangle$\xspace}
\newcommand{\meantaa}      {$\langle$\ensuremath{T_\mathrm{AA}}$\rangle$\xspace}
\newcommand{\dEdx}         {\ensuremath{\textrm{d}E/\textrm{d}x}\xspace}
\newcommand{\RpPb}         {\ensuremath{R_{\rm pPb}}\xspace}
\newcommand{\raa}          {$R_{AA}$\xspace}
\newcommand{\vtwo}         {$v_{2}$\xspace}
\newcommand{\vtwoinitial}  {$v_{2}^{\mathrm{initial}}$\xspace}
\newcommand{\vtwofinal}    {$v_{2}^{\mathrm{final}}$\xspace}
\newcommand{\vtwofourfinal}{$v_{2}^{\mathrm{final}}\{4\}$\xspace}
\newcommand{\vtwofit}      {$v_{2}^{\mathrm{Fit}}$\xspace}
\newcommand{\vtwotwo}      {$v_{2}\{2\}$\xspace}
\newcommand{\vtwofour}     {$v_{2}\{4\}$\xspace}
\newcommand{\vtwopt}       {$v_{2}(p_{\textnormal{T}})$\xspace}
\newcommand{\vtwoptfit}    {$v_{2}^{\mathrm{Fit}}(p_{\textnormal{T}})$\xspace}
\newcommand{\nch}          {\ensuremath{N_\mathrm{ch}}\xspace}
\newcommand{\psireactionplane}          {$\Psi_{\textnormal{RP}}$\xspace}
\newcommand{\deltaphireactionplane}     {$\Delta\upphi = \phi - \Psi_{\textnormal{RP}}$\xspace}
\newcommand{\nevdnchddeltaphi}     {(1/N$_{\textnormal{ev}}$)dN$_{\textnormal{ch}}$/d$\Delta\upphi$\xspace}
\newcommand{\meannch}      {\ensuremath{\langle N_\mathrm{ch}\rangle}\xspace}
\newcommand{\etamodule}    {\ensuremath{|\eta|}\xspace}
\newcommand{\qbar}         {$\bar{\textnormal{q}}$\xspace}
\newcommand{\qqbar}        {$\textnormal{q}\bar{\textnormal{q}}$\xspace}
\newcommand{\qqbarzero}    {$\textnormal{q}_{0}\bar{\textnormal{q}}_{0}$\xspace}
\newcommand{\qqqbars}      {$\bar{\textnormal{q}}\bar{\textnormal{q}}\bar{\textnormal{q}}$\xspace}
\newcommand{\alphastrong}  {$\alpha_{\textnormal{s}}$\xspace}
\newcommand{\alphastrongdistance}  {$\alpha_{\textnormal{s}}$(R)\xspace}
\newcommand{\qtwo}         {Q$^2$\xspace}
\newcommand{\alphastrongqtwo}  {$\alpha_{\textnormal{s}}$(Q$^2$)\xspace}
\newcommand{\lambdaqcd}        {$\Lambda_{\textnormal{QCD}}$\xspace}
\newcommand{\sectionpp}        {$\sigma^{\textnormal{pp}}_{\textnormal{inel}}$\xspace}

% 3) ENERGIES, UNITS
\newcommand{\sqrts}        {$\sqrt{s}$\xspace}
\newcommand{\sqrtsnn}      {$\sqrt{s_{\mathrm{NN}}}$\xspace}
\newcommand{\nineH}        {$\sqrt{s}~=~0.9$~Te\kern-.1emV\xspace}
\newcommand{\seven}        {$\sqrt{s}~=~7$~Te\kern-.1emV\xspace}
\newcommand{\twoH}         {$\sqrt{s}~=~0.2$~Te\kern-.1emV\xspace}
\newcommand{\twosevensix}  {$\sqrt{s}~=~2.76$~Te\kern-.1emV\xspace}
\newcommand{\five}         {$\sqrt{s}~=~5.02$~Te\kern-.1emV\xspace}
\newcommand{\twohundrernn} {$\sqrt{s_{\mathrm{NN}}}=200$~Ge\kern-.1emV\xspace}
\newcommand{\twosevensixnn} {$\sqrt{s_{\mathrm{NN}}}=2.76$~Te\kern-.1emV\xspace}
\newcommand{\fivenn}       {$\sqrt{s_{\mathrm{NN}}}~=~5.02$~Te\kern-.1emV\xspace}
\newcommand{\fivefourfournn} {$\sqrt{s_{\mathrm{NN}}}=5.44$~Te\kern-.1emV\xspace}
\newcommand{\LT}           {L{\'e}vy-Tsallis\xspace}
\newcommand{\GeVc}         {Ge\kern-.1emV/$c$\xspace}
\newcommand{\MeVc}         {Me\kern-.1emV/$c$\xspace}
\newcommand{\TeV}          {Te\kern-.1emV\xspace}
\newcommand{\GeV}          {Ge\kern-.1emV\xspace}
\newcommand{\MeV}          {Me\kern-.1emV\xspace}
\newcommand{\GeVmass}      {Ge\kern-.2emV/$c^2$\xspace}
\newcommand{\MeVmass}      {Me\kern-.2emV/$c^2$\xspace}
\newcommand{\lumi}         {\ensuremath{\mathcal{L}}\xspace}
\newcommand{\fmc}         {fm\kern-.1em/$c$\xspace}

% 4) PARTICLE SPECIES 
\newcommand{\ee}           {\ensuremath{e^{+}e^{-}}} 
\newcommand{\pip}          {\ensuremath{\pi^{+}}\xspace}
\newcommand{\pim}          {\ensuremath{\pi^{-}}\xspace}
\newcommand{\kap}          {\ensuremath{\rm{K}^{+}}\xspace}
\newcommand{\kam}          {\ensuremath{\rm{K}^{-}}\xspace}
\newcommand{\pbar}         {\ensuremath{\rm\overline{p}}\xspace}
\newcommand{\kzero}        {\ensuremath{{\rm K}^{0}_{\rm{S}}}\xspace}
\newcommand{\lmb}          {\ensuremath{\Lambda}\xspace}
\newcommand{\almb}         {\ensuremath{\overline{\Lambda}}\xspace}
\newcommand{\Om}           {\ensuremath{\Omega^-}\xspace}
\newcommand{\Mo}           {\ensuremath{\overline{\Omega}^+}\xspace}
\newcommand{\X}            {\ensuremath{\Xi^-}\xspace}
\newcommand{\Ix}           {\ensuremath{\overline{\Xi}^+}\xspace}
\newcommand{\Xis}          {\ensuremath{\Xi^{\pm}}\xspace}
\newcommand{\Oms}          {\ensuremath{\Omega^{\pm}}\xspace}
\newcommand{\degree}       {\ensuremath{^{\rm o}}\xspace}
\newcommand{\comment}[1]{}

% two-particle angular correlation
\newcommand{\deltaphitriggassoc}    {$\Delta\upphi = |\phi_{\textnormal{trigger}} - \phi_{\textnormal{assoc}}|$\xspace}
\newcommand{\deltaetatriggassoc}    {$\Delta\upeta = |\eta_{\textnormal{trigger}} - \eta_{\textnormal{assoc}}|$\xspace}
\newcommand{\etatrigg}    {$\eta_{\textnormal{trigger}}$\xspace}
\newcommand{\etaassoc}    {$\eta_{\textnormal{assoc}}$\xspace}
\newcommand{\deltaphideltaeta}      {$\Delta\upphi-\Delta\upeta$\xspace}
\newcommand{\deltaphi}              {$\Delta\upphi$\xspace}
\newcommand{\moduledeltaphipitwo}   {$|\Delta\upphi| < \pi/2 $\xspace}
\newcommand{\deltaeta}              {$\Delta\upeta$\xspace}
\newcommand{\moduledeltaeta}        {$|\Delta\upeta|$\xspace}
\newcommand{\deltaphiapproxzero}    {$\Delta\upphi = 0$\xspace}
\newcommand{\deltaphiapproxpi}      {$\Delta\upphi = \pi$\xspace}
\newcommand{\deltaetaapproxzero}    {$\Delta\upeta = 0$\xspace}
\newcommand{\corrfunc}              {C($\Delta\upphi$, $\Delta\upeta$)\xspace}
\newcommand{\corrfunccorrect}              {C$_{\mathrm{correct}}(\Delta\upphi$, $\Delta\upeta$)\xspace}
\newcommand{\corrfuncmix}              {C$_{\mathrm{mix}}(\Delta\upphi$, $\Delta\upeta$)\xspace}
\newcommand{\corrfuncdeltaphi}      {C($\Delta\upphi$)\xspace}
\newcommand{\pttrigger}             {$p_{\textnormal{T}}^{\textnormal{trigger}}$\xspace}
\newcommand{\ptassoc}               {$p_{\textnormal{T}}^{\textnormal{assoc}}$\xspace}
\newcommand{\ratioyieldawaynearside}{Y$_{\textnormal{Away}}$/Y$_{\textnormal{Near}}$\xspace}

% 4) definition to references, biblatex and hyperlink
\usepackage[backend=bibtex, 
style=nature,  %style reference.
sorting=none,
firstinits=true %first name abbreviate
]{biblatex}

\usepackage{hyperref}
\hypersetup{
    colorlinks=true, %set "true" if you want colored links
    linktoc=all,     %set to "all" if you want both sections and subsections linked
    linkcolor=blue,  %choose some color if you want links to stand out
    citecolor= blue, % color of \cite{} in the text.
    urlcolor  = blue, % color of the link for the paper in references.
}

% 5) Tikz and figures
\usepackage{epsfig}
\usepackage{lmodern}
\usepackage{mathtools}
\usepackage[utf8]{luainputenc}
\usepackage{xspace}
\usepackage{tikz}
\usepackage{pgfplots}
\pgfplotsset{compat=newest}

\usetikzlibrary{positioning}
\usepackage{subcaption}

% 6) colors:
\usepackage{xcolor}
\definecolor{ao(english)}{rgb}{0.0, 0.5, 0.0} % dark green

% 7) Add lines numbers
%\usepackage{lineno}

% add pdf file to thesis:
\usepackage{pdfpages}

\hypersetup{
    colorlinks=true,% make the links colored
    linkcolor=blue
}

\usepackage{setspace}
\addbibresource{bibliography.bib}

\newcommand{\printingbibliography}{%

    \pagestyle{myheadings}
    \markright{}
    \sloppy
    \printbibliography[heading=bibintoc, % add to table of contents
                   title=Refer\^encias % Chapter name
                  ]
    \fussy%
}
\PassOptionsToPackage{table}{xcolor}
\renewcommand{\baselinestretch}{1.5}

\pagestyle{fancy}
\fancyhf{}
\renewcommand{\headrulewidth}{0pt}
\fancyhead[R]{\thepage}

\geometry{a4paper,top=30mm,bottom=20mm,left=30mm,right=20mm}

\titleformat*{\section}{\bfseries\large}
\titleformat*{\subsection}{\bfseries\normalsize}

\title{ \textbf{\large Notas: Fun\c{c}\~ao} }
\author{Andr\'e V. Silva}
\date{\today}

\begin{document}

\maketitle

\textbf{Enunciado}

A trajetória de um salto de um golfinho nas proximidades de uma praia, do instante em que ele saiu da água ($t=0$) até o instante em que mergulhou ($t=T$), foi descrita por um observador através do seguinte modelo matemático:
\[
h(t)=4t - t\cdot 2^{0,2t},
\]
com $t$ em segundos, $h(t)$ em metros e $0\le t\le T$. O tempo, em segundos, em que o golfinho esteve fora da água durante este salto foi:

\begin{itemize}
    \item[a)] 1
    \item[b)] 2
    \item[c)] 4
    \item[d)] 8
    \item[e)] 10
\end{itemize}

{\color{blue}
\textbf{Resolução detalhada}

O golfinho está fora da água sempre que $h(t) > 0$. Sabemos que $h(0)=0$, portanto o instante inicial é $t=0$. Para descobrir quando ele volta a mergulhar, devemos encontrar o próximo instante $T>0$ tal que
\[
h(t)=4t - t\cdot 2^{0,2t}=0.
\]

Primeiro, fatoramos $t$:
\[
h(t)=t\left(4 - 2^{0,2t}\right).
\]

Igualando a zero:
\[
t\left(4 - 2^{0,2t}\right)=0.
\]

Daí, temos duas possibilidades:
\[
t=0 
\quad\text{ou}\quad 
4 - 2^{0,2t}=0.
\]

A segunda equação fornece o tempo de mergulho:
\[
4 = 2^{0,2t}.
\]

Como $4 = 2^2$, obtemos:
\[
2^{0,2t}=2^{2}.
\]

Igualando os expoentes:
\[
0,2t = 2.
\]

Resolvendo para $t$:
\[
t = \frac{2}{0,2} = 10.
\]

Portanto, o golfinho permaneceu fora da água durante:
\[
T = 10\ \text{segundos}.
\]

Logo, a alternativa correta é \textbf{(e) 10}.
}

\begin{figure}[!ht]
    \centering
    \includegraphics[scale=0.8]{images/dolphin_jump_plot.png}
    \caption{fun\c{c}\~ao do 2 Grau para altura do saldo do golfinho}
    \label{fig:golfheight}
\end{figure}

\newpage

--------------------------------------------------------------------------------------------------

\textbf{Enunciado}

Resolver a inequação:
\[
\left(\frac13\right)^{x^{2}-3} \le \frac13.
\]

{\color{blue}
\textbf{Resolução}

Escrevemos o número do lado direito como potência da mesma base:
\[
\frac13 = \left(\frac13\right)^1.
\]

Como $0 < \frac13 < 1$, a função $a^x$ é decrescente.  
Assim,
\[
\left(\tfrac13\right)^{A} \le \left(\tfrac13\right)^{B}
\quad\Longleftrightarrow\quad
A \ge B.
\]

Aplicando à inequação:
\[
x^{2} - 3 \ge 1.
\]

Logo:
\[
x^{2} \ge 4.
\]

Portanto:
\[
x \le -2 \quad\text{ou}\quad x \ge 2.
\]

A solução é:
\[
\boxed{x \le -2 \;\; \text{ou} \;\; x \ge 2}.
\]
}

\bigskip

\textbf{Gráfico da função } $\displaystyle \left(\frac13\right)^{x^{2}-3}$

\begin{center}
\begin{tikzpicture}
\begin{axis}[
    width=12cm,
    height=7cm,
    axis lines=middle,
    xlabel=$x$,
    ylabel={$y$},
    grid=both,
    xmin=-5, xmax=5,
    ymin=0, ymax=1.2,
    samples=300,
    legend style={draw=none, at={(0.98,0.98)}, anchor=north east}
]

% função f(x)
\addplot[thick, blue] {(1/3)^(x^2 - 3)};
\addlegendentry{$\left(\frac13\right)^{x^2-3}$}

% linha y=1/3
\addplot[thick, dashed, red] {1/3};
\addlegendentry{$y=\frac13$}

% marcas nas soluções
\addplot[mark=*] coordinates {(-2,{1/3})};
\addplot[mark=*] coordinates {(2,{1/3})};

\end{axis}
\end{tikzpicture}
\end{center}

--------------------------------------------------------------------------------
\begin{center}
\textbf{Gráfico da função } $y = \log_{4}(x)$
\end{center}

\begin{tikzpicture}[scale=3]

    % Axes
    \draw[->] (-0.1,0) -- (1.3,0) node[right] {$x$};
    \draw[->] (0,-1.3) -- (0,0.4) node[above] {$y$};

    % Logarithm curve (base 4)
    \draw[red, thick, domain=0.1:1.3, samples=200]
        plot (\x,{ln(\x)/ln(4)});

    % Dashed helper lines
    \draw[dashed] (0.25,0) -- (0.25,-1);
    \draw[dashed] (0,-1) -- (0.25,-1);

    % Labels
    \node[below] at (0.25,0.2) {$0.25$};
    \node[below] at (1,0) {$1$};
    \node[left]  at (0,-1) {$-1$};

\end{tikzpicture}

\[
\text{Sabemos que o gráfico representa a função } y = \log_b(x).
\]

Observe que o ponto destacado na figura é
\[
(0{,}25,\,-1).
\]

Isso significa que
\[
\log_b(0{,}25) = -1.
\]

Usando a definição de logaritmo:
\[
\log_b(0{,}25) = -1 
\quad \Longleftrightarrow \quad
b^{-1} = 0{,}25.
\]

Mas
\[
0{,}25 = \frac{1}{4}.
\]

Logo,
\[
b^{-1} = \frac{1}{4}
\quad \Longleftrightarrow \quad
b = 4.
\]

\[
\boxed{b = 4}
\]

---------------------------------------------------------------------------

\begin{center}
    \Large \textbf{Resoluções -- Questões de Funções Quadráticas}
\end{center}

\bigskip

\section*{Questão 2}
Seja \(f:\mathbb{R}\to\mathbb{R}\) tal que
\[
f(x)=x^{2}-18x+65.
\]

\medskip

{\color{blue}\textbf{(a) Resolva a inequação } \(f(x)\le 0\).}

\medskip

\begin{itemize}
    \item Primeiro encontramos as raízes da equação quadrática \(f(x)=0\):
    \[
    x^{2}-18x+65=0.
    \]
    Calcule o discriminante:
    \[
    \Delta=b^{2}-4ac = (-18)^{2}-4\cdot 1\cdot 65 = 324 - 260 = 64.
    \]
    \item Raízes:
    \[
    x=\frac{18\pm\sqrt{64}}{2}=\frac{18\pm 8}{2}.
    \]
    Assim
    \[
    x_1=\frac{18-8}{2}=5,\qquad x_2=\frac{18+8}{2}=13.
    \]
    \item Como o coeficiente \(a=1>0\), a parábola abre para cima; portanto \(f(x)\le 0\) entre as raízes:
    \[
    \boxed{\,5 \le x \le 13\,.}
    \]
\end{itemize}

\vspace{8pt}

{\color{blue}\textbf{(b) Determine a ordenada do vértice } \((y_V)\) diretamente pela fórmula.}

\medskip

Para um polinômio quadrático \(ax^{2}+bx+c\), a ordenada do vértice pode ser calculada por
\[
y_V = f\!\left(-\frac{b}{2a}\right).
\]
Aqui \(a=1\) e \(b=-18\), então a abscissa do vértice é
\[
x_V = -\frac{b}{2a} = -\frac{-18}{2\cdot 1} = \frac{18}{2}=9.
\]
Calcule \(y_V=f(9)\):
\[
y_V = 9^{2}-18\cdot 9 + 65 = 81 - 162 + 65 = -16.
\]
Logo,
\[
\boxed{\,y_V = -16\,.}
\]

\vspace{8pt}

{\color{blue}\textbf{(c) Confirme o resultado determinando primeiro } \(x_V\) e depois impondo que \(y_V\) seja imagem de \(x_V\) por \(f\).}

\medskip

\begin{itemize}
    \item Já calculamos \(x_V=9\).
    \item Agora avaliamos \(f(9)\) (repetindo o cálculo da forma indicada):
    \[
    f(9)=9^{2}-18\cdot 9 +65 = 81 -162 +65 = -16.
    \]
    \item Assim confirma-se que a ordenada do vértice é \(y_V=-16\), exatamente como em (b).
\end{itemize}

\bigskip
\hrule
\bigskip

\section*{Questão 3}
Uma função quadrática \(f:\mathbb{R}\to\mathbb{R}\) tem seu gráfico passando pelos pontos \((-3,0)\), \((11,0)\) e \((1,-80)\).

\medskip

{\color{blue}\textbf{(a) Determine essa função, em forma fatorada.}}

\medskip

\begin{itemize}
    \item Como a função anula em \(x=-3\) e \(x=11\), a forma fatorada é
    \[
    f(x) = k(x+3)(x-11),
    \]
    em que \(k\) é uma constante multiplicativa a determinar.
    \item Use o ponto \((1,-80)\) para achar \(k\):
    \[
    -80 = f(1) = k(1+3)(1-11) = k\cdot 4\cdot(-10) = k\cdot(-40).
    \]
    \item Logo \(k = \dfrac{-80}{-40} = 2\).
    \item Portanto a função é
    \[
    \boxed{\,f(x)=2(x+3)(x-11)\,}
    \]
    que, desenvolvida, é
    \[
    f(x)=2\bigl(x^{2}-8x-33\bigr)=2x^{2}-16x-66.
    \]
\end{itemize}

\vspace{8pt}

{\color{blue}\textbf{(b) Essa função tem valor máximo ou mínimo para seu conjunto imagem? Justifique e determine-o.}}

\medskip

\begin{itemize}
    \item O coeficiente quadrático é \(a=2>0\), portanto a parábola abre para cima e a função possui \emph{valor mínimo} (não máximo).
    \item A abscissa do vértice é
    \[
    x_V = -\frac{b}{2a} = -\frac{-16}{2\cdot 2} = \frac{16}{4}=4.
    \]
    \item A ordenada do vértice (valor mínimo) é
    \[
    y_V = f(4) = 2(4+3)(4-11) = 2\cdot 7\cdot(-7) = 2\cdot(-49) = -98.
    \]
    (Ou usando a forma desenvolvida: \(f(4)=2\cdot 16 -16\cdot 4 -66 =32 -64 -66 = -98\).)
    \item Portanto o valor mínimo é \(\boxed{\,y_{\min}=-98\,}\) atingido em \(x=4\).
\end{itemize}

\bigskip

\hrule

\bigskip
\noindent\textbf{Resumo das respostas:}
\begin{itemize}
    \item Questão 2: (a) \(5\le x\le 13\). (b) \(y_V=-16\). (c) Confirmação mostrando \(x_V=9\) e \(f(9)=-16\).
    \item Questão 3: (a) \(f(x)=2(x+3)(x-11)\). (b) Valor mínimo \(y_{\min}=-98\) em \(x=4\).
\end{itemize}

\section*{Questão 1}

\subsection*{a) \(\sqrt{21 - x} = 9 + x\)}

Primeiro, observamos o domínio:
\[
21 - x \ge 0 \quad \Longrightarrow \quad x \le 21.
\]
Além disso, o lado direito deve ser não negativo:
\[
9 + x \ge 0 \quad \Longrightarrow \quad x \ge -9.
\]
Logo, o domínio é:
\[
-9 \le x \le 21.
\]

Elevamos ambos os lados ao quadrado:
\[
21 - x = (9 + x)^2.
\]
Desenvolvendo:
\[
21 - x = 81 + 18x + x^2.
\]
Trazendo tudo para o mesmo lado:
\[
0 = x^2 + 19x + 60.
\]

Fatorando:
\[
x^2 + 19x + 60 = (x+4)(x+15).
\]

Assim:
\[
x = -4 \quad \text{ou} \quad x = -15.
\]

Agora testamos no domínio:

\(x=-4\):  
\(\sqrt{21-(-4)} = \sqrt{25}=5\) e \(9+(-4)=5\) 

\(x=-15\):  
\(\sqrt{21-(-15)}=\sqrt{36}=6\) mas \(9+(-15)=-6\) (negativo $\Rightarrow$ impossível)

Portanto, a solução é:
\[
\boxed{\{-4\}}.
\]

\bigskip

\subsection*{b) \(\sqrt{5x + 1} = \sqrt{4x - 3} + 1\)}

Domínios:
\[
5x + 1 \ge 0 \Rightarrow x \ge -\tfrac{1}{5},
\]
\[
4x - 3 \ge 0 \Rightarrow x \ge \tfrac{3}{4}.
\]
Portanto, o domínio é:
\[
x \ge \frac{3}{4}.
\]

Isolamos uma das raízes:
\[
\sqrt{5x + 1} - 1 = \sqrt{4x - 3}.
\]

Elevamos ao quadrado:
\[
(\sqrt{5x+1} - 1)^2 = 4x - 3.
\]

Desenvolvendo:
\[
(5x+1) - 2\sqrt{5x+1} + 1 = 4x - 3.
\]

Simplificando:
\[
5x + 2 - 2\sqrt{5x+1} = 4x - 3.
\]

\[
x + 5 = 2\sqrt{5x+1}.
\]

Elevando novamente ao quadrado:
\[
(x+5)^2 = 4(5x+1).
\]

Expansão:
\[
x^2 + 10x + 25 = 20x + 4.
\]

Reorganizando:
\[
x^2 - 10x + 21 = 0.
\]

Fatorando:
\[
x^2 - 10x + 21 = (x-3)(x-7).
\]

Logo:
\[
x = 3 \quad \text{ou} \quad x = 7.
\]

Ambos satisfazem o domínio \(x \ge \tfrac{3}{4}\).  
Testando:

- \(x=3\):  
LHS: \(\sqrt{5\cdot 3+1}=\sqrt{16}=4\)  
RHS: \(\sqrt{12-3}+1=3+1=4\) 

- \(x=7\):  
LHS: \(\sqrt{36}=6\)  
RHS: \(\sqrt{28-3}+1=5+1=6\)

Portanto:
\[
\boxed{\{3, 7\}}.
\]


\end{document}
