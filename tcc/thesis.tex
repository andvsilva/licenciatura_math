% document class thesis   
\documentclass[Portugues]{tese-IFCH}

%%%%%%%%%%%%%%%%%%%%%%%%%%%%%%%%%%%%%%%%%%%%%%%%%%
% These are some new commands that may be useful 
% for paper writing in general. If other new commands
% are needed for your specific paper, please feel 
% free to add here. 
%
% The currently available commands are organized in: 
% 1) Systems
% 2) Quantities
% 3) Energies and units
% 4) particle species
% 5) Colors package
% 6) hyperlink
%%%%%%%%%%%%%%%%%%%%%%%%%%%%%%%%%%%%%%%%%%%%%%%%%%

\usepackage{amsmath}
\usepackage{amssymb}
\usepackage{upgreek}
\usepackage{multirow}
\usepackage{setspace}% http://ctan.org/pkg/setspace
\usepackage{fancyhdr}
\usepackage{datetime}

% 1) SYSTEMS
\newcommand{\btc}               {\textbf{BTC}}
\newcommand{\btcspace}          {\textbf{BTC} }
\newcommand{\pow}               {\textbf{PoW}}

% 4) definition to references, biblatex and hyperlink
\usepackage[backend=bibtex, 
style=nature,  %style reference.
sorting=none,
firstinits=true %first name abbreviate
]{biblatex}

\usepackage{hyperref}
\hypersetup{
    colorlinks=true, %set "true" if you want colored links
    linktoc=all,     %set to "all" if you want both sections and subsections linked
    linkcolor=blue,  %choose some color if you want links to stand out
    citecolor= blue, % color of \cite{} in the text.
    urlcolor  = blue, % color of the link for the paper in references.
}

% 5) Tikz and figures
\usepackage{epsfig}
\usepackage{lmodern}
\usepackage{mathtools}
\usepackage[utf8]{luainputenc}
\usepackage{xspace}
\usepackage{tikz}
\usepackage{pgfplots}
\pgfplotsset{compat=newest}

\usetikzlibrary{positioning}
\usepackage{subcaption}

% 6) colors:
\usepackage{xcolor}
\definecolor{ao(english)}{rgb}{0.0, 0.5, 0.0} % dark green

% 7) Add lines numbers
%\usepackage{lineno}

% add pdf file to thesis:
\usepackage{pdfpages}

\usepackage{setspace}
\usepackage{fancyhdr}
\doublespacing

\usetikzlibrary{patterns}
\usetikzlibrary{plotmarks}
\usetikzlibrary{shapes.geometric}
\usetikzlibrary{angles}
\usepackage[portuguese]{babel}

\usetikzlibrary{external}
%\tikzexternalize[prefix=figures_tikz/]

\newcommand{\fake}[1]      {\textbf{\textcolor{red}{#1}}}
\newcommand{\inputtikz}[1]{%
  \tikzsetnextfilename{#1}%
  \input{#1.tex}%
}
\def \FigPath {./figures}

\addbibresource{bibliography.bib}

\newcommand{\printingbibliography}{%

    \pagestyle{myheadings}
    \markright{}
    \sloppy
    \printbibliography[heading=bibintoc, % add to table of contents
                   title=Bibliography % Chapter name
                  ]
    \fussy%
}

\begin{document}

 % Add lines numbers
%\linenumbers

% title
\title{\Large \textnormal{\textbf{Análise do Fluxo de Caixa e sua Importância na Gestão Financeira Empresarial}}}

% Author
\autor{\large André Vieira da Silva}
%\autora{Nome da Autora}


\titulo{\textnormal{\textbf{}}}

% Escolha entre orientador ou orientadora e inclua os títulos:
\orientador{}
%\orientadora{Profa. Dra. Nome da Orientadora}

% Escolha entre coorientador ou coorientadora, se houver, 
% e inclua os títulos:
%\coorientador{Prof. Dr. Eng. Lic. Nome do Co-Orientador}
%\coorientadora{Prof. Dra. Eng. Lic. Nome da Co-Orientadora}
 
% Escolha entre uma das quatro opções a seguir (comente as demais):
%\bsi         % para Trabalho de Conclusão de Curso em BSI
%\tads       % para Trabalho de Conclusão de Curso em TADS 
%\qualificacaoMestrado  % Para textos de qualificação de mestrado.
\qualificacaoDoutorado % Para textos de qualificação de doutorado.
%\mestrado   % para Dissertação de Mestrado em Tecnologia
%\doutorado  % para Tese de Doutorado em Tecnologia

%Defina a área de concentração. Se for TCC, deixe comentado
\areaConcentracao{Demografia}
%\areaConcentracao{Ambiente}
%\areaConcentracao{Ciência dos Materiais}

% Se houve cotutela, defina:
%\cotutela{Universidade Nova de Plutão}

%Defina a data da defesa no formato {Dia}{Mês}{Ano}
\datadadefesa{01}{12}{2023}

% Para a versão final defina:
% Repita o nome do Orientador(a) no primeiro avaliador
%\avaliadorA{avaliador Nome} 

% Para incluir a ficha catalográfica em PDF na versão final, 
% copie o arquivo PDF, descomente e ajuste a linha a seguir:
%\fichacatalografica{} 

% Este comando deve ficar aqui:
\paginasiniciais

% capa de aprovação da defesa
%\includepdf[pages=-]{aprovacao.pdf}

% Adicione no arquivo "agradecimentos.tex" os seus agradecimentos
% Caso prefira omitir os agradecimentos, comente a linha a seguir.
%%%%%%%%%%%%%%%%%%%%%%%%%%%%%%%%%%%%%%%%%%%%
%% Acknowledgements
%%%%%%%%%%%%%%%%%%%%%%%%%%%%%%%%%%%%%%%%%%%
\chapter*{Acknowledgements}\label{chp:Acknowledgements}
%\addcontentsline{toc}{chapter}{Acknowledgements}

\noindent
{\setstretch{1.5}
\hspace{0.635cm}...
}

%%%%%%%%%%%%%%%%%%%%%%%%%%%%%%%%%%%%%%%%%%%%%%%%%%%%%%%%%%%%%%%%%%%%%%%%%%%%%%%%
%% Chapter 1 -- Resumo
%%%%%%%%%%%%%%%%%%%%%%%%%%%%%%%%%%%%%%%%%%%%%%%%%%%%%%%%%%%%%%%%%%%%%%%%%%%%%%%%
\chapter*{\hspace{6.2cm}Resumo}\label{chp:resumo}
%\addcontentsline{toc}{chapter}{resumo}
\thispagestyle{headings}
% remove the number page for the first page of chapter.
\thispagestyle{empty} 

{\setstretch{1.5}
Em condições normais...

\textbf{Palavras-chave}: Cromodinâmica quântica...
}


%%%%%%%%%%%%%%%%%%%%%%%%%%%%%%%%%%%%%%%%%%%%%%%%%%%%%%%%%%%%%%%%%%%%%%%%%%%%%%%%
%% Chapter 1 -- Abstract
%%%%%%%%%%%%%%%%%%%%%%%%%%%%%%%%%%%%%%%%%%%%%%%%%%%%%%%%%%%%%%%%%%%%%%%%%%%%%%%%
\chapter*{\hspace{6.2cm}Abstract}\label{chp:abstract}
%\addcontentsline{toc}{chapter}{abstract}
\thispagestyle{headings}
% remove the number page for the first page of chapter.
\thispagestyle{empty} 

{\setstretch{1.5}
...
\textbf{Keywords}: ...
}


% A lista de figuras:
%\listoffigures

% A lista de tabelas:
%\listoftables

% adicionar abreviações e siglas no texto.
%\renewcommand{\nomname}{Lista de Abreviações e Siglas}
\printnomenclature[3cm]

% O sumário vem aqui:
\fancyfoot[L]{Thesis version \today\     \currenttime}
\tableofcontents

% E esta linha deve ficar bem aqui:
\fimdaspaginasiniciais

% O corpo da dissertação ou tese começa aqui:
%
% O comando a seguir inclui o arquivo introducao.tex
% que contém o capítulo de Introdução. 
% Detalhe: não precisa incluir a extensão .tex
%%%%%%%%%%%%%%%%%%%%%%%%%%%%%%%%
%% Chapter 1 -- Introduction
%%%%%%%%%%%%%%%%%%%%%%%%%%%%%%%%
\chapter{Introdução}\label{chp:introduction}
\thispagestyle{headings}
\hspace{0.7cm}O fluxo de caixa é uma ferramenta de gestão financeira que 
registra todas as entradas e saídas de dinheiro de uma organização em um 
determinado período de tempo. Ele monitora o fluxo de caixa da empresa, 
ou o dinheiro que entra e sai, permitindo que os gestores tenham uma visão 
clara e precisa da situação financeira da empresa.

A análise do fluxo de caixa é um método para analisar os dados do fluxo de 
caixa para encontrar padrões, tendências e oportunidades de melhoria. Como 
essa análise mostra como an empresa pode pagar dívidas, investir, gerar caixa 
e crescer, ela é importante para as decisões financeiras estratégicas.

É possível realizar uma análise do fluxo de caixa usando uma variedade de 
métricas, como:

\begin{itemize}
\item O dinheiro gerado pelas atividades operacionais da empresa, como vendas, 
despesas e custos, é chamado de fluxo de caixa operacional.
\item Fluxo de caixa de investimento: é o dinheiro que an empresa ganha com suas 
atividades de investimento, como comprar ativos financeiros e imobilizados.
\item Fluxo de caixa de financiamento: é o dinheiro que an empresa ganha com 
empréstimos, ações e outras atividades de financiamento.
\end{itemize}

Na complexa dança financeira que caracteriza o mundo empresarial, 
a análise do fluxo de caixa emerge como um confidente confiável, revelando os 
segredos das entradas e saídas de recursos de uma empresa. Ela não é apenas uma 
ferramenta numérica, mas sim um guia que nos leva pelas trilhas financeiras, 
desvendando histórias e proporcionando insights essenciais para a saúde do 
empreendimento.

Imagine o fluxo de caixa como um eco das transações diárias, um relato fiel das 
correntes monetárias que fluem dentro e fora do coração pulsante de uma organização. 
É nesse pulsar financeiro que reside a sua importância na gestão empresarial, 
moldando as decisões, desvendando desafios e, ao mesmo tempo, abrindo portas para 
oportunidades.

Esta análise vai além de números e gráficos; ela é um mergulho profundo nas águas 
da liquidez, permitindo que gestores entendam não apenas onde o dinheiro está,
mas também como ele flui. Nesse cenário, a humanização da análise do fluxo de caixa 
surge como uma necessidade, uma vez que cada cifrão representa não apenas valor 
monetário, mas o esforço, a dedicação e os sonhos de uma equipe.

Ao explorar o fluxo de caixa, estamos diante de mais do que simples transações 
financeiras. Estamos diante de histórias de superação, de estratégias bem-sucedidas e, 
por vezes, de desafios enfrentados com coragem. Entender esse fluxo não é apenas uma 
tarefa contábil; é uma jornada que nos permite compreender o pulso financeiro da empresa, 
reconhecendo suas virtudes e preparando-a para enfrentar as marés instáveis do mercado.

Neste contexto, esta análise não é apenas uma ferramenta, mas sim um parceiro na tomada 
de decisões. Um aliado que, ao revelar a capacidade de pagamento imediato, permite que 
gestores desenhem um futuro financeiro mais resiliente. A análise do fluxo de caixa, 
portanto, transcende o ambiente frio das finanças para se tornar um elemento vital 
na construção do sucesso empresarial, uma bússola que orienta as decisões com uma 
perspectiva humana, refletindo não apenas números, mas a essência e a trajetória de 
uma organização.

\section{Apresentação do problema}
\hspace{0.7cm}Company Bankuptcy Prediction

\section{Objetivos}
\section{Justificativa}

\section{Revisão de Literatura}

\subsection{Conceitos de matemática financeira}
\subsection{Importância do fluxo de caixa na gestão financeira}
\subsection{Métodos de análise de fluxo de caixa}
\subsection{Relação entre contabilidade financeira e fluxo de caixa}

% O comando a seguir inclui o arquivo revisao_bibliografica.tex
% que contém o capítulo de levantamento bibliográfico. 
% Detalhe: não precisa incluir a extensão .tex
%%%%%%%%%%%%%%%%%%%%%%%%%%%%%%%%%%%%%%%%%%%%%%%%%%%%%%%%%%%%%%%
%% Chapter 4 -- Methodology
%%%%%%%%%%%%%%%%%%%%%%%%%%%%%%%%%%%%%%%%%%%%%%%%%%%%%%%%%%%%%%%
\chapter{Metodologia}\label{chp:Methodology}
%\addcontentsline{toc}{Analysis Methodology}

\section{Descrição da metodologia de pesquisa}
\section{Coleta de dados}
\section{Análise de dados}

% O comando a seguir inclui o arquivo metodologia.tex
% que contém o capítulo de metodologia. 
% Detalhe: não precisa incluir a extensão .tex
%\include{chain}

% O comando a seguir inclui o arquivo analise.tex
% que contém o capítulo de analise. 
% Detalhe: não precisa incluir a extensão .tex
%%%%%%%%%%%%%%%%%%%%%%%%%%%%%%%%%%%%%%%%%%%%%%%%%%%%%%%%%%%%%%%
%% Chapter 4 -- Analysis Methodology
%%%%%%%%%%%%%%%%%%%%%%%%%%%%%%%%%%%%%%%%%%%%%%%%%%%%%%%%%%%%%%%
\chapter{Anal\'ise}\label{chp:analysis}
%\addcontentsline{toc}{Analysis Methodology}

\noindent
\hspace{0.635cm}


\section{Estudo de Caso}

\subsection{Aplicação dos conceitos em um caso real de uma empresa}
\subsection{Análise do fluxo de caixa e sua relação com o desempenho financeiro}

% O comando a seguir inclui o arquivo resultados.tex
% que contém o capítulo de resultados. 
% Detalhe: não precisa incluir a extensão .tex
%\include{results}

% O comando a seguir inclui o arquivo conclusoes.tex
% que contém o capítulo de conclusoes. 
% Detalhe: não precisa incluir a extensão .tex
%%%%%%%%%%%%%%%%%%%%%%%%%%%%%%%%%%%%%%%%%%%%%%%%%%%%%%%%%%%%%%%
%% Chapter 4 -- Analysis Methodology
%%%%%%%%%%%%%%%%%%%%%%%%%%%%%%%%%%%%%%%%%%%%%%%%%%%%%%%%%%%%%%%
\chapter{Conclusão}\label{chp:conclusion}
%\addcontentsline{toc}{Analysis Methodology}

\section{Considerações Finais}

\subsection{Resumo das principais conclusões}

\subsection{Sugestões para futuras pesquisas}

%%%%%%%% Bibliography 
% Os comandos para incluir as referências bibliográficas
\printingbibliography

% Os anexos, se houver, vêm depois das referências:
\appendix


\end{document}