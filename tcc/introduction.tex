%%%%%%%%%%%%%%%%%%%%%%%%%%%%%%%%
%% Chapter 1 -- Introduction
%%%%%%%%%%%%%%%%%%%%%%%%%%%%%%%%
\chapter{Introdução}\label{chp:introduction}
\thispagestyle{headings}
\hspace{0.7cm}O fluxo de caixa é uma ferramenta de gestão financeira que 
registra todas as entradas e saídas de dinheiro de uma organização em um 
determinado período de tempo. Ele monitora o fluxo de caixa da empresa, 
ou o dinheiro que entra e sai, permitindo que os gestores tenham uma visão 
clara e precisa da situação financeira da empresa.

A análise do fluxo de caixa é um método para analisar os dados do fluxo de 
caixa para encontrar padrões, tendências e oportunidades de melhoria. Como 
essa análise mostra como an empresa pode pagar dívidas, investir, gerar caixa 
e crescer, ela é importante para as decisões financeiras estratégicas.

É possível realizar uma análise do fluxo de caixa usando uma variedade de 
métricas, como:

\begin{itemize}
\item O dinheiro gerado pelas atividades operacionais da empresa, como vendas, 
despesas e custos, é chamado de fluxo de caixa operacional.
\item Fluxo de caixa de investimento: é o dinheiro que an empresa ganha com suas 
atividades de investimento, como comprar ativos financeiros e imobilizados.
\item Fluxo de caixa de financiamento: é o dinheiro que an empresa ganha com 
empréstimos, ações e outras atividades de financiamento.
\end{itemize}

Na complexa dança financeira que caracteriza o mundo empresarial, 
a análise do fluxo de caixa emerge como um confidente confiável, revelando os 
segredos das entradas e saídas de recursos de uma empresa. Ela não é apenas uma 
ferramenta numérica, mas sim um guia que nos leva pelas trilhas financeiras, 
desvendando histórias e proporcionando insights essenciais para a saúde do 
empreendimento.

Imagine o fluxo de caixa como um eco das transações diárias, um relato fiel das 
correntes monetárias que fluem dentro e fora do coração pulsante de uma organização. 
É nesse pulsar financeiro que reside a sua importância na gestão empresarial, 
moldando as decisões, desvendando desafios e, ao mesmo tempo, abrindo portas para 
oportunidades.

Esta análise vai além de números e gráficos; ela é um mergulho profundo nas águas 
da liquidez, permitindo que gestores entendam não apenas onde o dinheiro está,
mas também como ele flui. Nesse cenário, a humanização da análise do fluxo de caixa 
surge como uma necessidade, uma vez que cada cifrão representa não apenas valor 
monetário, mas o esforço, a dedicação e os sonhos de uma equipe.

Ao explorar o fluxo de caixa, estamos diante de mais do que simples transações 
financeiras. Estamos diante de histórias de superação, de estratégias bem-sucedidas e, 
por vezes, de desafios enfrentados com coragem. Entender esse fluxo não é apenas uma 
tarefa contábil; é uma jornada que nos permite compreender o pulso financeiro da empresa, 
reconhecendo suas virtudes e preparando-a para enfrentar as marés instáveis do mercado.

Neste contexto, esta análise não é apenas uma ferramenta, mas sim um parceiro na tomada 
de decisões. Um aliado que, ao revelar a capacidade de pagamento imediato, permite que 
gestores desenhem um futuro financeiro mais resiliente. A análise do fluxo de caixa, 
portanto, transcende o ambiente frio das finanças para se tornar um elemento vital 
na construção do sucesso empresarial, uma bússola que orienta as decisões com uma 
perspectiva humana, refletindo não apenas números, mas a essência e a trajetória de 
uma organização.

\section{Apresentação do problema}
\hspace{0.7cm}Company Bankuptcy Prediction

\section{Objetivos}
\section{Justificativa}

\section{Revisão de Literatura}

\subsection{Conceitos de matemática financeira}
\subsection{Importância do fluxo de caixa na gestão financeira}
\subsection{Métodos de análise de fluxo de caixa}
\subsection{Relação entre contabilidade financeira e fluxo de caixa}